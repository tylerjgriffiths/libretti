\begin{drama}
  \item
  \scene

\StageDir{Auf dem Grunde des Rheines. Grünliche Dämmerung, nach oben zu lichter, nach unten zu dunkler. Die Höhe ist von wogendem Gewässer erfüllt, das rastlos von rechts nach links zu strömt. Nach der Tiefe zu lösen die Fluten sich in einen immer feineren feuchten Nebel auf, so daß der Raum in Manneshöhe vom Boden auf gänzlich frei vom Wasser zu sein scheint, welches wie in Wolkenzügen über den nächtlichen Grund dahinfließt. Überall ragen schroffe Felsenriffe aus der Tiefe auf und grenzen den Raum der Bühne ab; der ganze Boden ist in ein wildes Zackengewirr zerspalten, so daß er nirgends vollkommen eben ist und nach allen Seiten hin in dichtester Finsternis tiefere Schlüfte annehmen läßt.
\\ Um ein Riff in der Mitte der Bühne, welches mit seiner schlanken Spitze bis in die dichtere, heller dämmernde Wasserflut hinaufragt, kreist in anmutig schwimmender Bewegung eine der Rheintöchter}

\Woglindespeaks
Weia! Waga! Woge, du Welle,
walle zur Wiege! Wagalaweia!
Wallala, weiala weia!
 

\Wellgundespeaks

\direct{Stimme von oben}

Woglinde, wachst du allein?
 

\Woglindespeaks
Mit Wellgunde wär' ich zu zwei.
 

\Wellgundespeaks

\direct{taucht aus der Flut zum Riff herab}

Laß sehn, wie du wachst!
 


\direct{sie sucht Woglinde zu erhaschen}


\Woglindespeaks

\direct{entweicht ihr schwimmend}

Sicher vor dir!
 


\direct{sie necken sich und suchen sich spielend zu fangen}



\Flosshildespeaks

\direct{Stimme von oben}

Heiaha weia! Wildes Geschwister!
 

\Wellgundespeaks
Flosshilde, schwimm'! Woglinde flieht:
hilf mir die Fließende fangen!
 

\Flosshildespeaks

\direct{taucht herab und fährt zwischen die Spielenden}

Des Goldes Schlaf hütet ihr schlecht!
Besser bewacht des schlummernden Bett,
sonst büßt ihr beide das Spiel!
 

\StageDir{Mit muntrem Gekreisch fahren die beiden auseinander. Flosshilde sucht bald die eine, bald die andere zu erhaschen; sie entschlüpfen ihr und vereinigen sich endlich, um gemeinschaftlich auf Flosshilde Jagd zu machen. So schnellen sie gleich Fischen von Riff zu Riff, scherzend und lachend.
  
Aus einer finstern Schluft ist währenddem Alberich, an einem Riffe klimmend, dem Abgrunde entstiegen. Er hält, noch vom Dunkel umgeben, an und schaut dem Spiele der Rheintöchter mit steigendem Wohlgefallen zu.}

\Alberichspeaks
Hehe! Ihr Nicker!
Wie seid ihr niedlich, neidliches Volk!
Aus Nibelheims Nacht naht' ich mich gern,
neigtet ihr euch zu mir!
 


\direct{die Mädchen halten, sobald sie Alberichs Stimme hören, mit dem Spiele ein}


\Woglindespeaks
Hei! Wer ist dort?


\Wellgundespeaks
Es dämmert und ruft!
 

\Flosshildespeaks
Lugt, wer uns lauscht!
 

\speaker{\Woglinde und \Wellgunde}

\direct{sie tauchen tiefer herab und erkennen den Nibelung}

Pfui! Der Garstige!
 

\Flosshildespeaks

\direct{schnell auftauchend}

Hütet das Gold!
Vater warnte vor solchem Feind.
 


\direct{Die beiden andern folgen ihr, und alle drei versammeln sich schnell um das mittlere Riff}


\Alberichspeaks
Ihr, da oben!
 

\speaker{Die drei Rheintöchter}
Was willst du dort unten?
 

\Alberichspeaks
Stör' ich eu'r Spiel,
wenn staunend ich still hier steh'?
Tauchtet ihr nieder, mit euch tollte
und neckte der Niblung sich gern!
 

\Woglindespeaks
Mit uns will er spielen?
 

\Wellgundespeaks
Ist ihm das Spott?
 

\Alberichspeaks
Wie scheint im Schimmer ihr hell und schön!
Wie gern umschlänge der Schlanken eine mein Arm,
schlüpfte hold sie herab!
 

\Flosshildespeaks
Nun lach' ich der Furcht: der Feind ist verliebt!
 


\direct{Sie lachen}


\Wellgundespeaks
Der lüsterne Kauz!
 

\Woglindespeaks
Laßt ihn uns kennen!
 


\direct{Sie läßt sich auf die Spitze des Riffes hinab, an dessen Fuße Alberich angelangt ist}


\Alberichspeaks
Die neigt sich herab.
 

\Woglindespeaks
Nun nahe dich mir!
 


\direct{Alberich klettert mit koboldartiger Behendigkeit, doch wiederholt aufgehalten, der Spitze des Riffes zu}


\Alberichspeaks
Garstig glatter glitschiger Glimmer!
Wie gleit' ich aus! Mit Händen und Füßen
nicht fasse noch halt' ich das schlecke Geschlüpfer!
 


\direct{er prustet}

Feuchtes Naß füllt mir die Nase:
verfluchtes Niesen!
 


\direct{er ist in Woglindes Nähe angelangt}


\Woglindespeaks

\direct{lachend}

Prustend naht meines Freiers Pracht!
 

\Alberichspeaks
Mein Friedel sei, du fräuliches Kind!
 


\direct{er sucht sie zu umfassen}


\Woglindespeaks

\direct{sich ihm entwindend}

Willst du mich frei'n, so freie mich hier!
 


\direct{sie taucht auf einem andern Riff auf, die Schwestern lachen}


\Alberichspeaks

\direct{kratzt sich den Kopf}

O weh! Du entweichst? Komm' doch wieder!
Schwer ward mir, was so leicht du erschwingst.
 

\Woglindespeaks

\direct{schwingt sich auf ein drittes Riff in größerer Tiefe}

Steig' nur zu Grund, da greifst du mich sicher!
 

\Alberichspeaks

\direct{hastig hinab kletternd}

Wohl besser da unten!
 

\Woglindespeaks

\direct{schnellt sich rasch aufwärts nach einem hohen Seitenriffe}

Nun aber nach oben!
 

\speaker{\Wellgunde und \Flosshilde}
Hahahahaha!
 

\Alberichspeaks
Wie fang' ich im Sprung den spröden Fisch?
Warte, du Falsche!
 


\direct{er will ihr eilig nachklettern}


\Wellgundespeaks

\direct{hat sich auf ein tieferes Riff auf der anderen Seite gesenkt}

Heia, du Holder! Hörst du mich nicht?
 

\Alberichspeaks

\direct{sich umwendend}

Rufst du nach mir?
 

\Wellgundespeaks
Ich rate dir wohl: zu mir wende dich,
Woglinde meide!
 

\Alberichspeaks

\direct{klettert hastig über den Bodengrund zu Wellgunde}

Viel schöner bist du als jene Scheue,
die minder gleißend und gar zu glatt.
Nur tiefer tauche, willst du mir taugen.
 

\Wellgundespeaks

\direct{noch etwas mehr sich zu ihm herabsenkend}

Bin nun ich dir nah?
 

\Alberichspeaks
Noch nicht genug!
Die schlanken Arme schlinge um mich,
daß ich den Nacken dir neckend betaste,
mit schmeichelnder Brunst
an die schwellende Brust mich dir schmiege.
 

\Wellgundespeaks
Bist du verliebt und lüstern nach Minne,
laß sehn, du Schöner, wie bist du zu schau'n?
Pfui! Du haariger, höckriger Geck!
Schwarzes, schwieliges Schwefelgezwerg!
Such' dir ein Friedel, dem du gefällst!
 

\Alberichspeaks

\direct{sucht sie mit Gewalt zu halten}

Gefall' ich dir nicht, dich fass' ich doch fest!
 

\Wellgundespeaks

\direct{schnell zum mittleren Riffe auftauchend}

Nur fest, sonst fließ ich dir fort!
 

\speaker{\Woglinde und \Flosshilde}
Hahahahaha!
 

\Alberichspeaks

\direct{Wellgunde erbost nachzankend}

Falsches Kind! Kalter, grätiger Fisch!
Schein' ich nicht schön dir,
niedlich und neckisch, glatt und glau---
hei, so buhle mit Aalen, ist dir eklig mein Balg!
 

\Flosshildespeaks
Was zankst du, Alp? Schon so verzagt?
Du freitest um zwei: frügst du die dritte,
süßen Trost schüfe die Traute dir!
 

\Alberichspeaks
Holder Sang singt zu mir her!
Wie gut, daß ihr eine nicht seid!
Von vielen gefall' ich wohl einer:
bei einer kieste mich keine!
Soll ich dir glauben, so gleite herab!
 

\Flosshildespeaks

\direct{taucht zu Alberich hinab}

Wie törig seid ihr, dumme Schwestern,
dünkt euch dieser nicht schön!
 

\Alberichspeaks

\direct{ihr nahend}

Für dumm und häßlich darf ich sie halten,
seit ich dich Holdeste seh'.
 

\Flosshildespeaks

\direct{schmeichelnd}

O singe fort so süß und fein,
wie hehr verführt es mein Ohr!
 

\Alberichspeaks

\direct{zutraulich sie berührend}

Mir zagt, zuckt und zehrt sich das Herz,
lacht mir so zierliches Lob.
 

\Flosshildespeaks

\direct{ihn sanft abwehrend}

Wie deine Anmut mein Aug' erfreut,
deines Lächelns Milde den Mut mir labt!
 


\direct{Sie zieht ihn selig an sich}

Seligster Mann!
 

\Alberichspeaks
Süßeste Maid!
 

\Flosshildespeaks
Wärst du mir hold!
 

\Alberichspeaks
Hielt dich immer!
 

\Flosshildespeaks

\direct{ihn ganz in ihren Armen haltend}

Deinen stechenden Blick, deinen struppigen Bart,
o säh ich ihn, faßt' ich ihn stets!
Deines stachligen Haares strammes Gelock,
umflöß es Flosshilde ewig!
Deine Krötengestalt, deiner Stimme Gekrächz,
o dürft' ich staunend und stumm
sie nur hören und sehn!
 

\speaker{\Woglinde und \Wellgunde}
Hahahahaha!
 

\Alberichspeaks

\direct{erschreckt aus Flosshildes Armen auffahrend}

Lacht ihr Bösen mich aus?
 

\Flosshildespeaks

\direct{sich plötzlich ihm entreissend}

Wie billig am Ende vom Lied!
 


\direct{sie taucht mit den Schwestern schnell auf}


\speaker{\Woglinde und \Wellgunde}
Hahahahaha!
 

\Alberichspeaks

\direct{mit kreischender Stimme}

Wehe! Ach wehe! O Schmerz! O Schmerz!
Die dritte, so traut, betrog sie mich auch?
Ihr schmählich schlaues, lüderlich schlechtes Gelichter!
Nährt ihr nur Trug, ihr treuloses Nickergezücht?
 

\speaker{Die drei Rheintöchter}
Wallala! Lalaleia! Leialalei!
Heia! Heia! Haha!
Schäme dich, Albe! Schilt nicht dort unten!
Höre, was wir dich heißen!
Warum, du Banger, bandest du nicht
das Mädchen, das du minnst?
Treu sind wir und ohne Trug
dem Freier, der uns fängt.
Greife nur zu, und grause dich nicht!
In der Flut entflieh'n wir nicht leicht!
Wallala! Lalaleia! Leialalei!
Heia! Heia! Haha!
 


\direct{Sie schwimmen auseinander, hierher und dorthin, bald tiefer, bald höher, um Alberich zur Jagd auf sie zu reizen}


\Alberichspeaks
Wie in den Gliedern brünstige Glut
mir brennt und glüht!
Wut und Minne, wild und mächtig,
wühlt mir den Mut auf!
Wie ihr auch lacht und lügt,
lüstern lechz' ich nach euch,
und eine muß mir erliegen!
 


\direct{Er macht sich mit verzweifelter Anstrengung zur Jagd auf: mit grauenhafter Behendigkeit erklimmt er Riff für Riff, springt von einem zum andern, sucht bald dieses, bald jenes der Mädchen zu erhaschen, die mit lustigem Gekreisch stets ihm entweichen. Er strauchelt, stürzt in den Abgrund hinab, klettert dann hastig wieder in die Höhe zu neuer Jagd. Sie neigen sich etwas herab. Fast erreicht er sie, stürzt abermals zurück und versucht es nochmals. Er hält endlich, vor Wut schäumend, atemlos an und streckt die geballte Faust nach den Mädchen hinauf.}


\Alberichspeaks

\direct{kaum seiner mächtig}

Fing' eine diese Faust!...
 


\direct{Er verbleibt in sprachloser Wut, den Blick aufwärts gerichtet, wo er dann plötzlich von dem folgenden Schauspiele angezogen und gefesselt wird. Durch die Flut ist von oben her ein immer lichterer Schein gedrungen, der sich an einer hohen Stelle des mittelsten Riffes allmählich zu einem blendend hell strahlenden Goldglanze entzündet: ein zauberisch goldenes Licht bricht von hier durch das Wasser}


\Woglindespeaks
Lugt, Schwestern!
Die Weckerin lacht in den Grund.
 

\Wellgundespeaks
Durch den grünen Schwall
den wonnigen Schläfer sie grüßt.
 

\Flosshildespeaks
Jetzt küßt sie sein Auge, daß er es öffne.
 

\Wellgundespeaks
Schaut, er lächelt in lichtem Schein.
 

\Woglindespeaks
Durch die Fluten hin fließt sein strahlender Stern!
 

\speaker{Die drei Rheintöchter}

\direct{zusammen das Riff anmutig umschwimmend}

Heiajaheia! Heiajaheia!
Wallalalalala leiajahei!
Rheingold! Rheingold!
Leuchtende Lust, wie lachst du so hell und hehr!
Glühender Glanz entgleißet dir weihlich im Wag'!
Heiajaheia! Heiajaheia!
Wache, Freund, Wache froh!
Wonnige Spiele spenden wir dir:
flimmert der Fluß, flammet die Flut,
umfließen wir tauchend, tanzend und singend
im seligem Bade dein Bett!
Rheingold! Rheingold!
Heiajaheia! Wallalalalala leiajahei!
 


\direct{Mit immer ausgelassenerer Lust umschwimmen die Mädchen das Riff. Die ganze Flut flimmert in hellem Goldglanze}


\Alberichspeaks

\direct{dessen Augen, mächtig vom Glanze angezogen, starr an dem Golde haften}

Was ist's, ihr Glatten, das dort so glänzt und gleißt?
 

\speaker{Die drei Rheintöchter}
Wo bist du Rauher denn heim,
daß vom Rheingold nie du gehört?
 

\Wellgundespeaks
Nichts weiß der Alp von des Goldes Auge,
das wechselnd wacht und schläft?
 

\Woglindespeaks
Von der Wassertiefe wonnigem Stern,
der hehr die Wogen durchhellt?
 

\speaker{Die drei Rheintöchter}
Sieh, wie selig im Glanze wir gleiten!
Willst du Banger in ihm dich baden,
so schwimm' und schwelge mit uns!
Wallalalala leialalai! Wallalalala leiajahei!
 

\Alberichspeaks
Eurem Taucherspiele nur taugte das Gold?
Mir gält' es dann wenig!
 

\Woglindespeaks
Des Goldes Schmuck schmähte er nicht,
wüßte er all seine Wunder!
 

\Wellgundespeaks
Der Welt Erbe gewänne zu eigen,
wer aus dem Rheingold schüfe den Ring,
der maßlose Macht ihm verlieh'.
 

\Flosshildespeaks
Der Vater sagt' es, und uns befahl er,
klug zu hüten den klaren Hort,
daß kein Falscher der Flut ihn entführe:
drum schweigt, ihr schwatzendes Heer!
 

\Wellgundespeaks
Du klügste Schwester, verklagst du uns wohl?
Weißt du denn nicht, wem nur allein
das Gold zu schmieden vergönnt?
 

\Woglindespeaks
Nur wer der Minne Macht entsagt,
nur wer der Liebe Lust verjagt,
nur der erzielt sich den Zauber,
zum Reif zu zwingen das Gold.
 

\Wellgundespeaks
Wohl sicher sind wir und sorgenfrei:
denn was nur lebt, will lieben,
meiden will keiner die Minne.
 

\Woglindespeaks
Am wenigsten er, der lüsterne Alp;
vor Liebesgier möcht' er vergehn!
 

\Flosshildespeaks
Nicht fürcht' ich den, wie ich ihn erfand:
seiner Minne Brunst brannte fast mich.
 

\Wellgundespeaks
Ein Schwefelbrand in der Wogen Schwall:
vor Zorn der Liebe zischt er laut!
 

\speaker{Die drei Rheintöchter}
Wallala! Wallaleialala!
Lieblichster Albe! Lachst du nicht auch?
In des Goldes Scheine wie leuchtest du schön!
O komm', Lieblicher, lache mit uns!
Heiajaheia! Heiajaheia! Wallalalala leiajahei!
 


\direct{Sie schwimmen lachend im Glanze auf und ab}


\Alberichspeaks

\direct{die Augen starr auf das Gold gerichtet, hat dem Geplauder der Schwestern wohl gelauscht}

Der Welt Erbe
gewänn' ich zu eigen durch dich?
Erzwäng' ich nicht Liebe,
doch listig erzwäng' ich mir Lust?
 


\direct{furchtbar laut}

Spottet nur zu!
Der Niblung naht eurem Spiel!
 


\direct{wütend springt er nach dem mittleren Riff hinüber und klettert in grausiger Hast nach dessen Spitze hinauf. Die Mädchen fahren kreischend auseinander und tauchen nach verschiedenen Seiten hin auf}


Die drei Rheintöchter
Heia! Heia! Heiajahei!
Rettet euch! Es raset der Alp:
in den Wassern sprüht's, wohin er springt:
die Minne macht ihn verrückt!
 


\direct{sie lachen im tollsten Übermut}


\Alberichspeaks

\direct{gelangt mit einem letzten Satze zur Spitze des Riffes}

Bangt euch noch nicht?
So buhlt nun im Finstern, feuchtes Gezücht!
 


\direct{er streckt die Hand nach dem Golde aus}

Das Licht lösch' ich euch aus, entreiße dem Riff das Gold,
schmiede den rächende Ring;
denn hör' es die Flut: so verfluch' ich die Liebe!
 


\direct{Er reißt mit furchtbarer Gewalt das Gold aus dem Riffe und stürzt damit hastig in die Tiefe, wo er schnell verschwindet. Dichte Nacht bricht plötzlich überall herein. Die Mädchen tauchen dem Räuber in die Tiefe nach}


\Flosshildespeaks
Haltet den Räuber!
 

\Wellgundespeaks
Rettet das Gold!
 

\speaker{\Woglinde und \Wellgunde}
Hilfe! Hilfe!
 

\speaker{Die drei Rheintöchter}
Weh! Weh!
 

\StageDir{Die Flut fällt mit ihnen nach der Tiefe hinab. Aus dem untersten Grunde hört man Alberichs gellendes Hohngelächter. In dichtester Finsternis verschwinden die Riffe; die ganze Bühne ist von der Höhe bis zur Tiefe von schwarzem Wassergewoge erfüllt, das eine Zeitlang immer noch abwärts zu sinken scheint.}

\scene

\StageDir{Allmählich sind die Wogen in Gewölke übergegangen, welches, als eine immer heller dämmernde Beleuchtung dahinter tritt, zu feinerem Nebel sich abklärt. Als der Nebel in zarten Wölkchen gänzlich sich in der Höhe verliert, wird im Tagesgrauen eine
freie Gegend auf Bergeshöhen
sichtbar. Der hervorbrechende Tag beleuchtet mit wachsendem Glanze eine Burg mit blinkenden Zinnen, die auf einem Felsgipfel im Hintergrunde steht; zwischen diesem burggekrönten Felsgipfel und dem Vordergrunde der Szene ist ein tiefes Tal, durch welches der Rhein fließt, anzunehmen. Zur Seite auf blumigem Grunde liegt Wotan, neben ihm Fricka, beide schlafend. Die Burg ist ganz sichtbar geworden.}

\Frickaspeaks

\direct{erwacht; ihr Blick fällt auf die Burg; sie staunt und erschrickt}

Wotan, Gemahl, erwache!
 

\Wotanspeaks

\direct{im Traume leise}

Der Wonne seligen Saal
bewachen mir Tür und Tor:
Mannes Ehre, ewige Macht,
ragen zu endlosem Ruhm!
 

\Frickaspeaks

\direct{rüttelt ihn}

Auf, aus der Träume wonnigem Trug!
Erwache, Mann, und erwäge!
 

\Wotanspeaks

\direct{erwacht und erhebt sich ein wenig, sein Auge wird sogleich vom Anblick der Burg gefesselt}

Vollendet das ewige Werk!
Auf Berges Gipfel die Götterburg;
prächtig prahlt der prangende Bau!
Wie im Traum ich ihn trug,
wie mein Wille ihn wies, stark und schön
steht er zur Schau; hehrer, herrlicher Bau!
 

\Frickaspeaks
Nur Wonne schafft dir, was mich erschreckt?
Dich freut die Burg, mir bangt es um Freia!
Achtloser, laß mich erinnern
des ausbedungenen Lohns!
Die Burg ist fertig, verfallen das Pfand:
vergaßest du, was du vergabst?
 

\Wotanspeaks
Wohl dünkt mich's, was sie bedangen,
die dort die Burg mir gebaut;
durch Vertrag zähmt' ich ihr trotzig Gezücht,
daß sie die hehre Halle mir schüfen;
die steht nun, dank den Starken:
um den Sold sorge dich nicht.
 

\Frickaspeaks
O lachend frevelnder Leichtsinn!
Liebelosester Frohmut!
Wußt' ich um euren Vertrag,
dem Truge hätt' ich gewehrt;
doch mutig entferntet ihr Männer die Frauen,
um taub und ruhig vor uns,
allein mit den Riesen zu tagen:
so ohne Scham verschenktet ihr Frechen
Freia, mein holdes Geschwister,
froh des Schächergewerbs!
Was ist euch Harten doch heilig und wert,
giert ihr Männer nach Macht!
 

\Wotanspeaks

\direct{ruhig}

Gleiche Gier war Fricka wohl fremd,
als selbst um den Bau sie mich bat?
 

\Frickaspeaks
Um des Gatten Treue besorgt,
muß traurig ich wohl sinnen,
wie an mich er zu fesseln,
zieht's in die Ferne ihn fort:
herrliche Wohnung, wonniger Hausrat
sollten dich binden zu säumender Rast.
Doch du bei dem Wohnbau sannst auf Wehr und Wall allein;
Herrschaft und Macht soll er dir mehren;
nur rastlosern Sturm zu erregen,
erstand dir die ragende Burg.
 

\Wotanspeaks

\direct{lächelnd}

Wolltest du Frau in der Feste mich fangen,
mir Gotte mußt du schon gönnen,
daß, in der Burg gebunden, ich mir
von außen gewinne die Welt.
Wandel und Wechsel liebt, wer lebt;
das Spiel drum kann ich nicht sparen!
 

\Frickaspeaks
Liebeloser, leidigster Mann!
Um der Macht und Herrschaft müßigen Tand
verspielst du in lästerndem Spott
Liebe und Weibes Wert?
 

\Wotanspeaks

\direct{ernst}

Um dich zum Weib zu gewinnen,
mein eines Auge setzt' ich werbend daran;
wie törig tadelst du jetzt!
Ehr' ich die Frauen doch mehr als dich freut;
und Freia, die gute, geb' ich nicht auf;
nie sann dies ernstlich mein Sinn.
 

\Frickaspeaks

\direct{mit ängstlicher Spannung in die Szene blickend}

So schirme sie jetzt: in schutzloser Angst
läuft sie nach Hilfe dort her!
 

\Freiaspeaks

\direct{tritt wie in hastiger Flucht auf}

Hilf mir, Schwester! Schütze mich, Schwäher!
Vom Felsen drüben drohte mir Fasolt,
mich Holde käm' er zu holen.
 

\Wotanspeaks
Laß ihn droh'n! Sahst du nicht Loge?
 

\Frickaspeaks
Daß am liebsten du immer dem Listigen traust!
Viel Schlimmes schuf er uns schon,
doch stets bestrickt er dich wieder.
 

\Wotanspeaks
Wo freier Mut frommt,
allein frag' ich nach keinem.
Doch des Feindes Neid zum Nutz sich fügen,
lehrt nur Schlauheit und List,
wie Loge verschlagen sie übt.
Der zum Vertrage mir riet,
versprach mir, Freia zu lösen:
auf ihn verlass' ich mich nun.
 

\Frickaspeaks
Und er läßt dich allein!
Dort schreiten rasch die Riesen heran:
wo harrt dein schlauer Gehilf'?
 

\Freiaspeaks
Wo harren meine Brüder, daß Hilfe sie brächten,
da mein Schwäher die Schwache verschenkt?
Zu Hilfe, Donner! Hieher, hieher!
Rette Freia, mein Froh!
 

\Frickaspeaks
Die in bösem Bund dich verrieten,
sie alle bergen sich nun!
 


\direct{Fasolt und Fafner, beide in riesiger Gestalt, mit starken Pfählen bewaffnet, treten auf}


\Fasoltspeaks
Sanft schloß Schlaf dein Aug';
wir beide bauten Schlummers bar die Burg.
Mächt'ger Müh' müde nie,
stauten starke Stein' wir auf;
steiler Turm, Tür und Tor,
deckt und schließt im schlanken Schloß den Saal.
 


\direct{Auf die Burg deutend}

Dort steht's, was wir stemmten,
schimmernd hell, bescheint's der Tag:
zieh nun ein, uns zahl' den Lohn!
 

\Wotanspeaks
Nennt, Leute, den Lohn:
was dünkt euch zu bedingen?
 

\Fasoltspeaks
Bedungen ist, was tauglich uns dünkt:
gemahnt es dich so matt?
Freia, die Holde, Holda, die Freie,
vertragen ist's, sie tragen wir heim.
 

\Wotanspeaks

\direct{schnell}

Seid ihr bei Trost mit eurem Vertrag?
Denkt auf andern Dank: Freia ist mir nicht feil.
 

\Fasoltspeaks

\direct{steht, in höchster Bestürzung, einen Augenblick sprachlos}

Was sagst du? Ha, sinnst du Verrat?
Verrat am Vertrag? Die dein Speer birgt,
sind sie dir Spiel, des berat'nen Bundes Runen?
 

\Fafnerspeaks

\direct{höhnisch}

Getreu'ster Bruder,
merkst du Tropf nun Betrug?
 

\Fasoltspeaks
Lichtsohn du, leicht gefügter!
Hör' und hüte dich: Verträgen halte Treu'!
Was du bist, bist du nur durch Verträge;
bedungen ist, wohl bedacht deine Macht.
Bist weiser du, als witzig wir sind,
bandest uns Freie zum Frieden du:
all deinem Wissen fluch' ich,
fliehe weit deinen Frieden,
weißt du nicht offen, ehrlich und frei
Verträgen zu wahren die Treu'!
Ein dummer Riese rät dir das:
Du Weiser, wiss' es von ihm.
 

\Wotanspeaks
Wie schlau für Ernst du achtest,
was wir zum Scherz nur beschlossen!
Die liebliche Göttin, licht und leicht,
was taugt euch Tölpeln ihr Reiz?
 

\Fasoltspeaks
Höhnst du uns? Ha, wie unrecht!
Die ihr durch Schönheit herrscht,
schimmernd hehres Geschlecht,
wir törig strebt ihr nach Türmen von Stein,
setzt um Burg und Saal
Weibes Wonne zum Pfand!
Wir Plumpen plagen uns
schwitzend mit schwieliger Hand,
ein Weib zu gewinnen, das wonnig und mild
bei uns Armen wohne;
und verkehrt nennst du den Kauf?
 

\Fafnerspeaks
Schweig' dein faules Schwatzen,
Gewinn werben wir nicht:
Freias Haft hilft wenig,
doch viel gilt's den Göttern sie zu entreißen.
 


\direct{leise}

Goldene Äpfel wachsen in ihrem Garten;
sie allein weiß die Äpfel zu pflegen!
Der Frucht Genuß frommt ihren Sippen
zu ewig nie alternder Jugend:
siech und bleich doch sinkt ihre Blüte,
alt und schwach schwinden sie hin,
müssen Freia sie missen.
 


\direct{grob}

Ihrer Mitte drum sei sie entführt!
 

\Wotanspeaks

\direct{für sich}

Loge säumt zu lang!
 

\Fasoltspeaks
Schlicht gib nun Bescheid!
 

\Wotanspeaks
Sinnt auf andern Sold!
 

\Fasoltspeaks
Kein andrer: Freia allein!
 

\Fafnerspeaks
Du da! Folg' uns fort!
 


\direct{Sie dringen auf Freia zu}


\Freiaspeaks

\direct{fliehend}

Helft! Helft, vor den Harten!
 

\Frohspeaks

\direct{Freia in seine Arme fassend}

Zu mir, Freia! Meide sie, Frecher!
Froh schützt die Schöne.
 

\Donnerspeaks

\direct{sich vor die beiden Riesen stellend}

Fasolt und Fafner,
fühltet ihr schon meines Hammers harten Schlag?
 

\Fafnerspeaks
Was soll das Drohn?
 

\Fasoltspeaks
Was dringst du her?
Kampf kiesten wir nicht,
verlangen nur unsern Lohn.
 

\Donnerspeaks
Schon oft zahlt' ich Riesen den Zoll.
Kommt her, des Lohnes Last
wäg' ich mit gutem Gewicht!
 


\direct{er schwingt den Hammer}


\Wotanspeaks

\direct{seinen Speer zwischen den Streitenden ausstreckend}

Halt, du Wilder! Nichts durch Gewalt!
Verträge schützt meines Speeres Schaft:
spar' deines Hammers Heft!
 

\Freiaspeaks
Wehe! Wehe! Wotan verläßt mich!
 

\Frickaspeaks
Begreif' ich dich noch, grausamer Mann?
 

\Wotanspeaks

\direct{wendet sich ab und sieht Loge kommen}

Endlich Loge! Eiltest du so,
den du geschlossen,
den schlimmen Handel zu schlichten?
 

\Logespeaks

\direct{ist im Hintergrunde aus dem Tale heraufgestiegen}

Wie? Welchen Handel hätt' ich geschlossen?
Wohl was mit den Riesen dort im Rate du dangst?
In Tiefen und Höhen treibt mich mein Hang;
Haus und Herd behagt mir nicht.
Donner und Froh,
die denken an Dach und Fach,
wollen sie frei'n,
ein Haus muß sie erfreu'n.
Ein stolzer Saal, ein starkes Schloß,
danach stand Wotans Wunsch.
Haus und Hof, Saal und Schloß,
die selige Burg, sie steht nun fest gebaut.
Das Prachtgemäuer prüft' ich selbst,
ob alles fest, forscht' ich genau:
Fasolt und Fafner fand ich bewährt:
kein Stein wankt in Gestemm'.
Nicht müßig war ich, wie mancher hier;
der lügt, wer lässig mich schilt!
 

\Wotanspeaks
Arglistig weichst du mir aus:
mich zu betrügen hüte in Treuen dich wohl!
Von allen Göttern dein einz'ger Freund,
nahm ich dich auf in der übel trauenden Troß.
Nun red' und rate klug!
Da einst die Bauer der Burg
zum Dank Freia bedangen,
du weißt, nicht anders willigt' ich ein,
als weil auf Pflicht du gelobtest,
zu lösen das hehre Pfand.
 

\Logespeaks
Mit höchster Sorge drauf zu sinnen,
wie es zu lösen, das hab' ich gelobt.
Doch, daß ich fände,
was nie sich fügt, was nie gelingt,
wie ließ sich das wohl geloben?
 

\Frickaspeaks

\direct{zu Wotan}

Sieh, welch trugvollem Schelm du getraut!
 

\Frohspeaks
Loge heißt du,
doch nenn' ich dich Lüge!
 

\Donnerspeaks
Verfluchte Lohe, dich lösch' ich aus!
 

\Logespeaks
Ihre Schmach zu decken,
schmähen mich Dumme!
 


\direct{Donner holt auf Loge aus}


\Wotanspeaks

\direct{tritt dazwischen}

In Frieden laßt mir den Freund!
Nicht kennt ihr Loges Kunst:
reicher wiegt seines Rates Wert,
zahlt er zögernd ihn aus.
 

\Fafnerspeaks
Nichts gezögert! Rasch gezahlt!
 

\Fasoltspeaks
Lang währt's mit dem Lohn!
 

\Wotanspeaks

\direct{wendet sich hart zu Loge, drängend}

Jetzt hör', Störrischer! Halte Stich!
Wo schweiftest du hin und her?
 

\Logespeaks
Immer ist Undank Loges Lohn!
Für dich nur besorgt, sah ich mich um,
durchstöbert' im Sturm alle Winkel der Welt,
Ersatz für Freia zu suchen,
wie er den Riesen wohl recht.
Umsonst sucht' ich, und sehe nun wohl:
in der Welten Ring nichts ist so reich,
als Ersatz zu muten dem Mann
für Weibes Wonne und Wert!
 


\direct{Alle geraten in Erstaunen und verschiedenartige Betroffenheit}

So weit Leben und Weben,
In Wasser, Erd' und Luft,
viel frug' ich, forschte bei allen,
wo Kraft nur sich rührt, und Keime sich regen:
was wohl dem Manne mächt'ger dünk',
als Weibes Wonne und Wert?
Doch so weit Leben und Weben,
verlacht nur ward meine fragende List:
in Wasser, Erd' und Luft,
lassen will nichts von Lieb' und Weib.
Nur einen sah' ich, der sagte der Liebe ab:
um rotes Gold entriet er des Weibes Gunst.
Des Rheines klare Kinder
klagten mir ihre Not:
der Nibelung, Nacht-Alberich,
buhlte vergebens um der Badenden Gunst;
das Rheingold da
raubte sich rächend der Dieb:
das dünkt ihn nun das teuerste Gut,
hehrer als Weibes Huld.
Um den gleißenden Tand,
der Tiefe entwandt,
erklang mir der Töchter Klage:
an dich, Wotan, wenden sie sich,
daß zu Recht du zögest den Räuber,
das Gold dem Wasser wieder gebest,
und ewig es bliebe ihr Eigen.
 


\direct{Hingebende Bewegung aller}

Dir's zu melden, gelobt' ich den Mädchen:
nun löste Loge sein Wort.
 

\Wotanspeaks
Törig bist du, wenn nicht gar tückisch!
Mich selbst siehst du in Not:
wie hülft' ich andern zum Heil?
 

\Fasoltspeaks

\direct{der aufmerksam zugehört, zu Fafner}

Nicht gönn' ich das Gold dem Alben;
viel Not schon schuf uns der Niblung,
doch schlau entschlüpfte unserm
Zwange immer der Zwerg.
 

\Fafnerspeaks
Neue Neidtat sinnt uns der Niblung,
gibt das Gold ihm Macht.
Du da, Loge! Sag' ohne Lug:
was Großes gilt denn das Gold,
daß dem Niblung es genügt?
 

\Logespeaks
Ein Tand ist's in des Wassers Tiefe,
lachenden Kindern zur Lust,
doch ward es zum runden Reife geschmiedet,
hilft es zur höchsten Macht,
gewinnt dem Manne die Welt.
 

\Wotanspeaks

\direct{sinnend}

Von des Rheines Gold hört' ich raunen:
Beute-Runen berge sein roter Glanz;
Macht und Schätze schüf ohne Maß ein Reif.
 

\Frickaspeaks

\direct{leise zu Loge}

Taugte wohl des goldnen Tandes
gleißend Geschmeid
auch Frauen zu schönem Schmuck?
 

\Logespeaks
Des Gatten Treu' ertrotzte die Frau,
trüge sie hold den hellen Schmuck,
den schimmernd Zwerge schmieden,
rührig im Zwange des Reifs.
 

\Frickaspeaks

\direct{schmeichelnd zu Wotan}

Gewänne mein Gatte sich wohl das Gold?
 

\Wotanspeaks

\direct{wie in einem Zustande wachsender Bezauberung}

Des Reifes zu walten,
rätlich will es mich dünken.
Doch wie, Loge, lernt' ich die Kunst?
Wie schüf' ich mir das Geschmeid'?
 

\Logespeaks
Ein Runenzauber zwingt das Gold zum Reif;
keiner kennt ihn;
doch einer übt ihn leicht,
der sel'ger Lieb' entsagt.
 


\direct{Wotan wendet sich unmutig ab}

Das sparst du wohl; zu spät auch kämst du:
Alberich zauderte nicht.
Zaglos gewann er des Zaubers Macht:
 


\direct{grell}

geraten ist ihm der Ring!
 

\Donnerspeaks

\direct{zu Wotan}

Zwang uns allen schüfe der Zwerg,
würd' ihm der Reif nicht entrissen.
 

\Wotanspeaks
Den Ring muß ich haben!
 

\Frohspeaks
Leicht erringt ohne Liebesfluch er sich jetzt.
 

\Logespeaks
Spottleicht, ohne Kunst, wie im Kinderspiel!
 

\Wotanspeaks

\direct{grell}

So rate, wie?
 

\Logespeaks
Durch Raub!
Was ein Dieb stahl, das stiehlst du dem Dieb;
ward leichter ein Eigen erlangt?
Doch mit arger Wehr wahrt sich Alberich;
klug und fein mußt du verfahren,
ziehst den Räuber du zu Recht,
um des Rheines Töchtern, den roten Tand,
 


\direct{mit Wärme}

das Gold wiederzugeben;
denn darum flehen sie dich.
 

\Wotanspeaks
Des Rheines Töchtern? Was taugt mir der Rat?
 

\Frickaspeaks
Von dem Wassergezücht mag ich nichts wissen:
schon manchen Mann---mir zum Leid---
verlockten sie buhlend im Bad.
 


\direct{Wotan steht stumm mit sich kämpfend; die übrigen Götter heften in schweigender Spannung die Blicke auf ihn. Währenddem hat Fafner beiseite mit Fasolt beraten}


\Fafnerspeaks

\direct{zu Fasolt}

Glaub' mir, mehr als Freia
frommt das gleißende Gold:
auch ew'ge Jugend erjagt,
wer durch Goldes Zauber sie zwingt.
 


\direct{Fasolts Gebärde deutet an, daß er sich wider Willen überredet fühlt. Fafner tritt mit Fasolt wieder an Wotan heran.}

Hör', Wotan, der Harrenden Wort!
Freia bleib' euch in Frieden;
leicht'ren Lohn fand ich zur Lösung:
uns rauhen Riesen genügt
des Niblungen rotes Gold.
 

\Wotanspeaks
Seid ihr bei Sinn?
Was nicht ich besitze,
soll ich euch Schamlosen schenken?
 

\Fafnerspeaks
Schwer baute dort sich die Burg;
leicht wird dir's mit list'ger Gewalt
was im Neidspiel nie uns gelang,
den Niblungen fest zu fahn.
 

\Wotanspeaks
Für euch müht' ich mich um den Alben?
Für euch fing' ich den Feind?
Unverschämt und überbegehrlich,
macht euch Dumme mein Dank!
 

\Fasoltspeaks

\direct{ergreift plötzlich Freia und führt sie mit Fafner zur Seite}

Hieher, Maid! In unsre Macht!
Als Pfand folgst du uns jetzt,
bis wir Lösung empfah'n!
 

\Freiaspeaks

\direct{wehklagend}

Wehe! Wehe! Wehe!
 


\direct{alle Götter sind in höchster Bestürzung}


\Fafnerspeaks
Fort von hier sei sie entführt!
Bis Abend---achtet's wohl---
pflegen wir sie als Pfand;
wir kehren wieder; doch kommen wir,
und bereit liegt nicht als Lösung
das Rheingold licht und rot---
 

\Fasoltspeaks
Zu End' ist die Frist dann,
Freia verfallen:
für immer folge sie uns!
 

\Freiaspeaks

\direct{schreiend}

Schwester! Brüder! Rettet! Helft!
 


\direct{sie wird von den hastig enteilenden Riesen fortgetragen}


\Frohspeaks
Auf, ihnen nach!
 

\Donnerspeaks
Breche denn alles!
 


\direct{Sie blicken Wotan fragend an}


\Freiaspeaks

\direct{aus weiter Ferne}

Rettet! Helft!
 

\Logespeaks

\direct{den Riesen nachsehend}

Über Stock und Stein zu Tal
stapfen sie hin:
durch des Rheines Wasserfurt
waten die Riesen.
Fröhlich nicht hängt Freia
den Rauhen über dem Rücken! -
Heia! Hei! Wie taumeln die Tölpel dahin!
Durch das Tal talpen sie schon.
Wohl an Riesenheims Mark
erst halten sie Rast. -
 


\direct{er wendet sich zu den Göttern}

Was sinnt nun Wotan so wild?
Den sel'gen Göttern wie geht's?
 


\direct{Ein fahler Nebel erfüllt mit wachsender Dichtheit die Bühne; in ihm erhalten die Götter ein zunehmend bleiches und ältliches Aussehen: alle stehen bang und erwartungsvoll auf Wotan blickend, der sinnend die Augen an den Boden heftet}


\Logespeaks
Trügt mich ein Nebel?
Neckt mich ein Traum?
Wie bang und bleich verblüht ihr so bald!
Euch erlischt der Wangen Licht;
der Blick eures Auges verblitzt!
Frisch, mein Froh, noch ist's ja früh!
Deiner Hand, Donner, entsinkt ja der Hammer!
Was ist's mit Fricka? Freut sie sich wenig
ob Wotans grämlichem Grau,
das schier zum Greisen ihn schafft?
 

\Frickaspeaks
Wehe! Wehe! Was ist geschehen?
 

\Donnerspeaks
Mir sinkt die Hand!
 

\Frohspeaks
Mir stockt das Herz!
 

\Logespeaks
Jetzt fand' ich's: hört, was euch fehlt!
Von Freias Frucht genosset ihr heute noch nicht.
Die goldnen Äpfel in ihrem Garten,
sie machten euch tüchtig und jung,
aßt ihr sie jeden Tag.
Des Gartens Pflegerin ist nun verpfändet;
an den Ästen darbt und dorrt das Obst,
bald fällt faul es herab. -
Mich kümmert's minder;
an mir ja kargte Freia von je
knausernd die köstliche Frucht:
denn halb so echt nur bin ich wie, Selige, ihr!
Doch ihr setztet alles auf das jüngende Obst:
das wußten die Riesen wohl;
auf eurer Leben legten sie's an:
nun sorgt, wie ihr das wahrt!
Ohne die Äpfel,
alt und grau, greis und grämlich,
welkend zum Spott aller Welt,
erstirbt der Götter Stamm.
 

\Frickaspeaks

\direct{bang}

Wotan, Gemahl, unsel'ger Mann!
Sieh, wie dein Leichtsinn lachend uns allen
Schimpf und Schmach erschuf!
 

\Wotanspeaks

\direct{mit plötzlichem Entschluß auffahrend}

Auf, Loge, hinab mit mir!
Nach Nibelheim fahren wir nieder:
gewinnen will ich das Gold.
 

\Logespeaks
Die Rheintöchter riefen dich an:
so dürfen Erhörung sie hoffen?
 

\Wotanspeaks

\direct{heftig}

Schweige, Schwätzer!
Freia, die Gute, Freia gilt es zu lösen!
 

\Logespeaks
Wie du befiehlst
führ' ich dich gern
steil hinab
steigen wir denn durch den Rhein?
 

\Wotanspeaks
Nicht durch den Rhein!
 

\Logespeaks
So schwingen wir uns durch die Schwefelkluft.
Dort schlüpfe mit mir hinein!
 


\direct{Er geht voran und verschwindet seitwärts in einer Kluft, aus der sogleich ein schwefliger Dampf hervorquillt}


\Wotanspeaks
Ihr andern harrt bis Abend hier:
verlorner Jugend erjag' ich erlösendes Gold!
 


\direct{Er steigt Loge nach in die Kluft hinab: der aus ihr dringende Schwefeldampf verbreitet sich über die ganze Bühne und erfüllt diese schnell mit dickem Gewölk. Bereits sind die Zurückbleibenden unsichtbar.}


\Donnerspeaks
Fahre wohl, Wotan!
 

\Frohspeaks
Glück auf! Glück auf!
 

\Frickaspeaks
O kehre bald zur bangenden Frau!
 

\StageDir{Der Schwefeldampf verdüstert sich bis zu ganz schwarzem Gewölk, welches von unten nach oben steigt; dann verwandelt sich dieses in festes, finsteres Steingeklüft, das sich immer aufwärts bewegt, so daß es den Anschein hat, als sänke die Szene immer tiefer in die Erde hinab. Wachsendes Geräusch wie von Schmiedenden wird überallher vernommen.}
 
\scene
 

\StageDir{Von verschiedenen Seiten her dämmert aus der Ferne dunkelroter Schein auf: eine unabsehbar weit sich dahinziehende unterirdische Kluft wird erkennbar, die nach allen Seiten hin in enge Schachte auszumünden scheint.

  Alberich zerrt den kreischenden Mime an den Ohren aus einer Seitenschlucht herbei.}

\Alberichspeaks
Hehe! Hehe!
Hieher! Hieher! Tückischer Zwerg!
Tapfer gezwickt sollst du mir sein,
schaffst du nicht fertig, wie ich's bestellt,
zur Stund' das feine Geschmeid'!
 

\Mimespeaks

\direct{heulend}

Ohe! Ohe! Au! Au!
Laß mich nur los!
Fertig ist's, wie du befahlst,
mit Fleiß und Schweiß ist es gefügt:
nimm nur
 


\direct{grell}

die Nägel vom Ohr!
 

\Alberichspeaks

\direct{loslassend}

Was zögerst du dann
und zeigst es nicht?
 

\Mimespeaks
Ich Armer zagte,
daß noch was fehle.
 

\Alberichspeaks
Was wär' noch nicht fertig?
 

\Mimespeaks

\direct{verlegen}

Hier - und da -
 

\Alberichspeaks
Was hier und da? Her das Geschmeid'!
 


\direct{Er will ihm wieder an das Ohr fahren; vor Schreck läßt Mime ein metallenes Gewirke, das er krampfhaft in den Händen hielt, sich entfallen. Alberich hebt es hastig auf und prüft es genau.}

Schau, du Schelm! Alles geschmiedet
und fertig gefügt, wie ich's befahl!
So wollte der Tropf schlau mich betrügen?
Für sich behalten das hehre Geschmeid',
das meine List ihn zu schmieden gelehrt?
Kenn' ich dich dummen Dieb?
 


\direct{Er setzt das Gewirk als ``Tarnhelm'' auf den Kopf}

Dem Haupt fügt sich der Helm:
ob sich der Zauber auch zeigt?
 


\direct{sehr leise}

``Nacht und Nebel - niemand gleich!''
 


\direct{seine Gestalt verschwindet; statt ihrer gewahrt man eine Nebelsäule}

Siehst du mich, Bruder?
 

\Mimespeaks

\direct{blickt sich verwundert um}

Wo bist du? Ich sehe dich nicht.
 

\Alberichspeaks

\direct{unsichtbar}

So fühle mich doch, du fauler Schuft!
Nimm das für dein Diebesgelüst!
 

\Mimespeaks

\direct{schreit und windet sich unter empfangenen Geißelhieben, deren Fall man vernimmt, ohne die Geißel selbst zu sehen}

Ohe, Ohe! Au! Au! Au!
 

\Alberichspeaks

\direct{lachend, unsichtbar}

Hahahahahaha!
Hab' Dank, du Dummer!
Dein Werk bewährt sich gut!
Hoho! Hoho!
Niblungen all', neigt euch nun Alberich!
Überall weilt er nun, euch zu bewachen;
Ruh' und Rast ist euch zerronnen;
ihm müßt ihr schaffen wo nicht ihr ihn schaut;
wo nicht ihr ihn gewahrt, seid seiner gewärtig!
Untertan seid ihr ihm immer!
 


\direct{grell}

Hoho! Hoho! Hört' ihn, er naht:
der Niblungen Herr!
 


\direct{Die Nebelsäule verschwindet dem Hintergrunde zu: man hört in immer weiterer Ferne Alberichs Toben und Zanken; Geheul und Geschrei antwortet ihm, das sich endlich in immer weiterer Ferne unhörbar verliert. Mime ist vor Schmerz zusammengesunken. Wotan und Loge lassen sich aus einer Schlucht von oben herab.}


\Logespeaks
Nibelheim hier:
Durch bleiche Nebel
was blitzen dort feurige Funken?
 

\Mimespeaks
Au! Au! Au!
 

\Wotanspeaks
Hier stöhnt es laut:
was liegt im Gestein?
 

\Logespeaks

\direct{neigt sich zu Mime}

Was Wunder wimmerst du hier?
 

\Mimespeaks
Ohe! Ohe! Au! Au!
 

\Logespeaks
Hei, Mime! Munt'rer Zwerg!
Was zwickt und zwackt dich denn so?
 

\Mimespeaks
Laß mich in Frieden!
 

\Logespeaks
Das will ich freilich,
und mehr noch, hör':
helfen will ich dir, Mime!
 


\direct{Er stellt ihn mühsam aufrecht}


\Mimespeaks
Wer hälfe mir?
Gehorchen muß ich dem leiblichen Bruder,
der mich in Bande gelegt.
 

\Logespeaks
Dich, Mime, zu binden,
was gab ihm die Macht?
 

\Mimespeaks
Mit arger List schuf sich Alberich
aus Rheines Gold einem gelben Reif:
seinem starken Zauber zittern wir staunend;
mit ihm zwingt er uns alle,
der Niblungen nächt'ges Heer.
Sorglose Schmiede, schufen wir sonst wohl
Schmuck unsern Weibern, wonnig Geschmeid',
niedlichen Niblungentand;
wir lachten lustig der Müh'.
Nun zwingt uns der Schlimme,
in Klüfte zu schlüpfen,
für ihn allein uns immer zu müh'n.
Durch des Ringes Gold errät seine Gier,
wo neuer Schimmer in Schachten sich birgt:
da müssen wir spähen, spüren und graben,
die Beute schmelzen und schmieden den Guß,
ohne Ruh' und Rast
dem Herrn zu häufen den Hort.
 

\Logespeaks
Dich Trägen so eben traf wohl sein Zorn?
 

\Mimespeaks
Mich Ärmsten, ach, mich zwang er zum Ärgsten:
ein Helmgeschmeid' hieß er mich schweißen;
genau befahl er, wie es zu fügen.
Wohl merkt' ich klug, welch mächtige Kraft
zu eigen dem Werk, das aus Erz ich wob;
für mich drum hüten wollt' ich dem Helm;
durch seinen Zauber
Alberichs Zwang mich entzieh'n:
vielleicht - ja vielleicht
den Lästigen selbst überlisten,
in meine Gewalt ihn zu werfen,
den Ring ihm zu entreißen,
daß, wie ich Knecht jetzt dem Kühnen,
 


\direct{grell}

mir Freien er selber dann frön'!
 

\Logespeaks
Warum, du Kluger, glückte dir's nicht?
 

\Mimespeaks
Ach, der das Werk ich wirkte,
den Zauber, der ihm entzuckt,
den Zauber erriet ich nicht recht!
Der das Werk mir riet und mir's entriß,
der lehrte mich nun,
- doch leider zu spät, -
welche List läg' in dem Helm:
Meinem Blick entschwand er,
doch Schwielen dem Blinden
schlug unschaubar sein Arm.
 


\direct{heulend und schluchzend}

Das schuf ich mir Dummen schön zu Dank!
 


\direct{er streicht sich den Rücken. Wotan und Loge lachen}


\Logespeaks

\direct{zu Wotan}

Gesteh', nicht leicht gelingt der Fang.
 

\Wotanspeaks
Doch erliegt der Feind, hilft deine List!
 

\Mimespeaks

\direct{von dem Lachen der Götter betroffen, betrachtet diese aufmerksamer}

Mit eurem Gefrage,
wer seid denn ihr Fremde?
 

\Logespeaks
Freunde dir; von ihrer Not
befrei'n wir der Niblungen Volk!
 

\Mimespeaks

\direct{schrickt zusammen, da er Alberich sich wieder nahen hört}

Nehmt euch in acht! Alberich naht.
 

\Wotanspeaks
Sein' harren wir hier.
 


\direct{Er setzt sich ruhig auf einen Stein; Loge lehnt ihm zur Seite. Alberich, der den Tarnhelm vom Haupte genommen und an den Gürtel gehängt hat, treibt mit geschwungener Geißel aus der unteren, tiefer gelegenen Schlucht aufwärts eine Schar Nibelungen vor sich her: diese sind mit goldenem und silbernem Geschmeide beladen, das sie, unter Alberichs steter Nötigung, all auf einen Haufen speichern und so zu einem Horte häufen.}


\Alberichspeaks
Hieher! Dorthin! Hehe! Hoho!
Träges Heer, dort zu Hauf schichtet den Hort!
Du da, hinauf! Willst du voran?
Schmähliches Volk, ab das Geschmeide!
Soll ich euch helfen? Alle hieher!
 


\direct{er gewahrt plötzlich Wotan und Loge}

He! Wer ist dort? Wer drang hier ein?
Mime, zu mir, schäbiger Schuft!
Schwatztest du gar mit dem schweifenden Paar?
Fort, du Fauler!
Willst du gleich schmieden und schaffen?
 


\direct{Er treibt Mime mit Geißelhieben unter den Haufen der Nibelungen hinein}

He! An die Arbeit!
Alle von hinnen! Hurtig hinab!
Aus den neuen Schachten schafft mir das Gold!
Euch grüßt die Geißel, grabt ihr nicht rasch!
Daß keiner mir müßig, bürge mir Mime,
sonst birgt er sich schwer meiner Geißel Schwunge!
Daß ich überall weile, wo keiner mich wähnt,
das weiß er, dünkt mich, genau!
Zögert ihr noch? Zaudert wohl gar?
 


\direct{Er zieht seinen Ring vom Finger, küßt ihn und streckt ihn drohend aus}

Zittre und zage, gezähmtes Heer!
Rasch gehorcht des Ringes Herrn!
 


\direct{Unter Geheul und Gekreisch stieben die Nibelungen, unter ihnen Mime, auseinander und schlüpfen in die Schächte hinab}


\Alberichspeaks

\direct{betrachtet lange und mißtrauisch Wotan und Loge}

Was wollt ihr hier?
 

\Wotanspeaks
Von Nibelheims nächt'gem Land
vernahmen wir neue Mär':
mächtige Wunder wirke hier Alberich;
daran uns zu weiden, trieb uns Gäste die Gier.
 

\Alberichspeaks
Nach Nibelheim führt euch der Neid:
so kühne Gäste, glaubt, kenn' ich gut!
 

\Logespeaks
Kennst du mich gut, kindischer Alp?
Nun sag', wer bin ich, daß du so bellst?
Im kalten Loch, da kauern du lagst,
wer gab dir Licht und wärmende Lohe,
wenn Loge nie dir gelacht?
Was hülf' dir dein Schmieden,
heizt' ich die Schmiede dir nicht?
Dir bin ich Vetter, und war dir Freund:
nicht fein drum dünkt mich dein Dank!
 

\Alberichspeaks
Den Lichtalben lacht jetzt Loge,
der list'ge Schelm:
bist du falscher ihr Freund,
wie mir Freund du einst warst:
haha! Mich freut's!
Von ihnen fürcht' ich dann nichts.
 

\Logespeaks
So denk' ich, kannst du mir traun?
 

\Alberichspeaks
Deiner Untreu trau' ich, nicht deiner Treu'!
 


\direct{eine herausfordernde Stellung einnehmend}

Doch getrost trotz' ich euch allen!
 

\Logespeaks
Hohen Mut verleiht deine Macht;
grimmig groß wuchs dir die Kraft!
 

\Alberichspeaks
Siehst du den Hort,
den mein Heer dort mir gehäuft?
 

\Logespeaks
So neidlichen sah ich noch nie.
 

\Alberichspeaks
Das ist für heut, ein kärglich Häufchen:
Kühn und mächtig soll er künftig sich mehren.
 

\Wotanspeaks
Zu was doch frommt dir der Hort,
da freudlos Nibelheim,
und nichts für Schätze hier feil?
 

\Alberichspeaks
Schätze zu schaffen und Schätze zu bergen,
nützt mir Nibelheims Nacht.
Doch mit dem Hort, in der Höhle gehäuft,
denk' ich dann Wunder zu wirken:
die ganze Welt gewinn' ich mit ihm mir zu eigen!
 

\Wotanspeaks
Wie beginnst du, Gütiger, das?
 

\Alberichspeaks
Die in linder Lüfte Weh'n da oben ihr lebt,
lacht und liebt: mit goldner Faust
euch Göttliche fang' ich mir alle!
Wie ich der Liebe abgesagt,
alles, was lebt, soll ihr entsagen!
Mit Golde gekirrt,
nach Gold nur sollt ihr noch gieren!
Auf wonnigen Höhn,
in seligem Weben wiegt ihr euch;
den Schwarzalben
verachtet ihr ewigen Schwelger!
Habt acht! Habt acht!
Denn dient ihr Männer erst meiner Macht,
eure schmucken Frau'n, die mein Frei'n verschmäht,
sie zwingt zur Lust sich der Zwerg,
lacht Liebe ihm nicht!
 


\direct{wild lachend}

Hahahaha! Habt ihr's gehört?
Habt acht vor dem nächtlichen Heer,
entsteigt des Niblungen Hort
aus stummer Tiefe zu Tag!
 

\Wotanspeaks

\direct{auffahrend}

Vergeh, frevelnder Gauch!
 

\Alberichspeaks
Was sagt der?
 

\Logespeaks

\direct{ist dazwischengetreten}

Sei doch bei Sinnen!
 


\direct{zu Alberich}

Wen doch faßte nicht Wunder,
erfährt er Alberichs Werk?
Gelingt deiner herrlichen List,
was mit dem Horte du heischest:
den Mächtigsten muß ich dich rühmen;
denn Mond und Stern', und die strahlende Sonne,
sie auch dürfen nicht anders,
dienen müssen sie dir.
Doch - wichtig acht' ich vor allem,
daß des Hortes Häufer, der Niblungen Heer,
neidlos dir geneigt.
Einen Reif rührtest du kühn;
dem zagte zitternd dein Volk: -
doch, wenn im Schlaf ein Dieb dich beschlich',
den Ring schlau dir entriss', -
wie wahrtest du, Weiser, dich dann?
 

\Alberichspeaks
Der Listigste dünkt sich Loge;
andre denkt er immer sich dumm:
daß sein' ich bedürfte zu Rat und Dienst,
um harten Dank,
das hörte der Dieb jetzt gern!
Den hehlenden Helm ersann ich mir selbst;
der sorglichste Schmied,
Mime, mußt' ihn mir schmieden:
schnell mich zu wandeln, nach meinem Wunsch
die Gestalt mir zu tauschen, taugt der Helm.
Niemand sieht mich, wenn er mich sucht;
doch überall bin ich, geborgen dem Blick.
So ohne Sorge
bin ich selbst sicher vor dir,
du fromm sorgender Freund!
 

\Logespeaks
Vieles sah ich, Seltsames fand ich,
doch solches Wunder gewahrt' ich nie.
Dem Werk ohnegleichen kann ich nicht glauben;
wäre das eine möglich,
deine Macht währte dann ewig!
 

\Alberichspeaks
Meinst du, ich lüg' und prahle wie Loge?
 

\Logespeaks
Bis ich's geprüft,
bezweifl' ich, Zwerg, dein Wort.
 

\Alberichspeaks
Vor Klugheit bläht sich
zum Platzen der Blöde!
Nun plage dich Neid!
Bestimm', in welcher Gestalt
soll ich jach vor dir stehn?
 

\Logespeaks
In welcher du willst;
nur mach' vor Staunen mich stumm.
 

\Alberichspeaks

\direct{hat den Helm aufgesetzt}

``Riesen-Wurm winde sich ringelnd!''
 


\direct{Sogleich verschwindet er: eine ungeheure Riesenschlange windet sich statt seiner am Boden; sie bäumt sich und streckt den aufgesperrten Rachen nach Wotan und Loge hin.}


\Logespeaks

\direct{stellt sich von Furcht ergriffen}

Ohe! Ohe!
Schreckliche Schlange, verschlinge mich nicht!
Schone Logen das Leben!
 

\Wotanspeaks
Hahaha! Gut, Alberich!
Gut, du Arger!
Wie wuchs so rasch
zum riesigen Wurme der Zwerg!
 

\direct{Die Schlange verschwindet; statt ihrer erscheint sogleich Alberich wieder in seiner wirklichen Gestalt.}

\Alberichspeaks
Hehe! Ihr Klugen, glaubt ihr mir nun?
 

\Logespeaks
Mein Zittern mag dir's bezeugen.
Zur großen Schlange schufst du dich schnell:
weil ich's gewahrt,
willig glaub' ich dem Wunder.
Doch, wie du wuchsest,
kannst du auch winzig
und klein dich schaffen?
Das Klügste schien' mir das,
Gefahren schlau zu entfliehn:
das aber dünkt mich zu schwer!
 

\Alberichspeaks
Zu schwer dir, weil du zu dumm!
Wie klein soll ich sein?
 

\Logespeaks
Daß die feinste Klinze dich fasse,
wo bang die Kröte sich birgt.
 

\Alberichspeaks
Pah! Nichts leichter! Luge du her!
 


\direct{Er setzt den Tarnhelm wieder auf}

``Krumm und grau krieche Kröte!''
 


\direct{Er verschwindet; die Götter gewahren im Gestein eine Kröte auf sich zukriechen.}


\Logespeaks
\direct{zu Wotan}
Dort, die Kröte, greife sie rasch!
 


\direct{Wotan setzt seinen Fuß auf die Kröte, Loge fährt ihr nach dem Kopfe und hält den Tarnhelm in der Hand. Alberich wird plötzlich in seiner wirklichen Gestalt sichtbar, wie er sich unter Wotans Fuße windet}


\Alberichspeaks
Ohe! Verflucht! Ich bin gefangen!
 

\Logespeaks
Halt' ihn fest, bis ich ihn band.
 


\direct{Er hat ein Bastseil hervorgeholt und bindet Alberich damit Hände und Beine; den Geknebelten, der sich wütend zu wehren sucht, fassen dann beide und schleppen ihn mit sich nach der Kluft, aus der sie herauskamen.}


\Logespeaks
Nun schnell hinauf: dort ist er unser!
 


\direct{Sie verschwinden, aufwärts steigend}

 
\scene

\StageDir{Die Szene verwandelt sich, nur in umgekehrter Weise, wie zuvor; die Verwandlung führt wieder an den Schmieden vorüber. Fortdauernde Verwandlung nach oben. Schließlich erscheint wieder die
freie Gegend auf Bergeshöhen
wie in der zweiten Szene; nur ist sie jetzt noch in fahle Nebel verhüllt, wie vor der zweiten Verwandlung nach Freias Abführung.

Wotan und Loge, den gebundenen Alberich mit sich führend, steigen aus der Kluft herauf.}

\Logespeaks
Da, Vetter, sitze du fest!
Luge Liebster, dort liegt die Welt,
die du Lungrer gewinnen dir willst:
welch Stellchen, sag',
bestimmst du drin mir zu Stall?
 


\direct{er schlägt ihm tanzend Schnippchen}


\Alberichspeaks
Schändlicher Schächer! Du Schalk! Du Schelm!
Löse den Bast, binde mich los,
den Frevel sonst büßest du Frecher!
 

\Wotanspeaks
Gefangen bist du, fest mir gefesselt,
wie du die Welt, was lebt und webt,
in deiner Gewalt schon wähntest,
in Banden liegst du vor mir,
du Banger kannst es nicht leugnen!
Zu ledigen dich, bedarf 's nun der Lösung.
 

\Alberichspeaks
O ich Tropf, ich träumender Tor!
Wie dumm traut' ich dem diebischen Trug!
Furchtbare Rache räche den Fehl!
 

\Logespeaks
Soll Rache dir frommen,
vor allem rate dich frei:
dem gebundnen Manne
büßt kein Freier den Frevel.
Drum, sinnst du auf Rache,
rasch ohne Säumen
sorg' um die Lösung zunächst!
 


\direct{er zeigt ihm, mit den Fingern schnalzend, die Art der Lösung an}


\Alberichspeaks

\direct{barsch}

So heischt, was ihr begehrt!
 

\Wotanspeaks
Den Hort und dein helles Gold.
 

\Alberichspeaks
Gieriges Gaunergezücht!
 


\direct{für sich}

Doch behalt' ich mir nur den Ring,
des Hortes entrat' ich dann leicht;
denn von neuem gewonnen
und wonnig genährt
ist er bald durch des Ringes Gebot:
eine Witzigung wär 's,
die weise mich macht;
zu teuer nicht zahl' ich,
lass' für die Lehre ich den Tand.
 

\Wotanspeaks
Erlegst du den Hort?
 

\Alberichspeaks
Löst mir die Hand, so ruf' ich ihn her.
 


\direct{Loge löst ihm die Schlinge an der rechten Hand. Alberich berührt den Ring mit den Lippen und murmelt heimlich einen Befehl.}

Wohlan, die Nibelungen rief ich mir nah'.
Ihrem Herrn gehorchend, hör' ich den Hort
aus der Tiefe sie führen zu Tag:
nun löst mich vom lästigen Band!
 

\Wotanspeaks
Nicht eh'r, bis alles gezahlt.
 


\direct{Die Nibelungen steigen aus der Kluft herauf, mit den Geschmeiden des Hortes beladen. Während des Folgenden schichten sie den Hort auf.}


\Alberichspeaks
O schändliche Schmach!
Daß die scheuen Knechte
geknebelt selbst mich ersch'aun!
 


\direct{zu den Nibelungen}

Dorthin geführt, wie ich's befehlt'!
All zu Hauf schichtet den Hort!
Helf' ich euch Lahmen?
Hieher nicht gelugt!
Rasch da, rasch!
Dann rührt euch von hinnen,
daß ihr mir schafft!
Fort in die Schachten!
Weh' euch, find' ich euch faul!
Auf den Fersen folg' ich euch nach!
 


\direct{er küßt seinen Ring und streckt ihn gebieterisch aus. Wie von einem Schlage getroffen, drängen sich die Nibelungen scheu und ängstlich der Kluft zu, in die sie schnell hinabschlüpfen.}

Gezahlt hab' ich;
nun laßt mich zieh'n:
und das Helmgeschmeid',
das Loge dort hält,
das gebt mir nun gütlich zurück!
 

\Logespeaks

\direct{den Tarnhelm zum Horte werfend}

Zur Buße gehört auch die Beute.
 

\Alberichspeaks
Verfluchter Dieb!
 


\direct{leise}

Doch nur Geduld!
Der den alten mir schuf, schafft einen andern:
noch halt' ich die Macht, der Mime gehorcht.
Schlimm zwar ist's, dem schlauen Feind
zu lassen die listige Wehr!
Nun denn! Alberich ließ euch alles:
jetzt löst, ihr Bösen, das Band.
 

\Logespeaks

\direct{zu Wotan}

Bist du befriedigt? Lass' ich ihn frei?
 

\Wotanspeaks
Ein goldner Ring ragt dir am Finger;
hörst du, Alp?
Der, acht' ich, gehört mit zum Hort.
 

\Alberichspeaks

\direct{entsetzt}

Der Ring?
 

\Wotanspeaks
Zu deiner Lösung mußt du ihn lassen.
 

\Alberichspeaks

\direct{bebend}

Das Leben, doch nicht den Ring!
 

\Wotanspeaks

\direct{heftiger}

Den Reif' verlang' ich,
mit dem Leben mach', was du willst!
 

\Alberichspeaks
Lös' ich mir Leib und Leben,
den Ring auch muß ich mir lösen;
Hand und Haupt, Aug' und Ohr
sind nicht mehr mein Eigen,
als hier dieser rote Ring!
 

\Wotanspeaks
Dein Eigen nennst du den Ring?
Rasest du, schamloser Albe?
Nüchtern sag',
wem entnahmst du das Gold,
daraus du den schimmernden schufst?
War's dein Eigen, was du Arger
der Wassertiefe entwandt?
Bei des Rheines Töchtern hole dir Rat,
ob ihr Gold sie zu eigen dir gaben,
das du zum Ring dir geraubt!
 

\Alberichspeaks
Schmähliche Tücke! Schändlicher Trug!
Wirfst du Schächer die Schuld mir vor,
die dir so wonnig erwünscht?
Wie gern raubtest
du selbst dem Rheine das Gold,
war nur so leicht
die Kunst, es zu schmieden, erlangt?
Wie glückt es nun dir Gleißner zum Heil,
daß der Niblung, ich, aus schmählicher Not,
in des Zornes Zwange,
den schrecklichen Zauber gewann,
dess' Werk nun lustig dir lacht?
Des Unseligen, Angstversehrten
fluchfertige, furchtbare Tat,
zu fürstlichem Tand soll sie fröhlich dir taugen,
zur Freude dir frommen mein Fluch?
Hüte dich, herrischer Gott!
Frevelte ich, so frevelt' ich frei an mir:
doch an allem, was war,
ist und wird,
frevelst, Ewiger, du,
entreißest du frech mir den Ring!
 

\Wotanspeaks
Her der Ring!
Kein Recht an ihm
schwörst du schwatzend dir zu.
 


\direct{er ergreift Alberich und entzieht seinem Finger mit heftiger Gewalt den Ring.}


\Alberichspeaks
\direct{gräßlich aufschreiend}
Ha! Zertrümmert! Zerknickt!
Der Traurigen traurigster Knecht!
 

\Wotanspeaks

\direct{den Ring betrachtend}

Nun halt' ich, was mich erhebt,
der Mächtigen mächtigsten Herrn!
 


\direct{er steckt den Ring an}


\Logespeaks
Ist er gelöst?
 

\Wotanspeaks
Bind' ihn los!
 

\Logespeaks

\direct{löst Alberich vollends die Bande}

Schlüpfe denn heim!
Keine Schlinge hält dich:
frei fahre dahin!
 

\Alberichspeaks

\direct{sich vom Boden erhebend}

Bin ich nun frei?
 


\direct{mit wütendem Lachen}

Wirklich frei?
So grüß' euch denn
meiner Freiheit erster Gruß! -
Wie durch Fluch er mir geriet,
verflucht sei dieser Ring!
Gab sein Gold mir Macht ohne Maß,
nun zeug' sein Zauber Tod dem, der ihn trägt!
Kein Froher soll seiner sich freun,
keinem Glücklichen lache sein lichter Glanz!
Wer ihn besitzt, den sehre die Sorge,
und wer ihn nicht hat, den nage der Neid!
Jeder giere nach seinem Gut,
doch keiner genieße mit Nutzen sein!
Ohne Wucher hüt' ihn sein Herr;
doch den Würger zieh' er ihm zu!
Dem Tode verfallen, feßle den Feigen die Furcht:
solang er lebt, sterb' er lechzend dahin,
des Ringes Herr als des Ringes Knecht:
bis in meiner Hand den geraubten wieder ich halte! -
So segnet in höchster Not
der Nibelung seinen Ring!
Behalt' ihn nun,
 


\direct{lachend}

hüte ihn wohl:
 


\direct{grimmig}

meinem Fluch fliehest du nicht!
 


\direct{er verschwindet schnell in der Kluft. Der dichte Nebelduft des Vordergrundes klärt sich allmählich auf}


\Logespeaks
Lauschtest du seinem Liebesgruß?
 

\Wotanspeaks

\direct{in den Anblick des Ringes an seiner Hand versunken}

Gönn' ihm die geifernde Lust!
 


\direct{es wird immer heller}


\Logespeaks

\direct{nach rechts in die Szene blickend}

Fasolt und Fafner nahen von fern:
Freia führen sie her.
 


\direct{Aus dem sich immer mehr zerteilenden Nebel erscheinen Donner, Froh und Fricka und eilen dem Vordergrunde zu.}


\Frohspeaks
Sie kehren zurück!
 

\Donnerspeaks
Willkommen, Bruder!
 

\Frickaspeaks

\direct{besorgt zu Wotan}

Bringst du gute Kunde?
 

\Logespeaks

\direct{auf den Hort deutend}

Mit List und Gewalt gelang das Werk:
dort liegt, was Freia löst.
 

\Donnerspeaks
Aus der Riesen Haft naht dort die Holde.
 

\Frohspeaks
Wie liebliche Luft wieder uns weht,
wonnig' Gefühl die Sinne erfüllt!
Traurig ging es uns allen,
getrennt für immer von ihr,
die leidlos ewiger Jugend
jubelnde Lust uns verleiht.
 


\direct{Der Vordergrund ist wieder hell geworden; das Aussehen der Götter gewinnt wieder die erste Frische: über dem Hintergrunde haftet jedoch noch der Nebelschleier, so daß die Burg unsichtbar bleibt. Fasolt und Fafner treten auf, Freia zwischen sich führend.}


\Frickaspeaks
\direct{eilt freudig auf die Schwester zu, um sie zu umarmen}
Lieblichste Schwester, süßeste Lust!
Bist du mir wieder gewonnen?
 

\Fasoltspeaks

\direct{ihr wehrend}

Halt! Nicht sie berührt!
Noch gehört sie uns.
Auf Riesenheims ragender Mark
rasteten wir; mit treuem Mut
des Vertrages Pfand pflegten wir.
So sehr mich's reut, zurück doch bring' ich's,
erlegt uns Brüdern die Lösung ihr.
 

\Wotanspeaks
Bereit liegt die Lösung:
des Goldes Maß sei nun gütlich gemessen.
 

\Fasoltspeaks
Das Weib zu missen, wisse, gemutet mich weh:
soll aus dem Sinn sie mir schwinden
des Geschmeides Hort häufet denn so,
daß meinem Blick die Blühende ganz er verdeck'!
 

\Wotanspeaks
So stellt das Maß nach Freias Gestalt!
 


\direct{Freia wird von den beiden Riesen in die Mitte gestellt. Darauf stoßen sie ihre Pfähle zu Freias beiden Seiten so in den Boden, daß sie gleiche Höhe und Breite mit ihrer Gestalt messen.}


\Fafnerspeaks
Gepflanzt sind die Pfähle nach Pfandes Maß;
Gehäuft nun füll' es der Hort!
 

\Wotanspeaks
Eilt mit dem Werk: widerlich ist mir's!
 

\Logespeaks
Hilf mir, Froh!
 

\Frohspeaks
Freias Schmach eil' ich zu enden.
 


\direct{Loge und Froh häufen hastig zwischen den Pfählen die Geschmeide}


\Fafnerspeaks
Nicht so leicht und locker gefügt!
 


\direct{er drückt mit roher Kraft die Geschmeide dicht zusammen}

Fest und dicht füll' er das Maß.
 


\direct{er beugt sich, um nach Lücken zu spähen}

Hier lug' ich noch durch:
verstopft mir die Lücken!
 

\Logespeaks
Zurück, du Grober!
 

\Fafnerspeaks
Hierher!
 

\Logespeaks
Greif' mir nichts an!
 

\Fafnerspeaks
Hierher! Die Klinze verklemmt!
 

\Wotanspeaks

\direct{unmutig sich abwendend}

Tief in der Brust brennt mir die Schmach!
 

\Frickaspeaks

\direct{den Blick auf Freia geheftet}

Sieh, wie in Scham schmählich die Edle steht:
um Erlösung fleht stumm der leidende Blick.
Böser Mann! Der Minnigen botest du das!
 

\Fafnerspeaks
Noch mehr! Noch mehr hierher!
 

\Donnerspeaks
Kaum halt' ich mich: schäumende Wut
weckt mir der schamlose Wicht!
Hierher, du Hund! Willst du messen,
so miß dich selber mit mir!
 

\Fafnerspeaks
Ruhig, Donner! Rolle, wo's taugt:
hier nützt dein Rasseln dir nichts!
 

\Donnerspeaks

\direct{holt aus}

Nicht dich Schmähl'chen zu zerschmettern?
 

\Wotanspeaks
Friede doch!
Schon dünkt mich Freia verdeckt.
 

\Logespeaks
Der Hort ging auf.
 

\Fafnerspeaks

\direct{mißt den Hort genau mit dem Blick und späht nach Lücken}

Noch schimmert mir Holdas Haar:
dort das Gewirk wirf auf den Hort!
 

\Logespeaks
Wie? Auch den Helm?
 

\Fafnerspeaks
Hurtig, her mit ihm!
 

\Wotanspeaks
Laß ihn denn fahren!
 

\Logespeaks

\direct{wirft den Tarnhelm auf den Hort}

So sind wir denn fertig!
Seid ihr zufrieden?
 

\Fasoltspeaks
Freia, die Schöne, schau' ich nicht mehr:
so ist sie gelöst? Muß ich sie lassen?
 


\direct{er tritt nahe hinzu und späht durch den Hort}

Weh! Noch blitzt ihr Blick zu mir her;
des Auges Stern strahlt mich noch an:
durch eine Spalte muß ich's erspäh'n.
 


\direct{außer sich}

Seh' ich dies wonnige Auge,
von dem Weibe lass' ich nicht ab!
 

\Fafnerspeaks
He! Euch rat' ich,
verstopft mir die Ritze!
 

\Logespeaks
Nimmersatte! Seht ihr denn nicht,
ganz schwand uns der Hort?
 

\Fafnerspeaks
Mitnichten, Freund! An Wotans Finger
glänzt von Gold noch ein Ring:
den gebt, die Ritze zu füllen!
 

\Wotanspeaks
Wie! Diesen Ring?
 

\Logespeaks
Laßt euch raten!
Den Rheintöchtern gehört dies Gold;
ihnen gibt Wotan es wieder.
 

\Wotanspeaks
Was schwatztest du da?
Was schwer ich mir erbeutet,
ohne Bangen wahr' ich's für mich!
 

\Logespeaks
Schlimm dann steht's um mein Versprechen,
das ich den Klagenden gab!
 

\Wotanspeaks
Dein Versprechen bindet mich nicht;
als Beute bleibt mir der Reif.
 

\Fafnerspeaks
Doch hier zur Lösung mußt du ihn legen.
 

\Wotanspeaks
Fordert frech, was ihr wollt,
alles gewähr' ich;
um alle Welt,
doch nicht fahren lass' ich den Ring!
 

\Fasoltspeaks

\direct{zieht wütend Freia hinter dem Horte hervor}

Aus denn ist's, beim Alten bleibt's;
nun folgt uns Freia für immer!
 

\Freiaspeaks
Hilfe! Hilfe!
 

\Frickaspeaks
Harter Gott, gib ihnen nach!
 

\Frohspeaks
Spare das Gold nicht!
 

\Donnerspeaks
Spende den Ring doch!
 


\direct{Fafner hält den fortdrängenden Fasolt noch auf; alle stehen bestürzt}


\Wotanspeaks
Laßt mich in Ruh'! Den Reif geb' ich nicht!
 


\direct{Wotan wendet sich zürnend zur Seite. Die Bühne hat sich von neuem verfinstert; aus der Felskluft zur Seite bricht ein bläulicher Schein hervor: in ihm wird plötzlich Erda sichtbar, die bis zu halber Leibeshöhe aus der Tiefe aufsteigt; sie ist von edler Gestalt, weithin von schwarzem Haar umwallt.}


\Erdaspeaks

\direct{die Hand mahnend gegen Wotan ausstreckend}

Weiche, Wotan! Weiche!
Flieh' des Ringes Fluch!
Rettungslos dunklem Verderben
weiht dich sein Gewinn.
 

\Wotanspeaks
Wer bist du, mahnendes Weib?
 

\Erdaspeaks
Wie alles war - weiß ich;
wie alles wird, wie alles sein wird,
seh' ich auch, -
der ew'gen Welt Ur-Wala,
Erda, mahnt deinen Mut. Drei der Töchter,
ur-erschaff'ne, gebar mein Schoß;
was ich sehe, sagen dir nächtlich die Nornen.
Doch höchste Gefahr führt mich heut'
selbst zu dir her.
Höre! Höre! Höre!
Alles was ist, endet.
Ein düst'rer Tag dämmert den Göttern:
dir rat' ich, meide den Ring!
 


\direct{sie versinkt langsam bis an die Brust, während der bläuliche Schein zu dunkeln beginnt}


\Wotanspeaks
Geheimnis-hehr
hallt mir dein Wort:
weile, daß mehr ich wisse!
 

\Erdaspeaks
\direct{im Versinken}
Ich warnte dich; du weißt genug:
sinn' in Sorg' und Furcht!
 


\direct{sie verschwindet gänzlich}


\Wotanspeaks
Soll ich sorgen und fürchten,
dich muß ich fassen, alles erfahren!
 


\direct{er will der Verschwindenden in die Kluft nach, um sie zu halten. Froh und Fricka werfen sich ihm entgegen und halten ihn zurück}


\Frickaspeaks
Was willst du, Wütender?
 

\Frohspeaks
Halt' ein, Wotan!
Scheue die Edle, achte ihr Wort!
 


\direct{Wotan starrt sinnend vor sich hin}


\Donnerspeaks

\direct{sich entschlossen zu den Riesen wendend}

Hört, ihr Riesen! Zurück, und harret:
das Gold wird euch gegeben.
 

\Freiaspeaks
Darf ich es hoffen?
Dünkt euch Holda wirklich der Lösung wert?
 

\direct{Alle blicken gespannt auf Wotan; dieser nach tiefem Sinnen zu sich kommend, erfaßt seinen Speer und schwenkt ihn wie zum Zeichen eines mutigen Entschlusses}

\Wotanspeaks
Zu mir, Freia! Du bist befreit.
Wieder gekauft kehr' uns die Jugend zurück!
Ihr Riesen, nehmt euren Ring!
 


\direct{er wirft den Ring auf den Hort}


\direct{Die Riesen lassen Freia los; sie eilt freudig auf die Götter zu, die sie abwechselnd längere Zeit in höchster Freude liebkosen. Fafner breitet sogleich einen ungeheuren Sack aus und macht sich über den Hort her, um ihn da hineinzuschichten}


\Fasoltspeaks

\direct{dem Bruder sich entgegenwerfend}

Halt, du Gieriger! Gönne mir auch was!
Redliche Teilung taugt uns beiden.
 

\Fafnerspeaks
Mehr an der Maid als am Gold
lag dir verliebtem Geck:
mit Müh' zum Tausch vermocht' ich dich Toren;
Ohne zu teilen, hättest du Freia gefreit:
teil' ich den Hort,
billig behalt' ich die größte Hälfte für mich.
 

\Fasoltspeaks
Schändlicher du! Mir diesen Schimpf?
 


\direct{zu den Göttern}

Euch ruf' ich zu Richtern:
teilet nach Recht uns redlich den Hort!
 


\direct{Wotan wendet sich verächtlich ab}


\Logespeaks
Den Hort laß ihn raffen;
halte du nur auf den Ring!
 

\Fasoltspeaks

\direct{stürzt sich auf Fafner, der immerzu eingesackt hat}

Zurück, du Frecher! Mein ist der Ring;
mir blieb er für Freias Blick!
 


\direct{Er greift hastig nach dem Reif. Sie ringen.}


\Fafnerspeaks
Fort mit der Faust! Der Ring ist mein!
 


\direct{Fasolt entreißt Fafner den Ring}


\Fasoltspeaks
Ich halt' ihn, mir gehört er!
 

\Fafnerspeaks

\direct{mit einem Pfahle nach Fasolt ausholend}

Halt' ihn fest, daß er nicht fall'!
 


\direct{Er streckt Fasolt mit einem Streiche zu Boden, dem Sterbenden entreißt er dann hastig den Ring}


\Fafnerspeaks
Nun blinzle nach Freias Blick!
An den Reif rührst du nicht mehr!
 


\direct{Er steckt den Ring in den Sack und rafft dann gemächlich den Hort vollends ein. Alle Götter stehen entsetzt. Langes, feierliches Schweigen}


\Wotanspeaks
Furchtbar nun erfind' ich des Fluches Kraft!
 

\Logespeaks
Was gleicht, Wotan, wohl deinem Glücke?
Viel erwarb dir des Ringes Gewinn;
daß er nun dir genommen, nützt dir noch mehr:
deine Feinde - sieh - fällen sich selbst
um das Gold, das du vergabst.
 

\Wotanspeaks

\direct{tief erschüttert}

Wie doch Bangen mich bindet!
Sorg' und Furcht fesseln den Sinn:
wie sie zu enden, lehre mich Erda:
zu ihr muß ich hinab!
 

\Frickaspeaks

\direct{schmeichelnd sich an ihn schmiegend}

Wo weilst du, Wotan?
Winkt dir nicht hold die hehre Burg,
die des Gebieters gastlich bergend nun harrt?
 

\Wotanspeaks

\direct{düster}

Mit bösem Zoll zahlt' ich den Bau.
 

\Donnerspeaks

\direct{auf den Hintergrund deutend, der noch in Nebel gehüllt ist}

Schwüles Gedünst schwebt in der Luft;
lästig ist mir der trübe Druck!
Das bleiche Gewölk
samml' ich zu blitzendem Wetter,
das fegt den Himmel mir hell.
 


\direct{er besteigt einen hohen Felsstein am Talabhange und schwingt dort seinen Hammer; Nebel ziehen sich um ihn zusammen}

He da! He da! He do!
Zu mir, du Gedüft! Ihr Dünste, zu mir!
Donner, der Herr, ruft euch zu Heer!
 


\direct{er schwingt den Hammer}

Auf des Hammers Schwung schwebet herbei!
Dunstig Gedämpf! Schwebend Gedüft!
Donner, der Herr, ruft euch zu Heer!
He da! He da! He do!
 


\direct{er verschwindet völlig in einer immer finsterer sich ballenden Gewitterwolke. Man hört Donners Hammerschlag schwer auf den Felsstein fallen: ein starker Blitz entfährt der Wolke; ein heftiger Donnerschlag folgt. Froh ist mit dem Gewölk verschwunden.}


\Donnerspeaks

\direct{unsichtbar}

Bruder, hieher! Weise der Brücke den Weg!
 


\direct{Plötzlich verzieht sich die Wolke; Donner und Froh werden sichtbar: von ihren Füßen aus zieht sich, mit blendendem Leuchten, eine Regenbogenbrücke über das Tal hinüber bis zur Burg, die jetzt, von der Abendsonne beschienen, im hellsten Glanze erstrahlt. Fafner, der neben der Leiche seines Bruders endlich den ganzen Hort eingerafft, hat, den ungeheuren Sack auf dem Rücken, während Donners Gewitterzauber die Bühne verlassen.}


\Frohspeaks

\direct{der der Brücke mit der ausgestreckten Hand den Weg über das Tal angewiesen, zu den Göttern}

Zur Burg führt die Brücke,
leicht, doch fest eurem Fuß:
beschreitet kühn ihren schrecklosen Pfad!
 


\direct{Wotan und die anderen Götter sind sprachlos in den prächtigen Anblick verloren}


\Wotanspeaks
Abendlich strahlt der Sonne Auge;
in prächtiger Glut prangt glänzend die Burg.
In des Morgens Scheine mutig erschimmernd,
lag sie herrenlos, hehr verlockend vor mir.
Von Morgen bis Abend, in Müh' und Angst,
nicht wonnig ward sie gewonnen!
Es naht die Nacht: vor ihrem Neid
biete sie Bergung nun.
 


\direct{Wie von einem großen Gedanken ergriffen, sehr entschlossen}

So grüß' ich die Burg,
sicher vor Bang' und Grau'n!
 


\direct{er wendet sich feierlich zu Fricka}

Folge mir, Frau:
in Walhall wohne mit mir!
 

\Frickaspeaks
Was deutet der Name?
Nie, dünkt mich, hört' ich ihn nennen.
 

\Wotanspeaks
Was, mächtig der Furcht,
mein Mut mir erfand,
wenn siegend es lebt,
leg' es den Sinn dir dar!
 


\direct{er faßt Fricka an der Hand und schreitet mit ihr langsam der Brücke zu; Froh, Freia und Donner folgen}


\Logespeaks

\direct{im Vordergrunde verharrend und den Göttern nachblickend}

Ihrem Ende eilen sie zu,
die so stark in Bestehen sich wähnen.
Fast schäm' ich mich, mit ihnen zu schaffen;
zur leckenden Lohe mich wieder zu wandeln,
spür' ich lockende Lust:
sie aufzuzehren, die einst mich gezähmt,
statt mit den Blinden blöd zu vergehn,
und wären es göttlichste Götter!
Nicht dumm dünkte mich das!
Bedenken will ich's: wer weiß, was ich tu'!
 


\direct{er geht, um sich den Göttern in nachlässiger Haltung anzuschließen. Aus der Tiefe hört man den Gesang der Rheintöchter heraufschallen}


Die drei Rheintöchter
\direct{in der Tiefe des Tales, unsichtbar}
Rheingold! Rheingold! Reines Gold!
Wie lauter und hell leuchtest hold du uns!
Um dich, du klares, wir nun klagen:
gebt uns das Gold!
O gebt uns das reine zurück!
 

\Wotanspeaks

\direct{im Begriff, den Fuß auf die Brücke zu setzen, hält an und wendet sich um}

Welch' Klagen klingt zu mir her?
 

\Logespeaks

\direct{späht in das Tal hinab}

Des Rheines Kinder beklagen des Goldes Raub!
 

\Wotanspeaks
Verwünschte Nicker!
 


\direct{zu Loge}

Wehre ihrem Geneck!
 

\Logespeaks

\direct{in das Tal hinabrufend}

Ihr da im Wasser, was weint ihr herauf?
Hört, was Wotan euch wünscht!
Glänzt nicht mehr euch Mädchen das Gold,
in der Götter neuem Glanze
sonnt euch selig fortan!
 


\direct{Die Götter lachen und beschreiten dann die Brücke}


\speaker{Die drei Rheintöchter}

\direct{aus der Tiefe}

Rheingold! Rheingold! Reines Gold!
O leuchtete noch in der Tiefe dein laut'rer Tand!
Traulich und treu ist's nur in der Tiefe:
falsch und feig ist, was dort oben sich freut!
 


\direct{während die Götter auf der Brücke der Burg zuschreiten, fällt der Vorhang}


\end{drama}