 \begin{drama}
   \let\oldthescene\thescene
   \let\oldscenename\scenename
   \let\oldscenecontentsline\scenecontentsline
   \renewcommand{\scenecontentsline}{}
   \renewcommand{\thescene}{}
   \renewcommand{\scenename}{}
   \let\oldsaythescene\saythescene
   \renewcommand{\saythescene}{Vorspiel}
\Scene{Vorspiel}

\StageDir{Auf dem Walkürenfelsen
  
Die Szene ist dieselbe wie am Schlusse des zweiten Tages. Nacht. Aus der Tiefe des Hintergrundes leuchtet Feuerschein. Die drei Nornen, hohe Frauengestalten in langen, dunklen und schleierartigen Faltengewändern. Die erste (älteste) lagert im Vordergrunde rechts unter der breitästigen Tanne; die zweite (jüngere) ist an einer Steinbank vor dem Felsengemache hingestreckt; die dritte (jüngste) sitzt in der Mitte des Hintergrundes auf einem Felssteine des Höhensaumes. Eine Zeitlang herrscht düsteres Schweigen.}

\DieErsteNornspeaks


\direct{ohne sich zu bewegen}

Welch Licht leuchtet dort?

\DieZweiteNornspeaks

Dämmert der Tag schon auf?
 

\DieDritteNornspeaks

Loges Heer lodert feurig um den Fels.
Noch ist's Nacht.
Was spinnen und singen wir nicht?
 

\DieZweiteNornspeaks


\direct{zu der ersten}

Wollen wir spinnen und singen,
woran spannst du das Seil?
 

\DieErsteNornspeaks


\direct{erhebt sich, während sie ein goldenes Seil von sich löst und mit dem einen Ende es an einen Ast der Tanne knüpft}

So gut und schlimm es geh',
schling' ich das Seil und singe.
An der Weltesche wob ich einst,
da groß und stark dem Stamm entgrünte
weihlicher Äste Wald.
Im kühlen Schatten rauscht' ein Quell,
Weisheit raunend rann sein Gewell';
da sang ich heil'gen Sinn.
Ein kühner Gott
trat zum Trunk an den Quell;
seiner Augen eines
zahlt' er als ewigen Zoll.
Von der Weltesche
brach da Wotan einen Ast;
eines Speeres Schaft
entschnitt der Starke dem Stamm.
In langer Zeiten Lauf
zehrte die Wunde den Wald;
falb fielen die Blätter,
dürr darbte der Baum,
traurig versiegte des Quelles Trank:
trüben Sinnes ward mein Gesang.
Doch, web' ich heut'
an der Weltesche nicht mehr,
muß mir die Tanne
taugen zu fesseln das Seil:
singe, Schwester, dir werf' ich's zu.
Weißt du, wie das wird?
 

\DieZweiteNornspeaks


\direct{windet das zugeworfene Seil um einen hervorspringenden Felsstein am Eingange des Gemaches}

Treu beratner Verträge Runen
schnitt Wotan in des Speeres Schaft:
den hielt er als Haft der Welt.
Ein kühner Held
zerhieb im Kampfe den Speer;
in Trümmer sprang
der Verträge heiliger Haft.
Da hieß Wotan Walhalls Helden
der Weltesche welkes Geäst
mit dem Stamm in Stücke zu fällen.
Die Esche sank;
ewig versiegte der Quell!
Fessle ich heut'
an den scharfen Fels das Seil:
singe, Schwester, dir werf' ich's zu.
Weißt du, wie das wird?
 

\DieDritteNornspeaks


\direct{das Seil auffangend und dessen Ende hinter sich werfend}

Es ragt die Burg, von Riesen gebaut:
mit der Götter und Helden heiliger Sippe
sitzt dort Wotan im Saal.
Gehau'ner Scheite hohe Schicht
ragt zuhauf rings um die Halle:
die Weltesche war dies einst!
Brennt das Holz
heilig brünstig und hell,
sengt die Glut
sehrend den glänzenden Saal:
der ewigen Götter Ende
dämmert ewig da auf.
Wisset ihr noch,
so windet von neuem das Seil;
von Norden wieder werf' ich's dir nach.
 


\direct{Sie wirft das Seil der zweiten Norn zu}


\DieZweiteNornspeaks


\direct{schwingt das Seil der ersten hin, die es vom Zweige löst und es an einen andern Ast wieder anknüpft}

Spinne, Schwester, und singe!
 

\DieErsteNornspeaks


\direct{nach hinten blickend}

Dämmert der Tag?
Oder leuchtet die Lohe?
Getrübt trügt sich mein Blick;
nicht hell eracht' ich das heilig Alte,
da Loge einst entbrannte in lichter Brunst.
Weißt du, was aus ihm ward?
 

\DieZweiteNornspeaks


\direct{das zugeworfene Seil wieder um den Stein windend}

Durch des Speeres Zauber
zähmte ihn Wotan;
Räte raunt' er dem Gott.
An des Schaftes Runen,
frei sich zu raten,
nagte zehrend sein Zahn:
da, mit des Speeres
zwingender Spitze
bannte ihn Wotan,
Brünnhildes Fels zu umbrennen.
Weißt du, was aus ihm wird?
 

\DieDritteNornspeaks


\direct{das zugeschwungene Seil wieder hinter sich werfend}

Des zerschlagnen Speeres
stechende Splitter
taucht einst Wotan
dem Brünstigen tief in die Brust:
zehrender Brand zündet da auf;
den wirft der Gott in der Weltesche
zuhauf geschichtete Scheite.
 


\direct{Sie wirft das Seil zurück, die zweite Norn windet es auf und wirft es der ersten wieder zu}


\DieZweiteNornspeaks

Wollt ihr wissen,
wann das wird?
Schwinget, Schwestern, das Seil!
 

\DieErsteNornspeaks


\direct{das Seil von neuem anknüpfend}

Die Nacht weicht;
nichts mehr gewahr' ich:
des Seiles Fäden find' ich nicht mehr;
verflochten ist das Geflecht.
Ein wüstes Gesicht wirrt mir wütend den Sinn:
das Rheingold raubte Alberich einst:
weißt du, was aus ihm ward?
 

\DieZweiteNornspeaks


\direct{mit mühevoller Hand das Seil um den zackigen Stein des Gemaches windend}

Des Steines Schärfe schnitt in das Seil;
nicht fest spannt mehr der Fäden Gespinst;
verwirrt ist das Geweb'.
Aus Not und Neid
ragt mir des Niblungen Ring:
ein rächender Fluch
nagt meiner Fäden Geflecht.
Weißt du, was daraus wird?
 

\DieDritteNornspeaks


\direct{das zugeworfene Seil hastig fassend}

Zu locker das Seil, mir langt es nicht.
Soll ich nach Norden neigen das Ende,
straffer sei es gestreckt!

\direct{Sie zieht gewaltsam das Seil an: dieses reißt in der Mitte}

Es riß!
 

\DieZweiteNornspeaks

Es riß!
 

\DieErsteNornspeaks

Es riß!
 


\direct{Erschreckt sind die drei Nornen aufgefahren und nach der Mitte der Bühne zusammengetreten: sie fassen die Stücke des zerrissenen Seiles und binden damit ihre Leiber aneinander}


\speaker{Die Drei Nornen}
Zu End' ewiges Wissen!
Der Welt melden Weise nichts mehr.
Hinab! Zur Mutter! Hinab!
 


\direct{Sie verschwinden}


\direct{Tagesgrauen. Wachsende Morgenröte, immer schwächeres Leuchten des Feuerscheines aus der Tiefe}


\direct{Tagesgrauen---Sonnenaufgang---Heller Tag.}


\direct{treten aus dem Steingemache auf. Siegfried ist in vollen Waffen, Brünnhilde führt ihr Roß am Zaume}


\Brunnhildespeaks

Zu neuen Taten, teurer Helde,
wie liebt' ich dich, ließ ich dich nicht?
Ein einzig' Sorgen läßt mich säumen:
daß dir zu wenig mein Wert gewann!
Was Götter mich wiesen, gab ich dir:
heiliger Runen reichen Hort;
doch meiner Stärke magdlichen Stamm
nahm mir der Held, dem ich nun mich neige.
Des Wissens bar, doch des Wunsches voll:
an Liebe reich, doch ledig der Kraft:
mögst du die Arme nicht verachten,
die dir nur gönnen, nicht geben mehr kann!
 

\Siegfriedspeaks

Mehr gabst du, Wunderfrau,
als ich zu wahren weiß.
Nicht zürne, wenn dein Lehren
mich unbelehret ließ!
Ein Wissen doch wahr' ich wohl:
daß mir Brünnhilde lebt;
eine Lehre lernt' ich leicht:
Brünnhildes zu gedenken!
 

\Brunnhildespeaks

Willst du mir Minne schenken,
gedenke deiner nur,
gedenke deiner Taten:
gedenk' des wilden Feuers,
das furchtlos du durchschrittest,
da den Fels es rings umbrann.
 

\Siegfriedspeaks

Brünnhilde zu gewinnen!
 

\Brunnhildespeaks

Gedenk' der beschildeten Frau,
die in tiefem Schlaf du fandest,
der den festen Helm du erbrachst.
 

\Siegfriedspeaks

Brünnhilde zu erwecken!
 

\Brunnhildespeaks

Gedenk' der Eide, die uns einen;
gedenk' der Treue, die wir tragen;
gedenk' der Liebe, der wir leben:
Brünnhilde brennt dann ewig
heilig dir in der Brust!
 


\direct{Sie umarmt Siegfried}


\Siegfriedspeaks

Laß ich, Liebste, dich hier
in der Lohe heiliger Hut;

\direct{Er hat den Ring Alberichs von seinem Finger gezogen und reicht ihn jetzt Brünnhilde dar}

zum Tausche deiner Runen
reich' ich dir diesen Ring.
Was der Taten je ich schuf,
des Tugend schließt er ein.
Ich erschlug einen wilden Wurm,
der grimmig lang' ihn bewacht.
Nun wahre du seine Kraft
als Weihegruß meiner Treu'!
 

\Brunnhildespeaks


\direct{voll Entzücken den Ring sich ansteckend}

Ihn geiz' ich als einziges Gut!
Für den Ring nimm nun auch mein Roß!
Ging sein Lauf mit mir
einst kühn durch die Lüfte,
mit mir verlor es die mächt'ge Art;
über Wolken hin auf blitzenden Wettern
nicht mehr schwingt es sich mutig des Wegs;
doch wohin du ihn führst---
sei es durchs Feuer---
grauenlos folgt dir Grane;
denn dir, o Helde,
soll er gehorchen!
Du hüt' ihn wohl;
er hört dein Wort:
o bringe Grane oft Brünnhildes Gruß!
 

\Siegfriedspeaks

Durch deine Tugend allein
soll so ich Taten noch wirken?
Meine Kämpfe kiesest du,
meine Siege kehren zu dir:
auf deines Rosses Rücken,
in deines Schildes Schirm,
nicht Siegfried acht' ich mich mehr,
ich bin nur Brünnhildes Arm.
 

\Brunnhildespeaks

O wäre Brünnhild' deine Seele!
 

\Siegfriedspeaks

Durch sie entbrennt mir der Mut.
 

\Brunnhildespeaks

So wärst du Siegfried und Brünnhild'?
 

\Siegfriedspeaks

Wo ich bin, bergen sich beide.
 

\Brunnhildespeaks


\direct{lebhaft}

So verödet mein Felsensaal?
 

\Siegfriedspeaks

Vereint, faßt er uns zwei!
 

\Brunnhildespeaks


\direct{in großer Ergriffenheit}

O heilige Götter!
Hehre Geschlechter!
Weidet eu'r Aug' an dem weihvollen Paar!
Getrennt---wer will es scheiden?
Geschieden---trennt es sich nie!
 

\Siegfriedspeaks

Heil dir, Brünnhilde, prangender Stern!
Heil, strahlende Liebe!
 

\Brunnhildespeaks

Heil dir, Siegfried, siegendes Licht!
Heil, strahlendes Leben!
 

\speaker{Beide}
Heil! Heil! Heil! Heil!
 

\StageDir{(Siegfried geleitet schnell das Roß dem Felsenabhange zu, wohin ihm Brünnhilde folgt. Siegfried ist mit dem Rosse hinter dem Felsenvorsprunge abwärts verschwunden, so daß der Zuschauer ihn nicht mehr sieht: Brünnhilde steht so plötzlich allein am Abhange und blickt Siegfried in die Tiefe nach. Man hört Siegfrieds Horn aus der Tiefe. Brünnhilde lauscht. Sie tritt weiter auf den Abhang hinaus und erblickt Siegfried nochmals in der Tiefe: sie winkt ihm mit entzückter Gebärde zu. Aus ihrem freudigen Lächeln deutet sich der Anblick des lustig davonziehenden Helden. Der Vorhang fällt schnell).
  
(Das Orchester nimmt die Weise des Hornes auf und führt sie in einem kräftigen Satze durch. Darauf beginnt sogleich der erste Aufzug)}


\let\saythescene\oldsaythescene
\let\thescene\oldthescene
\let\scenename\oldscenename
\let\scenecontentsline\oldscenecontentsline
\act

\scene

\StageDir{Die Halle der Gibichungen am Rhein.
  
Diese ist dem Hintergrunde zu ganz offen; den Hintergrund selbst nimmt ein freier Uferraum bis zum Flusse hin ein; felsige Anhöhen umgrenzen das Ufer.}
 

\direct{Gunther und Gutrune auf dem Hochsitze zur Seite, vor welchem ein Tisch mit Trinkgerät steht; davor sitzt Hagen}


\Guntherspeaks

Nun hör', Hagen, sage mir, Held:
sitz' ich herrlich am Rhein,
Gunther zu Gibichs Ruhm?
 

\Hagenspeaks

Dich echt genannten acht' ich zu neiden:
die beid' uns Brüder gebar,
Frau Grimhild' hieß mich's begreifen.
 

\Guntherspeaks

Dich neide ich: nicht neide mich du!
Erbt' ich Erstlingsart,
Weisheit ward dir allein:
Halbbrüderzwist bezwang sich nie besser.
Deinem Rat nur red' ich Lob,
frag' ich dich nach meinem Ruhm.
 

\Hagenspeaks

So schelt' ich den Rat,
da schlecht noch dein Ruhm;
denn hohe Güter weiß ich,
die der Gibichung noch nicht gewann.
 

\Guntherspeaks

Verschwiegest du sie,
so schelt' auch ich.
 

\Hagenspeaks

In sommerlich reifer Stärke
seh' ich Gibichs Stamm,
dich, Gunther, unbeweibt,
dich, Gutrun', ohne Mann.
 


\direct{Gunther und Gutrune sind in schweigendes Sinnen verloren}


\Guntherspeaks

Wen rätst du nun zu frein,
daß unsrem Ruhm' es fromm'?
 

\Hagenspeaks

Ein Weib weiß ich,
das herrlichste der Welt:
auf Felsen hoch ihr Sitz;
ein Feuer umbrennt ihren Saal;
nur wer durch das Feuer bricht,
darf Brünnhildes Freier sein.
 

\Guntherspeaks

Vermag das mein Mut zu bestehn?
 

\Hagenspeaks

Einem Stärkren noch ist's nur bestimmt.
 

\Guntherspeaks

Wer ist der streitlichste Mann?
 

\Hagenspeaks

Siegfried, der Wälsungen Sproß:
der ist der stärkste Held.
Ein Zwillingspaar,
von Liebe bezwungen,
Siegmund und Sieglinde,
zeugten den echtesten Sohn.
Der im Walde mächtig erwuchs,
den wünsch' ich Gutrun' zum Mann.
 

\Gutrunespeaks


\direct{schüchtern beginnend}

Welche Tat schuf er so tapfer,
daß als herrlichster Held er genannt?
 

\Hagenspeaks

Vor Neidhöhle den Niblungenhort
bewachte ein riesiger Wurm:
Siegfried schloß ihm den freislichen Schlund,
erschlug ihn mit siegendem Schwert.
Solch ungeheurer Tat
enttagte des Helden Ruhm.
 

\Guntherspeaks


\direct{in Nachsinnen}

Vom Niblungenhort vernahm ich:
er birgt den neidlichsten Schatz?
 

\Hagenspeaks

Wer wohl ihn zu nützen wüßt',
dem neigte sich wahrlich die Welt.
 

\Guntherspeaks

Und Siegfried hat ihn erkämpft?
 

\Hagenspeaks

Knecht sind die Niblungen ihm.
 

\Guntherspeaks

Und Brünnhild' gewänne nur er?
 

\Hagenspeaks

Keinem andren wiche die Brunst.
 

\Guntherspeaks


\direct{unwillig sich vom Sitze erhebend}

Wie weckst du Zweifel und Zwist!
Was ich nicht zwingen soll,
darnach zu verlangen machst du mir Lust?
 


\direct{Er schreitet bewegt in der Halle auf und ab. Hagen, ohne seinen Sitz zu verlassen, hält Gunther, als dieser wieder in seine Nähe kommt, durch einen geheimnisvollen Wink fest.}


\Hagenspeaks

Brächte Siegfried die Braut dir heim,
wär' dann nicht Brünnhilde dein?
 

\Guntherspeaks


\direct{wendet sich wieder zweifelnd und unmutig ab}

Was zwänge den frohen Mann,
für mich die Braut zu frein?
 

\Hagenspeaks


\direct{wie vorher}

Ihn zwänge bald deine Bitte,
bänd' ihn Gutrun' zuvor.
 

\Gutrunespeaks

Du Spötter, böser Hagen!
Wie sollt' ich Siegfried binden?
Ist er der herrlichste Held der Welt,
der Erde holdeste Frauen
friedeten längst ihn schon.
 

\Hagenspeaks


\direct{sehr vertraulich zu Gutrune hinneigend}

Gedenk' des Trankes im Schrein;

\direct{heimlicher}

vertraue mir, der ihn gewann:
den Helden, des du verlangst,
bindet er liebend an dich.

\direct{Gunther ist wieder an den Tisch getreten und hört, auf ihn gelehnt, jetzt aufmerksam zu}

Träte nun Siegfried ein,
genöss' er des würzigen Tranks,
daß vor dir ein Weib er ersah,
daß je ein Weib ihm genaht,
vergessen müßt' er des ganz.
Nun redet: wie dünkt euch Hagens Rat?
 

\Guntherspeaks


\direct{lebhaft auffahrend}

Gepriesen sei Grimhild',
die uns den Bruder gab!
 

\Gutrunespeaks

Möcht' ich Siegfried je ersehn!
 

\Guntherspeaks

Wie suchten wir ihn auf?
 


\direct{Ein Horn auf dem Theater klingt aus dem Hintergrunde von links her. Hagen lauscht}


\Hagenspeaks

Jagt er auf Taten wonnig umher,
zum engen Tann wird ihm die Welt:
wohl stürmt er in rastloser Jagd
auch zu Gibichs Strand an den Rhein.
 

\Guntherspeaks

Willkommen hieß' ich ihn gern!

\direct{Horn auf dem Theater, näher, aber immer noch fern. Beide lauschen}

Vom Rhein ertönt das Horn.
 

\Hagenspeaks


\direct{ist an das Ufer gegangen, späht den Fluß hinab und ruft zurück}

In einem Nachen Held und Roß!
Der bläst so munter das Horn!

\direct{Gunther bleibt auf halbem Wege lauschend zurück}

Ein gemächlicher Schlag,
wie von müßiger Hand,
treibt jach den Kahn wider den Strom;
so rüstiger Kraft in des Ruders Schwung
rühmt sich nur der, der den Wurm erschlug.
Siegfried ist es, sicher kein andrer!
 

\Guntherspeaks

Jagt er vorbei?
 

\Hagenspeaks


\direct{durch die hohlen Hände nach dem Flusse rufend}

Hoiho! Wohin,
du heitrer Held?
 

\speaker{\Siegfried (Stimme)}

\direct{aus der Ferne, vom Flusse her}

Zu Gibichs starkem Sohne.
 

\Hagenspeaks

Zu seiner Halle entbiet' ich dich.
 


\direct{Siegfried erscheint im Kahn am Ufer}

Hieher! Hier lege an!

\scene

\StageDir{Siegfried legt mit dem Kahne an und springt, nachdem Hagen den Kahn mit der Kette am Ufer festgeschlossen hat, mit dem Rosse auf den Strand}


\Hagenspeaks

Heil! Siegfried, teurer Held!
 


\direct{Gunther ist zu Hagen an das Ufer getreten. Gutrune blickt vom Hochsitze aus in staunender Bewunderung auf Siegfried. Gunther will freundlichen Gruß bieten. Alle sind in gegenseitiger stummer Betrachtung gefesselt}


\Siegfriedspeaks


\direct{auf sein Roß gelehnt, bleibt ruhig am Kahne stehen}

Wer ist Gibichs Sohn?
 

\Guntherspeaks

Gunther, ich, den du suchst.
 

\Siegfriedspeaks

Dich hört' ich rühmen weit am Rhein:
nun ficht mit mir, oder sei mein Freund!
 

\Guntherspeaks

Laß den Kampf!
Sei willkommen!
 

\Siegfriedspeaks


\direct{sieht sich ruhig um}

Wo berg' ich mein Roß?
 

\Hagenspeaks

Ich biet' ihm Rast.
 

\Siegfriedspeaks


\direct{zu Hagen gewendet}

Du riefst mich Siegfried:
sahst du mich schon?
 

\Hagenspeaks

Ich kannte dich nur an deiner Kraft.
 

\Siegfriedspeaks


\direct{indem er an Hagen das Roß übergibt}

Wohl hüte mir Grane! Du hieltest nie
von edlerer Zucht am Zaume ein Roß.
 


\direct{Hagen führt das Roß rechts hinter die Halle ab. Während Siegfried ihm gedankenvoll nachblickt, entfernt sich auch Gutrune, durch einen Wink Hagens bedeutet, von Siegfried unbemerkt, nach links durch eine Tür in ihr Gemach}


\direct{Gunther schreitet mit Siegfried, den er dazu einlädt, in die Halle vor}


\Guntherspeaks

Begrüße froh, o Held,
die Halle meines Vaters;
wohin du schreitest,
was du ersiehst,
das achte nun dein Eigen:
dein ist mein Erbe, Land und Leut',
hilf, mein Leib, meinem Eide!
Mich selbst geb' ich zum Mann.
 

\Siegfriedspeaks

Nicht Land noch Leute biete ich,
noch Vaters Haus und Hof:
einzig erbt' ich den eignen Leib;
lebend zehr' ich den auf.
Nur ein Schwert hab' ich,
selbst geschmiedet:
hilf, mein Schwert, meinem Eide!
Das biet' ich mit mir zum Bund.
 

\Hagenspeaks


\direct{der zurückgekommen ist und jetzt hinter Siegfried steht}

Doch des Niblungenhortes
nennt die Märe dich Herrn?
 

\Siegfriedspeaks


\direct{sich zu Hagen umwendend}

Des Schatzes vergaß ich fast:
so schätz' ich sein müß'ges Gut!
In einer Höhle ließ ich's liegen,
wo ein Wurm es einst bewacht'.
 

\Hagenspeaks

Und nichts entnahmst du ihm?
 

\Siegfriedspeaks


\direct{auf das stählerne Netzgewirk deutend, das er im Gürtel hängen hat}

Dies Gewirk, unkund seiner Kraft.
 

\Hagenspeaks

Den Tarnhelm kenn' ich,
der Niblungen künstliches Werk:
er taugt, bedeckt er dein Haupt,
dir zu tauschen jede Gestalt;
verlangt dich's an fernsten Ort,
er entführt flugs dich dahin.
Sonst nichts entnahmst du dem Hort?
 

\Siegfriedspeaks

Einen Ring.
 

\Hagenspeaks

Den hütest du wohl?
 

\Siegfriedspeaks

Den hütet ein hehres Weib.
 

\Hagenspeaks


\direct{für sich}

Brünnhild'!...
 

\Guntherspeaks

Nicht, Siegfried, sollst du mir tauschen:
Tand gäb' ich für dein Geschmeid,
nähmst all' mein Gut du dafür.
Ohn' Entgelt dien' ich dir gern.
 


\direct{Hagen ist zu Gutrunes Türe gegangen und öffnet sie jetzt. Gutrune tritt heraus, sie trägt ein gefülltes Trinkhorn und naht damit Siegfried}


\Gutrunespeaks

Willkommen, Gast, in Gibichs Haus!
Seine Tochter reicht dir den Trank.
 

\Siegfriedspeaks


\direct{neigt sich ihr freundlich und ergreift das Horn; er hält es gedankenvoll vor sich hin und sagt leise}

Vergäß' ich alles, was du mir gabst,
von einer Lehre lass' ich doch nie:
den ersten Trunk zu treuer Minne,
Brünnhilde, bring' ich dir!
 


\direct{Er setzt das Trinkhorn an und trinkt in einem langen Zuge. Er reicht das Horn an Gutrune zurück, die verschämt und verwirrt ihre Augen vor ihm niederschlägt}


\Siegfriedspeaks


\direct{heftet den Blick mit schnell entbrannter Leidenschaft auf sie}

Die so mit dem Blitz den Blick du mir sengst,
was senkst du dein Auge vor mir?
 


\direct{Gutrune schlägt errötend das Auge zu ihm auf}


\Siegfriedspeaks

Ha, schönstes Weib!
Schließe den Blick;
das Herz in der Brust
brennt mir sein Strahl:
zu feurigen Strömen fühl' ich
ihn zehrend zünden mein Blut!

\direct{mit bebender Stimme}

Gunther, wie heißt deine Schwester?
 

\Guntherspeaks

Gutrune.
 

\Siegfriedspeaks


\direct{leise}

Sind's gute Runen,
die ihrem Aug' ich entrate?

\direct{Er faßt Gutrune mit feurigem Ungestüm bei der Hand}

Deinem Bruder bot ich mich zum Mann:
der Stolze schlug mich aus;
trügst du, wie er, mir Übermut,
böt' ich mich dir zum Bund?
 


\direct{Gutrune trifft unwillkürlich auf Hagens Blick. Sie neigt demütig das Haupt, und mit einer Gebärde, als fühle sie sich seiner nicht wert, verläßt sie schwankenden Schrittes wieder die Halle}


\Siegfriedspeaks


\direct{von Hagen und Gunther aufmerksam beobachtet, blickt ihr, wie festgezaubert, nach; dann, ohne sich umzuwenden, fragt er}

Hast du, Gunther, ein Weib?
 

\Guntherspeaks

Nicht freit' ich noch,
und einer Frau soll ich mich schwerlich freun!
Auf eine setzt' ich den Sinn,
die kein Rat mir je gewinnt.
 

\Siegfriedspeaks


\direct{wendet sich lebhaft zu Gunther}

Was wär' dir versagt, steh' ich zu dir?
 

\Guntherspeaks

Auf Felsen hoch ihr Sitz---
 

\Siegfriedspeaks


\direct{mit verwunderungsvoller Hast einfallend}

``Auf Felsen hoch ihr Sitz;''
 

\Guntherspeaks

ein Feuer umbrennt den Saal---
 

\Siegfriedspeaks

``ein Feuer umbrennt den Saal''... ?
 

\Guntherspeaks

Nur wer durch das Feuer bricht---
 

\Siegfriedspeaks


\direct{mit der heftigsten Anstrengung, um eine Erinnerung festzuhalten}

``Nur wer durch das Feuer bricht''... ?
 

\Guntherspeaks

---darf Brünnhildes Freier sein.
 


\direct{Siegfried drückt durch eine Gebärde aus, daß bei Nennung von Brünnhildes Namen die Erinnerung ihm vollends ganz schwindet}


\Guntherspeaks

Nun darf ich den Fels nicht erklimmen;
das Feuer verglimmt mir nie!
 

\Siegfriedspeaks


\direct{kommt aus einem traumartigen Zustand zu sich und wendet sich mit übermütiger Lustigkeit zu Gunther}

Ich fürchte kein Feuer,
für dich frei ich die Frau;
denn dein Mann bin ich,
und mein Mut ist dein,
gewinn' ich mir Gutrun' zum Weib.
 

\Guntherspeaks

Gutrune gönn' ich dir gerne.
 

\Siegfriedspeaks

Brünnhilde bring' ich dir.
 

\Guntherspeaks

Wie willst du sie täuschen?
 

\Siegfriedspeaks

Durch des Tarnhelms Trug
tausch' ich mir deine Gestalt.
 

\Guntherspeaks

So stelle Eide zum Schwur!
 

\Siegfriedspeaks

Blut-Brüderschaft schwöre ein Eid!
 


\direct{Hagen füllt ein Trinkhorn mit frischem Wein; dieses hält er dann Siegfried und Gunther hin, welche sich mit ihren Schwertern die Arme ritzen und diese eine kurze Zeit über die Öffnung des Trinkhornes halten}


\direct{Siegfried und Gunther legen zwei ihrer Finger auf das Horn, welches Hagen fortwährend in ihrer Mitte hält}


\Siegfriedspeaks

Blühenden Lebens labendes Blut
träufelt' ich in den Trank.
 

\Guntherspeaks

Bruder-brünstig mutig gemischt,
blüh' im Trank unser Blut.
 

\speaker{Beide}
Treue trink' ich dem Freund.
Froh und frei entblühe dem Bund,
Blut-Brüderschaft heut'!
 

\Guntherspeaks

Bricht ein Bruder den Bund,
 

\Siegfriedspeaks

Trügt den Treuen der Freund,
 

\speaker{Beide}
Was in Tropfen heut' hold wir tranken,
in Strahlen ström' es dahin,
fromme Sühne dem Freund!
 

\Guntherspeaks


\direct{trinkt und reicht das Horn Siegfried}

So biet' ich den Bund.
 

\Siegfriedspeaks

So trink' ich dir Treu'!
 


\direct{Er trinkt und hält das geleerte Trinkhorn Hagen hin. Hagen zerschlägt mit seinem Schwerte das Horn in zwei Stücke. Siegfried und Gunther reichen sich die Hände}


\Siegfriedspeaks


\direct{betrachtet Hagen, welcher während des Schwures hinter ihm gestanden}

Was nahmst du am Eide nicht teil?
 

\Hagenspeaks

Mein Blut verdürb' euch den Trank;
nicht fließt mir's echt und edel wie euch;
störrisch und kalt stockt's in mir;
nicht will's die Wange mir röten.
Drum bleibt ich fern vom feurigen Bund.
 

\Guntherspeaks


\direct{zu Siegfried}

Laß den unfrohen Mann!
 

\Siegfriedspeaks


\direct{hängt sich den Schild wieder über}

Frisch auf die Fahrt!
Dort liegt mein Schiff;
schnell führt es zum Felsen.

\direct{Er tritt näher zu Gunther und bedeutet diesen}

Eine Nacht am Ufer harrst du im Nachen;
die Frau fährst du dann heim.

\direct{Er wendet sich zum Fortgehen und winkt Gunther, ihm zu folgen}


\Guntherspeaks

Rastest du nicht zuvor?
 

\Siegfriedspeaks

Um die Rückkehr ist mir's jach!
 


\direct{Er geht zum Ufer, um das Schiff loszubinden}


\Guntherspeaks

Du, Hagen, bewache die Halle!
 


\direct{Er folgt Siegfried zum Ufer.  Während Siegfried und Gunther, nachdem sie ihre Waffen darin niedergelegt, im Schiff das Segel aufstecken und alles zur Abfahrt bereit machen, nimmt Hagen seinen Speer und Schild}


\direct{Gutrune erscheint an der Tür ihres Gemachs, als soeben Siegfried das Schiff abstößt, welches sogleich der Mitte des Stromes zutreibt}


\Gutrunespeaks

Wohin eilen die Schnellen?
 

\Hagenspeaks


\direct{während er sich gemächlich mit Schild und Speer vor der Halle niedersetzt}

Zu Schiff---Brünnhild' zu frein.
 

\Gutrunespeaks

Siegfried?
 

\Hagenspeaks

Sieh', wie's ihn treibt,
zum Weib dich zu gewinnen!
 

\Gutrunespeaks

Siegfried mein!
 


\direct{Sie geht, lebhaft erregt, in ihr Gemach zurück}


\direct{Siegfried hat das Ruder erfaßt und treibt jetzt mit dessen Schlägen den Nachen stromabwärts, so daß dieser bald gänzlich außer Gesicht kommt}


\Hagenspeaks


\direct{sitzt mit dem Rücken an den Pfosten der Halle gelehnt, bewegungslos}

Hier sitz' ich zur Wacht, wahre den Hof,
wehre die Halle dem Feind.
Gibichs Sohne wehet der Wind,
auf Werben fährt er dahin.
lhm führt das Steuer ein starker Held,
Gefahr ihm will er bestehn:
Die eigne Braut ihm bringt er zum Rhein;
mir aber bringt er den Ring!
Ihr freien Söhne, frohe Gesellen,
segelt nur lustig dahin!
Dünkt er euch niedrig, ihr dient ihm doch,
des Niblungen Sohn.
 


\direct{Ein Teppich, welcher dem Vordergrunde zu die Halle einfaßte, schlägt zusammen und schließt die Bühne vor dem Zuschauer ab. Nachdem während eines kurzen Orchester-Zwischenspieles der Schauplatz verwandelt ist, wird der Teppich gänzlich aufgezogen}

 

\scene

\StageDir{Die Felsenhöhle (wie im Vorspiel)

  Brünnhilde sitzt am Eingange des Steingemaches, in stummen Sinnen Siegfrieds Ring betrachtend; von wonniger Erinnerung überwältigt, bedeckt sie ihn mit Küssen. Ferner Donner läßt sich vernehmen, sie blickt auf und lauscht. Dann wendet sie sich wieder zu dem Ring. Ein feuriger Blitz. Sie lauscht von neuem und späht nach der Ferne, von woher eine finstre Gewitterwolke dem Felsensaume zuzieht}


\Brunnhildespeaks

Altgewohntes Geräusch
raunt meinem Ohr die Ferne.
Ein Luftroß jagt im Laufe daher;
auf der Wolke fährt es wetternd zum Fels.
Wer fand mich Einsame auf?
 

\speaker{\Waltraute (Stimme)}

\direct{aus der Ferne}

Brünnhilde! Schwester!
Schläfst oder wachst du?
 

\Brunnhildespeaks


\direct{fährt vom Sitze auf}

Waltrautes Ruf, so wonnig mir kund!

\direct{in die Szene rufend}

Kommst du, Schwester?
Schwingst dich kühn zu mir her?

\direct{Sie eilt nach dem Felsrande}

Dort im Tann
---dir noch vertraut---
steige vom Roß
und stell' den Renner zur Rast!


\direct{Sie stürmt in den Tann, von wo ein starkes Geräusch, gleich einem Gewitterschlage, sich vernehmen läßt. Dann kommt sie in heftiger Bewegung mit Waltraute zurück; sie bleibt freudig erregt, ohne Waltrautes ängstliche Scheu zu beachten}


Kommst du zu mir?
Bist du so kühn,
magst ohne Grauen
Brünnhild' bieten den Gruß?
 

\Waltrautespeaks

Einzig dir nur galt meine Eil'!
 

\Brunnhildespeaks


\direct{in höchster freudiger Aufgeregtheit}

So wagtest du, Brünnhild' zulieb,
Walvaters Bann zu brechen?
Oder wie---o sag'---
wär' wider mich Wotans Sinn erweicht?
Als dem Gott entgegen Siegmund ich schützte,
fehlend---ich weiß es---
erfüllt' ich doch seinen Wunsch.
Daß sein Zorn sich verzogen,
weiß ich auch;
denn verschloß er mich gleich in Schlaf,
fesselt' er mich auf den Fels,
wies er dem Mann mich zur Magd,
der am Weg mich fänd' und erweckt',
meiner bangen Bitte doch gab er Gunst:
mit zehrendem Feuer umzog er den Fels,
dem Zagen zu wehren den Weg.
So zur Seligsten schuf mich die Strafe:
der herrlichste Held
gewann mich zum Weib!
In seiner Liebe leucht' und lach' ich heut' auf.
 


\direct{Sie umarmt Waltraute unter stürmischen Freudenbezeigungen, welche diese mit scheuer Ungeduld abzuwehren sucht}

Lockte dich, Schwester, mein Los?
An meiner Wonne willst du dich weiden,
teilen, was mich betraf?
 

\Waltrautespeaks


\direct{heftig}

Teilen den Taumel, der dich Törin erfaßt?
Ein andres bewog mich in Angst,
zu brechen Wotans Gebot.
 


\direct{Brünnhilde gewahrt hier erst mit Befremdung die wildaufgeregte Stimmung Waltrautes}


\Brunnhildespeaks

Angst und Furcht fesseln dich Arme?
So verzieh der Strenge noch nicht?
Du zagst vor des Strafenden Zorn?
 

\Waltrautespeaks


\direct{düster}

Dürft' ich ihn fürchten,
meiner Angst fänd' ich ein End'!
 

\Brunnhildespeaks

Staunend versteh' ich dich nicht!
 

\Waltrautespeaks

Wehre der Wallung:
achtsam höre mich an!
Nach Walhall wieder
drängt mich die Angst,
die von Walhall hierher mich trieb.
 

\Brunnhildespeaks


\direct{erschrocken}

Was ist's mit den ewigen Göttern?
 

\Waltrautespeaks

Höre mit Sinn, was ich dir sage!
Seit er von dir geschieden,
zur Schlacht nicht mehr schickte uns Wotan;
irr und ratlos ritten wir ängstlich zu Heer;
Walhalls mutige Helden mied Walvater.
Einsam zu Roß, ohne Ruh' noch Rast,
durchschweift er als Wandrer die Welt.
Jüngst kehrte er heim;
in der Hand hielt er seines Speeres Splitter:
die hatte ein Held ihm geschlagen.
Mit stummem Wink Walhalls Edle
wies er zum Forst, die Weltesche zu fällen.
Des Stammes Scheite hieß er sie schichten
zu ragendem Hauf rings um der Seligen Saal.
Der Götter Rat ließ er berufen;
den Hochsitz nahm heilig er ein:
ihm zu Seiten hieß er die Bangen sich setzen,
in Ring und Reih' die Hall' erfüllen die Helden.
So sitzt er, sagt kein Wort,
auf hehrem Sitze stumm und ernst,
des Speeres Splitter fest in der Faust;
Holdas Äpfel rührt er nicht an.
Staunen und Bangen binden starr die Götter.
Seine Raben beide sandt' er auf Reise:
kehrten die einst mit guter Kunde zurück,
dann noch einmal zum letztenmal
lächelte ewig der Gott.
Seine Knie umwindend, liegen wir Walküren;
blind bleibt er den flehenden Blicken;
uns alle verzehrt Zagen und endlose Angst.
An seine Brust preßt' ich mich weinend:
da brach sich sein Blick---
er gedachte, Brünnhilde, dein'!
Tief seufzt' er auf, schloß das Auge,
und wie im Traume
raunt' er das Wort:
``Des tiefen Rheines Töchtern
gäbe den Ring sie wieder zurück,
von des Fluches Last
erlöst wär' Gott und Welt!''
Da sann ich nach: von seiner Seite
durch stumme Reihen stahl ich mich fort;
in heimlicher Hast bestieg ich mein Roß
und ritt im Sturme zu dir.
Dich, o Schwester, beschwör' ich nun:
was du vermagst, vollend' es dein Mut!
Ende der Ewigen Qual!
 


\direct{Sie hat sich vor Brünnhilde niedergeworfen}


\Brunnhildespeaks


\direct{ruhig}

Welch' banger Träume Mären
meldest du Traurige mir!
Der Götter heiligem Himmelsnebel
bin ich Törin enttaucht:
nicht faß ich, was ich erfahre.
Wirr und wüst scheint mir dein Sinn;
in deinem Aug'---so übermüde---
glänzt flackernde Glut.
Mit blasser Wange, du bleiche Schwester,
was willst du Wilde von mir?
 

\Waltrautespeaks


\direct{heftig}

An deiner Hand, der Ring,
er ist's; hör' meinen Rat:
für Wotan wirf ihn von dir!
 

\Brunnhildespeaks

Den Ring? Von mir?
 

\Waltrautespeaks

Den Rheintöchtern gib ihn zurück!
 

\Brunnhildespeaks

Den Rheintöchtern---ich---den Ring?
Siegfrieds Liebespfand?
Bist du von Sinnen?
 

\Waltrautespeaks

Hör' mich! Hör' meine Angst!
Der Welt Unheil haftet sicher an ihm.
Wirf ihn von dir, fort in die Welle!
Walhalls Elend zu enden,
den verfluchten wirf in die Flut!
 

\Brunnhildespeaks

Ha! Weißt du, was er mir ist?
Wie kannst du's fassen, fühllose Maid!
Mehr als Walhalls Wonne,
mehr als der Ewigen Ruhm
ist mir der Ring:
ein Blick auf sein helles Gold,
ein Blitz aus dem hehren Glanz
gilt mir werter
als aller Götter ewig währendes Glück!
Denn selig aus ihm leuchtet mir Siegfrieds Liebe:
Siegfrieds Liebe!
O ließ' sich die Wonne dir sagen!
Sie wahrt mir der Reif.
Geh' hin zu der Götter heiligem Rat!
Von meinem Ringe raune ihnen zu:
die Liebe ließe ich nie,
mir nähmen nie sie die Liebe,
stürzt' auch in Trümmern
Walhalls strahlende Pracht!
 

\Waltrautespeaks

Dies deine Treue?
So in Trauer
entlässest du lieblos die Schwester?
 

\Brunnhildespeaks

Schwinge dich fort!
Fliege zu Roß!
Den Ring entführst du mir nicht!
 

\Waltrautespeaks

Wehe! Wehe!
Weh' dir, Schwester!
Walhalls Göttern weh'!
 


\direct{Sie stürzt fort. Bald erhebt sich unter Sturm eine Gewitterwolke aus dem Tann}


\Brunnhildespeaks


\direct{während sie der davonjagenden, hell erleuchteten Gewitterwolke, die sich bald gänzlich in der Ferne verliert, nachblickt}

Blitzend Gewölk,
vom Wind getragen,
stürme dahin:
zu mir nie steure mehr her!

\direct{Es ist Abend geworden. Aus der Tiefe leuchtet der Feuerschein allmählich heller auf. Brünnhilde blickt ruhig in die Landschaft hinaus}

Abendlich Dämmern deckt den Himmel;
heller leuchtet die hütende Lohe herauf.

\direct{Der Feuerschein nähert sich aus der Tiefe. Immer glühendere Flammenzungen lecken über den Felsensaum auf}

Was leckt so wütend
die lodernde Welle zum Wall?
Zur Felsenspitze wälzt sich der feurige Schwall.

\direct{Man hört aus der Tiefe Siegfrieds Hornruf nahen. Brünnhilde lauscht und fährt entzückt auf}

Siegfried! Siegfried zurück?
Seinen Ruf sendet er her!
Auf! Auf! Ihm entgegen!
In meines Gottes Arm!
 


\direct{Sie eilt in höchstem Entzücken dem Felsrande zu. Feuerflammen schlagen herauf: aus ihnen springt Siegfried auf einen hochragenden Felsstein empor, worauf die Flammen sogleich wieder zurückweichen und abermals nur aus der Tiefe heraufleuchten. Siegfried, auf dem Haupte den Tarnhelm, der ihm bis zur Hälfte das Gesicht verdeckt und nur die Augen freiläßt, erscheint in Gunthers Gestalt}


\Brunnhildespeaks


\direct{voll Entsetzen zurückweichend}

Verrat! Wer drang zu mir?
 


\direct{Sie flieht bis in den Vordergrund und heftet von da aus in sprachlosem Erstaunen ihren Blick auf Siegfried}


\Siegfriedspeaks


\direct{im Hintergrunde auf dem Steine verweilend, betrachtet sie lange, regungslos auf seinen Schild gelehnt; dann redet er sie mit verstellter---tieferer---Stimme an}

Brünnhild'! Ein Freier kam,
den dein Feuer nicht geschreckt.
Dich werb' ich nun zum Weib:
du folge willig mir!
 

\Brunnhildespeaks


\direct{heftig zitternd}

Wer ist der Mann,
der das vermochte,
was dem Stärksten nur bestimmt?
 

\Siegfriedspeaks


\direct{unverändert wie zuvor}

Ein Helde, der dich zähmt,
bezwingt Gewalt dich nur.
 

\Brunnhildespeaks


\direct{von Grausen erfaßt}

Ein Unhold schwang sich auf jenen Stein!
Ein Aar kam geflogen,
mich zu zerfleischen!
Wer bist du, Schrecklicher?
 


\direct{Langes Schweigen}

Stammst du von Menschen?
Kommst du von Hellas nächtlichem Heer?
 

\Siegfriedspeaks


\direct{wie zuvor, mit etwas bebender Stimme beginnend, alsbald aber wieder sicherer fortfahrend}

Ein Gibichung bin ich,
und Gunther heißt der Held,
dem, Frau, du folgen sollst.
 

\Brunnhildespeaks


\direct{in Verzweiflung ausbrechend}

Wotan! Ergrimmter, grausamer Gott!
Weh'! Nun erseh' ich der Strafe Sinn:
zu Hohn und Jammer jagst du mich hin!
 

\Siegfriedspeaks


\direct{springt vom Stein herab und tritt näher heran}

Die Nacht bricht an:
in diesem Gemach
mußt du dich mir vermählen!
 

\Brunnhildespeaks


\direct{indem sie den Finger, an dem sie Siegfrieds Ring trägt, drohend ausstreckt}

Bleib' fern! Fürchte dies Zeichen!
Zur Schande zwingst du mich nicht,
solang' der Ring mich beschützt.
 

\Siegfriedspeaks

Mannesrecht gebe er Gunther,
durch den Ring sei ihm vermählt!
 

\Brunnhildespeaks

Zurück, du Räuber!
Frevelnder Dieb!
Erfreche dich nicht, mir zu nahn!
Stärker als Stahl macht mich der Ring:
nie raubst du ihn mir!
 

\Siegfriedspeaks

Von dir ihn zu lösen,
lehrst du mich nun!
 


\direct{Er dringt auf sie ein; sie ringen miteinander. Brünnhilde windet sich los, flieht und wendet sich um, wie zur Wehr. Siegfried greift sie von neuem an. Sie flieht, er erreicht sie. Beide ringen heftig miteinander. Er faßt sie bei der Hand und entzieht ihrem Finger den Ring. Sie schreit heftig auf. Als sie wie zerbrochen in seinen Armen niedersinkt, streift ihr Blick bewußtlos die Augen Siegfrieds}


\Siegfriedspeaks


\direct{läßt die Machtlose auf die Steinbank vor dem Felsengemach niedergleiten}

Jetzt bist du mein,
Brünnhilde, Gunthers Braut.
Gönne mir nun dein Gemach!
 

\Brunnhildespeaks


\direct{starrt ohnmächtig vor sich hin, matt}

Was könntest du wehren, elendes Weib!
 


\direct{Siegfried treibt sie mit einer gebietenden Bewegung an. Zitternd und wankenden Schrittes geht sie in das Gemach}


\Siegfriedspeaks


\direct{das Schwert ziehend, mit seiner natürlichen Stimme}

Nun, Notung, zeuge du,
daß ich in Züchten warb.
Die Treue wahrend dem Bruder,
trenne mich von seiner Braut!
 


\direct{Er folgt Brünnhilde}


\direct{Der Vorhang fällt}


   

\act

\scene

\StageDir{Uferraum vor der Halle der Gibichungen: rechts der offene Eingang zur Halle; links das Rheinufer; von diesem aus erhebt sich eine durch verschiedene Bergpfade gespaltene, felsige Anhöhe quer über die Bühne, nach rechts dem Hintergrunde zu aufsteigend. Dort sieht man einen der Fricka errichteten Weihstein, welchem höher hinauf ein größerer für Wotan, sowie seitwärts ein gleicher dem Donner geweihter entspricht. Es ist Nacht.}
 


\direct{Hagen, den Speer im Arm, den Schild zur Seite, sitzt schlafend an einen Pfosten der Halle gelehnt. Der Mond wirft plötzlich ein grelles Licht auf ihn und seine nächste Umgebung; man gewahrt Alberich vor Hagen kauernd, die Arme auf dessen Knie gelehnt}


\Alberichspeaks


\direct{leise}

Schläfst du, Hagen, mein Sohn?
Du schläfst und hörst mich nicht,
den Ruh' und Schlaf verriet?
 

\Hagenspeaks


\direct{leise, ohne sich zu rühren, so daß er immerfort zu schlafen scheint, obwohl er die Augen offen hat}

Ich höre dich, schlimmer Albe:
was hast du meinem Schlaf zu sagen?
 

\Alberichspeaks

Gemahnt sei der Macht,
der du gebietest,
bist du so mutig,
wie die Mutter dich mir gebar!
 

\Hagenspeaks


\direct{immer wie zuvor}

Gab mir die Mutter Mut,
nicht mag ich ihr doch danken,
daß deiner List sie erlag:
frühalt, fahl und bleich,
hass' ich die Frohen, freue mich nie!
 

\Alberichspeaks


\direct{wie zuvor}

Hagen, mein Sohn! Hasse die Frohen!
Mich Lustfreien, Leidbelasteten
liebst du so, wie du sollst!
Bist du kräftig, kühn und klug:
die wir bekämpfen mit nächtigem Krieg,
schon gibt ihnen Not unser Neid.
Der einst den Ring mir entriß,
Wotan, der wütende Räuber,
vom eignen Geschlechte ward er geschlagen:
an den Wälsung verlor er Macht und Gewalt;
mit der Götter ganzer Sippe
in Angst ersieht er sein Ende.
Nicht ihn fürcht' ich mehr:
fallen muß er mit allen!
Schläfst du, Hagen, mein Sohn?
 

\Hagenspeaks


\direct{bleibt unverändert wie zuvor}

Der Ewigen Macht, wer erbte sie?
 

\Alberichspeaks

Ich und du! Wir erben die Welt.
Trüg' ich mich nicht in deiner Treu',
teilst du meinen Gram und Grimm.
Wotans Speer zerspellte der Wälsung,
der Fafner, den Wurm, im Kampfe gefällt
und kindisch den Reif sich errang.
Jede Gewalt hat er gewonnen;
Walhall und Nibelheim neigen sich ihm.

\direct{immer heimlich}

An dem furchtlosen Helden
erlahmt selbst mein Fluch:
denn nicht kennt er des Ringes Wert,
zu nichts nützt er die neidlichste Macht.
Lachend in liebender Brunst,
brennt er lebend dahin.
Ihn zu verderben, taugt uns nun einzig!
Schläfst du, Hagen, mein Sohn?
 

\Hagenspeaks


\direct{wie zuvor}

Zu seinem Verderben dient er mir schon.
 

\Alberichspeaks

Den goldnen Ring,
den Reif gilt's zu erringen!
Ein weises Weib lebt dem Wälsung zulieb':
riet es ihm je des Rheines Töchtern,
die in Wassers Tiefen einst mich betört,
zurückzugeben den Ring,
verloren ging' mir das Gold,
keine List erlangte es je.
Drum, ohne Zögern ziel' auf den Reif!
Dich Zaglosen zeugt' ich mir ja,
daß wider Helden hart du mir hieltest.
Zwar stark nicht genug, den Wurm zu bestehn,
---was allein dem Wälsung bestimmt---
zu zähem Haß doch erzog ich Hagen,
der soll mich nun rächen,
den Ring gewinnen
dem Wälsung und Wotan zum Hohn!
Schwörst du mir's, Hagen, mein Sohn?
 


\direct{Von hier an bedeckt ein immer finsterer werdender Schatten wieder Alberich. Zugleich beginnt das erste Tagesgrauen}


\Hagenspeaks


\direct{immer wie zuvor}

Den Ring soll ich haben:
harre in Ruh'!
 

\Alberichspeaks

Schwörst du mir's, Hagen, mein Held?
 

\Hagenspeaks

Mir selbst schwör' ich's;
schweige die Sorge!
 

\Alberichspeaks


\direct{wie er allmählich immer mehr dem Blicke entschwindet, wird auch seine Stimme immer unvernehmbarer}

Sei treu, Hagen, mein Sohn!
Trauter Helde! Sei treu!
Sei treu! Treu!
 


\direct{Alberich ist gänzlich verschwunden. Hagen, der unverändert in seiner Stellung verblieben, blickt regungslos und starren Auges nach dem Rheine hin, auf welchem sich die Morgendämmerung ausbreitet}


\scene

\StageDir{Der Rhein färbt sich immer stärker vom erglühenden Morgenrot. Hagen macht eine zuckende Bewegung. Siegfried tritt plötzlich, dicht am Ufer, hinter einem Busche hervor. Er ist in seiner eignen Gestalt; nur den Tarnhelm hat er noch auf dem Haupte: er zieht ihn jetzt ab und hängt ihn, während er hervorschreitet, in den Gürtel}


\Siegfriedspeaks

Hoiho, Hagen! Müder Mann!
Siehst du mich kommen?
 

\Hagenspeaks


\direct{gemächlich sich erhebend}

Hei, Siegfried?
Geschwinder Helde?
Wo brausest du her?
 

\Siegfriedspeaks

Vom Brünnhildenstein!
Dort sog ich den Atem ein,
mit dem ich dich rief:
so schnell war meine Fahrt!
Langsamer folgt mir ein Paar:
zu Schiff gelangt das her!
 

\Hagenspeaks

So zwangst du Brünnhild'?
 

\Siegfriedspeaks

Wacht Gutrune?
 

\Hagenspeaks


\direct{in die Halle rufend}

Hoiho, Gutrune! Komm' heraus!
Siegfried ist da:
was säumst du drin?
 

\Siegfriedspeaks


\direct{zur Halle sich wendend}

Euch beiden meld' ich,
wie ich Brünnhild' band.
 


\direct{Gutrune tritt ihm aus der Halle entgegen}


\Siegfriedspeaks

Heiß' mich willkommen, Gibichskind!
Ein guter Bote bin ich dir.
 

\Gutrunespeaks

Freia grüße dich zu aller Frauen Ehre!
 

\Siegfriedspeaks

Frei und hold sei nun mir Frohem:
zum Weib gewann ich dich heut'.
 

\Gutrunespeaks

So folgt Brünnhild' meinem Bruder?
 

\Siegfriedspeaks

Leicht ward die Frau ihm gefreit.
 

\Gutrunespeaks

Sengte das Feuer ihn nicht?
 

\Siegfriedspeaks

Ihn hätt' es auch nicht versehrt,
doch ich durchschritt es für ihn,
da dich ich wollt' erwerben.
 

\Gutrunespeaks

Und dich hat es verschont?
 

\Siegfriedspeaks

Mich freute die schwelende Brunst.
 

\Gutrunespeaks

Hielt Brünnhild' dich für Gunther?
 

\Siegfriedspeaks

Ihm glich ich auf ein Haar:
der Tarnhelm wirkte das,
wie Hagen tüchtig es wies.
 

\Hagenspeaks

Dir gab ich guten Rat.
 

\Gutrunespeaks

So zwangst du das kühne Weib?
 

\Siegfriedspeaks

Sie wich Gunthers Kraft.
 

\Gutrunespeaks

Und vermählte sie sich dir?
 

\Siegfriedspeaks

Ihrem Mann gehorchte Brünnhild'
eine volle bräutliche Nacht.
 

\Gutrunespeaks

Als ihr Mann doch galtest du?
 

\Siegfriedspeaks

Bei Gutrune weilte Siegfried.
 

\Gutrunespeaks

Doch zur Seite war ihm Brünnhild'?
 

\Siegfriedspeaks


\direct{auf sein Schwert deutend}

Zwischen Ost und West der Nord:
so nah war Brünnhild' ihm fern.
 

\Gutrunespeaks

Wie empfing Gunther sie nun von dir?
 

\Siegfriedspeaks

Durch des Feuers verlöschende Lohe,
im Frühnebel vom Felsen folgte sie mir zu Tal;
dem Strande nah,
flugs die Stelle tauschte Gunther mit mir:
durch des Geschmeides Tugend
wünscht' ich mich schnell hieher.
Ein starker Wind nun treibt
die Trauten den Rhein herauf:
drum rüstet jetzt den Empfang!
 

\Gutrunespeaks

Siegfried, mächtigster Mann!
Wie faßt mich Furcht vor dir!
 

\Hagenspeaks


\direct{von der Höhe im Hintergrunde den Fluß hinabspähend}

In der Ferne seh' ich ein Segel.
 

\Siegfriedspeaks

So sagt dem Boten Dank!
 

\Gutrunespeaks

Lasset uns sie hold empfangen,
daß heiter sie und gern hier weile!
Du, Hagen, minnig rufe die Mannen
nach Gibichs Hof zur Hochzeit!
Frohe Frauen ruf' ich zum Fest:
der Freudigen folgen sie gern.
 


\direct{Nach der Halle schreitend, wendet sie sich wieder um}

Rastest du, schlimmer Held?
 

\Siegfriedspeaks

Dir zu helfen, ruh' ich aus.
 


\direct{Er reicht ihr die Hand und geht mit ihr in die Halle}



\scene

\Hagenspeaks

\direct{er hat einen Felsstein in der Höhe des Hintergrundes erstiegen; dort setzt er, der Landseite zugewendet, sein Stierhorn zum Blasen an}

Hoiho! Hoihohoho!
Ihr Gibichsmannen, machet euch auf!
Wehe! Wehe! Waffen! Waffen!
Waffen durchs Land! Gute Waffen!
Starke Waffen! Scharf zum Streit.
Not ist da! Not! Wehe! Wehe!
Hoiho! Hoihohoho!
 


\direct{Hagen bleibt immer in seiner Stellung auf der Anhöhe. Er bläst abermals. Aus verschiedenen Gegenden vom Lande her antworten Heerhörner. Auf den verschiedenen Höhenpfaden stürmen in Hast und Eile gewaffnete Mannen herbei, erst einzelne, dann immer mehrere zusammen, welche sich dann auf dem Uferraum vor der Halle anhäufen}


\speaker{Die Mannen}

\direct{erst einzelne, dann immer neu hinzukommende}

Was tost das Horn?
Was ruft es zu Heer?
Wir kommen mit Wehr,
Wir kommen mit Waffen!
Hagen! Hagen!
Hoiho! Hoiho!
Welche Not ist da?
Welcher Feind ist nah?
Wer gibt uns Streit?
Ist Gunther in Not?
Wir kommen mit Waffen,
mit scharfer Wehr.
Hoiho! Ho! Hagen!
 

\Hagenspeaks


\direct{immer von der Anhöhe herab}

Rüstet euch wohl und rastet nicht;
Gunther sollt ihr empfahn:
ein Weib hat der gefreit.
 

\speaker{Die Mannen}
Drohet ihm Not?
Drängt ihn der Feind?
 

\Hagenspeaks

Ein freisliches Weib führet er heim.
 

\speaker{Die Mannen}
Ihm folgen der Magen feindliche Mannen?
 

\Hagenspeaks

Einsam fährt er: keiner folgt.
 

\speaker{Die Mannen}
So bestand er die Not?
So bestand er den Kampf?
Sag' es an!
 

\Hagenspeaks

Der Wurmtöter wehrte der Not:
Siegfried, der Held, der schuf ihm Heil!
 

\speaker{Ein Mann}
Was soll ihm das Heer nun noch helfen?
 

\speaker{Zehn Weitere}
Was hilft ihm nun das Heer?
 

\Hagenspeaks

Starke Stiere sollt ihr schlachten;
am Weihstein fließe Wotan ihr Blut!
 

\speaker{Ein Mann}
Was, Hagen, was heißest du uns dann?
 

\speaker{Acht Mannen}
Was heißest du uns dann?
 

\speaker{Vier Weitere}
Was soll es dann?
 

\speaker{Alle}
Was heißest du uns dann?
 

\Hagenspeaks

Einen Eber fällen sollt ihr für Froh!
Einen stämmigen Bock stechen für Donner!
Schafe aber schlachtet für Fricka,
daß gute Ehe sie gebe!
 

\speaker{Die Mannen}

\direct{mit immer mehr ausbrechender Heiterkeit}

Schlugen wir Tiere,
was schaffen wir dann?
 

\Hagenspeaks

Das Trinkhorn nehmt,
von trauten Frau'n
mit Met und Wein wonnig gefüllt!
 

\speaker{Die Mannen}
Das Trinkhorn zur Hand,
wie halten wir es dann?
 

\Hagenspeaks

Rüstig gezecht, bis der Rausch euch zähmt!
Alles den Göttern zu Ehren,
daß gute Ehe sie geben!
 

\speaker{Die Mannen}

\direct{brechen in ein schallendes Gelächter aus}

Groß Glück und Heil lacht nun dem Rhein,
da Hagen, der Grimme, so lustig mag sein!
Der Hagedorn sticht nun nicht mehr;
zum Hochzeitsrufer ward er bestellt.
 

\Hagenspeaks


\direct{der immer sehr ernst geblieben, ist zu den Mannen herabgestiegen und steht jetzt unter ihnen}

Nun laßt das Lachen, mut'ge Mannen!
Empfangt Gunthers Braut!
Brünnhilde naht dort mit ihm.
 


\direct{Er deutet die Mannen nach dem Rhein hin: diese eilen zum Teil nach der Anhöhe, während andere sich am Ufer aufstellen, um die Ankommenden zu erblicken}


\direct{Näher zu einigen Mannen tretend}

Hold seid der Herrin,
helfet ihr treu:
traf sie ein Leid,
rasch seid zur Rache!
 


\direct{Er wendet sich langsam zur Seite, in den Hintergrund}


\direct{Während des Folgenden kommt der Nachen mit Gunther und Brünnhilde auf dem Rheine an}


\speaker{Die Mannen}

\direct{diejenigen, welche von der Höhe ausgeblickt hatten, kommen zum Ufer herab}

Heil! Heil!
Willkommen! Willkommen!
 


\direct{Einige der Mannen springen in den Fluß und ziehen den Kahn an das Land. Alles drängt sich immer dichter an das Ufer}

Willkommen, Gunther!
Heil! Heil!



\direct{Gunther steigt mit Brünnhilde aus dem Kahne; die Mannen reihen sich ehrerbietig zu ihren Empfange. Während des Folgenden geleitet Gunther Brünnhilde feierlich an der Hand}


\speaker{Die Mannen}
Heil dir, Gunther!
Heil dir und deiner Braut!
Willkommen!
 


\direct{Sie schlagen die Waffen tosend zusammen}


\Guntherspeaks


\direct{Brünnhilde, welche bleich und gesenkten Blickes ihm folgt, den Mannen vorstellend}

Brünnhild', die hehrste Frau,
bring' ich euch her zum Rhein.
Ein edleres Weib ward nie gewonnen.
Der Gibichungen Geschlecht,
gaben die Götter ihm Gunst,
zum höchsten Ruhm rag' es nun auf!
 

\speaker{Die Mannen}

\direct{feierlich an ihre Waffen schlagend}

Heil! Heil dir,
glücklicher Gibichung!
 


\direct{Gunther geleitet Brünnhilde, die nie aufblickt, zur Halle, aus welcher jetzt Siegfried und Gutrune, von Frauen begleitet, heraustreten}


\Guntherspeaks


\direct{hält vor der Halle an}

Gegrüßt sei, teurer Held;
gegrüßt, holde Schwester!
Dich seh' ich froh ihm zur Seite,
der dich zum Weib gewann.
Zwei sel'ge Paare
seh ich hier prangen:
 


\direct{Er führt Brünnhilde näher heran}

Brünnhild' und Gunther,
Gutrun' und Siegfried!
 


\direct{Brünnhilde schlägt erschreckt die Augen auf und erblickt Siegfried; wie in Erstaunen bleibt ihr Blick auf ihn gerichtet. Gunther, welcher Brünnhildes heftig zuckende Hand losgelassen hat, sowie alle übrigen zeigen starre Betroffenheit über Brünnhildes Benehmen}


\speaker{Mannen und Frauen}
Was ist ihr? Ist sie entrückt?
 


\direct{Brünnhilde beginnt zu zittern}


\Siegfriedspeaks


\direct{geht ruhig einige Schritte auf Brünnhilde zu}

Was müht Brünnhildes Blick?
 

\Brunnhildespeaks


\direct{kaum ihrer mächtig}

Siegfried... hier...! Gutrune...?
 

\Siegfriedspeaks

Gunthers milde Schwester:
mir vermählt wie Gunther du.
 

\Brunnhildespeaks


\direct{furchtbar heftig}

Ich.... Gunther... ? Du lügst!

\direct{Sie schwankt und droht umzusinken: Siegfried, ihr zunächst, stützt sie}

Mir schwindet das Licht ....

\direct{Sie blickt in seinen Armen matt zu Siegfried auf}

Siegfried kennt mich nicht!
 

\Siegfriedspeaks

Gunther, deinem Weib ist übel!

\direct{Gunther tritt hinzu}

Erwache, Frau!
Hier steht dein Gatte.
 

\Brunnhildespeaks


\direct{erblickt am ausgestreckten Finger Siegfrieds den Ring und schrickt mit furchtbarer Heftigkeit auf}

Ha! Der Ring 
an seiner Hand!
Er? Siegfried?
 

\speaker{Mannen und Frauen}
Was ist?
 

\Hagenspeaks


\direct{aus dem Hintergrunde unter die Mannen tretend}

Jetzt merket klug,
was die Frau euch klagt!
 

\Brunnhildespeaks


\direct{sucht sich zu ermannen, indem sie die schrecklichste Aufregung gewaltsam zurückhält}

Einen Ring sah ich an deiner Hand,
nicht dir gehört er,
ihn entriß mir

\direct{auf Gunther deutend}

dieser Mann!
Wie mochtest von ihm
den Ring du empfahn?
 

\Siegfriedspeaks


\direct{aufmerksam den Ring an seiner Hand betrachtend}

Den Ring empfing ich nicht von ihm.
 

\Brunnhildespeaks


\direct{zu Gunther}

Nahmst du von mir den Ring,
durch den ich dir vermählt;
so melde ihm dein Recht,
fordre zurück das Pfand!
 

\Guntherspeaks


\direct{in großer Verwirrung}

Den Ring? Ich gab ihm keinen:
doch kennst du ihn auch gut?
 

\Brunnhildespeaks

Wo bärgest du den Ring,
den du von mir erbeutet?
 


\direct{Gunther schweigt in höchster Betroffenheit}


\Brunnhildespeaks


\direct{wütend auffahrend}

Ha! Dieser war es,
der mir den Ring entriß:
Siegfried, der trugvolle Dieb!
 


\direct{Alles blickt erwartungsvoll auf Siegfried, welcher über der Betrachtung des Ringes in fernes Sinnen entrückt ist}


\Siegfriedspeaks

Von keinem Weib kam mir der Reif;
noch war's ein Weib, dem ich ihn abgewann:
genau erkenn' ich des Kampfes Lohn,
den vor Neidhöhl' einst ich bestand,
als den starken Wurm ich erschlug.
 

\Hagenspeaks


\direct{zwischen sie tretend}

Brünnhild', kühne Frau,
kennst du genau den Ring?
Ist's der, den du Gunthern gabst,
so ist er sein,
und Siegfried gewann ihn durch Trug,
den der Treulose büßen sollt'!
 

\Brunnhildespeaks


\direct{in furchtbarstem Schmerze aufschreiend}

Betrug! Betrug! Schändlichster Betrug!
Verrat! Verrat! Wie noch nie er gerächt!
 

\Gutrunespeaks

Verrat? An wem?
 

\speaker{Mannen und Frauen}
Verrat? Verrat?
 

\Brunnhildespeaks

Heil'ge Götter, himmlische Lenker!
Rauntet ihr dies in eurem Rat?
Lehrt ihr mich Leiden, wie keiner sie litt?
Schuft ihr mir Schmach, wie nie sie geschmerzt?
Ratet nun Rache, wie nie sie gerast!
Zündet mir Zorn, wie noch nie er gezähmt!
Heißet Brünnhild' ihr Herz zu zerbrechen,
den zu zertrümmern, der sie betrog!
 

\Guntherspeaks

Brünnhild', Gemahlin!
Mäß'ge dich!
 

\Brunnhildespeaks

Weich' fern, Verräter!
Selbst Verrat'ner
Wisset denn alle: nicht ihm,
dem Manne dort bin ich vermählt.
 

\speaker{Frauen}
Siegfried? Gutruns Gemahl?
 

\speaker{Mannen}
Gutruns Gemahl?
 

\Brunnhildespeaks

Er zwang mir Lust und Liebe ab.
 

\Siegfriedspeaks

Achtest du so der eignen Ehre?
Die Zunge, die sie lästert,
muß ich der Lüge sie zeihen?
Hört, ob ich Treue brach!
Blutbrüderschaft
hab' ich Gunther geschworen:
Notung, das werte Schwert,
wahrte der Treue Eid;
mich trennte seine Schärfe
von diesem traur'gen Weib.
 

\Brunnhildespeaks

Du listiger Held, sieh', wie du lügst!
Wie auf dein Schwert du schlecht dich berufst!
Wohl kenn' ich seine Schärfe,
doch kenn' auch die Scheide,
darin so wonnig ruht' an der Wand
Notung, der treue Freund,
als die Traute sein Herr sich gefreit.
 

\speaker{Die Mannen}

\direct{in lebhafter Entrüstung zusammentretend}

Wie? Brach er die Treue?
Trübte er Gunthers Ehre?
 

\speaker{Die Frauen}
Brach er die Treue?
 

\Guntherspeaks


\direct{zu Siegfried}

Geschändet wär' ich, schmählich bewahrt,
gäbst du die Rede nicht ihr zurück!
 

\Gutrunespeaks

Treulos, Siegfried, sannest du Trug?
Bezeuge, daß jene falsch dich zeiht!
 

\speaker{Die Mannen}
Reinige dich, bist du im Recht!
Schweige die Klage!
Schwöre den Eid!
 

\Siegfriedspeaks

Schweig' ich die Klage,
schwör' ich den Eid:
wer von euch wagt seine Waffe daran?
 

\Hagenspeaks

Meines Speeres Spitze wag' ich daran:
sie wahr' in Ehren den Eid.
 


\direct{Die Mannen schließen einen Ring um Siegfried und Hagen. Hagen hält den Speer hin; Siegfried legt zwei Finger seiner rechten Hand auf die Speerspitze}


\Siegfriedspeaks

Helle Wehr! Heilige Waffe!
Hilf meinem ewigen Eide!
Bei des Speeres Spitze sprech' ich den Eid:
Spitze, achte des Spruchs!
Wo Scharfes mich schneidet,
schneide du mich;
wo der Tod mich soll treffen,
treffe du mich:
klagte das Weib dort wahr,
brach ich dem Bruder den Eid!
 

\Brunnhildespeaks


\direct{tritt wütend in den Ring, reißt Siegfrieds Hand vom Speere hinweg und faßt dafür mit der ihrigen die Spitze}

Helle Wehr! Heilige Waffe!
Hilf meinem ewigen Eide!
Bei des Speeres Spitze sprech' ich den Eid:
Spitze, achte des Spruchs!
Ich weihe deine Wucht,
daß sie ihn werfe!
Deine Schärfe segne ich,
daß sie ihn schneide:
denn, brach seine Eide er all',
schwur Meineid jetzt dieser Mann!
 

\speaker{Die Mannen}

\direct{im höchsten Aufruhr}

Hilf, Donner, tose dein Wetter,
zu schweigen die wütende Schmach!
 

\Siegfriedspeaks

Gunther! Wehr' deinem Weibe,
das schamlos Schande dir lügt!
Gönnt ihr Weil' und Ruh',
der wilden Felsenfrau,
daß ihre freche Wut sich lege,
die eines Unholds arge List
wider uns alle erregt!
Ihr Mannen, kehret euch ab!
Laßt das Weibergekeif'!
Als Zage weichen wir gern,
gilt es mit Zungen den Streit.

\direct{Er tritt dicht zu Gunther}

Glaub', mehr zürnt es mich als dich,
daß schlecht ich sie getäuscht:
der Tarnhelm, dünkt mich fast,
hat halb mich nur gehehlt.
Doch Frauengroll friedet sich bald:
daß ich dir es gewann,
dankt dir gewiß noch das Weib.

\direct{Er wendet sich wieder zu den Mannen}

Munter, ihr Mannen!
Folgt mir zum Mahl!

\direct{zu den Frauen}

Froh zur Hochzeit, helfet, ihr Frauen!
Wonnige Lust lache nun auf!
In Hof und Hain,
heiter vor allen sollt ihr heute mich sehn.
Wen die Minne freut,
meinem frohen Mute
tu' es der Glückliche gleich!
 


\direct{Er schlingt in ausgelassenem Übermute seinen Arm um Gutrune und zieht sie mit sich in die Halle fort. Die Mannen und Frauen, von seinem Beispiele hingerissen, folgen ihm nach}


\direct{Die Bühne ist leer geworden. Nur Brünnhilde, Gunther und Hagen bleiben zurück. Gunther hat sich in tiefer Scham und furchtbarer Verstimmung mit verhülltem Gesichte abseits niedergesetzt. Brünnhilde, im Vordergrunde stehend, blickt Siegfried und Gutrune noch eine Zeitlang schmerzlich nach und senkt dann das Haupt.}


\Brunnhildespeaks


\direct{in starrem Nachsinnen befangen}

Welches Unholds List liegt hier verhohlen?
Welches Zaubers Rat regte dies auf?
Wo ist nun mein Wissen gegen dies Wirrsal?
Wo sind meine Runen gegen dies Rätsel?
Ach Jammer! Jammer! Weh', ach Wehe!
All mein Wissen wies ich ihm zu!
In seiner Macht hält er die Magd;
in seinen Banden faßt er die Beute,
die, jammernd ob ihrer Schmach,
jauchzend der Reiche verschenkt!
Wer bietet mir nun das Schwert,
mit dem ich die Bande zerschnitt'?
 

\Hagenspeaks


\direct{dicht an sie herantretend}

Vertraue mir, betrog'ne Frau!
Wer dich verriet, das räche ich.
 

\Brunnhildespeaks


\direct{matt sich umblickend}

An wem?
 

\Hagenspeaks

An Siegfried, der dich betrog.
 

\Brunnhildespeaks

An Siegfried?... Du?

\direct{bitter lächelnd}

Ein einz'ger Blick seines blitzenden Auges,
das selbst durch die Lügengestalt
leuchtend strahlte zu mir,
deinen besten Mut
machte er bangen!
 

\Hagenspeaks

Doch meinem Speere
spart ihn sein Meineid?
 

\Brunnhildespeaks

Eid und Meineid, müßige Acht!
Nach Stärkrem späh',
deinen Speer zu waffnen,
willst du den Stärksten bestehn!
 

\Hagenspeaks

Wohl kenn' ich Siegfrieds siegende Kraft,
wie schwer im Kampf er zu fällen;
drum raune nun du mir klugen Rat,
wie doch der Recke mir wich'?
 

\Brunnhildespeaks

O Undank, schändlichster Lohn!
Nicht eine Kunst war mir bekannt,
die zum Heil nicht half seinem Leib'!
Unwissend zähmt' ihn mein Zauberspiel,
das ihn vor Wunden nun gewahrt.
 

\Hagenspeaks

So kann keine Wehr ihm schaden?
 

\Brunnhildespeaks

Im Kampfe nicht; doch
träfst du im Rücken ihn....
Niemals---das wußt ich---
wich' er dem Feind,
nie reicht' er fliehend ihm den Rücken:
an ihm drum spart' ich den Segen.
 

\Hagenspeaks

Und dort trifft ihn mein Speer!

\direct{Er wendet sich rasch von Brünnhilde ab zu Gunther}

Auf, Gunther, edler Gibichung!
Hier steht dein starkes Weib:
was hängst du dort in Harm?
 

\Guntherspeaks


\direct{leidenschaftlich auffahrend}

O Schmach! O Schande!
Wehe mir, dem jammervollsten Manne!
 

\Hagenspeaks

In Schande liegst du;
leugn' ich das?
 

\Brunnhildespeaks


\direct{zu Gunther}

O feiger Mann! Falscher Genoss'!
Hinter dem Helden hehltest du dich,
daß Preise des Ruhmes er dir erränge!
Tief wohl sank das teure Geschlecht,
das solche Zagen gezeugt!
 

\Guntherspeaks


\direct{außer sich}

Betrüger ich und betrogen!
Verräter ich und verraten!
Zermalmt mir das Mark!
Zerbrecht mir die Brust!
Hilf, Hagen! Hilf meiner Ehre!
Hilf deiner Mutter,
die mich auch ja gebar!
 

\Hagenspeaks

Dir hilft kein Hirn,
dir hilft keine Hand:
dir hilft nur Siegfrieds Tod!
 

\Guntherspeaks


\direct{von Grausen erfaßt}

Siegfrieds Tod!
 

\Hagenspeaks

Nur der sühnt deine Schmach!
 

\Guntherspeaks


\direct{vor sich hinstarrend}

Blutbrüderschaft schwuren wir uns!
 

\Hagenspeaks

Des Bundes Bruch sühne nun Blut!
 

\Guntherspeaks

Brach er den Bund?
 

\Hagenspeaks

Da er dich verriet!
 

\Guntherspeaks

Verriet er mich?
 

\Brunnhildespeaks

Dich verriet er,
und mich verrietet ihr alle!
Wär' ich gerecht, alles Blut der Welt
büßte mir nicht eure Schuld!
Doch des einen Tod taugt mir für alle:
Siegfried falle zur Sühne für sich und euch!
 

\Hagenspeaks


\direct{heimlich zu Gunther}

Er falle---dir zum Heil!
Ungeheure Macht wird dir,
gewinnst von ihm du den Ring,
den der Tod ihm wohl nur entreißt.
 

\Guntherspeaks


\direct{leise}

Brünnhildes Ring?
 

\Hagenspeaks

Des Nibelungen Reif.
 

\Guntherspeaks


\direct{schwer seufzend}

So wär' es Siegfrieds Ende!
 

\Hagenspeaks

Uns allen frommt sein Tod.
 

\Guntherspeaks

Doch Gutrune, ach, der ich ihn gönnte!
Straften den Gatten wir so,
wie bestünden wir vor ihr?
 

\Brunnhildespeaks


\direct{wild auffahrend}

Was riet mir mein Wissen?
Was wiesen mich Runen?
Im hilflosen Elend achtet mir's hell:
Gutrune heißt der Zauber,
der den Gatten mir entrückt!
Angst treffe sie!
 

\Hagenspeaks


\direct{zu Gunther}

Muß sein Tod sie betrüben,
verhehlt sei ihr die Tat.
Auf muntres Jagen ziehen wir morgen:
der Edle braust uns voran,
ein Eber bracht' ihn da um.
 

\speaker{\Gunther und \Brunnhilde}
So soll es sein! Siegfried falle!
Sühn' er die Schmach, die er mir schuf!
Des Eides Treue hat er getrogen:
mit seinem Blut büß' er die Schuld!
Allrauner, rächender Gott!
Schwurwissender Eideshort!
Wotan! Wende dich her!
Weise die schrecklich heilige Schar,
hieher zu horchen dem Racheschwur!
 

\Hagenspeaks

Sterb' er dahin, der strahlende Held!
Mein ist der Hort, mir muß er gehören.
Drum sei der Reif ihm entrissen.
Alben-Vater, gefallner Fürst!
Nachthüter! Niblungenherr!
Alberich! Achte auf mich!
Weise von neuem der Niblungen Schar,
dir zu gehorchen, des Ringes Herrn!
 

\StageDir{Als Gunther mit Brünnhilde heftig der Halle sich zuwendet, tritt ihnen der von dort heraustretende Brautzug entgegen. Knaben und Mädchen, Blumenstäbe schwingend, springen lustig voraus. Siegfried wird auf einem Schilde, Gutrune auf einem Sessel von den Männern getragen. Auf der Anhöhe des Hintergrundes führen Knechte und Mägde auf verschiedenen Bergpfaden Opfergeräte und Opfertiere zu den Weihsteinen herbei und schmücken diese mit Blumen. Siegfried und die Mannen blasen auf ihren Hörnern den Hochzeitsruf. Die Frauen fordern Brünnhilde auf, an Gutrunes Seite sie zu geleiten. Brünnhilde blickt starr zu Gutrune auf, welche ihr mit freundlichem Lächeln zuwinkt. Als Brünnhilde heftig zurücktreten will, tritt Hagen rasch dazwischen und drängt sie an Gunther, der jetzt von neuem ihre Hand erfaßt, worauf er selbst von den Männern sich auf den Schild heben läßt. Während der Zug, kaum unterbrochen, schnell der Höhe zu sich wieder in Bewegung setzt, fällt der Vorhang}
   

\act
 
\scene

\StageDir{Wildes Wald- und Felsental am Rheine, welcher im Hintergrunde an einem steilen Abhange vorbeifließt.}


\direct{Die drei Rheintöchter, Woglinde, Wellgunde und Flosshilde, tauchen aus der Flut auf und schwimmen, wie im Reigentanze, im Kreise umher}


\speaker{Die Drei Rheintöchter}

\direct{im Schwimmen mäßig einhaltend}

Frau Sonne sendet lichte Strahlen;
Nacht liegt in der Tiefe:
einst war sie hell,
da heil und hehr
des Vaters Gold noch in ihr glänzte.
Rheingold! Klares Gold!
Wie hell du einstens strahltest,
hehrer Stern der Tiefe!

\direct{Sie schließen wieder den Schwimmreigen}

Weialala leia, wallala leialala.

\direct{Ferner Hornruf. Sie lauschen. Sie schlagen jauchzend das Wasser}

Frau Sonne, sende uns den Helden,
der das Gold uns wiedergäbe!
Ließ' er es uns, dein lichtes Auge
neideten dann wir nicht länger.
Rheingold! Klares Gold!
Wie froh du dann strahltest,
freier Stern der Tiefe!

\direct{man hört Siegfrieds Horn von der Höhe her}


\Woglindespeaks

Ich höre sein Horn.
 

\Wellgundespeaks

Der Helde naht.
 

\Flosshildespeaks

Laßt uns beraten!
 


\direct{Sie tauchen alle drei schnell unter}


\direct{Siegfried erscheint auf dem Abhange in vollen Waffen}


\Siegfriedspeaks

Ein Albe führte mich irr,
daß ich die Fährte verlor:
He, Schelm, in welchem Berge
bargst du so schnell mir das Wild?
 

\speaker{Die Drei Rheintöchter}

\direct{tauchen wieder auf und schwimmen im Reigen}

Siegfried!
 

\Flosshildespeaks

Was schiltst du so in den Grund?
 

\Wellgundespeaks

Welchem Alben bist du gram?
 

\Woglindespeaks

Hat dich ein Nicker geneckt?
 

\speaker{Alle Drei}
Sag' es, Siegfried, sag' es uns!
 

\Siegfriedspeaks


\direct{sie lächelnd betrachtend}

Entzücktet ihr zu euch den zottigen Gesellen,
der mir verschwand?
Ist's euer Friedel,
euch lustigen Frauen lass' ich ihn gern.

\direct{Die Mädchen lachen laut auf}


\Woglindespeaks

Siegfried, was gibst du uns,
wenn wir das Wild dir gönnen?
 

\Siegfriedspeaks

Noch bin ich beutelos;
so bittet, was ihr begehrt.
 

\Wellgundespeaks

Ein goldner Ring ragt dir am Finger!
 

\speaker{Die Drei Rheintöchter}
Den gib uns!
 

\Siegfriedspeaks

Einen Riesenwurm erschlug ich um den Reif:
für eines schlechten Bären Tatzen
böt' ich ihn nun zum Tausch?
 

\Woglindespeaks

Bist du so karg?
 

\Wellgundespeaks

So geizig beim Kauf?
 

\Flosshildespeaks

Freigebig solltest Frauen du sein.
 

\Siegfriedspeaks

Verzehrt' ich an euch mein Gut,
des zürnte mir wohl mein Weib.
 

\Flosshildespeaks

Sie ist wohl schlimm?
 

\Wellgundespeaks

Sie schlägt dich wohl?
 

\Woglindespeaks

Ihre Hand fühlt schon der Held!
 


\direct{Sie lachen unmäßig}


\Siegfriedspeaks

Nun lacht nur lustig zu!
In Harm lass' ich euch doch:
denn giert ihr nach dem Ring,
euch Nickern geb' ich ihn nie!
 


\direct{Die Rheintöchter haben sich wieder zum Reigen gefaßt}


\Flosshildespeaks

So schön!
 

\Wellgundespeaks

So stark!
 

\Woglindespeaks

So gehrenswert!
 

\speaker{Alle Drei}
Wie schade, daß er geizig ist!
 


\direct{Sie lachen und tauchen unter}


\Siegfriedspeaks


\direct{tiefer in den Grund hinabsteigend}

Was leid' ich doch das karge Lob?
Lass' ich so mich schmähn?
Kämen sie wieder zum Wasserrand,
den Ring könnten sie haben.

\direct{laut rufend}

He! He, he! Ihr muntren Wasserminnen!
Kommt rasch! Ich schenk' euch den Ring!


\direct{Er hat den Ring vom Finger gezogen und hält ihn in die Höhe}


\direct{Die drei Rheintöchter tauchen wieder auf. Sie zeigen sich ernst und feierlich}


\Flosshildespeaks

Behalt' ihn, Held, und wahr' ihn wohl,
bis du das Unheil errätst -
 

\speaker{\Woglinde und \Wellgunde}
das in dem Ring du hegst.
 

\speaker{Alle Drei}
Froh fühlst du dich dann,
befrein wir dich von dem Fluch.
 

\Siegfriedspeaks


\direct{steckt gelassen den Ring wieder an seinen Finger}

So singet, was ihr wißt!
 

\speaker{Die Rheintöchter}
Siegfried! Siegfried! Siegfried!
Schlimmes wissen wir dir.
 

\Wellgundespeaks

Zu deinem Unheil wahrst du den Reif!
 

\speaker{Alle Drei}
Aus des Rheines Gold ist der Reif geglüht.
 

\Wellgundespeaks

Der ihn listig geschmiedet und schmählich verlor -
 

\speaker{Alle Drei}
der verfluchte ihn, in fernster Zeit
zu zeugen den Tod dem, der ihn trüg'.
 

\Flosshildespeaks

Wie den Wurm du fälltest -
 

\speaker{\Wellgunde und \Flosshilde}
so fällst auch du -
 

\speaker{Alle Drei}
und heute noch:
So heißen wir's dir,
tauschest den Ring du uns nicht -
 

\speaker{\Wellgunde und \Flosshilde}
im tiefen Rhein ihn zu bergen:
 

\speaker{Alle Drei}
Nur seine Flut sühnet den Fluch!
 

\Siegfriedspeaks

Ihr listigen Frauen, laßt das sein!
Traut' ich kaum eurem Schmeicheln,
euer Drohen schreckt mich noch minder!
 

\speaker{Die Drei Rheintöchter}
Siegfried! Siegfried!
Wir weisen dich wahr.
Weiche, weiche dem Fluch!
Ihn flochten nächtlich webende Nornen
in des Urgesetzes Seil!
 

\Siegfriedspeaks

Mein Schwert zerschwang einen Speer:
des Urgesetzes ewiges Seil,
flochten sie wilde Flüche hinein,
Notung zerhaut es den Nornen!
Wohl warnte mich einst
vor dem Fluch ein Wurm,
doch das Fürchten lehrt' er mich nicht!

\direct{Er betrachtet den Ring}

Der Welt Erbe gewänne mir ein Ring:
für der Minne Gunst miss' ich ihn gern;
ich geb' ihn euch, gönnt ihr mir Lust.
Doch bedroht ihr mir Leben und Leib:
faßte er nicht eines Fingers Wert,
den Reif entringt ihr mir nicht!
Denn Leben und Leib,
seht:so werf' ich sie weit von mir!
 


\direct{Er hebt eine Erdscholle vom Boden auf, hält sie über seinem Haupte und wirft sie mit den letzten Worten hinter sich}


\speaker{Die Drei Rheintöchter}
Kommt, Schwestern!
Schwindet dem Toren!
So weise und stark verwähnt sich der Held,
als gebunden und blind er doch ist.

\direct{Sie schwimmen, wild aufgeregt, in weiten Schwenkungen dicht an das Ufer heran}

Eide schwur er---und achtet sie nicht.

\direct{Wieder heftige Bewegung}

Runen weiß er---und rät sie nicht!
 

\speaker{\Flosshilde, dann \Woglinde}
Ein hehrstes Gut ward ihm vergönnt.
 

\speaker{Alle Drei}
Daß er's verworfen, weiß er nicht;
 

\Flosshildespeaks

nur den Ring, -
 

\Wellgundespeaks

der zum Tod ihm taugt, -
 

\speaker{Alle Drei}
den Reif nur will er sich wahren!
Leb' wohl, Siegfried!
Ein stolzes Weib
wird noch heute dich Argen beerben:
sie beut uns besseres Gehör:
Zu ihr! Zu ihr! Zu ihr!
 


\direct{Sie wenden sich schnell zum Reigen, mit welchem sie gemächlich dem Hintergrunde zu fortschwimmen}


\direct{Siegfried sieht ihnen lächelnd nach, stemmt ein Bein auf ein Felsstück am Ufer und verweilt mit auf der Hand gestütztem Kinne}


\speaker{Alle Drei}
Weialala leia, wallala leialala.
 

\Siegfriedspeaks

Im Wasser, wie am Lande
lernte nun ich Weiberart:
wer nicht ihrem Schmeicheln traut,
den schrecken sie mit Drohen;
wer dem kühnlich trotzt,
dem kommt dann ihr Keifen dran.

\direct{Die Rheintöchter sind hier gänzlich verschwunden}

Und doch, trüg' ich nicht Gutrun' Treu, -
der zieren Frauen eine
hätt' ich mir frisch gezähmt!

\direct{Er blickt ihnen unverwandt nach}


\speaker{Die Rheintöchter}

\direct{in größerer Entfernung}

La, la!

\direct{Jagdhornrufe kommen von der Höhe näher}




\speaker{\Hagen (Stimme)}

\direct{von fern}

Hoiho!
 


\direct{Siegfried fährt aus seiner träumerischen Entrücktheit auf und antwortet dem vernommenen Rufe auf seinem Horne}

\scene

\speaker{Die Mannen}

\direct{außerhalb der Szene}

Hoiho! Hoiho!
 

\Siegfriedspeaks


\direct{antwortend}

Hoiho! Hoiho! Hoihe!
 

\Hagenspeaks


\direct{kommt auf der Höhe hervor. Gunther folgt ihm. Siegfried erblickend}

Finden wir endlich,
wohin du flogest?
 

\Siegfriedspeaks

Kommt herab! Hier ist's frisch und kühl!
 


\direct{Die Mannen kommen alle auf der Höhe an und steigen nun mit Hagen und Gunther herab}


\Hagenspeaks

Hier rasten wir und rüsten das Mahl.

\direct{Jagdbeute wird zuhauf gelegt}

Laßt ruhn die Beute und bietet die Schläuche!

\direct{Trinkhörner und Schläuche werden hervorgeholt, dann lagert sich alles}

Der uns das Wild verscheuchte,
nun sollt ihr Wunder hören,
was Siegfried sich erjagt.
 

\Siegfriedspeaks


\direct{lachend}

Schlimm steht es um mein Mahl:
von eurer Beute bitte ich für mich.
 

\Hagenspeaks

Du beutelos?
 

\Siegfriedspeaks

Auf Waldjagd zog ich aus,
doch Wasserwild zeigte sich nur.
War ich dazu recht beraten,
drei wilde Wasservögel
hätt' ich euch wohl gefangen,
die dort auf dem Rheine mir sangen,
erschlagen würd' ich noch heut'.
 


\direct{Er lagert sich zwischen Gunther und Hagen}


\direct{Gunther erschrickt und blickt düster auf Hagen}


\Hagenspeaks

Das wäre üble Jagd,
wenn den Beutelosen selbst
ein lauernd Wild erlegte!
 

\Siegfriedspeaks

Mich dürstet!
 

\Hagenspeaks


\direct{indem er für Siegfried ein Trinkhorn füllen läßt und es diesem dann darreicht}

Ich hörte sagen, Siegfried,
der Vögel Sangessprache
verstündest du wohl:
so wäre das wahr?
 

\Siegfriedspeaks

Seit lange acht' ich des Lallens nicht mehr.
 


\direct{Er faßt das Trinkhorn und wendet sich damit zu Gunther. Er trinkt und reicht das Horn Gunther hin}

Trink', Gunther, trink'!
 

Dein Bruder bringt es dir!
 

\Guntherspeaks


\direct{gedankenvoll und schwermütig in das Horn blickend, dumpf}

Du mischtest matt und bleich:

\direct{noch gedämpfter}

dein Blut allein darin!
 

\Siegfriedspeaks


\direct{lachend}

So misch' ich's mit dem deinen!

\direct{Er gießt aus Gunthers Horn in das seine, so daß dieses überläuft}

Nun floß gemischt es über:
der Mutter Erde laß das ein Labsal sein!
 

\Guntherspeaks


\direct{mit einem heftigen Seufzer}

Du überfroher Held!
 

\Siegfriedspeaks


\direct{leise zu Hagen}

Ihm macht Brünnhilde Müh?
 

\Hagenspeaks


\direct{leise zu Siegfried}

Verstünd' er sie so gut,
wie du der Vögel Sang!
 

\Siegfriedspeaks

Seit Frauen ich singen hörte,
vergaß ich der Vöglein ganz.
 

\Hagenspeaks

Doch einst vernahmst du sie?
 

\Siegfriedspeaks


\direct{sich lebhaft zu Gunther wendend}

Hei! Gunther, grämlicher Mann!
Dankst du es mir,
so sing' ich dir Mären
aus meinen jungen Tagen.
 

\Guntherspeaks

Die hör' ich so gern.
 


\direct{Alle lagern sich nah an Siegfried, welcher allein aufrecht sitzt, während die andern tiefer gestreckt liegen}


\Hagenspeaks

So singe, Held!
 

\Siegfriedspeaks

Mime hieß ein mürrischer Zwerg:
in des Neides Zwang zog er mich auf,
daß einst das Kind, wann kühn es erwuchs,
einen Wurm ihm fällt' im Wald,
der faul dort hütet' einen Hort.
Er lehrte mich schmieden und Erze schmelzen;
doch was der Künstler selber nicht konnt',
des Lehrlings Mute mußt' es gelingen:
eines zerschlagnen Stahles Stücke
neu zu schmieden zum Schwert.
Des Vaters Wehr fügt' ich mir neu:
nagelfest schuf ich mir Notung.
Tüchtig zum Kampf dünkt' er dem Zwerg;
der führte mich nun zum Wald:
dort fällt' ich Fafner, den Wurm.
Jetzt aber merkt wohl auf die Mär':
Wunder muß ich euch melden.
Von des Wurmes Blut
mir brannten die Finger;
sie führt' ich kühlend zum Mund:
kaum netzt' ein wenig
die Zunge das Naß, -
was da die Vöglein sangen,
das konnt' ich flugs verstehn.
Auf den Ästen saß es und sang:
``Hei! Siegfried gehört nun
der Niblungen Hort!
Oh! Fänd' in der Höhle
den Hort er jetzt!
Wollt' er den Tarnhelm gewinnen,
der taugt' ihm zu wonniger Tat!
Doch möcht' er den Ring sich erraten,
der macht ihn zum Walter der Welt!''
 

\Hagenspeaks

Ring und Tarnhelm trugst du nun fort?
 

\speaker{Die Mannen}
Das Vöglein hörtest du wieder?
 

\Siegfriedspeaks

Ring und Tarnhelm hatt' ich gerafft:
da lauscht' ich wieder dem wonnigen Laller;
der saß im Wipfel und sang:
``Hei, Siegfried gehört nun der Helm und der Ring.
O traute er Mime, dem Treulosen, nicht!
Ihm sollt' er den Hort nur erheben;
nun lauert er listig am Weg:
nach dem Leben trachtet er Siegfried.
Oh, traute Siegfried nicht Mime!''
 

\Hagenspeaks

Es mahnte dich gut?
 

\speaker{Vier Mannen}
Vergaltest du Mime?
 

\Siegfriedspeaks

Mit tödlichem Tranke trat er zu mir;
bang und stotternd gestand er mir Böses:
Notung streckte den Strolch!
 

\Hagenspeaks


\direct{grell lachend}

Was er nicht geschmiedet,
schmeckte doch Mime!
 

\speaker{Zwei Mannen}

\direct{nacheinander}

Was wies das Vöglein dich wieder?
 

\Hagenspeaks


\direct{läßt ein Trinkhorn neu füllen und träufelt den Saft eines Krautes hinein}

Trink' erst, Held, aus meinem Horn:
ich würzte dir holden Trank,
die Erinnerung hell dir zu wecken,

\direct{er reicht Siegfried das Horn}

daß Fernes nicht dir entfalle!
 

\Siegfriedspeaks


\direct{blickt gedankenvoll in das Horn und trinkt dann langsam}

In Leid zu dem Wipfel lauscht' ich hinauf;
da saß es noch und sang:
``Hei, Siegfried erschlug nun den schlimmen Zwerg!
Jetzt wüßt' ich ihm noch das herrlichste Weib.
Auf hohem Felsen sie schläft,
Feuer umbrennt ihren Saal;
durchschritt' er die Brunst,
weckt' er die Braut -
Brünnhilde wäre dann sein!''
 

\Hagenspeaks

Und folgtest du des Vögleins Rate?
 

\Siegfriedspeaks

Rasch ohne Zögern zog ich nun aus,

\direct{Gunther hört mit wachsendem Erstaunen zu}

bis den feurigen Fels ich traf:
die Lohe durchschritt ich
und fand zum Lohn -

\direct{in immer größere Verzückung geratend}

schlafend ein wonniges Weib
in lichter Waffen Gewand.
Den Helm löst' ich der herrlichen Maid;
mein Kuß erweckte sie kühn:
oh, wie mich brünstig da umschlang
der schönen Brünnhilde Arm!
 

\Guntherspeaks


\direct{in höchstem Schrecken aufspringend}

Was hör' ich!
 


\direct{Zwei Raben fliegen aus einem Busche auf, kreisen über Siegfried und fliegen dann, dem Rheine zu, davon}


\Hagenspeaks

Errätst du auch dieser Raben Geraun'?
 


\direct{Siegfried fährt heftig auf und blickt, Hagen den Rücken zukehrend, den Raben nach}


\Hagenspeaks

Rache rieten sie mir!
 


\direct{Er stößt seinen Speer in Siegfrieds Rücken: Gunther fällt ihm---zu spät---in den Arm. Siegfried schwingt mit beiden Händen seinen Schild hoch empor, um Hagen damit zu zerschmettern: die Kraft verläßt ihn, der Schild entsinkt ihm rückwärts; er selbst stürzt krachend über dem Schilde zusammen}


\speaker{Vier Mannen}

\direct{welche vergebens Hagen zurückzuhalten versucht}

Hagen! Was tust du?
 

\speaker{Zwei Andere}
Was tatest du?
 

\Guntherspeaks

Hagen, was tatest du?
 

\Hagenspeaks


\direct{auf den zu Boden Gestreckten deutend}

Meineid rächt' ich!
 


\direct{Er wendet sich ruhig zur Seite ab und verliert sich dann einsam über die Höhe, wo man ihn langsam durch die bereits mit der Erscheinung der Raben eingebrochenen Dämmerung von dannen schreiten sieht. Gunther beugt sich schmerzergriffen zu Siegfrieds Seite nieder. Die Mannen umstehen teilnahmsvoll den Sterbenden}


\Siegfriedspeaks


\direct{von zwei Mannen sitzend erhalten, schlägt die Augen glanzvoll auf}

Brünnhilde! Heilige Braut!
Wach' auf! Öffne dein Auge!
Wer verschloß dich wieder in Schlaf?
Wer band dich in Schlummer so bang?
Der Wecker kam; er küßt dich wach,
und aber der Braut bricht er die Bande:
da lacht ihm Brünnhildes Lust!
Ach! Dieses Auge, ewig nun offen!
Ach, dieses Atems wonniges Wehen!
Süßes Vergehen, seliges Grauen:
Brünnhild' bietet mir---Gruß!
 


\StageDir{Er sinkt zurück und stirbt. Regungslose Trauer der Umstehenden. Die Nacht ist hereingebrochen. Auf die stumme Ermahnung Gunthers erheben die Mannen Siegfrieds Leiche und geleiten mit dem Folgenden sie in feierlichem Zuge über die Felsenhöhe langsam von dannen. Gunther folgt der Leiche zunächst.

Der Mond bricht durch die Wolken hervor und beleuchtet immer heller den die Berghöhe erreichenden Trauerzug. Dann steigen Nebel aus dem Rheine auf und erfüllen allmählich die ganze Bühne, auf welcher der Trauerzug bereits unsichtbar geworden ist, bis nach vorne, so daß diese während des Zwischenspiels gänzlich verhüllt bleibt. Als sich die Nebel wieder verteilen, tritt die Halle der Gibichungen, wie im ersten Aufzuge, immer erkennbarer hervor}


\scene

\StageDir{Es ist Nacht. Mondschein spiegelt sich auf dem Rheine. Gutrune tritt aus ihrem Gemache in die Halle hinaus}


\Gutrunespeaks

War das sein Horn?

\direct{Sie lauscht}

Nein!---Noch kehrt er nicht heim.---
Schlimme Träume störten mir den Schlaf!
Wild wieherte sein Roß;
Lachen Brünnhildes weckte mich auf.
Wer war das Weib,
das ich zum Ufer schreiten sah?
Ich fürchte Brünnhild'!
Ist sie daheim?

\direct{Sie lauscht an der Tür rechts und ruft dann leise}

Brünnhild'! Brünnhild'!
Bist du wach?

\direct{Sie öffnet schüchtern und blickt in das innere Gemach}

Leer das Gemach.
So war es sie,
die ich zum Rheine schreiten sah!

\direct{Sie erschrickt und lauscht nach der Ferne}

War das sein Horn?
Nein! Öd' alles!
Säh' ich Siegfried nur bald!
 


\direct{Sie will sich wieder ihrem Gemache zuwenden: als sie jedoch Hagens Stimme vernimmt, hält sie an und bleibt, von Furcht gefesselt, eine Zeitlang unbeweglich stehen}


\speaker{\Hagen (Stimme)}

\direct{von außen sich nähernd}

Hoiho! Hoiho!
Wacht auf! Wacht auf!
Lichte! Lichte! Helle Brände!
Jagdbeute bringen wir heim.
Hoiho! Hoiho!

\direct{Licht und wachsender Feuerschein von außen}


\Hagenspeaks


\direct{tritt in die Halle}

Auf, Gutrun'! Begrüße Siegfried!
Der starke Held, er kehret heim!
 

\Gutrunespeaks


\direct{im großer Angst}

Was geschah? Hagen!
Nicht hört' ich sein Horn!
 


\direct{Männer und Frauen, mit Lichtern und Feuerbränden, geleiten den Zug der mit Siegfrieds Leiche Heimkehrenden, unter denen Gunther}


\Hagenspeaks

Der bleiche Held,
nicht bläst er es mehr;
nicht stürmt er zur Jagd,
zum Streite nicht mehr,
noch wirbt er um wonnige Frauen.
 

\Gutrunespeaks


\direct{mit wachsendem Entsetzen}

Was bringen die?
 


\direct{Der Zug gelangt in die Mitte der Halle, und die Mannen setzen dort die Leiche auf einer schnell errichteten Erhöhung nieder}


\Hagenspeaks

Eines wilden Ebers Beute:
Siegfried, deinen toten Mann.
 


\direct{Gutrune schreit auf und stürzt über die Leiche hin. Allgemeine Erschütterung und Trauer}


\Guntherspeaks


\direct{bemüht sich um die Ohnmächtige}

Gutrun'! Holde Schwester,
hebe dein Auge, schweige mir nicht!
 

\Gutrunespeaks


\direct{wieder zu sich kommend}

Siegfried! Siegfried erschlagen!

\direct{Sie stößt Gunther heftig zurück}

Fort, treuloser Bruder,
du Mörder meines Mannes!
O Hilfe! Hilfe! Wehe! Wehe!
Sie haben Siegfried erschlagen!
 

\Guntherspeaks

Nicht klage wider mich!
Dort klage wider Hagen.
Er ist der verfluchte Eber,
der diesen Edlen zerfleischt'.
 

\Hagenspeaks

Bist du mir gram darum?
 

\Guntherspeaks

Angst und Unheil greife dich immer!
 

\Hagenspeaks


\direct{mit furchtbarem Trotze herantretend}

Ja denn! Ich hab' ihn erschlagen!
Ich, Hagen, schlug ihn zu Tod.
Meinem Speer war er gespart,
bei dem er Meineid sprach.
Heiliges Beuterecht
hab' ich mir nun errungen:
drum fordr' ich hier diesen Ring.
 

\Guntherspeaks

Zurück! Was mir verfiel,
sollst nimmer du empfahn.
 

\Hagenspeaks

Ihr Mannen, richtet mein Recht!
 

\Guntherspeaks

Rührst du an Gutrunes Erbe,
schamloser Albensohn?
 

\Hagenspeaks


\direct{sein Schwert ziehend}

Des Alben Erbe fordert so sein Sohn!
 


\direct{Er dringt auf Gunther ein, dieser wehrt sich; sie fechten. Die Mannen werfen sich dazwischen. Gunther fällt von einem Streiche Hagens darnieder}

Her den Ring!
 


\direct{Er greift nach Siegfrieds Hand; diese hebt sich drohend empor. Gutrune und die Frauen schreien entsetzt laut auf. Alles bleibt in Schauder regungslos gefesselt}


\direct{Vom Hintergrunde her schreitet Brünnhilde fest und feierlich dem Vordergrunde zu}


\Brunnhildespeaks


\direct{noch im Hintergrunde}

Schweigt eures Jammers
jauchzenden Schwall!
Das ihr alle verrietet,
zur Rache schreitet sein Weib.

\direct{Sie schreitet ruhig weiter vor}

Kinder hört' ich greinen nach der Mutter,
da süße Milch sie verschüttet:
doch nicht erklang mir würdige Klage,
des hehrsten Helden wert.
 

\Gutrunespeaks


\direct{vom Boden heftig sich aufrichtend}

Brünnhilde! Neiderboste!
Du brachtest uns diese Not:
die du die Männer ihm verhetztest,
weh, daß du dem Haus genaht!
 

\Brunnhildespeaks

Armselige, schweig'!
Sein Eheweib warst du nie,
als Buhlerin bandest du ihn.
Sein Mannesgemahl bin ich,
der ewige Eide er schwur,
eh' Siegfried je dich ersah.
 

\Gutrunespeaks


\direct{in jähe Verzweiflung ausbrechend}

Verfluchter Hagen!
Daß du das Gift mir rietest,
das ihr den Gatten entrückt!
Ach, Jammer!
Wie jäh nun weiß ich's,
Brünnhilde war die Traute,
die durch den Trank er vergaß! -
 


\direct{Sie wendet sich voll Scheu von Siegfried ab und beugt sich, im Schmerz aufgelöst, über Gunthers Leiche; so verbleibt sie regungslos bis zum Ende. Hagen steht, trotzig auf Speer und Schild gelehnt, in finsteres Sinnen versunken, auf der entgegengesetzen Seite}


\Brunnhildespeaks


\direct{allein in der Mitte; nachdem sie lange, zuerst mit tiefer Erschütterung, dann mit fast überwältigender Wehmut das Angesicht Siegfrieds betrachtet, wendet sie sich mit feierlicher Erhebung an die Männer und Frauen. Zu den Mannen}

Starke Scheite schichtet mir dort
am Rande des Rheins zuhauf!
Hoch und hell lodre die Glut,
die den edlen Leib
des hehrsten Helden verzehrt.
Sein Roß führet daher,
daß mit mir dem Recken es folge:
denn des Helden heiligste Ehre zu teilen,
verlangt mein eigener Leib.
Vollbringt Brünnhildes Wunsch!
 


\direct{Die jüngeren Männer errichten während des Folgenden vor der Halle nahe am Rheinufer einen mächtigen Scheiterhaufen, Frauen schmücken ihn mit Decken, auf die sie Kräuter und Blumen streuen}


\Brunnhildespeaks


\direct{versinkt von neuem in die Betrachtung des Antlitzes der Leiche Siegfrieds. Ihre Mienen nehmen eine immer sanftere Verklärung an}

Wie Sonne lauter strahlt mir sein Licht:
der Reinste war er, der mich verriet!
Die Gattin trügend,---treu dem Freunde,---
von der eignen Trauten---einzig ihm teuer---
schied er sich durch sein Schwert.
Echter als er schwur keiner Eide;
treuer als er hielt keiner Verträge;
lautrer als er liebte kein andrer:
und doch, alle Eide, alle Verträge,
die treueste Liebe trog keiner wie er! -
Wißt ihr, wie das ward?

\direct{nach oben blickend}

O ihr, der Eide ewige Hüter!
Lenkt euren Blick auf mein blühendes Leid:
erschaut eure ewige Schuld!
Meine Klage hör', du hehrster Gott!
Durch seine tapferste Tat,
dir so tauglich erwünscht,
weihtest du den, der sie gewirkt,
dem Fluche, dem du verfielest:
mich mußte der Reinste verraten,
daß wissend würde ein Weib!
Weiß ich nun, was dir frommt? -
Alles, alles, alles weiß ich,
alles ward mir nun frei!
Auch deine Raben hör' ich rauschen;
mit bang ersehnter Botschaft
send' ich die beiden nun heim.
Ruhe, ruhe, du Gott! -


\direct{Sie winkt den Mannen, Siegfrieds Leiche auf den Scheiterhaufen zu tragen; zugleich zieht sie von Siegfrieds Finger den Ring ab und betrachtet ihn sinnend}


Mein Erbe nun nehm' ich zu eigen.
Verfluchter Reif! Furchtbarer Ring!
Dein Gold fass' ich und geb' es nun fort.
Der Wassertiefe weise Schwestern,
des Rheines schwimmende Töchter,
euch dank' ich redlichen Rat.
Was ihr begehrt, ich geb' es euch:
aus meiner Asche nehmt es zu eigen!
Das Feuer, das mich verbrennt,
rein'ge vom Fluche den Ring!
Ihr in der Flut löset ihn auf,
und lauter bewahrt das lichte Gold,
das euch zum Unheil geraubt.
 


\direct{Sie hat sich den Ring angesteckt und wendet sich jetzt zu dem Scheiterhaufen, auf welchem Siegfrieds Leiche ausgestreckt liegt. Sie entreißt einem Manne den mächtigen Feuerbrand, schwingt diesen und deutet nach dem Hintergrunde}


Fliegt heim, ihr Raben!
Raunt es eurem Herren,
was hier am Rhein ihr gehört!
An Brünnhildes Felsen fahrt vorbei! -
Der dort noch lodert,
weiset Loge nach Walhall!
Denn der Götter Ende dämmert nun auf.
So---werf' ich den Brand
in Walhalls prangende Burg.
 


\direct{Sie schleudert den Brand in den Holzstoß, der sich schnell hell entzündet. Zwei Raben sind vom Felsen am Ufer aufgeflogen und verschwinden nach den Hintergrunde zu}


\direct{Brünnhilde gewahrt ihr Roß, welches zwei junge Männer hereinführen. Sie ist ihm entgegengesprungen, faßt es und entzäumt es schnell; dann neigt sie sich traulich zu ihm}


Grane, mein Roß!
Sei mir gegrüßt!
Weißt du auch, mein Freund,
wohin ich dich führe?
Im Feuer leuchtend, liegt dort dein Herr,
Siegfried, mein seliger Held.
Dem Freunde zu folgen, wieherst du freudig?
Lockt dich zu ihm die lachende Lohe?
Fühl' meine Brust auch, wie sie entbrennt;
helles Feuer das Herz mir erfaßt,
ihn zu umschlingen, umschlossen von ihm,
in mächtigster Minne vermählt ihm zu sein!
Heiajoho! Grane!
Grüß' deinen Herren!
Siegfried! Siegfried! Sieh!
Selig grüßt dich dein Weib!
 

\StageDir{Sie hat sich auf das Roß geschwungen und hebt es jetzt zum Sprunge. Sie sprengt es mit einem Satze in den brennenden Scheiterhaufen. Sogleich steigt prasselnd der Brand hoch auf, so daß das Feuer den ganzen Raum vor der Halle erfüllt und diese selbst schon zu ergreifen scheint. Entsetzt drängen sich Männer und Frauen nach dem äußersten Vordergrunde.}



\StageDir{Als der ganze Bühnenraum nur noch von Feuer erfüllt erscheint, verlischt plötzlich der Glutschein, so daß bald bloß ein Dampfgewölk zurückbleibt, welches sich dem Hintergrunde zu verzieht und dort am Horizont sich als finstere Wolkenschicht lagert. Zugleich ist vom Ufer her der Rhein mächtig angeschwollen und hat seine Flut über die Brandstätte gewälzt. Auf den Wogen sind die drei Rheintöchter herbeigeschwommen und erscheinen jetzt über der Brandstätte. Hagen, der seit dem Vorgange mit dem Ringe Brünnhildes Benehmen mit wachsender Angst beobachtet hat, gerät beim Anblick der Rheintöchter in höchsten Schreck. Er wirft hastig Speer, Schild und Helm von sich und stürzt wie wahnsinnig sich in die Flut.}



\Hagenspeaks


Zurück vom Ring!
 


\StageDir{Woglinde und Wellgunde umschlingen mit ihren Armen seinen Nacken und ziehen ihn so, zurückschwimmend, mit sich in die Tiefe. Flosshilde, den anderen voran dem Hintergrunde zuschwimmend, hält jubelnd den gewonnenen Ring in die Höhe. Durch die Wolkenschicht, welche sich am Horizont gelagert, bricht ein rötlicher Glutschein mit wachsender Helligkeit aus. Von dieser Helligkeit beleuchtet, sieht man die drei Rheintöchter auf den ruhigeren Wellen des allmählich wieder in sein Bett zurückgetretenen Rheines, lustig mit dem Ringe spielend, im Reigen schwimmen. Aus den Trümmern der zusammengestürzten Halle sehen die Männer und Frauen in höchster Ergriffenheit dem wachsenden Feuerschein am Himmel zu. Als dieser endlich in lichtester Helligkeit leuchtet, erblickt man darin den Saal Walhalls, in welchem die Götter und Helden, ganz nach der Schilderung Waltrautes im ersten Aufzuge, versammelt sitzen. Helle Flammen scheinen in dem Saal der Götter aufzuschlagen. Als die Götter von den Flammen gänzlich verhüllt sind, fällt der Vorhang}




\end{drama}
