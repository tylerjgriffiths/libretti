\Character{Sachs}{Sachs}
\Character{Walther}{Walther}
\Character{Beckmesser}{Beckmesser}
\Character{David}{David}
\Character{Pogner}{Pogner}
\Character{Vogelgesang}{Vogelgesang}
\Character{Magdalene}{Magdalene}
\Character{Eva}{Eva}
\Character{Kothner}{Kothner}
\Character{Nachtigall}{Nachtigall}
\Character{Ortel}{Ortel}
\Character{Moser}{Moser}
\Character{Zorn}{Zorn}
\Character{Eisslinger}{Eisslinger}
\Character{Foltz}{Foltz}
\Character{Schwarz}{Schwarz}



\begin{drama}

  \act

  \scene
  
  \StageDir{Die Bühne stellt das Innere der Katharinenkirche in schrägem Durchschnitt dar. Von dem Hauptschiff, welches links ab dem Hintergrunde zu sich ausdehnend anzunehmen ist, sind nur noch die letzten Reihen der Kirchenstuhlbänke sichtbar. Den Vordergrund nimmt der freie Raum vor dem Chor ein; dieser wird später durch einen schwarzen Vorhang gegen das Schiff zu gänzlich geschlossen. In der letzten Reihe der Kirchenstühle sitzen Eva und Magdalene; Walther von Stolzing steht, in einiger Entfernung, zur Seite an eine Säule gelehnt, die Blicke auf Eva heftend, die sich mit stummem Gebärdenspiel wiederholt zu ihm umkehrt.}

  
\speaker{Die Gemeinde}

Da zu dir der Heiland kam

\direct{Walther drückt durch Gebärde eine schmachtende Frage an Eva aus}

willig deine Taufe nahm,

\direct{Evas Blick und Gebärde sucht zu antworten; doch beschämt schlägt sie das Auge wieder nieder}

weihte sich dem Opfertod,

\direct{Walther zärtlich, dann dringender}

gab er uns des Heils Gebot: 

\direct{Eva, Walther schüchtern abweisend, aber schnell wieder seelenvoll zu ihm aufblickend}

dass wir durch ein' Tauf' uns weih'n, 

\direct{Walther entzückt, höchste Beteuerungen, Hoffnung.}

seines Opfers wert zu sein.

\direct{Eva lächelnd, dann beschämt die Augen senkend. Walther dringend, aber schnell sich unterbrechend}

Edler Täufer, Christ's Vorläufer! 

\direct{Walther nimmt die dringende Gebärde wieder auf, mildert sie aber sogleich, um sanft um eine Unterredung zu bitten}

Nimm uns freundlich an, dort am Fluss Jordan.

\StageDir{Die Gemeinde erhebt sich, wendet sich dem Ausgange zu und verlässt unter dem Nachspiel allmählich die Kirche. Walther heftet in höchster Spannung seinen Blick auf Eva, welche ihren Sitz ebenfalls verlässt und, von Magdalene gefolgt, langsam in seine Nähe kommt. Da Walther Eva sich nähern sieht, drängt er sich gewaltsam durch die Kirchgänger zu ihr.}
  

\Waltherspeaks
\direct{leise, doch feurig zu Eva}

Verweilt! Ein Wort! Ein einzig Wort!

\Evaspeaks
\direct{sich schnell zu Magdalena umwendend}

Mein Brusttuch! Schau! Wohl liegt's im Ort?

\Magdalenespeaks

Vergesslich' Kind! Nun heisst es: such!

Sie kehrt nach den Kirchenstühlen zurück

\Waltherspeaks

Fräulein! Verzeiht der Sitte Bruch!
Eines zu wissen, eines zu fragen,
was müsst' ich nicht zu brechen wagen?
Ob Leben oder Tod, ob Segen oder Fluch?
Mit einem Worte sei mir's vertraut:
mein Fräulein sagt---

\Magdalenespeaks

\direct{zurückkommend}

Hier ist das Tuch.

\Evaspeaks
O weh! Die Spange!

\Magdalenespeaks

Fiel sie wohl ab?

\direct{Sie geht suchend abermals nach hinten}

\Waltherspeaks
Ob Licht und Lust oder Nacht und Tod?
Ob ich erfahr, wonach ich verlange,
ob ich vernehme, wovor mir graut:
Mein Fräulein, sagt---

\Magdalenespeaks
\direct{wieder zurückkommend}

Da ist auch die Spange.
Komm, Kind! Nun hast du Spang' und Tuch \ldots
O weh! Da vergass ich selbst mein Buch!

\direct{Sie geht nochmals eilig nach hinten}

\Waltherspeaks
Dies eine Wort, Ihr sagt mir's nicht?
Die Silbe, die mein Urteil spricht?
Ja oder nein! Ein flücht'ger Laut:
mein Fräulein sagt,

\direct{entschlossen und hastig}

seid Ihr schon Braut?

\Magdalenespeaks

\direct{die wieder zurückgekehrt ist und sich vor Walther verneigt}

Sieh da, Herr Ritter,
wie sind wir hochgeehrt:
mit Evchens Schutze habt Ihr Euch gar beschwert?
Darf den Besuch des Helden
ich Meister Pogner melden?

\Waltherspeaks
\direct{bitter, leidenschaftlich}

Oh, betrat ich doch nie sein Haus!

\Magdalenespeaks
Ei, Junker! Was sagt Ihr da aus?
In Nürnberg eben nur angekommen,
wart Ihr nicht freundlich aufgenommen?
Was Küch' und Keller, Schrein und Schrank
Euch bot, verdient' es keinen Dank?

\Evaspeaks

Gut Lenchen, ach, das meint er ja nicht.
Doch von mir wohl wünscht er Bericht.
Wie sag ich's schnell? Versteh' ich's doch kaum!
Mir ist, als wär' ich gar wie im Traum!
Er frägt---ob ich schon Braut?

\Magdalenespeaks
\direct{heftig erschrocken}

Hilf Gott! Sprich nicht so laut!
Jetzt lass uns nach Hause gehn;
wenn uns die Leut' hier sehn!

\Waltherspeaks

Nicht eh'r, bis ich alles weiss!

\Evaspeaks
\direct{zu Magdalene}

's ist leer, die Leut' sind fort.

\Magdalenespeaks
Drum eben wird mir heiss!
Herr Ritter, an andrem Ort!

\direct{David tritt aus der Sakristei ein und macht sich darüber her, die, schwarzen Vorhänge zu schliessen}

\Waltherspeaks

\direct{dringend}

Nein! Erst dies Wort!

\Evaspeaks

\direct{bittend zu Magdalene}

Dies Wort!

\Magdalenespeaks
\direct{die sich bereits umgewendet, erblickt David, hält an und ruft zärtlich für sich}

David? Ei! David hier?

\direct{Sie wendet sich wieder zurück, und zu Walther.}


\Evaspeaks
\direct{zu Magdalene}
Was sag ich? Sag du's mir!


\Magdalenespeaks
\direct{zerstreut, öfter nach David sich umsehend}

Herr Ritter, was Ihr die Jungfer fragt,
das ist so leichtlich nicht gesagt;
fürwahr ist Evchen Pogner Braut

\Evaspeaks
\direct{lebhaft unterbrechend}

Doch hat noch keiner den Bräut'gam erschaut.

\Magdalenespeaks
Den Bräut'gam wohl noch niemand kennt,
bis morgen ihn das Gericht ernennt,
das dem Meistersinger erteilt den Preis---

\Evaspeaks
\direct{enthusiastisch}

Und selbst die Braut ihm reicht das Reis.

\Waltherspeaks
\direct{verwundert}

Dem Meistersinger?

\Evaspeaks
\direct{bang}

Seid Ihr das nicht?

\Waltherspeaks
Ein Werbgesang?

\Magdalenespeaks
Vor Wettgericht.

\Waltherspeaks
Den Preis gewinnt?

\Magdalenespeaks
Wen die Meister meinen.

\Waltherspeaks
Die Braut dann wählt?

\Evaspeaks

\direct{sich vergessend}

Euch oder keinen!

\direct{Walther wendet sich, in grosser Erregung auf und ab gehend, zur Seite}

\Magdalenespeaks
\direct{sehr erschrocken}

Was, Evchen! Evchen! Bist du von Sinnen?

\Evaspeaks
Gut' Lene, lass mich den Ritter gewinnen!

\Magdalenespeaks
Sahst ihn doch gestern zum erstenmal?

\Evaspeaks
Das eben schuf mir so schnelle Qual,
dass ich schon längst ihn im Bilde sah!
Sag, trat er nicht ganz wie David nah?

\Magdalenespeaks
\direct{höchst verwundert}

Bist du toll? Wie David?

\Evaspeaks
Wie David im Bild.

\Magdalenespeaks
Ach, meinst du den König mit der Harfen
und langem Bart in der Meister Schild?

\Evaspeaks
Nein! Der, dess' Kiesel den Goliath warfen,
das Schwert im Gurt, die Schleuder zur Hand,
das Haupt von lichten Locken umstrahlt,
wie ihn uns Meister Dürer gemalt.

\Magdalenespeaks
\direct{laut seufzend}

Ach, David! David!

\Davidspeaks
\direct{der hinausgegangen und jetzt wieder zurückkommt, ein Lineal im Gürtel und ein grosses Stück weisser Kreide an einer Schnur schwenkend}

Da bin ich! Wer ruft?

\Magdalenespeaks

Ach, David! Was Ihr für Unglück schuft!

\direct{für sich}

Der liebe Schelm! Wüsst' er's noch nicht?

\direct{laut}

Ei seht, da hat er uns gar verschlossen?

\Davidspeaks

\direct{zärtlich}

Ins Herz Euch allein!

\Magdalenespeaks
\direct{feurig}

Das treue Gesicht! Ei sagt!
Was treibt Ihr hier für Possen?

\Davidspeaks
Behüt es, Possen? Gar ernste Ding'!
Für die Meister hier richt' ich den Ring.

\Magdalenespeaks
Wie? Gäb' es ein Singen?

\Davidspeaks

Nur Freiung heut:
der Lehrling wird da losgesprochen,
der nichts wider die Tabulatur verbrochen;
Meister wird, wen die Prob' nicht reut.

\Magdalenespeaks
Da wär' der Ritter ja am rechten Ort.
Jetzt, Evchen, komm, wir müssen fort.

\Waltherspeaks
\direct{schnell sich zu den Frauen wendend}

Zu Meister Pogner lasst mich euch geleiten.

\Magdalenespeaks
Erwartet den hier; er ist bald da.
Wollt Ihr Evchens Hand erstreiten,
rückt Zeit und Ort das Glück Euch nah.
Zwei Lehrbuben kommen dazu und tragen Bänke herbei
Jetzt eilig von hinnen!

\Waltherspeaks
Was soll ich beginnen?

\Magdalenespeaks
Lasst David Euch lehren, die Freiung begehren.
Davidchen, hör, mein lieber Gesell,
den Ritter hier bewahr' mir wohl zur Stell'!
Was Fein's aus der Küch' bewahr' ich für dich;
und morgen begehr' du noch dreister,
wird hier der Junker heut' Meister.

\direct{Sie drängt Eva zum Fortgehen}


\Evaspeaks
\direct{zu Walther}

Seh' ich Euch wieder?

\Waltherspeaks
\direct{sehr feurig}

Heut abend, gewiss!
Was ich will wagen, wie könnt' ich's sagen?
Neu ist mein Herz, neu mein Sinn,
neu ist mir alles, was ich beginn'.
Eines nur weiss ich, eines begreif' ich:
Mit allen Sinnen Euch zu gewinnen!
Ist's mit dem Schwert nicht, muss es gelingen,
gilt es als Meister Euch zu ersingen.
Für Euch Gut und Blut!
Für Euch Dichters heil'ger Mut!

\Evaspeaks

\direct{mit grosser Wärme}

Mein Herz, sel'ger Glut,
für Euch liebesheil'ge Hut!

\Magdalenespeaks
Schnell heim, sonst geht's nicht gut!

\Davidspeaks
\direct{der Walther verwunderungsvoll gemessen}

Gleich Meister? Oho! Viel Mut!

\direct{Magdalene zieht Eva eilig durch die Vorhänge nach sich fort. Walther wirft sich, aufgeregt und brütend, in einen erhöhten kathederartigen Lehnstuhl, den zuvor zwei Lehrbuben von der Wand ab mehr nach der Mitte zu gerückt haben.}



\scene

\StageDir{Noch mehrere Lehrbuben sind eingetreten; sie tragen und stellen Bänke und richten alles zur Sitzung der Meistersinger her.}

\speaker{Zweiter Lehrbube}
David, was stehst?

\speaker{Erster Lehrbube}
Greif ans Werk!

\speaker{Zweiter Lehrbube}
Hilf uns richten das Gemerk!

\Davidspeaks
Zu eifrigst war ich vor euch allen;
schafft nun für euch:
hab ander Gefallen!

\speaker{Vier Lehrbuben}
Was der sich dünkt!

\speaker{Vier Lehrbuben}
Der Lehrling' Muster!

\speaker{Vier Lehrbuben}
Das macht, weil sein Meister ein Schuster.

\speaker{Vier Lehrbuben}
Beim Leisten sitzt er mit der Feder.

\speaker{Vier Lehrbuben}
Beim Dichten mit Draht und Pfriem.

\speaker{Vier Lehrbuben}
Sein' Verse schreibt er auf rohes Leder.

\speaker{Alle zwölf Lehrbuben}
\direct{mit entsprechender Gebärde}

Das---dächt' ich---gerbten wir ihm!

\direct{Sie machen sich lachend an die fernere Herrichtung}

\Davidspeaks
\direct{nachdem er den sinnenden Ritter eine Weile betrachtet}
Fanget an!

\Waltherspeaks
\direct{verwundert}
Was soll's?

\Davidspeaks
\direct{noch stärker}

Fanget an! So ruft der ``Merker''
Nun sollt Ihr singen! Wisst Ihr das nicht?

\Waltherspeaks
Wer ist der Merker?

\Davidspeaks
Wisst Ihr das nicht? Wart Ihr noch nie bei 'nem Sing-Gericht?

\Waltherspeaks
Noch nie, wo die Richter Handwerker!

\Davidspeaks
Seid Ihr ein ``Dichter''?

\Waltherspeaks
Wär' ich's doch!

\Davidspeaks
Seid Ihr ein ``Singer''?

\Waltherspeaks
Wüsst' ich's noch!

\Davidspeaks
Doch ``Schulfreund'' wart Ihr und ``Schüler'' zuvor?

\Waltherspeaks
Das klingt mir alles fremd vorm Ohr.

\Davidspeaks
Und so gradhin wollt Ihr Meister werden?

\Waltherspeaks
Wie, machte das so grosse Beschwerden?

\Davidspeaks
O Lene! Lene!

\Waltherspeaks
Wie Ihr doch tut!

\Davidspeaks
O Magdalene!

\Waltherspeaks
Ratet mir gut!

\Davidspeaks
\direct{setzt sich in Positur}

Mein Herr, der Singer Meister-Schlag
gewinnt sich nicht an einem Tag.
In Nüremberg der grösste Meister
mich lehrt die Kunst Hans Sachs!
Schon voll ein Jahr mich unterweist er,
dass ich als Schüler wachs'.
Schuhmacherei und Poeterei,
die lern' ich da alleinerlei:
hab ich das Leder glatt geschlagen,
lern' ich Vokal und Konsonanz sagen;
wichst' ich den Draht erst fest und steif,
was sich dann reimt, ich wohl begreif!
Den Pfriemen schwingend,
im Stich die Ahl',
was stumpf, was klingend,
was Mass, was Zahl---
den Leisten im Schurz, was lang, was kurz,
was hart, was lind, hell oder blind,
was Waisen, was Milben, was Klebsilben,
was Pausen, was Körner, was Blumen, was Dörner---
das alles lernt' ich mit Sorg' und Acht.
Wie weit nun, meint Ihr, dass ich's gebracht?

\Waltherspeaks
Wohl zu ‘nem Paar recht guter Schuh'?

\Davidspeaks
Ja, dahin hat's noch gute Ruh'!
Ein ``Bar'' hat manch Gesätz' und Gebänd';
wer da gleich die rechte Regel fänd',
die richt'ge Naht und den rechten Draht,
mit gutgefügten ``Stollen'' den Bar recht zu versohlen.
Und dann erst kommt der ``Abgesang'';
dass der nicht kurz und nicht zu lang
und auch keinen Reim enthält,
der schon im Stollen gestellt.
Wer alles das merkt, weiss und kennt,
wird doch immer noch nicht ``Meister'' genennt.

\Waltherspeaks
Hilf Gott! Will ich denn Schuster sein?
In die Singkunst lieber führ mich ein.

\Davidspeaks
Ja, hätt' ich's nur selbst schon zum ``Singer'' gebracht!
Wer glaubt wohl, was das für Mühe macht?
Der Meister Tön' und Weisen,
gar viel an Nam' und Zahl,
die starken und die leisen,
wer die wüsste allzumal!
Der ``kurze'', ``lang'\,'' und ``überlang'\,'' Ton,
die ``Schreibpapier''-, ``Schwarz-Tinten''-Weis';
der ``rote'', ``blau''' und ``grüne'' Ton;
die ``Hageblüh''-, ``Strohhalm''-, ``Fengel''-Weis';
der ``zarte'', der ``süsse'', der ``Rosen''-Ton;
der ``kurzen Liebe'', der ``vergessne'' Ton;
die ``Rosmarin''-, ``Gelbveiglein''-Weis',
die ``Regenbogen''-, die ``Nachtigall'' -Weis',
die ``englische Zinn''-, die ``Zimmtröhren''-Weis',
``frisch' Pomeranzen''-, ``grün' Lindenblüh''-Weis',
die ``Frösch'''-, die ``Kälber''-, die ``Stieglitz''-Weis',
die ``abgeschiedene Vielfrass''-Weis';
der ``Lerchen''-, der ``Schnecken''-, der ``Beller''-Ton,
die ``Melissenblümlein''-, die ``Meiran''-Weis',
``Gelblöwenhaut''-,
\direct{gefühlvoll}
``treu' Pelikan''-Weis',
\direct{prunkend}
die ``buttglänzende Draht''-Weis' \ldots 

\Waltherspeaks
Hilf Himmel! Welch endlos Tönegeleis'!

\Davidspeaks
Das sind nur die Namen:
nun lernt sie singen,
recht, wie die Meister sie gestellt!
Jed' Wort und Ton muss klärlich klingen, 
wo steigt die Stimm' und wo sie fällt;
fangt nicht zu hoch, zu tief nicht an,
als es die Stimm' erreichen kann;
mit dem Atem spart, dass er nicht knappt
und gar am End' Ihr überschnappt;
vor dem Wort mit der Stimme ja nicht summt,
nach dem Wort mit dem Mund auch nicht brummt.
Nicht ändert an ``Blum'\,'' und ``Koloratur'',
jed' Zierat fest nach des Meisters Spur.
Verwechseltet Ihr, würdet gar irr',
verlört Ihr Euch und kämt ins Gewirr:
wär' sonst Euch alles auch gelungen,
da hättet Ihr gar ``versungen!''
Trotz grossem Fleiss und Emsigkeit
ich selbst noch bracht' es nicht so weit.
So oft ich's versuch' und ‘s nicht gelingt,
die ``Knieriem-Schlag''-Weis' der Meister mir singt.

\direct{sanft}

Wenn dann Jungfer Lene nicht Hilfe weiss,

\direct{greinend}

sing' ich die ``eitel Brot- und Wasser''-Weis'!
Nehmt Euch ein Beispiel dran
und lasst vom Meister-Wahn!
Denn ``Singer'' und ``Dichter'' müsst Ihr sein,
eh' Ihr zum ``Meister'' kehret ein.

\speaker{Vier Lehrbuben}

\direct{während der Arbeit}

David!

\Waltherspeaks
Wer ist nun Dichter?

\speaker{Vier Lehrbuben}

David! Kommst her?

\Davidspeaks

\direct{zu den Lehrbuben}

Wartet nur, gleich!---
\direct{schnell wieder zu Walther sich wendend}
Wer der ``Dichter'' wär'?
Habt Ihr zum ``Singer'' Euch aufgeschwungen
und der Meister Töne richtig gesungen,
fügtet Ihr selbst nun Reim' und Wort',
dass sie genau an Stell' und Ort
passten zu eines Meisters Ton,
dann trügt Ihr den Dichterpreis davon.

\speaker{Vier Lehrbuben}

He, David! Soll man's dem Meister klagen?

\speaker{Alle Lehrbuben}

Wirst dich bald des/deines Schwatzens entschlagen?

\Davidspeaks

Oho! Jawohl! Denn helf' ich euch nicht,
ohne mich wird alles doch falsch gericht't.
\direct{Er will sich z ihnen wenden}

\Waltherspeaks

\direct{ihn zurückhaltend}

Nur dies noch:
wer wird ``Meister'' genannt?

\Davidspeaks

\direct{schnell wieder umkehrend}

Damit, Herr Ritter, ist's so bewandt:
\direct{mit sehr tiefsinniger Miene}

Der Dichter, der aus eig'nem Fleisse
zu Wort' und Reimen, die er erfand,

\direct{äusserst zart}

aus Tönen auch fügt eine neue Weise,
der wird als ``Meistersinger'' erkannt.

\Waltherspeaks
So bleibt mir einzig der Meisterlohn!
Muss ich singen,
kann's nur gelingen,
find' ich zum Vers auch den eig'nen Ton.

\Davidspeaks
\direct{der sich zu den Lehrbuben gewendet}

Was macht ihr denn da? Ja, fehl' ich beim Werk,
verkehrt nur richtet ihr Stuhl und Gemerk! 

\direct{Er wirft polternd und lärmend die Anordnungen der Lehrbuben in betreff des Gemerkes um}

Ist denn heut ``Singschul'\,''? Dass ihr's wisst,
das kleine Gemerk! Nur ``Freiung'' ist!


\direct{Die Lehrbuben, welche in der Mitte der Bühne ein grösseres Gerüst mit Vorhängen aufgeschlagen hatten, schaffen auf Davids Weisung dies schnell beiseite und stellen dafür ein geringeres Brettergerüst auf; daraufstellen sie einen Stuhl mit einem kleinen Pult davor, daneben eine grosse schwarze Tafel, daran die Kreide am Faden aufgehängt wird; um das Gerüst sind schwarze Vorhänge angebracht, die zunächst hinten und an beiden Seiten, dann auch vorn ganz zusammengezogen werden.}


\speaker{Alle Lehrbuben}

\direct{während der Herrichtung}

Aller End' ist doch David der Allergescheit'st,
nach hohen Ehren ganz sicher er geizt:
‘s ist Freiung heut;
gewiss er freit,
als vornehmer ``Singer'' er schon sich spreizt!
Die ``Schlag''-Reime fest er inne hat,
``Arm-Hunger''-Weise singt er glatt.

\speaker{Vier Lehrbuben}

Doch die ``harte-Tritt''-Weis', die kennt er am best'

\speaker{Alle}
Die trat ihm der Meister hart und fest!
\direct{Mit der Gebärde zweier Fusstrtte. Sie lachen}

\Davidspeaks

Ja, lacht nur zu! Heut bin ich's nicht;
ein andrer stellt sich zum Gericht:
der war nicht Schüler, ist nicht Singer,
den Dichter, sagt er, überspring' er;
denn er ist Junker, und mit einem Sprung er
denkt ohne weit're Beschwerden
heut' hier Meister zu werden.
Drum richtet nur fein das Gemerk dem ein!
Während die Lehrbuben vollends aufrichten.
Dorthin! Hierher! Die Tafel all die Wand,
so dass sie recht dem Merker zur Hand!

\direct{sich zu Walther um wendend}

Ja, ja, dem ``Merker''! Wird Euch wohl bang?
Vor ihm schon mancher Werber versang.
Sieben Fehler gibt er Euch vor,
die merkt er mit Kreide dort an;
wer über sieben Fehler verlor,
hat versungen und ganz vertan!
Nun nehmt Euch in acht!
Der Merker wacht.

\direct{Derb in die Hände schlagend}

Glück auf zum Meistersingen!
Mögt Euch das Kränzlein erschwingen!
Das Blumenkränzlein aus Seiden fein
wird das dem Herrn Ritter beschieden sein?

\speaker{Die Lehrbuben}

\direct{elche zu gleicher Zeit das Gemerk geschlossen haben, fassen sich an und tanzen einen verschlungenen Reigen um dasselbe}

Das Blumenkränzlein aus Seiden fein,
wird das dem Herrn Ritter beschieden sein?

\StageDir{Die Lehrbuben fahren sogleich erschrocken auseinander, als die Sakristei aufgeht und Pogner mit Beckmesser eintritt;sie ziehen sich nach hinten zurück.}

\scene

\StageDir{Die Einrichtung ist nun folgendermassen beendigt: Zur Seite rechts sind gepolsterte Bänke in der Weise ausgestellt, dass sie einen schwachen Halbkreis nach der Mitte zu bilden. Am Ende der Bänke, in der Mitte der Bühne, befindet sich das ``Gemerk'' benannte Gerüst, welches zuvor hergerichtet worden. Zur linken Seite steht nun der erhöhte, kathederartige Stuhl (``der Singstuhl'') der Versammlung gegenüber. Im Hintergrunde, den grossen Vorhang entlang, steht eine lange niedere Bank für die Lehrlinge. Walther, verdriesslich über das Gespött der Knaben, hat sich auf die vordere Bank niedergelassen. Pogner und Beckmesser sind im Gespräch aus der Sakristei aufgetreten. Die Lehrbuben harren, ehrerbietig vor der hinteren Bank stehend. Nur David stellt sich anfänglich am Eingang der Sakristei auf.}

\Pognerspeaks

\direct{zu Beckmesser}

Seid meiner Treue wohl versehen.
Was ich bestimmt, ist Euch zu Nutz:
im Wettgesang müsst Ihr bestehen;
wer böte Euch als Meister Trutz?

\Beckmesserspeaks

Doch wollt Ihr von dem Punkt nicht weichen,
der mich---ich sag's---bedenklich macht;
kann Evchens Wunsch den Werber streichen,
was nützt mir meine Meisterpracht?

\Pognerspeaks

Ei sagt! Ich mein, vor allen Dingen
sollt' Euch an dem gelegen sein.
Könnt Ihr der Tochter Wunsch nicht zwingen,
wie möchtet Ihr wohl um sie frei'n?

\Beckmesserspeaks
Ei ja! Gar wohl! Drum eben bitt' ich,
dass bei dem Kind Ihr für mich sprecht,
wie ich geworben zart und sittig
und wie Beckmesser grad Euch recht.

\Pognerspeaks
Das tu ich gern.

\Beckmesserspeaks
\direct{beiseite}

Er lässt nicht nach!
Wie wehrt' ich da 'nem Ungemach?

\Waltherspeaks

\direct{der, als er Pogner gewahrt, aufgestanden und ihm entgegengegangen ist, verneigt sich vor ihm}

Gestattet, Meister!

\Pognerspeaks

Wie, mein Junker?
Ihr sucht mich in der Singschul' hie?
Sie wechseln die Begrüssungen

\Beckmesserspeaks
\direct{immer beiseite}

Verstünden's die Frau'n! Doch schlechtes Geflunker
gilt ihnen mehr als all' Poesie.

\direct{Er geht verdriesslich im Hintergrunde auf und ab}


\Waltherspeaks

Hier eben bin ich am rechten Ort.
Gesteh' ich's frei, vom Lande fort
was mich nach Nürnberg trieb,
war nur zur Kunst die Lieb'.
Vergass ich's gestern Euch zu sagen,
heut muss ich's laut zu künden wagen:
ein Meistersinger möcht' ich sein.

\direct{Sehr innig}

Schliesst, Meister, in die Zunft mich ein!

\direct{Kunz Vogelgesang und Konrad Nachtigall sind eingetreten}


\Pognerspeaks
\direct{freudig zu den Hinzutretenden}

Kunz Vogelgesang! Freund Nachtigall!
Hört doch, welch' ganz besondrer Fall!
Der Ritter hier, mir wohlbekannt,
hat der Meisterkunst sich zugewandt.

\direct{Vorstellungen, Begrüssungen, andere Meister treten noch dazu}

\Beckmesserspeaks

\direct{wieder in den Vordergrund tretend, für sich}

Noch such' ich's zu wenden;
doch sollt's nicht gelingen,
versuch' ich des Mädchens Herz zu ersingen.
In stiller Nacht, von ihr nur gehört,
erfahr' ich, ob auf mein Lied sie schwört.

\direct{Walther erblickend}

Wer ist der Mensch?

\Pognerspeaks

\direct{sehr warm zu Walther fortfahrend}

Glaubt, wie mich's freut!
Die alte Zeit dünkt mich erneut.

\Beckmesserspeaks
Er gefällt mir nicht!

\Pognerspeaks
Was Ihr begehrt,

\Beckmesserspeaks
Was will er hier?

\Pognerspeaks
\ldots soviel an mir\ldots

\Beckmesserspeaks
Wie der Blick ihm lacht!

\Pognerspeaks
\ldots sei's Euch gewährt.
Half ich Euch gern bei des Guts Verkauf,

\Beckmesserspeaks
Holla, Sixtus!

\Pognerspeaks
in die Zunft nun nehm' ich Euch gleich gern auf.

\Beckmesserspeaks
Auf den hab acht!

\Waltherspeaks
Habt Dank der Güte aus tiefstem Gemüte!
Und darf ich denn hoffen, steht heut mir noch offen,
zu werben um den Preis, dass Meistersinger ich heiss'?

\Beckmesserspeaks
Oho! Fein sacht! Auf dem Kopf steht kein Kegel!

\Pognerspeaks
Herr Ritter, dies geh' nun nach der Regel.
Doch heut ist Freiung:
ich schlag' Euch vor;
mir leihen die Meister ein willig Ohr.

\direct{Die Meistersinger sind nun alle angelangt, zuletzt Hans Sachs}


\Sachsspeaks
Gott grüss Euch, Meister!

\Vogelgesangspeaks
Sind wir beisammen?

\Beckmesserspeaks
Der Sachs ist ja da!

\Nachtigallspeaks
So ruft die Namen!

\Kothnerspeaks

\direct{zieht eine Liste hervor, stellt sich zur Seite auf und ruft laut}

Zu einer Freiung und Zunftberatung
ging an die Meister ein' Einladung:
bei Nenn' und Nam', ob jeder kam,
ruf' ich nun auf als letztentbot'ner,
der ich mich nenn' und bin Fritz Kothner.
Seid Ihr da, Veit Pogner?

\Pognerspeaks
Hier zur Hand.

\direct{Er setzt sich}


\Kothnerspeaks
Kunz Vogelgesang?

\Vogelgesangspeaks
Ein sich fand.

\direct{Er setzt sich}


\Kothnerspeaks
Hermann Ortel?

\Ortelspeaks
Immer am Ort.

\direct{Er setzt sich}


\Kothnerspeaks
Balthasar Zorn?

\Zornspeaks
Bleibt niemals fort.

\direct{Er setzt sich}


\Kothnerspeaks
Konrad Nachtigall?

\Nachtigallspeaks
Treu seinem Schlag.

\direct{Er setzt sich}


\Kothnerspeaks
Augustin Moser?

\Moserspeaks
Nie fehlen mag.

\direct{Er setzt sich}


\Kothnerspeaks
Niklaus Vogel? Schweigt?

\speaker{Ein Lehrbube}

\direct{von der Bank aufstehend}

Ist krank.

\Kothnerspeaks
Gut' Bess'rung dem Meister!

\speaker{Die Meister}
\direct{ausser Kothner}

Walt's Gott!

\speaker{Der Lehrbube}
Schön' Dank!

\direct{Er setzt sich wieder nieder}


\Kothnerspeaks
Hans Sachs?

\Davidspeaks
\direct{vorlaut sich erhebend und auf Sachs zeigend}

Da steht er!

\Sachsspeaks
\direct{drohend zu David}

Juckt dich das Fell?
Verzeiht, Meister! Sachs ist zur Stell'.

\direct{Er setzt sich}


\Kothnerspeaks
Sixtus Beckmesser?

\Beckmesserspeaks
Immer bei Sachs

\direct{während er sich setzt}

dass den Reim ich lern' von ``blüh' und wachs''.


\direct{Sachs lacht}


\Kothnerspeaks
Ulrich Eisslinger?

\Eisslingerspeaks
Hier.

\direct{Er setzt sich}


\Kothnerspeaks
Hans Foltz?

\Foltzspeaks
Bin da.

\direct{Er setzt sich}


\Kothnerspeaks
Hans Schwarz?

\Schwarzspeaks
Zuletzt:
Gott wollt's!

\direct{Setzt sich}

\Kothnerspeaks
Zur Sitzung gut und voll die Zahl.
Beliebt's, wir schreiten zur Merkerwahl?

\Vogelgesangspeaks
Wohl eh'r nach dem Fest.

\Beckmesserspeaks
Pressiert's dem Herrn?
Mein Stell' und Amt lass ich ihm gern.

\Pognerspeaks
Nicht doch, Ihr Meister! Lasst das jetzt fort.
Für wichtigen Antrag bitt ich ums Wort.

\direct{Alle Meister stehen auf, nicken Kothner zu und setzen sich wieder}

\Kothnerspeaks
Das habt Ihr, Meister, sprecht!

\Pognerspeaks
Nun hört und versteht mich recht!
Das schöne Fest, Johannistag,
Ihr wisst, begeh'n wir morgen.
Auf grüner Au', am Blumenhang,
bei Spiel und Tanz im Lustgelag,
an froher Brust geborgen,
vergessen seiner Sorgen,
ein jeder freut sich, wie er mag.
Die Singschul' ernst im Kirchenchor
die Meister selbst vertauschen;
mit Kling und Klang hinaus zum Tor
auf offne Wiese ziehn sie vor
bei hellen Festes Rauschen;
das Volk sie lassen lauschen
dem Freigesang mit Laienohr.
Zu einem Werb- und Wettgesang
gestellt sind Siegespreise,
und beide preist man weit und lang,
die Gabe wie die Weise.
Nun schuf mich Gott zum reichen Mann;
und gibt ein jeder, wie er kann,
so musste ich wohl sinnen,
was ich gäb' zu gewinnen,
dass ich nicht käm' zu Schand':
so hört denn, was ich fand.
In deutschen Landen viel gereist,
hat oft es mich verdrossen,
dass man den Bürger wenig preist,
ihn karg nennt und verschlossen.
An Höfen wie an nied'rer Statt
des bitt'ren Tadels ward ich satt,
dass nur auf Schacher und Geld
sein Merk' der Bürger stellt.
Dass wir im weiten deutschen Reich
die Kunst einzig noch pflegen,
dran dünkt ihnen wenig gelegen.
Doch wie uns das zur Ehre gereich',
und dass mit hohem Mut
wir schätzen, was schön und gut,
was wert die Kunst und was sie gilt,
das ward ich der Welt zu zeigen gewillt.
Drum hört, Meister, die Gab',
die als Preis bestimmt ich hab.
Dem Sieger, der im Kunstgesang
vor allem Volk den Preis errang
am Sankt-Johannis-Tag,
sei er, wer er auch mag,
dem geh' ich, ein Kunstgewogner,
von Nürnberg Veit Pogner,
mit all meinem Gut, wie's geh' und steh',
Eva, mein einzig Kind, zur Eh'.

\speaker{Die Meister}

\direct{sich erhebend und sehr lebhaft durcheinander}

Das heisst ein Wort! Ein Mann!
Da sieht man, was ein Nürnberger kann!
Drob preist man Euch noch weit und breit,
den wack'ren Bürger Pogner Veit!

\speaker{Die Lehrbuben}

\direct{lustig aufspringend}

Alle Zeit, weit und breit:
Pogner Veit! Pogner Veit!

\Vogelgesangspeaks
Wer möchte da nicht ledig sein?

\Sachsspeaks
Sein Weib gäb' mancher gern wohl drein!

\Kothnerspeaks
Auf, ledig' Mann! Jetzt macht euch 'ran!

\Pognerspeaks
Nun hört noch, wie ich's ernstlich mein'!

\direct{Die Meister setzen sich allmählich wieder nieder, die Lehrbuben ebenfalls}

Ein' leblos' Gabe geh' ich nicht:
ein Mägdlein sitzt mit zu Gericht.
Den Preis erkennt die Meisterzunft;
doch gilt's der Eh', so will's Vernunft,
dass ob der Meister Rat
die Braut den Ausschlag hat.

\Beckmesserspeaks
\direct{zu Kothner gewandt}

Dünkt Euch das klug?

\Kothnerspeaks
\direct{laut}

Versteh' ich gut,
Ihr gebt uns in des Mägdleins Hut?

\Beckmesserspeaks
Gefährlich das!

\Kothnerspeaks
Stimmt es nicht bei,
wie wäre dann der Meister Urteil frei?

\Beckmesserspeaks
Lasst's gleich wählen nach Herzensziel
und lasst den Meistergesang aus dem Spiel!

\Pognerspeaks
Nicht so! Wie doch? Versteht mich recht!
Wem Ihr Meister den Preis zusprecht,
die Maid kann dem verwehren,
doch nie einen andren begehren.
Ein Meistersinger muss er sein:
nur wen Ihr krönt, den soll sie frei'n.

\Sachsspeaks
\direct{erhebt sich}

Verzeiht!
Vielleicht schon ginget Ihr zu weit.
Ein Mädchenherz und Meisterkunst
erglüh'n nicht stets in gleicher Brunst;
der Frauen Sinn, gar unbelehrt,
dünkt mich dem Sinn des Volks gleich wert.
Wollt Ihr nun vor dem Volke zeigen,
wie hoch die Kunst Ihr ehrt,
und lasst Ihr dem Kind die Wahl zu eigen,
wollt nicht, dass dem Spruch es wehrt:
so lasst das Volk auch Richter sein;
mit dem Kinde sicher stimmt's überein.

\speaker{Vogelgesang, Nachtigal}
Oho!

\speaker{Alle Meister}

\direct{ausser Sachs und Pogner}

Das Volk? Ja, das wäre schön!
Ade dann Kunst und Meistertön'!

\Kothnerspeaks
Nein, Sachs! Gewiss, das hat keinen Sinn,
gäbt Ihr dem Volk die Regeln hin?

\Sachsspeaks
Vernehmt mich recht! Wie Ihr doch tut!
Gesteht, ich kenn die Regeln gut;
und dass die Zunft die Regeln bewahr',
bemüh' ich mich selbst schon manches Jahr.
Doch einmal im Jahre fänd' ich's weise,
dass man die Regeln selbst probier',
ob in der Gewohnheit trägem Gleise
ihr' Kraft und Leben nicht sich verlier':
und ob Ihr der Natur noch seid auf rechter Spur,
das sagt Euch nur,
wer nichts weiss von der Tabulatur.

\direct{Die Lehrbuben springen auf und reiben sich die Hände}

\Beckmesserspeaks
Hei! Wie sich die Buben freuen!

\Sachsspeaks
\direct{eifrig fortfahrend}

Drum möcht' es Euch nie gereuen,
dass jährlich am Sankt-Johannis-Fest,
statt dass das Volk man kommen lässt,
herab aus hoher Meister Wolk'
Ihr selbst Euch wendet zu dem Volk.
Dem Volke wollt Ihr behagen;
nun dächt' ich, läg' es nah,
Ihr liesst es selbst Euch auch sagen,
ob das ihm zur Lust geschah.
Dass Volk und Kunst gleich blüh' und wachs',
bestellt Ihr so, mein' ich, Hans Sachs.

\Vogelgesangspeaks
Ihr meint's wohl recht!

\Kothnerspeaks
Doch steht's drum faul.

\Nachtigallspeaks
Wenn spricht das Volk, halt' ich das Maul.

\Kothnerspeaks
Der Kunst droht allweil Fall und Schmach,
läuft sie der Gunst des Volkes nach.

\Beckmesserspeaks
Drin bracht' er's weit, der hier so dreist:
Gassenhauer dichtet er meist.

\Pognerspeaks
Freund Sachs, was ich mein', ist schon neu:
zuviel auf einmal brächte Reu'!

\direct{Er wendet sich zu den Meistern.}

So frag' ich, ob den Meistern gefällt
Gab' und Regel, so wie ich's gestellt?

\direct{Die Meister erheben sich beistimmend.}

\Sachsspeaks
Mir genügt der Jungfer Ausschlagstimm'.

\Beckmesserspeaks
Der Schuster weckt doch stets mir Grimm!

\Kothnerspeaks
Wer schreibt sich als Werber ein?
Ein Junggesell' muss es sein.

\Beckmesserspeaks
Vielleicht auch ein Witwer? Fragt nur den Sachs!

\Sachsspeaks
Nicht doch, Herr Merker! Aus jüng'rem Wachs
als ich und Ihr muss der Freier sein,
soll Evchen ihm den Preis verleih'n.

\Beckmesserspeaks
Als wie auch ich? Grober Gesell!

\Kothnerspeaks
Begehrt wer Freiung, der komm' zur Stell'!
Ist jemand gemeld't, der Freiung begehrt?

\Pognerspeaks
Wohl, Meister! Zur Tagesordnung kehrt!
Und nehmt von mir Bericht,
wie ich auf Meisterpflicht
einen jungen Ritter empfehle,
der will, dass man ihn wähle
und heut als Meistersinger frei'.
Mein Junker Stolzing, kommt herbei!

\direct{Walther tritt hervor und verneigt sich}


\Beckmesserspeaks

\direct{bei Seite}

Dacht' ich mir's doch! Geht's da hinaus, Veit?

\direct{Laut}

Meister, ich mein', zu spät ist's der Zeit.

\speaker{\Schwarz und \Foltz}
Der Fall Soll man sich freu'n?

\speaker{Die übrigen Meister}
Ein Ritter gar?

\speaker{\Vogelgesang, \Moser, \Eisslinger}
Soll man sich freu'n?

\speakec{\Zorn, \Kothner, \Nachtigall, \Ortel}
Wäre da Gefahr?

\Vogelgesangspeaks
Oder wär' Gefahr?

\speaker{Alle Meister}
Immerhin hat's ein gross' Gewicht,
dass Meister Pogner für ihn spricht.

\Kothnerspeaks
Soll uns der Junker willkommen sein,
zuvor muss er wohl vernommen sein.

\Pognerspeaks
Vernehmt ihn wohl! Wünsch' ich ihm Glück,
nicht bleib' ich doch hinter der Regel zurück.
Tut, Meister, die Fragen!

\Kothnerspeaks
So mög' uns der Junker sagen:
ist er frei und ehrlich geboren?

\Pognerspeaks
Die Frage gebt verloren,
da ich Euch selbst des Bürge steh',
dass er aus frei' und edler Eh':
von Stolzing Walther aus Frankenland,
nach Brief und Urkund' mir wohlbekannt.
Als seines Stammes letzter Spross
verliess er neulich Hof und Schloss
und zog nach Nürnberg her,
dass er hier Bürger wär'.

\Beckmesserspeaks
Neu Junker-Unkraut! Tut nicht gut!

\Nachtigallspeaks
Freund Pogners Wort Genüge tut.

\Sachsspeaks
Wie längst von den Meistern beschlossen ist,
ob Herr, ob Bauer, hier nichts beschiesst:
hier fragt sich's nach der Kunst allein,
wer will ein Meistersinger sein.

\Kothnerspeaks
Drum nun frag' ich zur Stell':
welch Meisters seid Ihr Gesell'?

\Waltherspeaks

Am stillen Herd in Winterszeit,
wann Burg und Hof mir eingeschneit,
wie einst der Lenz so lieblich lacht'
und wie er bald wohl neu erwacht,
ein altes Buch, vom Ahn vermacht,
gab das mir oft zu lesen:
Herr Walther von der Vogelweid',
der ist mein Meister gewesen.

\Sachsspeaks
Ein guter Meister!

\Beckmesserspeaks
Doch lang' schon tot;
wie lehrt' ihn der wohl der Regeln Gebot?

\Kothnerspeaks
Doch in welcher Schul' das Singen
mocht' Euch zu lernen gelingen?

\Waltherspeaks
Wann dann die Flur vom Frost befreit
und wiederkehrt die Sommerszeit,
was einst in langer Winternacht
das alte Buch mir kundgemacht,
das schallte laut in Waldespracht,
das hört' ich hell erklingen:
im Wald dort auf der Vogelweid',
da lernt' ich auch das Singen.

\Beckmesserspeaks
Oho! Von Finken und Meisen
lerntet Ihr Meisterweisen?
Das wird dann wohl auch darnach sein!

\Vogelgesangspeaks
Zwei art'ge Stollen fasst' er da ein.

\Beckmesserspeaks
Ihr lobt ihn, Meister Vogelgesang,
wohl weil vom Vogel er lernt' den Gesang?

\Kothnerspeaks
Was meint Ihr, Meister? Frag' ich noch fort?
Mich dünkt, der Junker ist fehl am Ort.

\Sachsspeaks
Das wird sich bäldlich zeigen.
Wenn rechte Kunst ihm eigen
und gut er sie bewährt,
was gilt's, wer sie ihn gelehrt?

\Kothnerspeaks
\direct{zu Walther}

Seid Ihr bereit, ob Euch geriet
mit neuer Find' ein Meisterlied,
nach Dicht' und Weis' Eu'r eigen,
zur Stunde jetzt zu zeigen?

\Waltherspeaks
Was Winternacht, was Waldespracht,
was Buch und Hain mich wiesen;
was Dichtersanges Wundermacht
mir heimlich wollt' erschliessen;
was Rosses Schritt beim Waffenritt,
was Reihentanz bei heit'rem Schanz
mir sinnend gab zu lauschen:
gilt es des Lebens höchsten Preis,
um Sang mir einzutauschen,
zu eignem Wort und eigner Weis'
will einig mir es fliessen,
als Meistersang, ob den ich weiss,
Euch Meistern sich ergiessen.

\Beckmesserspeaks
Entnahmt Ihr was der Worte Schwall?

\Vogelgesangspeaks
Ei nun, er wagt's!

\Nachtigallspeaks

Merkwürd'ger Fall!

\Kothnerspeaks
Nun, Meister, wenn's gefällt,
werd' das Gemerk bestellt.

\direct{zu Walther}

Wählt der Herr einen heiligen Stoff?

\Waltherspeaks
Was heilig mir, der Liebe Panier
schwing' und sing' ich mir zu Hoff.

\Kothnerspeaks
Das gilt uns weltlich. Drum allein,
Meister Beckmesser, schliesst Euch ein!

\Beckmesserspeaks

\direct{erhebt sich und schreitet wie widerwillig dem Gemerke zu}

Ein sau'res Amt, und heut'zumal!
Wohl gibt's mit der Kreide manche Qual.

\direct{Er verneigt sich gegen Walther.}

Herr Ritter, wisst:
Sixtus Beckmesser Merker ist.
Hier im Gemerk
verrichtet er still sein strenges Werk.
Sieben Fehler gibt er Euch vor,
die merkt er mit Kreide dort an:
wenn er über sieben Fehler verlor,
dann versang der Herr Rittersmann.

\direct{Er setzt sich im Gemerk}

Gar fein er hört;
doch dass er Euch den Mut nicht stört,
säht Ihr ihm zu, so gibt er Euch Ruh'
und schliesst sich gar hier ein
lässt Gott Euch befohlen sein.

\direct{Er streckt den Kopf höhnisch freundlich nickend heraus und verschwindet hinter dem zugezogenen Vorhange des Gemerks gänzlich}


\Kothnerspeaks
\direct{winkt den Lehrbuben. Zu Walther.}

Was Euch zum Liede Richt' und Schnur,
vernehmt nun aus der Tabulatur.

\direct{Zwei Lehrbuben haben die an der Wand aufgehängte Tafel der ``Leges Tabulaturae'' herabgenommen und halten sie Kothner vor; dieser liest daraus}

``Ein jedes Meistergesanges Bar
stell' ordentlich ein Gemässe dar
aus unterschiedlichen Gesätzen,
die keiner soll verletzen.
Ein Gesätz besteht aus zweenen Stollen,
die gleiche Melodei haben sollen;
der Stoll' aus etlicher Vers' Gebänd',
der Vers hat seinen Reim am End'.
Darauf erfolgt der Abgesang,
der sei auch etlich' Verse lang
und hab' sein' besond're Melodei,
als nicht im Stollen zu finden sei.
Derlei Gemässes mehre Baren
soll ein jed'Meisterlied bewahren;
und wer ein neues Lied gericht't,
das über vier der Silben nicht
eingreift in andrer Meister Weis',
dess Lied erwerb' sich Meisterpreis.''
Er gibt die Tafel den Lehrbuben zurück; diese hängen sie wieder auf
Nun setzt Euch in den Singestuhl!

\Waltherspeaks

\direct{mit einem Schauer}

Hier, in den Stuhl?

\Kothnerspeaks
Wie's Brauch der Schul'.

\Waltherspeaks
\direct{besteigt den Stuhl und setzt sich mit Widerstreben. Beiseite}

Für dich, Geliebte, sei's getan!

\Kothnerspeaks
\direct{sehr laut}

Der Sänger sitzt.

\Beckmesserspeaks
\direct{unsichtbar im Gemerk, sehr grell}

Fanget an!

\Waltherspeaks
Fanget an!
So rief der Lenz in den Wald,
dass laut es ihn durchhallt;
und wie in fern'ren Wellen
der Hall von dannen flieht,
von weither naht ein Schwellen,
das mächtig näher zieht;
es schwillt und schallt,
es tönt der Wald
von holder Stimmen Gemenge;
nun laut und hell schon nah zur Stell',
wie wächst der Schwall! Wie Glockenhall
ertost des Jubels Gedränge!
Der Wald, wie bald
antwortet er dem Ruf,
der neu ihm Leben schuf,
stimmte an
das süsse Lenzeslied!

\direct{Man hört aus dem Gemerk unnzutige Seufzer des Merkers und heftiges Anstreichen mit der Kreide. Auch Walther hat es gehört; nach kurzer Störung fährt er fort}

In einer Dornenhecken,
von Neid und Gram verzehrt,
musst' er sich da verstecken,
der Winter, grimm-bewehrt.
Von dürrem Laub umrauscht
er lauert da und lauscht,
wie er das frohe Singen
zu Schaden könnte bringen.

\direct{Er steht vom Stuhle auf}

Doch fanget an!
So rief es mir in der Brust,
als noch ich von Liebe nicht wusst'.
Da fühlt' ich's tief sich regen,
als weckt' es mich aus dem Traum;
mein Herz mit bebenden Schlägen
erfüllte des Busens Raum:
das Blut, es wallt mit Allgewalt,
geschwellt von neuem Gefühle;
aus warmer Nacht mit Übermacht
schwillt mir zum Meer der Seufzer Heer
im wilden Wonnegewühle.
Die Brust wie bald
antwortet sie dem Ruf,
der neu ihr Leben schuf;
stimmt nun an
das hehre Liebeslied!

\Beckmesserspeaks
\direct{den Vorhang aufreissend}

Seid Ihr nun fertig?

\Waltherspeaks
Wie fraget Ihr?

\Beckmesserspeaks
Mit der Tafel ward ich fertig schier.

\direct{Er hält die ganz mit Kreidestrichen bedeckte Tafel heraus; die Meister brechen in ein Gelächter aus}


\Waltherspeaks
Hört doch! Zu meiner Frauen Preis
gelang' ich jetzt erst mit der Weis'.

\Beckmesserspeaks
\direct{das Gemerk verlassend}

Singt, wo Ihr wollt! Hier habt Ihr vertan.
Ihr Meister, schaut die Tafel Euch an:
so lang' ich leb', ward's nicht erhört;
ich glaubt's nicht, wenn Ihr's all auch schwört!

\Waltherspeaks
Erlaubt Ihr's, Meister, dass er mich stört?
Blieb ich von allen ungehört?

\Pognerspeaks
Ein Wort, Herr Merker! Ihr seid gereizt!

\Beckmesserspeaks
Sei Merker fortan, wer danach geizt!
Doch dass der Junker hier versungen hat,
beleg' ich erst noch vor der Meister Rat.
Zwar wird's 'ne harte Arbeit sein:
wo beginnen, da wo nicht aus noch ein?
Von falscher Zahl und falschem Gebänd'
schweig' ich schon ganz und gar;
zu kurz, zu lang, wer ein End' da fänd'!
Wer meint hier im Ernst einen Bar?
Auf ``blinde Meinung'' klag' ich allein:
sagt, konnt' ein Sinn unsinniger sein?

\speaker{Die Meister}

\direct{ohne Sachs und Pogner}

Man ward nicht klug! Ich muss gestehn.
Ein Ende konnte keiner erseh'n.

\Beckmesserspeaks
Und dann die Weis'! Welch tolles Gekreis'
aus ``Abenteuer''-, ``blau Rittersporn''-Weis',
``hoch Tannen''- und ``stolz Jüngling''-Ton!

\Kothnerspeaks
Ja, ich verstand gar nichts davon!

\Beckmesserspeaks
Kein Absatz wo, kein' Koloratur,
von Melodei auch nicht eine Spur!

\speaker{\Ortel, dann \Foltz}
Wer nennt das Gesang?

\Moserspeaks
Es ward einem bang'!

\Nachtigallspeaks
Ja, 's ward einem bang!

\Vogelgesangspeaks
Eitel Ohrgeschinder!

\Zornspeaks
Auch gar nichts dahinter!

\Kothnerspeaks
Und gar vom Singstuhl ist er gesprungen!

\Beckmesserspeaks
Wird erst auf die Fehlerprobe gedrungen?
Oder gleich erklärt, dass er versungen?

\Sachsspeaks

\direct{der vom Beginne an Walther mit wachsendem Ernst zugehört hat, schreitet vor}

Halt Meister! Nicht so geeilt!
Nicht jeder Eure Meinung teilt.
Des Ritters Lied und Weise,
sie fand ich neu, doch nicht verwirrt;
verliess er unsre Gleise,
schritt er doch fest und unbeirrt.
Wollt Ihr nach Regeln messen,
was nicht nach Eurer Regeln Lauf,
der eig'nen Spur vergessen,
sucht davon erst die Regeln auf!

\Beckmesserspeaks

Aha, schon recht! Nun hört Ihr's doch:
den Stümpern öffnet Sachs ein Loch,
da aus und ein nach Belieben
ihr Wesen leicht sie trieben.
Singet dem Volk auf Markt und Gassen;
hier wird nach den Regeln nur eingelassen!

\Sachsspeaks
Herr Merker, was doch solch ein Eifer?
Was doch so wenig Ruh'?
Eu'r Urteil, dünkt mich, wäre reifer,
hörtet Ihr besser zu.
Darum, so komm' ich jetzt zum Schluss,
dass den Junker man zu End' hören muss.

\Beckmesserspeaks
Der Meister Zunft, die ganze Schul',
gegen den Sachs da sind wir Null.

\Sachsspeaks
Verhüt' es Gott, was ich begehr',
dass das nicht nach den Gesetzen wär'!
Doch da nun steht geschrieben:
``Der Merker werde so bestellt,
dass weder Hass noch Lieben
das Urteil trübe, das er fällt.''
Geht der nun gar auf Freiersfüssen,
wie sollt' er da die Lust nicht büssen,
den Nebenbuhler auf dem Stuhl
zu schmähen vor der ganzen Schul'?

\direct{Walther flammt auf.}


\Nachtigallspeaks
Ihr geht zu weit!

\Kothnerspeaks
Persönlichkeit!

\Pognerspeaks
Vermeidet, Meister, Zwist und Streit!

\Beckmesserspeaks
Ei, was kümmert doch Meister Sachsen,
auf was für Füssen ich geh?
Liess er doch lieber Sorge sich wachsen,
dass mir nichts drück' die Zeh'!
Doch seit mein Schuster ein grosser Poet,
gar übel es um mein Schuhwerk steht.
Da seht, wie's schlappt und überall klappt!
All seine Vers' und Reim' liess ich ihm gern daheim,
Historien, Spiel' und Schwänke dazu,
brächt' er mir morgen die neuen Schuh'!

\Sachsspeaks

\direct{kratzt sich hinter den Ohren}

Ihr mahnt mich da gar recht:
doch schickt sich's, Meister, sprecht,
dass, find' ich selbst dem Eseltreiber
ein Sprüchlein auf die Sohl',
dem hochgelahrten Herrn Stadtschreiber
ich nichts drauf schreiben soll?
Das Sprüchlein, das Eu'r würdig sei,
mit all meiner armen Poeterei
fand ich noch nicht zur Stund';
doch wird's wohl jetzt mir kund,
wenn ich des Ritters Lied gehört:
drum sing' er nun weiter ungestört!

\direct{Walther steigt in grosser Aufregung auf den Singstuhl und blickt stehend herab}


\Beckmesserspeaks
Nicht weiter! Zum Schluss!

\speaker{\Ortel, \Moser, \Vogelgesang, \Nachtigall}
\direct{nacheinander}

Genug!

\speaker{\Zorn, \Eisslinger}
Zum Schluss!

\Kothnerspeaks
Genug! Zum Schluss.

\Sachsspeaks
\direct{zu Walther}

Singt dem Herrn Merker zum Verdruss!

\Beckmesserspeaks
Was sollte man da noch hören?
Wär's nicht Euch zu betören?

\direct{Er holt aus dem Gemerk die Tafel herbei und hält sie während des Folgenden, von einem zum andern sich wendend, zur Prüfung den Meistern vor}


\Waltherspeaks
Aus finst'rer Dornenhecken
die Eule rauscht' hervor,
tät' rings mit Kreischen wecken
der Raben heis'ren Chor:

\Beckmesserspeaks
Jeden Fehler gross und klein
seht genau auf der Tafel ein.

\speaker{Die Meister}
\direct{ohne \Sachs und \Pogner}

Jawohl, so ist's!

\Waltherspeaks
in nächt'gem Heer zu Hauf
wie krächzen all' da auf
mit ihren Stimmen, den hohlen,
die Elstern, Kräh'n und Dohlen!

\Beckmesserspeaks
``Falsch Gebänd'', ``unredbare Worte'',
``Klebsilben'', hier ``Laster'' gar;

\speaker{Die Meister}
\direct{ohne \Sachs und \Pogner}

Ich seh' es recht!
Mit dem Herrn Ritter steht es schlecht.
Mag Sachs von ihm halten, was er will,
hier in der Singschul' schweig' er still!

\Sachsspeaks
\direct{beobachtet Walther entzückt}

Ha, welch ein Mut!
Begeisterungsglut!

\Waltherspeaks
Auf da steigt
mit gold'nem Flügelpaar
ein Vogel wunderbar:
sein strahlend hell Gefieder
licht in den Lüften blinkt;

\Beckmesserspeaks
``Äquivoca'', ``Reim am falschen Orte'',
``verkehrt'', ``verstellt'' der ganze Bar;
ein ``Flickgesang'' hier zwischen den Stollen;

\Pognerspeaks
Jawohl, ich seh's, was mir nicht recht:
mit meinem Junker steht es schlecht!

\speaker{Die Meister}

\direct{ohne \Sachs und \Pogner}

Bleibt einem jeden doch unbenommen,
wen er sich zum Genossen begehrt!

\Sachsspeaks
Ihr Meister, schweigt doch und hört!

\Waltherspeaks
schwebt selig hin und wieder,
zu Flug und Flucht mir winkt.
Es schwillt das Herz
vor süssem Schmerz,

\Pognerspeaks
Weich' ich hier der Übermacht,
mir ahnet, dass mir's Sorge macht.

\speaker{Die Meister}
\direct{ohne \Sachs und \Pogner}

Wär' uns der erste best'willkommen,
was blieben die Meister dann wert?

\Sachsspeaks
\direct{inständig}

Hört, wenn Sachs Euch beschwört!

\Beckmesserspeaks
``blinde Meinung'' allüberall;

\Sachsspeaks
Herr Merker da, gönnt doch nur Ruh'!

\Beckmesserspeaks
``unklare Wort'\,'', ``Differenz'',
hier ``Schrollen'',
da ``falscher Atem'', hier ``Überfall''.

\Waltherspeaks
der Not entwachsen Flügel;
es schwingt sich auf
zum kühnen Lauf,
aus der Städte Gruft
zum Flug durch die Luft,
dahin zum heimischen Hügel;

\Sachsspeaks
Lasst and're hören, gebt das nur zu!
Umsonst! All eitel' Trachten!
Kaum vernimmt man sein eig'nes Wort!

\Beckmesserspeaks
Ganz unverständliche Melodei!
Aus allen Tönen ein Mischgebräu!

\Sachsspeaks
Des Junkers will keiner achten.
Das nenn' ich Mut, singt der noch fort!

\Pognerspeaks
Wie gern säh' ich ihn angenommen,

\Waltherspeaks
dahin zur grünen Vogelweid',
wo Meister Walther einst mich freit';
da sing' ich hell und hehr
der liebsten Frauen Ehr';

\speaker{\David und die Lehrbuben}
\direct{sind von der Bank aufgestanden und nähern sich dem Gemerk, um welches sie einen Ring schliessen und sich zum Reigen ordnen}

Glück auf zum Meistersingen,
mögt Ihr Euch das Kränzlein erschwingen!

\direct{Sie fassen sich an und tanzen im Ringe immer lustiger um das Gemerk}


\Beckmesserspeaks
Scheutet Ihr nicht das Ungemach,
Meister, zählt mir die Fehler nach!

\speaker{Die Meister}
\direct{ohne \Sachs und \Pogner}

Hei wie sich der Ritter da quält!

\Pognerspeaks
als Eidam wär' er mir gar wert;

\Sachsspeaks
Das Herz auf dem rechten Fleck:
ein wahrer Dichter-Reck'!

\Waltherspeaks
auf dann steigt,
ob Meister-Kräh'n ihm ungeneigt,
das stolze Minnelied.

\speaker{\David und die Lehrbuben}

Das Blumenkränzlein aus Seiden fein
wird das dem Herrn Ritter beschieden sein?

\Beckmesserspeaks
Verloren hätt' er schon mit dem acht':
doch so weit wie der hat's noch keiner gebracht!

\Pognerspeaks
nenn' ich den Sieger jetzt willkommen,
wer weiss, ob ihn mein Kind erwählt?

\speaker{Die Meister}
\direct{ohne \Sachs und \Pogner}

Der Sachs hat ihn sich erwählt!

\direct{lachend}

Hahaha!

\Sachsspeaks
Mach' ich, Hans Sachs, wohl Vers' und Schuh',
ist Ritter der und Poet dazu.

\speaker{Die Meister}
\direct{ohne \Sachs und \Pogner}

's ist ärgerlich gar!
Drum macht ein End'!

\Beckmesserspeaks
Wohl über fünfzig, schlecht gezählt!
Sagt, ob Ihr Euch den zum Meister wählt?

\Pognerspeaks
Gesteh ich's, dass mich das quält,
ob Eva den Meister wählt!

\speaker{Die Meister}
\direct{ohne \Sachs und \Pogner}

Auf, Meister, stimmt
und erhebt die Händ'!
Die Meister erheben die Hände

\Waltherspeaks
Ade, Ihr Meister, hienied'!

\Beckmesserspeaks
Nun, Meister, kündet's an!

\speaker{Die Meister}
\direct{ohne \Sachs und \Pogner}

Versungen und vertan!

\StageDir{Er verlässt mit einer stolzen verächtlichen Gebärde den Stuhl und wendet sich rasch zum Fortgehen.
  
Alles geht in Aufregung auseinander; lustiger Tumult der Lehrbuben, welche sich des Gemerks des Singstuhls und der Meisterbänke bemächtigen, wodurch Gedränge und Durcheinander der nach dem Ausgange sich wendenden Meister entsteht. Sachs, der allein im Vordergrunde geblieben, blickt noch gedankenvoll nach dem leeren Singestuhl, als die Lehrbuben auch diesen erfassen. Während Sachs mit humoristisch-unmutiger Gebärde sich abwendet, fällt der Vorhang.}


\act

\scene

\StageDir{Die Bühne stellt im Vordergrund eine Strasse im Längendurchschnitt dar, welche in der Mitte von einer schmalen Gasse, nach dem Hintergrunde zu krumm abbiegend, durchschnitten wird, so dass sich in Front zwei Eckhäuser darbieten, von denen das eine reichere, rechts, das Haus Pogners, das andere einfachere. Links, das des Hans \Sachs ist. Vor Pogners Haus eine Linde; vor dem Sachsens ein Fliederbaum. Heiterer Sommerabend, im Verlaufe der ersten Auftritte allmählich einbrechende Nacht. David ist darüber her, die Fensterläden nach der Gasse zu von aussen zu schliessen. Andere Lehrbuben tun das gleiche bei anderen Häusern.}

\speaker{Lehrbuben}

\direct{an der Arbeit}

Johannistag! Johannistag!
Blumen und Bänder, so viel man mag!

\Davidspeaks

\direct{leise für sich}

Das Blumenkränzlein von Seiden fein
möcht' es mir balde beschieden sein!

\Magdalenespeaks

\direct{ist mit einem Korbe am Arm aus Pogners Haus gekommen und sucht David unbemerkt sich zu nähern}

Pst, David!

\Davidspeaks

\direct{nach der Gasse zu sich umwendend, heftig}

Ruft ihr schon wieder?
Singt allein eure dummen Lieder!

\direct{Er wendet sich unwillig zur Seite}


\speaker{Lehrbuben}

\direct{zuerst Magdalenes Stimme nachahmend}

David, was soll's? Wärst nicht so stolz,
schaut'st besser um, wärst nicht so dumm!
Johannistag! Johannistag!
Wie der nur die Jungfer Lene nicht kennen mag!

\Magdalenespeaks
David, hör' doch! Kehr' dich zu mir!

\Davidspeaks
Ach, Jungfer Lene! Ihr seid hier?

\Magdalenespeaks

\direct{auf ihren Korb deutend}

Bring' dir was Gut's; schau nur hinein!
Das soll für mein lieb' Schätzel sein.
Erst aber schnell, wie ging's mit dem Ritter?
Du rietest ihm gut? Er gewann den Kranz?

\Davidspeaks
Ach, Jungfer Lene! Da steht's bitter; der hat versungen und ganz vertan!

\Magdalenespeaks

\direct{erschrocken}

Versungen? Vertan?

\Davidspeaks
Was geht's Euch nur an?

\Magdalenespeaks

\direct{den Korb, nach welchem David die Hand ausstreckt, heftig zurückziehend}

Hand von der Taschen! Nichts zu naschen!
Hilf Gott! Unser Junker vertan!


\direct{Sie geht mit Gebärden der Trostlosigkeit ins Haus zurück. David sieht verblüfft nach}


\speaker{Die Lehrbuben}

\direct{welche unbemerkt nähergeschlichen waren und gelauscht hatten, präsentieren sich jetzt, wie glückwünschend, \David}

Heil, Heil zur Eh' dem jungen Mann!
Wie glücklich hat er gefreit!
Wir hörten's all' und sahen's an:
der er sein Herz geweiht,
für die er lässt sein Leben,
die hat ihm den Korb nicht gegeben.

\Davidspeaks

\direct{auffahrend}

Was steht ihr hier faul?
Gleich haltet das Maul!

\speaker{Die Lehrbuben}

\direct{schliessen einen Ring um David und tanzen um ihn}

Johannistag! Johannistag!
Da freit ein jeder, wie er mag.
Der Meister freit, der Bursche freit!
Da gibt's Geschlamb und Geschlumbfer.
Der Alte freit die junge Maid,
der Bursche die alte Jumbfer!
Juchhei! Juchhei! Johannistag!


\direct{David ist im Begriff wütend dreinzuschlagen, als Sachs, der aus der Gasse hervorgekommen, dazwischentritt. Die Lehrbuben fahren auseinander.}


\Sachsspeaks

\direct{zu David}

Was gibt's? Treff' ich dich wieder am Schlag?

\Davidspeaks
Nicht ich! Schandlieder singen die.

\Sachsspeaks
Hör' nicht drauf! Lern's besser wie sie!
Zur Ruh'! Ins Haus! Schliess und mach Licht!


\direct{Die Lehrbuben zerstreuen sich}


\Davidspeaks
Hab ich heut Singstund'?

\Sachsspeaks
Nein, singst nicht
zur Straf' für dein heutig frech' Erdreisten.
Die neuen Schuh' steck mir auf den Leisten!


\direct{David und Sachs sind in die Werkstatt eingetreten und gehen durch eine innere Tür ab}



\scene

\StageDir{Pogner und Eva, vom Spaziergang heimkehrend, die Tochter leicht am Arme des Vaters eingehängt, sind schweigsam die Gasse heraufgekommen.}

\Pognerspeaks

\direct{noch auf der Gasse, durch eine Klinze im Fensterladen von Sachs' Werkstatt spähend}

Lass seh'n, ob Nachbar Sachs zu Haus?
Gern spräch' ich ihn. Trät' ich wohl ein?


\direct{David kommt mit Licht aus der Kammer, setzt sich damit an den Werktisch am Fenster und macht sich über die Arbeit her}


\Evaspeaks

\direct{spähend}

Er scheint daheim:
kommt Licht heraus.

\Pognerspeaks
Tu ich's? Zu was doch? Besser, nein!

\direct{Er wendet sich ab}

Will einer Selt'nes wagen,
was liess' er sich dann sagen?

\direct{Er sinnt nach}

War er's nicht, der meint', ich ging' zu weit?
Und blieb ich nicht im Geleise,
war's nicht auf seine Weise?
Doch war's vielleicht auch Eitelkeit?

\direct{Er wendet sich zu Eva}

Und du, mein Kind, du sagst mir nichts?

\Evaspeaks
Ein folgsam Kind, gefragt nur spricht's.

\Pognerspeaks
Wie klug! Wie gut! Komm, setz' dich hier
ein Weil' noch auf die Bank zu mir.
Er setzt sich auf die Steinbank unter der Linde

\Evaspeaks
Wird's nicht zu kühl?
‘s war heut' gar schwül.

\Pognerspeaks
Nicht doch, ‘s ist mild und labend; gar lieblich lind der Abend.

\direct{Eva setzt sich zögernd und beklommen Pogner zur Seite}

Das deutet auf den schönsten Tag,
der morgen soll erscheinen.
o Kind, sagt dir kein Herzensschlag,
welch Glück dich morgen treffen mag,
wenn Nüremberg, die ganze Stadt
mit Bürgern und Gemeinen,
mit Zünften, Volk und hohem Rat,
vor dir sich soll vereinen,
dass du den Preis, das edle Reis,
erteilest als Gemahl
dem Meister deiner Wahl?

\Evaspeaks
Lieb' Vater, muss es ein Meister sein?

\Pognerspeaks
Hör' wohl: ein Meister deiner Wahl.

\direct{Magdalene erscheint an der Tür und winkt Eva}


\Evaspeaks

\direct{zerstreut}

Ja meiner Wahl! Doch tritt nur ein.

\direct{Laut zu Magdalene gewandt}

Gleich, Lene, gleich! Zum Abendmahl.

\direct{Sie steht auf}


\Pognerspeaks

\direct{ärgerlich aufstehend}

‘s gibt doch keinen Gast?

\Evaspeaks

\direct{wie zuvor}

Wohl den Junker?

\Pognerspeaks

\direct{verwirrt}

Wieso?

\Evaspeaks
Sahst ihn heut' nicht?

\Pognerspeaks

\direct{halb für sich nachdenklich zerstreut}

Ward sein nicht froh.

\direct{Sich zusammennehmend}

Nicht doch! Was denn?

\direct{Sich vor die Stirn klopfend}

Ei, werd ich dumm?

\Evaspeaks
Lieb' Väterchen, komm! Geh', kleid' dich um!

\Pognerspeaks

\direct{während er ins Haus vorangeht}

Hm! Was geht mir im Kopf doch, rum?

\Magdalenespeaks

\direct{heimlich zu Eva}

Hast was heraus?

\Evaspeaks

\direct{ebenso}


Blieb still und stumm.

\Magdalenespeaks
Sprach David:
meint', er habe vertan.

\Evaspeaks

\direct{erschrocken}

Der Ritter! Hilf Gott, was fing' ich an?
Ach, Lene, die Angst! Wo was erfahren?

\Magdalenespeaks
Vielleicht vom Sachs?

\Evaspeaks

\direct{heiter}

Ach, der hat mich lieb! Gewiss, ich geh' hin.

\Magdalenespeaks
Lass drin nichts gewahren!
Der Vater merkt' es, wenn man jetzt blieb'.
Nach dem Mahl:
dann hab ich dir noch was zu sagen,

\direct{im Abgehen auf der Treppe}

was jemand geheim mir aufgetragen.

\Evaspeaks

\direct{sich umwendend}

Wer denn? Der Junker?

\Magdalenespeaks
Nichts da! Nein, Beckmesser!

\Evaspeaks
Das mag was Rechtes sein!

\direct{Sie geht in das Haus, Magdalene folgt ihr}



\scene

\StageDir{Sachs ist, in leichter Hauskleidung, von innen in die Werkstatt zurückgekommen. Er wendet sich zu David, der an seinem Werktische verblieben ist.}

\Sachsspeaks
Zeig her! 's ist gut. Dort an die Tür
%% TODO THIS
riick' mir Tisch und Schemel herfür!
Leg' dich zu Bett! Steh' auf beizeit'
verschlaf die Dummheit, sei morgen gescheit!

\Davidspeaks

\direct{während er den Tisch und Schemel richtet}

Schafft Ihr noch Arbeit?

\Sachsspeaks
Kümmert dich das?

\Davidspeaks

\direct{für sich}

Was war nur der Lene? Gott weiss, was!
Warum wohl der Meister heute wacht?

\Sachsspeaks
Was stehst noch?

\Davidspeaks
Schlaft wohl, Meister!

\Sachsspeaks
Gut' Nacht!


\direct{David geht in die der Gasse zu gelegene Kammer ab}


\Sachsspeaks

\direct{legt sich die Arbeit zurecht, setzt sich an der Tür auf den Schemel, lässt aber die Arbeit wieder liegen und lehnt, mit dem Arm auf den geschlossenen Unterteil des Türladens gestützt, sich zurück}

Was duftet doch der Flieder
so mild, so stark und voll!
Mir löst es weich die Glieder,
will, dass ich was sagen soll.
Was gilt's, was ich dir sagen kann?
Bin gar ein arm einfältig Mann!
Soll mir die Arbeit nicht schmecken,
gäbst, Freund, lieber mich frei;
tät' besser, das Leder zu strecken,
und liess alle Poeterei.

\direct{Er nimmt heftig und geräuschvoll die Schusterarbeit vor. Lässt wieder ab, lehnt sich von neuem zurück und sinnt nach}

%% TODO Weird apostrophe in 's.
Und doch, ‘s will halt nicht geh'n.
Ich fühl's, und kann's nicht versteh'n
kann's nicht behalten---doch auch nicht vergessen;
und fass ich es ganz---kann ich's nicht messen!
Doch wie wollt' ich auch messen,
was unermesslich mir schien?
Kein' Regel wollte da passen
und war doch kein Fehler drin.
Es klang so alt und war doch so neu
wie Vogelsang im süssen Mai!
Wer ihn hört und wahnbetört
sänge dem Vogel nach,
dem brächt' es Spott und Schmach.
Lenzes Gebot, die süsse Not,
die legt' es ihm in die Brust:
nun sang er, wie er musst'!
Und wie er musst'---so konnt' er's;
das merkt' ich ganz besonders.
Dem Vogel, der heut' sang,
dem war der Schnabel hold gewachsen:
macht' er den Meistern bang,
gar wohl gefiel' er doch Hans Sachsen.

\direct{Er nimmt mit heiterer Gelassenheit seine Arbeit vor}

\scene

\StageDir{\Eva ist auf die Strasse getreten, hat sich schüchtern der Werkstatt genähert und steht jetzt unbemerkt an der Tür bei \Sachs.}

\Evaspeaks
Gut'n Abend, Meister! Noch so fleissig?

\Sachsspeaks

\direct{fährt angenehm überrascht auf}

Ei, Kind!
Lieb Evchen! Noch so spät?
Und doch, warum so spät noch, weiss ich:
die neuen Schuh'?

\Evaspeaks
Wie fehl er rät!
Die Schuh' hab ich noch gar nicht probiert;
sie sind so schön und reich geziert,
dass ich sie noch nicht an die Füss' mir getraut.

\direct{Sie setzt sich dicht neben Sachs auf den Steinsitz}


\Sachsspeaks
Doch sollst sie morgen tragen als Braut?

\Evaspeaks
Wer wäre denn Bräutigam?

\Sachsspeaks
Weiss ich das?

\Evaspeaks
Wie wisst Ihr dann, dass ich Braut?

\Sachsspeaks
Ei was! Das weiss die Stadt.

\Evaspeaks
Ja, weiss es die Stadt,
Freund Sachs gute Gewähr dann hat.
Ich dacht', er wüsst' mehr.

\Sachsspeaks
Was sollt' ich wissen?

\Evaspeaks
Ei seht doch! Werd ich's ihm sagen müssen?
Ich bin wohl recht dumm?

\Sachsspeaks
Das sag ich nicht.

\Evaspeaks
Dann wärt Ihr wohl klug?

\Sachsspeaks
Das weiss ich nicht.

\Evaspeaks
Ihr wisst nichts? Ihr sagt nichts? Ei, Freund Sachs,
jetzt merk' ich wahrlich, Pech ist kein Wachs.
Ich hätt' Euch für feiner gehalten.

\Sachsspeaks
Kind,
beid', Wachs und Pech, vertraut mir sind.
Mit Wachs strich ich die seid'nen Fäden,
damit ich dir die zieren Schuh' gefasst:
heut fass ich die Schuh' mit dicht'ren Drähten,
da gilt's mit Pech für den derb'ren Gast.

\Evaspeaks
Wer ist denn der? Wohl was Recht's?

\Sachsspeaks
Das mein' ich!
Ein Meister, stolz auf Freiers Fuss,
denkt morgen zu siegen ganz alleinig:
Herrn Beckmessers Schuh' ich richten muss.

\Evaspeaks
So nehmt nur tüchtig Pech dazu:
da kleb' er drin und lass' mir Ruh'!

\Sachsspeaks
Er hofft dich sicher zu ersingen.

\Evaspeaks
Wieso denn der?

\Sachsspeaks
Ein Junggesell:
's gibt deren wenig dort zur Stell'.

\Evaspeaks
Könnt's einem Witwer nicht gelingen?

\Sachsspeaks
Mein Kind, der wär' zu alt für dich.

\Evaspeaks
Ei, was! Zu alt? Hier gilt's der Kunst,
wer sie versteht, der werb' um mich!

\Sachsspeaks
Lieb' Evchen! Machst mir blauen Dunst?

\Evaspeaks
Nicht ich! Ihr seid's; Ihr macht mir Flausen!
Gesteht nur, dass Ihr wandelbar;
Gott weiss, wer Euch jetzt im Herzen mag hausen,
glaubt' ich mich doch drin so manches Jahr.

\Sachsspeaks
Wohl, da ich dich gern auf den Armen trug?

\Evaspeaks
Ich seh', 's war nur, weil Ihr kinderlos.

\Sachsspeaks
Hatt' einst ein Weib und Kinder genug.

\Evaspeaks
Doch starb Eure Frau, so wuchs ich gross.

\Sachsspeaks
Gar gross und schön!

\Evaspeaks
Da dacht' ich aus,
Ihr nähmt mich für Weib und Kind ins Haus.

\Sachsspeaks
Da hätt' ich ein Kind und auch ein Weib!
's wär ein lieber Zeitvertreib!
Ja, ja! Das hast du dir schön erdacht.

\Evaspeaks
Ich glaub', der Meister mich gar verlacht?
Am End' auch liess' er sich gar gefallen,
dass unter der Nas' ihm weg vor allen
der Beckmesser morgen mich ersäng'?

\Sachsspeaks
Wer sollt's ihm wehren, wenn's ihm geläng'?
Dem wüsst' allein dein Vater Rat.

\Evaspeaks
Wo so ein Meister den Kopf nur hat!
Käm' ich zu Euch wohl, fänd' ich's zu Haus?

\Sachsspeaks

\direct{trocken}

Ach ja! Hast recht! 's ist im Kopf mir kraus.
Hab heut manch' Sorg' und Wirr' erlebt:
da mag's dann sein, dass was drin klebt.

\Evaspeaks

\direct{wieder näher rückend}

Wohl in der Singschul'? 's war heut Gebot.

\Sachsspeaks
Ja, Kind! Eine Freiung machte mir Not.

\Evaspeaks
Ja, Sachs! Das hättet Ihr gleich soll'n sagen;
quält Euch dann nicht mit unnützen Fragen.
Nun sagt, wer war's, der Freiung begehrt?

\Sachsspeaks
Ein Junker, Kind, gar unbelehrt.

\Evaspeaks

\direct{wie heimlich}

Ein Ritter? Mein, sagt!
Und ward er gefreit?

\Sachsspeaks
Nichts da, mein Kind! 's gab gar viel Streit.

\Evaspeaks
So sagt! Erzählt, wie ging es zu?
Macht's Euch Sorg', wie liess' mir es Ruh'?
So bestand er übel und hat vertan?

\Sachsspeaks
Ohne Gnad' versang der Herr Rittersmann.

\Magdalenespeaks

\direct{kommt zum Hause heraus und ruft leise}

Pst! Evchen! Pst!

\Evaspeaks

\direct{eifrig zu Sachs gewandt}

Ohne Gnade? Wie?
Kein Mittel gäb's, das ihm gedieh?
Sang er so schlecht, so fehlervoll,
dass nichts mehr zum Meister ihm helfen soll?

\Sachsspeaks
Mein Kind, für den ist alles verloren,
und Meister wird der in keinem Land;
denn wer als Meister geboren,
der hat unter Meistern den schlimmsten Stand.

\Magdalenespeaks

\direct{vernehmlicher rufend}

Der Vater verlangt.

\Evaspeaks

\direct{immer dringender zu Sachs}

So sagt mir noch an,
ob keinen der Meister zum Freund er gewann?

\Sachsspeaks
Das wär' nicht übel! Freund ihm noch sein!
Ihm, vor dem sich alle fühlten so klein?
Den Junker Hochmut, lasst ihn laufen,
mag er durch die Welt sich raufen;
was wir erlernt mit Not und Müh',
dabei lasst uns in Ruh' verschnaufen:
hier renn' er uns nichts über'n Haufen,
sein Glück ihm anderswo erblüh'!

\Evaspeaks

\direct{erhebt sich zornig}

Ja, anderswo soll's ihm erblühn
als bei euch garst'gen, neid'schen Mannsen;
wo warm die Herzen noch erglühen,
trotz allen tück'schen Meister Hansen!

\direct{zu Magdalene}

Gleich, Lene, gleich! Ich komme schon!
Was trüg' ich hier für Trost davon?
Da riecht's nach Pech, dass Gott erbarm'!
Brennt' er's lieber, da würd' er doch warm!

\direct{Sie geht sehr aufgeregt mit Magdalene über die Strasse hinüber und verweilt in grosser Unruhe unter der Tür des Hauses}

\Sachsspeaks

\direct{sieht ihr mit bedeutungsvollem Kopfnicken nach}

Das dacht' ich wohl. Nun heisst's:
schaff Rat!

\direct{Er ist während des Folgenden damit beschäftigt, auch die obere Ladentüre so weit zu schiessen dass sie nur ein wenig Licht noch durchlässt er selbst verschwindet so fast gänzlich}

\Magdalenespeaks
Hilf Gott! Wo bliebst du nur so spat? Der Vater rief.

\Evaspeaks
Geh zu ihm ein:
ich sei zu Bett im Kämmerlein.

\Magdalenespeaks
Nicht doch! Hör mich! Komm ich dazu?
Beckmesser fand mich, er lässt nicht Ruh',
zur Nacht sollst du dich ans Fenster neigen,
er will dir was Schönes singen und geigen,
mit dem er dich hofft zu gewinnen, das Lied,
ob das dir nach Gefallen geriet.

\Evaspeaks
Das fehlte auch noch! Käme nur er!

\Magdalenespeaks
Hast David gesehn?

\Evaspeaks
Was soll mir der?

\direct{Sie späht aus}


\Magdalenespeaks

\direct{für sich}

Ich war zu streng; er wird sich grämen.

\Evaspeaks
Siehst du noch nichts?

\Magdalenespeaks

\direct{tut, als spähe sie}

's ist, als ob Leut' dort kämen.

\Evaspeaks
Wär' er's?

\Magdalenespeaks
Mach und komm jetzt hinan!

\Evaspeaks
Nicht eh'r, bis ich sah den teuersten Mann!

\Magdalenespeaks
Ich täuschte mich dort, er war es nicht.
Jetzt komm, sonst merkt der Vater die Geschicht'!

\Evaspeaks
Ach, meine Angst!

\Magdalenespeaks
Auch lass uns beraten, wie wir des Beckmessers uns entladen.

\Evaspeaks
Zum Fenster gehst du für mich.

\direct{Sie lauscht}


\Magdalenespeaks
Wie, ich?

\direct{für sich}

Das machte wohl David eiferlich?
Er schläft nach der Gassen! Hihi, 's wär' fein!

\Evaspeaks
Da hör' ich Schritte.

\Magdalenespeaks

\direct{zu Eva}

Jetzt komm, es muss sein!

\Evaspeaks
Jetzt näher!

\Magdalenespeaks
Du irrst! 's ist nichts, ich wett'.
Ei, komm! Du musst, bis der Vater zu Bett.

\speaker{\Pogner (Stimme)}

\direct{von innen}

He! Lene! Eva!

\Magdalenespeaks

's ist höchste Zeit!
Hörst du's? Komm! Dein Ritter ist weit.

\direct{Sie zieht die sich sträubende Eva am Arm die Stufen zur Tür hinauf}



\scene

\StageDir{Walther ist die Gasse heraufgekommen; jetzt biegt er um die Ecke herum: Eva erblickt ihn, reisst sich von Magdalene los und stürzt Walther auf die Strasse entgegen.}

\Evaspeaks
Da ist er!

\Magdalenespeaks
Da haben wir's! Nun heisst's:
gescheit!

\direct{Sie geht eilig in das Haus}


\Evaspeaks

\direct{ausser sich}

Ja, Ihr seid es! Nein, du bist es!
Alles sag' ich, denn Ihr wisst es;
alles klag' ich, denn ich weiss es;
Ihr seid beides, Held des Preises
und mein einz'ger Freund!

\Waltherspeaks

\direct{leidenschaftlich}

Ach, du irrst! Bin nur dein Freund, doch des Preises
noch nicht würdig, nicht den Meistern ebenbürtig.
Mein Begeistern fand Verachten,
und, ich weiss es, darf nicht trachten
nach der Freundin Hand!

\Evaspeaks

Wie du irrst! Der Freundin Hand,
erteilt nur sie den Preis,
wie deinen Mut ihr Herz erfand,
reicht sie nur dir das Reis.

\Waltherspeaks

Ach nein, du irrst! Der Freundin Hand,
wär' keinem sie erkoren;
wie sie des Vaters Wille band,
mir war sie doch verloren.
``Ein Meistersinger muss er sein,
nur wen Ihr krönt, den darf sie frein!''
So sprach er festlich zu den Herr'n,
kann nicht zurück, möcht' er auch gern!
Das eben gab mir Mut;
wie ungewohnt mir alles schien,
ich sang voll Lieb' und Glut,
dass ich den Meisterschlag verdien'.
Doch diese Meister!

\direct{wütend}

Ha, diese Meister!
Dieser Reim-Gesetze Leimen und Kleister!
Mir schwillt die Galle,
das Herz mir stockt,
denk' ich der Falle,
darein ich gelockt!
Fort in die Freiheit!
Da hin gehör' ich,
da, wo ich Meister im Haus!
Soll ich dich frei'n heut,
dich nun beschwör' ich,
komm und folg mir hinaus!
Nichts steht zu hoffen;
keine Wahl ist offen!
Überall Meister,
wie böse Geister
seh' ich sich rotten,
mich zu verspotten:
mit den Gewerken,
aus den Gemerken,
aus allen Ecken,
auf allen Flecken
seh' ich zu Haufen
Meister nur laufen,
mit höhnendem Nicken
frech auf dich blicken,
in Kreisen und Ringeln
dich umzingeln,
näselnd und kreischend
zur Braut dich heischend,
als Meisterbuhle
auf dem Singestuhle,
zitternd und bebend,
hoch dich erhebend!
Und ich ertrüg' es, sollt' es nicht wagen,
gradaus tüchtig d'rein zu schlagen?

\direct{Man hört den starken Ruf eines Nachtwächterhorns}

Ha!

\direct{Er hat mit emphatischer Gebärde die Hand an das Schwert gelegt und starrt wild vor sich hin}

\Evaspeaks

\direct{fasst ihn besänftigend bei der Hand}

Geliebter, spare den Zorn!
's war nur des Nachtwächters Horn.
Unter der Linde birg dich geschwinde;
hier kommt der Wächter vorbei.

\Magdalenespeaks

\direct{ruft leise unter der Tür}

Evchen! 's ist Zeit:
mach dich frei!

\Waltherspeaks
Du fliehst?

\Evaspeaks

\direct{lächelnd}

Muss ich denn nicht?

\Waltherspeaks
Entweichst?

\Evaspeaks

\direct{mit zarter Bestimmtheit}

Dem Meistergericht.

\direct{Sie verschwindet mit Magdalene im Hause}

\speaker{Der Nachtwächter}

\direct{ist währenddem in der Gasse erschienen, kommt singend nach vorn, biegt um die Ecke von Pogners Haus und geht nach links ab}

Hört, ihr Leut', und lasst euch sagen,
die Glock' hat zehn geschlagen:
bewahrt das Feuer und auch das Licht,
damit niemand kein Schad' geschicht!
Lobet Gott den Herrn!

\Sachsspeaks

\direct{welcher hinter der Ladentür dem Gespräche gelauscht, öffnet jetzt, bei eingezogenem Lampenlicht, ein wenig mehr}

Üble Dinge, die ich da merk':
eine Entführung gar im Werk!
Aufgepasst! Das darf nicht sein!

\Waltherspeaks

\direct{hinter der Linde}

Käm' sie nicht wieder? O der Pein!

\direct{Eva kommt in Magdalenes Kleidung aus dem Hause; die Gestalt gewahrend}

Doch ja, sie kommt dort!
Weh mir, nein! Die Alte ist's!

\direct{Eva erblickt Walther und eilt auf ihn zu}

Doch aber ja!

\Evaspeaks
Das tör'ge Kind:
da hast du's! Da!

\direct{Sie wirft sich ihm heiter an die Brust}


\Waltherspeaks

\direct{hingerissen}

O Himmel! Ja, nun wohl ich weiss,
dass ich gewann den Meisterpreis!

\Evaspeaks
Doch nun kein Besinnen! Von hinnen! Von hinnen!
o wären wir schon fort!

\Waltherspeaks
Hier durch die Gasse:
dort finden wir vor dem Tor Knecht und Rosse vor.

\direct{Nachtwächterhorn entfernt. Als sich beide wenden, um in die Gasse einzubiegen, lässt Sachs, nachdem er die Lampe hinter eine Glaskugel gestellt, durch die ganz wieder geöffnete Ladentür einen grellen Lichtschein quer über die Strasse fallen, so dass Eva und Walther sich plötzlich hell beleuchtet sehen.}


\Evaspeaks

\direct{Walther hastig zurückziehend}

O weh, der Schuster!
Wenn er uns säh'!
Birg dich! Komm ihm nicht in die Näh'!

\Waltherspeaks
Welch and'rer Weg führt uns hinaus?

\Evaspeaks
Dort durch die Strasse:
doch der ist kraus,
ich kenn' ihn nicht gut;
auch stiessen wir dort auf den Wächter.

\Waltherspeaks
Nun denn:
durch die Gasse!

\Evaspeaks
Der Schuster muss erst vom Fenster fort.

\Waltherspeaks
Ich zwing' ihn, dass er's verlasse.

\Evaspeaks
Zeig dich ihm nicht:
er kennt dich!

\Waltherspeaks
Der Schuster?

\Evaspeaks
's ist Sachs!

\Waltherspeaks
Hans Sachs? Mein Freund!

\Evaspeaks
Glaub's nicht! Von dir Übles zu sagen nur wusst' er.

\Waltherspeaks
Wie, Sachs? Auch er? Ich lösch' ihm das Licht.


\scene

\StageDir{Beckmesser ist, dem Nachtwächter nachschleichend, die Gasse heraufgekommen, hat nach den Fenstern von Pogners Haus gespäht und, an Sachsens Haus gelehnt, stimmt er jetzt seine mitgebrachte Laute.}

\Evaspeaks

\direct{Walther zurückhaltend}

Tu's nicht! Doch horch!

\Waltherspeaks
Einer Laute Klang.

\direct{Als Sachs den ersten Ton der Laute vernommen, hat er, von einem plötzlichen Einfall erfasst, das Licht wieder etwas eingezogen und öffnet leise den unteren Teil des Ladens}


\Evaspeaks
Ach, meine Not!

\Waltherspeaks
Wie, wird dir bang'?
Der Schuster, sieh, zog ein das Licht. So sei's gewagt!

\Evaspeaks
Weh! Siehst du denn nicht?
Ein and'rer kam und nahm dort Stand.

\direct{Sachs hat unvermerkt seinen Werktisch ganz unter die Tür gestellt Jetzt erlauscht er Evas Ausruf}


\Waltherspeaks
Ich hör's und seh's: ein Musikant.
Was will der hier so spät des Nachts?

\Evaspeaks

\direct{in Verzweiflung}

's ist Beckmesser schon!

\Sachsspeaks
Aha, ich dacht's!

\direct{Er setzt sich leise zur Arbeit zurecht}


\Waltherspeaks
Der Merker? Er in meiner Gewalt?
Drauf zu! Den Lung'rer mach' ich kalt!

\Evaspeaks
Um Gott! So hör! Willst den Vater wecken?

\direct{Er singt ein Lied, dann zieht er ab.}

Lass dort uns im Gebüsch verstecken.
Was mit den Männern ich Müh' doch hab!

\direct{Sie zieht Walther hinter das Gebüsch auf die Bank unter der Linde. Beckmesser, eifrig nach dem Fenster lugend, klimpert voll Ungeduld heftig auf der Laute. Als er sich endlich auch zum Singen rüstet, schlägt Sachs sehr stark mit dem Hammer auf den Leisten, nachdem er soeben das Licht wieder hell auf die Strasse hat fallen lassen.}


\Sachsspeaks

Jerum! Jerum! Hallo hallo he!
O ho! Trallalei! Trallalei! O ho!

\Beckmesserspeaks

\direct{springt ärgerlich von dem Steinsitz auf und gewahrt Sachs bei der Arbeit}

Was soll das sein?
Verdammtes Schrein!

\Sachsspeaks
Als Eva aus dem Paradies
von Gott dem Herrn verstossen,
gar schuf ihr Schmerz der harte Kies
an ihrem Fuss, dem blossen.

\Beckmesserspeaks
Was fällt dem groben Schuster ein?

\Sachsspeaks
Das jammerte den Herrn,

\Waltherspeaks

\direct{flüsternd zu Eva}

Was heisst das Lied? Wie nennt er dich?

\Sachsspeaks
ihr Füsschen hatt' er gern,

\Evaspeaks

\direct{flüsternd zu Walther}

Ich hört' es schon:
's geht nicht auf mich.

\Sachsspeaks
und seinem Engel rief er zu:

\Evaspeaks
Doch eine Bosheit steckt darin.

\Sachsspeaks
``Da, mach der armen Sünd'rin Schuh'!
Und da der Adam, wie ich seh',
an Steinen dort sich stösst die Zeh',
dass recht fortan er wandeln kann,
so miss dem auch Stiefeln an!''

\Waltherspeaks
Welch Zögernis! Die Zeit geht hin!

\Beckmesserspeaks

\direct{tritt zu Sachs heran}

Wie, Meister? Auf? Noch so spät zur Nacht?

\Sachsspeaks
Herr Stadtschreiber! Was, Ihr wacht?
Die Schuh' machen Euch grosse Sorgen?
Ihr seht, ich bin dran:
Ihr habt sie morgen.

\direct{Er arbeitet}


\Beckmesserspeaks

\direct{zornig}

Hol' der Teufel die Schuh'! Hier will ich Ruh'!

\Sachsspeaks
Jerum! Jerum!
Hallo hallo he!
Oho! Trallalei! Trallalei! O he!
O Eva, Eva! Schlimmes Weib,
das hast du am Gewissen,

\Waltherspeaks

\direct{zu Eva}

Uns oder dem Merker? Wem spielt er den Streich?

\Sachsspeaks
dass ob der Füss' am Menschenleib

\Evaspeaks

\direct{zu Walther}

Ich fürcht', uns dreien
gilt er gleich.

\Sachsspeaks
jetzt Engel schustern müssen.

\Evaspeaks
O weh der Pein.
Mir ahnt nichts Gutes!

\Sachsspeaks
Blieb'st du im Paradies, da gab es keinen Kies.

\Waltherspeaks
Mein süsser Engel, sei guten Mutes!

\Sachsspeaks
Um deiner jungen Missetat
hantier' ich jetzt mit Ahl' und Draht

\Evaspeaks
Mich betrübt das Lied!

\Waltherspeaks
Ich hör' es kaum!
Du bist bei mir,
welch holder Traum!

\direct{Er zieht sie zärtlich an sich}


\Sachsspeaks
und ob Herrn Adams übler Schwäch'
versohl' ich Schuh' und streiche Pech.
Wär' ich nicht fein ein Engel rein,
Teufel möchte Schuster sein!


\direct{Beckmesser drohend auf Sachs zufahrend}


\Sachsspeaks
Je!

\direct{Er unterbricht sich}


\Beckmesserspeaks
Gleich höret auf!
Spielt Ihr mir Streich'?
Bleibt Ihr tags und nachts Euch gleich?

\Sachsspeaks
Wenn ich hier sing', was kümmert's Euch?
Die Schuhe sollen doch fertig werden?

\Beckmesserspeaks
So schliesst Euch ein und schweigt dazu still!

\Sachsspeaks
Des Nachts arbeiten macht Beschwerden;
wenn ich da munter bleiben will,
so brauch' ich Luft und frischen Gesang;
drum hört, wie der dritte Vers gelang!

\direct{Er wichst den Draht ersichtlich}


\Beckmesserspeaks
Er macht mich rasend!

\Sachsspeaks

\direct{fortarbeitend}

Jerum! Jerum!
Hallo hallo he!

\Beckmesserspeaks
Das grobe Geschrei!

\Sachsspeaks
O ho! Trallalei! Trallalei! O he!

\Beckmesserspeaks
Am End' denkt sie gar, dass ich das sei!

\direct{Er hält sich die Ohren zu und geht verzweiflungsvoll, sich mit sich beratend, die Gasse vor dem Fenster auf und ab}


\Sachsspeaks
O Eva! Hör mein' Klageruf,
mein' Not und schwer Verdrüssen!
Die Kunstwerk', die ein Schuster schuf,
sie tritt die Welt mit Füssen!
Gäb' nicht ein Engel Trost,
der gleiches Werk erlost,
und rief' mich oft ins Paradies,
wie ich da Schuh' und Stiefel liess'!
Doch wenn mich der im Himmel hält,
dann liegt zu Füssen mir die Welt,
und bin in Ruh'
Hans Sachs: ein Schuhmacher und Poet dazu.

\Beckmesserspeaks
Das Fenster geht auf!

\direct{Er späht nach dem Fenster, welches jetzt leise geöffnet wird und an welchem vorsichtig Magdalene in Evas Kleidung sich zeigt.}


\Evaspeaks

\direct{mit grosser Aufgeregtheit}

Mich schmerzt das Lied, ich weiss nicht wie!
O fort, lass uns fliehen!

\Waltherspeaks

\direct{auffahrend}

Nun denn:
mit dem Schwert!

\Evaspeaks
Nicht doch! Ach, halt!

\Beckmesserspeaks
Herrgott, 's ist sie!

\Waltherspeaks

\direct{die Hand vom Schwert nehmend}

Kaum wär' er's wert!

\Evaspeaks
Ja, besser Geduld!

\Beckmesserspeaks

\direct{der, während Sachs fortfährt zu arbeiten und zu singen, in grosser Aufregung mit sich beraten hat}

Jetzt bin ich verloren, singt der noch fort!

\Evaspeaks
O bester Mann,
dass ich so Not dir machen kann!

\Beckmesserspeaks

\direct{tritt zu Sachs an den Laden heran und klimpert, während des Folgenden mit dem Rücken der Gasse zugewandt, seitwärts auf der Laute, um Magdalene am Fenster festzuhalten}

Freund Sachs! So hört doch nur ein Wort!

\Waltherspeaks

\direct{leise zu Eva}

Wer ist am Fenster?

\Beckmesserspeaks
Wie seid Ihr auf die Schuh' versessen!

\Evaspeaks
's ist Magdalene.

\Beckmesserspeaks
Ich hatt' sie wahrlich schon vergessen.

\Waltherspeaks
Das heiss' ich vergelten!

\Beckmesserspeaks
Als Schuster seid Ihr mir wohl wert,

\Waltherspeaks
Fast muss ich lachen.

\Beckmesserspeaks
als Kunstfreund doch weit mehr verehrt.

\Evaspeaks
Wie ich ein End' und Flucht mir ersehne!

\Waltherspeaks
Ich wünscht', er möchte den Anfang machen.

\direct{\Walther und \Eva, auf der Bank sanft aneinandergelehnt, erfolgen des weiteren \Sachs und \Beckmesser mit wachsender Teilnahme}


\Beckmesserspeaks
Eu'r Urteil, glaubt, das halt' ich hoch;
drum bitt' ich:
hört das Liedlein doch,
mit dem ich morgen möcht' gewinnen,
ob das auch recht nach Euren Sinnen.

\direct{Er klimpert wiederholt seitwärts nach dem Fenster gewandt}


\Sachsspeaks
Oha! Wollt mich beim Wahne fassen?
Mag mich nicht wieder schelten lassen.
``Seit sich der Schuster dünkt Poet,
gar übel es um Eu'r Schuhwerk steht.''
Ich seh', wie's schlappt und überall klappt:
drum lass ich Vers und Reim'
gar billig nun daheim,
Verstand und Witz und Kenntnis dazu,
mach' Euch für morgen die neuen Schuh'.

\Beckmesserspeaks

\direct{kreischend}

Lasst das doch sein! Das war ja nur Scherz.
Vernehmt besser, wie's mir ums Herz!
Vom Volk seid Ihr geehrt,
auch der Pognerin seid Ihr wert.
Will ich vor aller Welt
nun morgen um die werben,
sagt, könnt's mich nicht verderben,
wenn mein Lied ihr nicht gefällt?
Drum hört mich ruhig an;
und sang ich, sagt mir dann,
was Euch gefällt, was nicht,
dass ich mich danach richt'.

\direct{Er klimpert wieder}

\Sachsspeaks
Ei, lasst mich doch in Ruh'!
Wie käme solche Ehr' mir zu?
Nur Gassenhauer dicht' ich zum meisten,
drum sing' ich zur Gassen und hau' auf den Leisten.
Jerum! Jerum!
Hallo hallo he!

\Beckmesserspeaks
Verfluchter Kerl! Den Verstand verlier' ich
mit seinem Lied voll Pech und Schmierich!

\Sachsspeaks
O ho! Trallalei! Trallalei! O he!

\Beckmesserspeaks
Schweigt doch! Weckt Ihr die Nachbarn auf?

\Sachsspeaks
Die sind's gewohnt:
's hört keiner drauf.
``O Eva, Eva!''

\Beckmesserspeaks

\direct{in höchste Wut ausbrechend}

O Ihr boshafter Geselle!
Ihr spielt mir heut' den letzten Streich!
Schweigt Ihr jetzt nicht auf der Stelle,
so denkt Ihr dran, das schwör' ich Euch.

\direct{Er klimpert wütend}

Neidisch seid Ihr, nichts weiter,
dünkt Ihr Euch auch gleich gescheiter.
Dass andre auch was sind, ärgert Euch schändlich!
Glaubt, ich kenne Euch aus- und inwendlich!
Dass man Euch noch nicht zum Merker gewählt,
das ist's, was den gallichten Schuster quält.
Nun gut! Solang' als Beckmesser lebt
und ihm noch ein Reim an den Lippen klebt,
solang' ich noch bei den Meistern was gelt',
ob Nürnberg ``blüh' und wachs',''
das schwör' ich Herrn Hans Sachs:
nie wird er je zum Merker bestellt!

\direct{Er klimpert in höchster Wut}


\Sachsspeaks

\direct{der ihm ruhig und aufmerksam zugehört hat}

War das Eu'r Lied?

\Beckmesserspeaks
Der Teufel hol's!

\Sachsspeaks
Zwar wenig Regel:
doch klang's recht stolz!

\Beckmesserspeaks
Wollt Ihr mich hören?

\Sachsspeaks
In Gottes Namen
singt zu:
ich schlag' auf die Sohl' die Rahmen.

\Beckmesserspeaks
Doch schweigt Ihr still?

\Sachsspeaks
Ei, singet Ihr,
die Arbeit, schaut, fördert's auch mir.

\Beckmesserspeaks
Das verfluchte Klopfen wollt Ihr doch lassen?

\Sachsspeaks
Wie sollt' ich die Sohl' Euch richtig fassen?

\Beckmesserspeaks
Was? Ihr wollt klopfen, und ich soll singen?

\Sachsspeaks
Euch muss das Lied, mir der Schuh gelingen.

\Beckmesserspeaks
Ich mag keine Schuh'!

\Sachsspeaks
Das sagt Ihr jetzt;
in der Singschul' Ihr mir's dann wieder versetzt.
Doch hört! Vielleicht sich's richten lässt:
zwei-einig geht der Mensch am best.
Darf ich die Arbeit nicht entfernen,
die Kunst des Merkers möcht' ich erlernen.
Darin kommt Euch nun keiner gleich;
ich lern' sie nie, wenn nicht von Euch.
Drum singt Ihr nun, ich acht' und merk'
und fördr' auch wohl dabei mein Werk.

\Beckmesserspeaks
Merkt immer zu; und was nicht gewann,
nehmt Eure Kreide und streicht mir's an.

\Sachsspeaks
Nein, Herr! Da fleckten die Schuh' mir nicht,
mit dem Hammer auf den Leisten halt' ich Gericht.

\Beckmesserspeaks
Verdammte Bosheit! Gott, und 's wird spät:
am End' mir die Jungfer vom Fenster geht!

\direct{Er klimpert eifrig}

\Sachsspeaks

\direct{aufschlagend}

Fanget an! 's pressiert! Sonst sing' ich für mich!

\Beckmesserspeaks

Haltet ein! Nur das nicht! Teufel, wie ärgerlich!
Wollt Ihr Euch denn als Merker erdreisten,
nun gut, so merkt mit dem Hammer auf den Leisten;
nur mit dem Beding, nach den Regeln scharf,
aber nichts, was nach den Regeln ich darf.

\Sachsspeaks

Nach den Regeln, wie sie der Schuster kennt,
dem die Arbeit unter den Händen brennt.

\Beckmesserspeaks
Auf Meisterehr'?

\Sachsspeaks
Und Schustermut!

\Beckmesserspeaks
Nicht einen Fehler:
glatt und gut!

\direct{Nachtwächterhorn sehr entfernt}


\Sachsspeaks
Dann gingt Ihr morgen unbeschuht.

\Waltherspeaks

\direct{leise zu Eva}

Welch toller Spuk!
Mich dünkt's ein Traum.

\Sachsspeaks

\direct{auf den Steinsitz vor der Ladentür deutend}

Setzt Euch denn hier!

\Beckmesserspeaks

\direct{zieht sich nach der Ecke des Hauses zurück}

Lasst hier mich stehen!

\Waltherspeaks
den Singstuhl, scheint's, verliess ich kaum!

\Sachsspeaks
Warum so weit?

\Beckmesserspeaks
Euch nicht zu seh'n,
wie's Brauch der Schul' vor dem Gemerk'.

\Evaspeaks

\direct{sanft an Walthers Brust gelehnt}

Die Schläf' umwebt mir's wie ein Wahn:
ob's Heil, ob Unheil, was ich ahn'?

\Sachsspeaks
Da hör' ich Euch schlecht.

\Beckmesserspeaks
Der Stimme Stärk'
ich so gar lieblich dämpfen kann.

\direct{Er stellt sich ganz um die Ecke, dem Fenster gegenüber, auf}


\Sachsspeaks
Wie fein! Nun gut denn! Fanget an!

\direct{Beckmesser stimmt die in der Wut unversehens heraufgeschraubte D-Saite wieder herunter. Sachs holt mit dem Hammer aus.}

\Beckmesserspeaks

\direct{zur Laute}

``Den Tag seh' ich erscheinen,
der mir wohlgefall'n tut\ldots''

\direct{Sachs schlägt auf, Beckmesser schüttelt sich}

``\ldots Da fasst mein Herz sich einen \ldots''

\direct{Sachs schlägt auf, Beckmesser setzt heftig ab, singt aber weiter}

``\ldots guten und frischen \ldots''

\direct{Sachs hat aufgeschlagen, Beckmesser wendet sich wütend um die Ecke herum}

Treibt Ihr hier Scherz? Was wär' nicht gelungen?

\Sachsspeaks
Besser gesungen:
``Da fasst mein Herz sich einen guten, frischen''

\Beckmesserspeaks
Wie sollt' sich das reimen
auf ``Seh ich erscheinen''?

\Sachsspeaks
Ist Euch an der Weise nichts gelegen?
Mich dünkt, sollt' passen Ton und Wort.

\Beckmesserspeaks
Mit Euch zu streiten?
Lasst von den Schlägen,
sonst denkt Ihr mir dran!

\Sachsspeaks
Jetzt fahret fort!

\Beckmesserspeaks
Bin ganz verwirrt!

\Sachsspeaks
So fangt noch mal an:
drei Schläg' ich jetzt pausieren kann.

\Beckmesserspeaks

\direct{für sich}

Am besten, wenn ich ihn gar nicht beacht'.
Wenn's nur die Jungfer nicht irre macht!
Den Tag seh' ich erscheinen,
der mir wohl gefall'n tut;
da fasst mein Herz sich einen
guten und frischen Mut.
Da denk' ich nicht an Sterben,

\direct{Sachs schlägt}

lieber an Werben
um jung' Mägdeleins Hand.

\direct{Sachs schlägt}

Warum wohl aller Tage
schönster mag dieser sein?

\direct{Schlag. Ärgerlich}

Allen hier ich es sage:

\direct{Schlag}

weil ein schönes Fräulein

\direct{zwei Schläge}

von ihrem lieb'n Herrn Vater,

\direct{Sachs schlägt und nickt ironisch beifällig}

wie gelobt hat er,

\direct{viele kleine Schläge}

ist bestimmt zum Eh'stand.

\direct{Fünf Schläge. Sehr ärgerlich}

Wer sich getrau',

\direct{Schlag}

der komm' und schau',
da steh'n die hold lieblich' Jungfrau,

\direct{drei Schläge}

auf die ich all mein' Hoffnung bau':

\direct{Schlag}

darum ist der Tag so schön blau,

\direct{viele Schläge}

als ich anfänglich fand.

\direct{Er bricht wütend um die Ecke auf \Sachs los}


\Beckmesserspeaks
Sachs! Seht, Ihr bringt mich um!
Wollt Ihr jetzt schweigen?

\Sachsspeaks
Ich bin ja stumm!
Die Zeichen merkt' ich; wir sprechen dann:
derweil lassen die Sohlen sich an.

\Beckmesserspeaks

\direct{gewahrt, dass \Magdalene sich vom Fenster entfernen will}

Sie entweicht? Pst, pst! Herrgott! Ich muss!

\direct{Um die Ecke herum die Faust gegen Sachs ballend}

Sachs, Euch gedenk' ich die Ärgernuss!

\direct{Er macht sich zum zweiten Vers fertig}


\Sachsspeaks

\direct{mit dem Hammer nach dem Leisten ausholend}

Merker am Ort! Fahret fort!

\Beckmesserspeaks

\direct{immer stärker und atemloser}

Will heut' mir das Herz hüpfen,

\direct{Schlag}

werben um Fräulein jung,

\direct{drei Schläge}

doch tät' der Vater knüpfen

\direct{Schlag}

daran ein' Bedingung

\direct{drei Schläge}

für den, wer ihn beerben
will und auch werben

\direct{zwei Schläge}

um sein Kindelein fein.

\direct{viele Schläge}

Der Zunft ein bied'rer Meister
wohl sein' Tochter er liebt,

\direct{drei Schläge}

doch zugleich auch beweist er,

\direct{zwei Schläge}

was er auf die Kunst gibt:

\direct{ununterbrochene Schläge}

zum Preise muss es bringen
im Meistersingen,
wer sein Eidam will sein.

\direct{Er stampft wütend mit den Füssen}

Nun gilt es Kunst, dass mit Vergunst,
ohn' all schädlich gemeinen Dunst,

\direct{fortwährende Schläge}

ihm glücke des Preises Gewunst,
war begehrt mit wahrer Inbrunst,

\direct{Sachs, welcher kopfschüttelnd es aufgibt, die einzelnen Fehler anzumerken, arbeitet hämmernd fort, um den Keil aus dem Leisten zu schlagen}

um die Jungfrau zu frei'n.

\Sachsspeaks

\direct{über den Laden weit herausgelehnt}

Seid Ihr nun fertig?

\Beckmesserspeaks

\direct{in höchster Angst}

Wie fraget Ihr?

\Sachsspeaks

\direct{hält die fertigen Schuhe triumphierend heraus}

Mit den Schuhen ward ich fertig schier.

\direct{Während er die Schuhe an den Bändern hoch in der Luft tanzen lässt}

Das heiss ich mir echte Merkerschuh:
mein Merkersprüchlein hört dazu!


\direct{sehr kräftig}

Mit lang und kurzen Hieben
steht's auf der Sohl geschrieben:
da lest es klar
und nehmt es wahr,
und merkt's Euch immerdar.
Gut Lied will Takt:
wer den verzwackt,
dem Schreiber mit der Feder
haut ihn der Schuster aufs Leder.
Nun lauft in Ruh:
habt gute Schuh,
der Fuss Euch drin nicht knackt,
ihn hält die Sohl im Takt!

\Beckmesserspeaks

\direct{der sich ganz in die Gasse zurückgezogen hat und an die Mauer mit dem Rücken sich anlehnt, singt, um Sachs zu übertäuben, mit grösster Anstrengung, schreiend und atemlos hastig, während er die Laute wütend nach Sachs schwingt}

``Darf ich mich Meister nennen,
das bewähr ich heut gern,
weil ich nach dem Preis brennen
muss, dursten und hungern.
Nun ruf ich die neun Musen,
dass an sie blusen
mein dicht'rischen Verstand.
Wohl kenn ich alle Regeln,
halte gut Mass und Zahl;
doch Sprung und Überkegeln
wohl passiert je einmal,
wann der Kopf ganz voll Zagen
zu frei'n will wagen
um jung Mägdeleins Hand.
Er verschnauft sich
Ein Junggesell,
trug ich mein Fell,
mein Ehr, Amt, Würd und Brot zur Stell,
dass Euch mein Gesang wohl gefällt,
und mich das Jungfräulein erwähl,
wenn sie mein Lied gut fand.'' 

\Davidspeaks

\direct{hat den Fensterladen, dicht hinter \Beckmesser, ein wenig geöffnet und lugt daraus hervor}

Wer Teufel, hier?
Er wird Magdalene gewahr
Und drüben gar?
Die Lene ist's, ich seh es klar!
Herrje, der war's, den hat sie bestellt.
Der ist's, der ihr besser als ich gefällt!
Nun warte, du kriegst's!
Dir streich ich das Fell!


\direct{Er entfernt sich nach innen}


\speaker{Nachbarn}

\direct{erst einige, dann immer mehr, öffnen während Beckmessers Lied in der Gasse die Fenster und gucken heraus}

Was heult denn da?
Wer kreischt mit Macht?
Ist das erlaubt so spät zur Nacht?
Gebt Ruhe hier! 's ist Schlafenszeit.
Mein', hört nur, wie dort der Esel schreit!
Ihr da! Seid still und schert Euch fort!
Heult, kreischt und schreit an andrem Ort!

\StageDir{Sie verlassen die Fenster und kommen nach und nach in Nachtkleidern einzeln auf die Strasse heraus. \Sachs beobachtet noch eine Zeitlang den wachsenden Tumult, löscht aber alsbald sein Licht aus und schliesst den Laden so weit, dass er, ungesehen, stets durch eine kleine Öffnung den Platz unter der Linde beobachten kann. \Walther und \Eva sehen mit wachsender Sorge dem anschwellenden Auflaufe zu; er schliesst sie in seinen Mantel fest an sich und birgt sich hart an der Linde im Gebüsch, so dass beide fast ungesehen bleiben.}

\Davidspeaks

\direct{ist, mit einem Knüppel bewaffnet, zurückgekommen, steigt aus dem Fenster und wirft sich auf Beckmesser}

Zum Teufel mit dir, verdammter Kerl!

\Magdalenespeaks

\direct{winkt David heftig zurück. Am Fenster, schreiend}

Ach, Himmel! David! Gott, welche Not!
Zu Hilfe! Zu Hilfe!
Sie schlagen sich tot!

\Beckmesserspeaks

\direct{wehrt sich, will fliehen; David hält ihn am Kragen}

Verfluchter Bursch!
Lässt du mich los?

\Davidspeaks
Gewiss! Die Glieder brech ich dir bloss!


\direct{Beckmesser und David balgen sich fortwährend; bald verschwinden sie gänzlich, bald kommen sie wieder in den Vordergrund, immer Beckmesser auf der Flucht. David ihn einholend, festhaltend und prügelnd}


\speaker{Nachbarn}

\direct{an den Fenstern}

Seht nach! Springt zu!
Da würgen sich zwei!
Sie kommen herab.
's gibt Schlägerei!

\speaker{Andere Nachbarn}

\direct{in die Gasse laut schreiend}

Heda! Herbei! 's gibt Schlägerei:
da würgen sich zwei.
Ihr da, lasst los! Gebt freien Lauf!
Lasst ihr nicht los, wir schlagen drauf.

\speaker{Ein Nachbar}
Ei, seht, auch Ihr hier?
Geht's Euch was an?

\speaker{Ein Zweiter}
Was sucht Ihr hier?
Hat man Euch was getan?

\speaker{Erster Nachbar}
Euch kennt man gut.

\speaker{Zweiter Nachbar}
Euch noch viel besser.

\speaker{Erster Nachbar}
Wieso denn?

\speaker{Zweiter Nachbar}

\direct{zuschlagend}

Ei, so!

\Magdalenespeaks

\direct{hinabschreiend}

David! Beckmesser!

\speaker{Lehrbuben}

\direct{einzeln, dann mehr, von allen Seiten dazukommend}

Herbei! Herbei! 's gibt Keilerei!


\direct{Einige}

's sind die Schuster!


\direct{Andere}

Nein, 's sind die Schneider!

\speaker{Die Ersteren}
Die Trunkenbolde!

\speaker{Die Anderen}
Die Hungerleider!

\speaker{Die Nachbarn}

\direct{auf der Gasse durcheinander}

Euch gönnt ich's schon lange
Wird euch wohl bange?
Das für die Klage!
Seht euch vor, wenn ich schlage!
Hat euch die Frau gehetzt?
Schau, wie es Prügel setzt!
Seid ihr noch nicht gewitzt?
Nun, schlagt doch! Das sitzt!
Dass dich Halunken
gleich ein Donnerwetter träf!
Wartet, ihr Racker!
Massabzwacker! 
Esel! Dummrian! 
Du Grobian! 
Lümmel du! 
Drauf und zu!

\speaker{Lehrbuben}

\direct{kommen von allen Seiten dazu}

Kennt man die Schlosser nicht?
Die haben's sicher angericht't!
Ich glaub, die Schmiede werden's sein!
Die Schreiner seh ich dort beim Schein!
Hei! Schaut die Schäffler dort beim Tanz!
Dort seh die Bader ich im Glanz;
herbei zum Tanz!
Krämer finden sich zur Hand
mit Gerstenstang und Zuckerkand,
mit Pfeffer, Zimt, Muskatennuss,
sie riechen schön,
doch machen viel Verdruss;
sie riechen schön,
und bleiben gern vom Schuss.
Seht nur, der Has
hat überall die Nas!
Meinst du damit etwa mich?
Mein ich damit etwa dich?
Immer mehr heran!
Lustig, wacker! jetzt geht's erst recht an!
Hei, nun geht's Plauz! hast du nicht gesehn!
Hast's auf die Schnauz!
Ha! nun geht's: Krach! Hagelwetterschlag!
Wo es sitzt, da wächst nichts so bald nach!
Keilt euch wacker! Keiner weiche!
Haltet selbst Gesellen mutig stand!
Wer wich, 's wär wahrlich eine Schand!
Wacker drauf und dran!
Wir stehen alle wie ein Mann!
Wie ein Mann
stehn wir alle fest zur Keilerei!


\direct{Bereits prügeln sich Nachbarn und Lebrbuben fast allgemein durcheinander}


\speaker{Gesellen}

\direct{mit Knitteln bewaffnet, kommen von verschiedenen Seiten dazu}

Heda! Gesellen 'ran!
Dort wird mit Streit und Zank getan;
da gibt's gewiss noch Schlägerei;
Gesellen, haltet euch dabei!
's sind die Weber! 's sind die Gerber!
Die Preisverderber!
Dacht ich mir's doch gleich:
spielen immer Streich!
Dort den Metzger Klaus
kenn ich heraus!
's brennt manchem im Haus!
's ist morgen der Fünfte!
Zünfte heraus!
Hei, hier setzt's Prügel!
Schneider mit dem Bügel!
Gürtler! Spengler! Zinngiesser! 
Leimsieder! Lichtgiesser!
Tuchscherer! Leinweber!
Immer dran! Immer drauf!
Schert euch selber fort
und macht euch heim!
Immer drauf und dran!
jetzt gilt's, keiner weiche hier!
Zünfte! Zünfte! Heraus! 

\speaker{Die Meister und älteren Bürger kommen von verschiedenen Seiten dazu}

Was gibt's denn da für Zank und Streit?
Das tost ja weit und breit!
Gebt Ruh und schert
euch jeder gleich nach Hause heim,
sonst schlag ein Hageldonnerwetter drein!
Stemmt euch hier nicht mehr zu Hauf,
oder sonst wir schlagen drein!

\speaker{Nachbarinnen}

\direct{haben die Fenster geöffnet und gucken heraus}

Was ist das für Zanken und Streit?
Da gibt's gewiss noch Schlägerei!
Wär nur der Vater nicht dabei!
's wird einem wahrlich angst und bang!
Heda! Ihr dort unten,
so seid doch nur gescheit!
Seid ihr denn Alle gleich
zu Streit und Zank bereit?
Seid ihr alle blind und toll?
Sind euch vom Wein denn
noch die Köpfe voll?
Mein! Dort schlägt sich mein Mann!
Hilfe! Der Vater! Der Vater!
Ach, sie haun ihn tot!
Hört keines mehr sein Wort!
Gott, welche Not!
Seht dort den Christian;
er walkt den Peter ab!
Auf, schreit zu Hilfe: Mord und Zeter! 
Gott, wie sie walken!
Die Köpf und Zöpfe wackeln hin und her!
Schafft Wasser, Wasser her! Wasser her!
das giesst ihn' auf die Köpf herab!


\direct{Die Rauferei ist allgemein geworden, Schreien und Toben}


\Magdalenespeaks

\direct{am Fenster, verzweifelt die Hände ringend}

Ach Himmel! David! Gott! Welche Not!
Zu Hilfe! Zu Hilfe! Sie schlagen sich tot!

\direct{mit grösster Anstrengung}

Hör doch nur, David!
So lass doch nur den Herrn dort los,
er hat mir nichts getan!

\direct{hinabspähend}

So hör mich doch nur an!
Herrgott, er hält ihn noch!
Nein! David, ist er toll?
mit höchster Anstrengung
Ach, David, hör:
's ist Herr Beckmesser!

\Pognerspeaks

\direct{ist im Nachtgewand oben an das Fenster getreten}

Um Gott! Eva! Schliess zu!
Ich seh, ob unt' im Hause Ruh!


\direct{Er zieht Magdalenen, welche jammernd die Hände nach der Gasse hinab gerungen, herein und schliesst das Fenster}


\Waltherspeaks

\direct{der bisher mit Eva sich hinter dem Gebüsch verborgen, fasst jetzt Eva dicht in den linken Arm und zieht mit der rechten Hand das Schwert}

Jetzt gilt's zu wagen,
sich durchzuschlagen!


\direct{Er dringt mit geschwungenem Schwert bis in die Mitte der Bühne vor, um sich mit Eva durch die Gasse durchzuhauen. Da springt Sachs mit einem kräftigen Satze aus dem Laden, bahnt sich mit geschwungenem Knieriemen den Weg bis zu Walther und packt diesen beim Arm.}


\Pognerspeaks

\direct{auf der Treppe}

He! Lene! Wo bist du?

\Sachsspeaks

\direct{die halb ohnmächtige Eva die Treppe hinaufstossend}

Ins Haus, Jungfer Lene!

\StageDir{\Pogner empfängt \Eva und zieht sie in das Haus. \Sachs, mit einem Knieriemen \David eines überhauend und mit einem Fusstritt ihn voran in den Laden stossend, zieht \Walther, den er mit der andren Hand fest gefasst hält, mit sich hinein und schliesst sogleich fest hinter sich zu. \Beckmesser, durch \Sachs von \David befreit, sucht sich eilig durch die Menge zu flüchten. Im gleichen Augenblick, wo Sachs auf die Strasse sprang, hörte man einen Hornruf des Nachtwächters. Alle suchen in eiliger Flucht nach allen Seiten hin das Weite, so dass die Bühne sehr bald gänzlich leer wird. Als die Strasse und Gasse leer geworden und alle Häuser geschlossen sind, betritt der Nachtwächter die Bühne, reibt sich die Augen, siebt sich verwundert um und schüttelt den Kopf.}


\speaker{Der Nachtwächter}

\direct{mit leise bebender Stimme}

Hört, ihr Leut,
und lasst euch sagen,
die Glock hat
eilfe geschlagen:
bewahrt euch vor Gespenstern und Spuk,
dass kein böser Geist eu'r Seel beruck!
Lobet Gott, den Herrn!


\StageDir{Hornruf. Der Vollmond tritt hervor und scheint hell in die Gasse hinein; der Nachtwächter schreitet langsam dieselbe hinab. Als der Nachtwächter um die Ecke biegt, fällt der Vorhang, genau mit dem letzten Takte.}

\act

\scene

\StageDir{In Sachs' Werkstatt. Kurzer Raum. Im Hintergrund die halb geöffnete Ladentür, nach der Strasse führend. Rechts zur Seite eine Kammertür. Links das nach der Gasse gehende Fenster, mit Blumenstöcken davor, zur Seite ein Werktisch. \Sachs sitzt auf einem grossen Lehnstuhle an diesem Fenster, durch welches die Morgensonne hell auf ihn hereinscheint: Er hat vor sich auf dem Schosse einen grossen Folianten und ist im Lesen vertieft. \David zeigt sich, von der Strasse kommend, unter der Ladentür, er lugt herein, und da er \Sachs gewahrt, fährt er zurück. Er versichert sich aber, dass \Sachs ihn nicht bemerkt, schlüpft herein, stellt seinen mitgebrachten Korb auf den hinteren Werktisch beim Laden und untersucht seinen Inhalt: er holt Blumen und Bänder und kramt sie auf dem Tische aus, endlich findet er auf dem Grunde eine Wurst und einen Kuchen und lässt sich sogleich an, diese zu verzehren, als \Sachs, der ihn fortwährend nicht beachtet, mit starkem Geräusch eines der grossen Blätter des Folianten umwendet.}

\Davidspeaks

\direct{fährt zusammen, verbirgt das Essen und wendet sich zurück}

Gleich, Meister! Hier!
Die Schuh' sind abgegeben
in Herrn Beckmessers Quartier.
Mir war's, als rieft Ihr mich eben?

\direct{beiseite}

Er tut, als säh' er mich nicht?
Da ist er bös', wenn er nicht spricht!

\direct{Er nähert sich sehr demütig langsam Sachs}

Ach, Meister, wollt mir verzeih'n!
Kann ein Lehrbub' vollkommen sein?
Kenntet Ihr die Lene wie ich,
dann vergäbt Ihr mir sicherlich.
Sie ist so gut, so sanft für mich
und blickt mich oft an so innerlich.
Wenn Ihr mich schlagt, streichelt sie mich
und lächelt dabei holdseliglich.
Muss ich karieren, füttert sie mich
und ist in allem gar liebelich.
Nur gestern, weil der Junker versungen,
hab ich den Korb ihr nicht abgerungen.
Das schmerzte mich; und da ich fand,
dass nachts einer vor dem Fenster stand
und sang zu ihr und schrie wie toll,
da hieb ich ihm den Buckel voll.
Wie käm' nun da was Grosses drauf an?
Auch hat's uns'rer Liebe gar wohl getan.
Die Lene hat mir eben alles erklärt
und zum Fest Blumen und Bänder beschert.

\direct{Er bricht in grössere Angst aus}

Ach, Meister, sprecht doch nur ein Wort!

\direct{beiseite}

Hätt' ich nur die Wurst und den Kuchen erst fort!

\Sachsspeaks

\direct{hat unbeirrt immer weitergelesen. Jetzt schlägt er den Folianten zu. Von dem Geräusch erschrickt David so, dass er strauchelt und unwillkürlich vor Sachs auf die Knie fällt. Sachs sieht über das Buch, das er noch auf dem Schosse behält, hinweg, über David, welcher immer auf den Knien furchtsam nach ihm aufblickt, hin und heftet seinen Blick unwillkürlich auf den hinteren Werktisch. Sehr leise}

Blumen und Bänder seh' ich dort!
Schaut hold und jugendlich aus!
Wie kamen mir die ins Haus?

\Davidspeaks

\direct{verwundert über Sachs' Freundlichkeit}

Ei, Meister! 's ist heut festlicher Tag;
da putzt sich jeder, so schön er mag.

\Sachsspeaks

\direct{immer leise, wie für sich}

Wär' heut Hochzeitsfest?

\Davidspeaks
Ja, käm's erst so weit, dass David die Lene freit!

\Sachsspeaks

\direct{immer wie zuvor}

's war Polterabend, dünkt mich doch?

\Davidspeaks

\direct{für sich}

Polterabend? Da krieg' ich's wohl noch?

\direct{laut}

Verzeiht das, Meister! Ich bitt', vergesst!
Wir feiern ja heut' Johannisfest.

\Sachsspeaks
Johannisfest?

\Davidspeaks

\direct{beiseite}

Hört er heut' schwer?

\Sachsspeaks
Kannst du dein Sprüchlein? Sag es her!

\Davidspeaks

\direct{ist allmählich zu stehen gekommen}

Mein Sprüchlein? Denk', ich kann es gut.

\direct{beiseite}

's setzt nichts! Der Meister ist wohlgemut! -

\direct{stark und grob}

``Am Jordan Sankt Johannes stand!''

\Sachsspeaks
Wa---was?

\Davidspeaks

\direct{lächelnd}

Verzeiht, das Gewirr! Mich machte der Polterabend irr.

\direct{Er sammelt sich und stellt sich gehörig auf}

``Am Jordan Sankt Johannes stand,
all' Volk der Welt zu taufen;
kam auch ein Weib aus fernem Land,
von Nürnberg gar gelaufen;
sein Söhnlein trug's zum Uferrand,
empfing da Tauf' und Namen;
doch als sie dann sich heimgewandt,
nach Nürnberg wieder kamen,
in deutschem Land gar bald sich fand's,
dass wer am Ufer des Jordans
Johannes war genannt,
an der Pegnitz hiess der Hans.''

\direct{sich besinnend}

Hans? Hans!
Herr! Meister!

\direct{feurig}

's ist heut Eu'r Namenstag!
Nein! Wie man so was vergessen mag!
Hier! Hier, die Blumen sind für Euch,
die Bänder, und was nur alles noch gleich?
Ja, hier schaut! Meister, herrlicher Kuchen!
Möchtet Ihr nicht auch die Wurst versuchen?

\Sachsspeaks

\direct{immer ruhig, ohne seine Stellung zu verändern}

Schön Dank, mein Jung', behalt's für dich!
Doch heut auf die Wiese begleitest du mich.
Mit Blumen und Bändern putz' dich fein;
sollst mein stattlicher Herold sein.

\Davidspeaks
Sollt' ich nicht lieber Brautführer sein?
Meister, ach Meister! Ihr müsst wieder frein!

\Sachsspeaks
Hätt'st wohl gern eine Meist'rin im Haus?

\Davidspeaks
Ich mein', es säh' doch viel stattlicher aus.

\Sachsspeaks
Wer weiss! Kommt Zeit, kommt Rat.

\Davidspeaks
's ist Zeit!

\Sachsspeaks
Dann wär' der Rat wohl auch nicht weit?

\Davidspeaks
Gewiss! Gehn schon Reden hin und wieder,
den Beckmesser, denk' ich, sängt Ihr doch nieder?
Ich mein', dass der heut' sich nicht wichtig macht.

\Sachsspeaks
Wohl möglich! Hab mir's auch schon bedacht.
Jetzt geh' und stör' mir den Junker nicht!
Komm wieder, wenn du schön gericht't.

\Davidspeaks

\direct{küsst Sachs gerührt die Hand}

So war er noch nie, wenn sonst auch gut!
Kann mir gar nicht mehr denken, wie der Knieriemen tut!

\direct{Er packt alles zusammen und geht in die Kammer ab}


\Sachsspeaks

\direct{immer noch den Folianten auf dem Schosse, lehnt sich, mit untergestütztem Arme, sinnend darauf; es scheint, dass ihn das Gespräch mit David gar nicht aus seinem Nachdenken gestört hat}

Wahn! Wahn! Überall Wahn!
Wohin ich forschend blick'
in Stadt- und Weltchronik,
den Grund mir aufzufinden,
warum gar bis aufs Blut
die Leut' sich quälen und schinden
in unnütz toller Wut!
Hat keiner Lohn noch Dank davon:
in Flucht geschlagen, wähnt er zu jagen.
Hört nicht sein eigen Schmerzgekreisch,
wenn er sich wühlt ins eig'ne Fleisch,
wähnt Lust sich zu erzeigen.
Wer gibt den Namen an?

\direct{kräftig}

's ist halt der alte Wahn,
ohn' den nichts mag geschehen,
's mag gehen oder stehen!
Steht's wo im Lauf,
er schläft nur neue Kraft sich an;
gleich wacht er auf,
dann schaut, wer ihn bemeistern kann!
Wie friedsam treuer Sitten
getrost in Tat und Werk,
liegt nicht in Deutschlands Mitten
mein liebes Nürenberg!

\direct{Er blickt mit freudiger Begeisterung ruhig vor sich hin}

Doch eines Abends spat,
ein Unglück zu verhüten,
bei jugendheissen Gemüten,
ein Mann weiss sich nicht Rat;
ein Schuster in seinem Laden
zieht an des Wahnes Faden.
Wie bald auf Gassen und Strassen
fängt der da an zu rasen!
Mann, Weib, Gesell und Kind
fällt sich da an wie toll und blind;
und will's der Wahn gesegnen,
nun muss es Prügel regnen,
mit Hieben, Stoss' und Dreschen
den Wutesbrand zu löschen.
Gott weiss, wie das geschah?
Ein Kobold half wohl da!
Ein Glühwurm fand sein Weibchen nicht;
der hat den Schaden angericht't.
Der Flieder war's:
Johannisnacht.
Nun aber kam Johannistag!
Jetzt schau'n wir, wie Hans Sachs es macht,
dass er den Wahn fein lenken kann,
ein edler' Werk zu tun.
Denn lässt er uns nicht ruh'n
selbst hier in Nürenberg,
so sei's um solche Werk',
die selten vor gemeinen Dingen
und nie ohn' ein'gen Wahn gelingen.


\scene

\StageDir{Walther tritt unter der Kammertür ein. Er bleibt einen Augenblick dort stehen und blickt auf Sachs. Dieser wendet sich und lässt den Folianten auf den Boden gleiten.}

\Sachsspeaks
Grüss Gott, mein Junker! Ruhtet Ihr noch?
Ihr wachtet lang: nun schlieft Ihr doch?

\Waltherspeaks

\direct{sehr ruhig}

Ein wenig, aber fest und gut.

\Sachsspeaks
So ist Euch nun wohl bass zumut?

\Waltherspeaks

\direct{immer sehr ruhig}

Ich hatt' einen wunderschönen Traum.

\Sachsspeaks
Das deutet Gut's! Erzählt mir den.

\Waltherspeaks
Ihn selbst zu denken wag' ich kaum;
ich fürcht' ihn mir vergeh'n zu sehn.

\Sachsspeaks
Mein Freund, das grad' ist Dichters Werk,
dass er sein Träumen deut' und merk'.
Glaubt mir, des Menschen wahrster Wahn
wird ihm im Traume aufgetan:
all Dichtkunst und Poeterei
ist nichts als Wahrtraumdeuterei.
Was gilt's, es gab der Traum Euch ein,
wie heut' Ihr sollet Meister sein?

\Waltherspeaks

\direct{sehr ruhig}

Nein, von der Zunft und ihren Meistern
wollt' sich mein Traumbild nicht begeistern.

\Sachsspeaks
Doch lehrt' es wohl den Zauberspruch,
mit dem Ihr sie gewännet?

\Waltherspeaks

\direct{etwas lebhafter}

Wie wähnt Ihr doch nach solchem Bruch,
wenn Ihr noch Hoffnung kennet!

\Sachsspeaks
Die Hoffnung lass ich mir nicht mindern,
nichts stiess sie noch über'n Haufen.
Wär's nicht, glaubt, statt Eure Flucht zu hindern,
wär' ich selbst mit Euch fortgelaufen!
Drum bitt ich, lasst den Groll jetzt ruh'n;
Ihr habt's mit Ehrenmännern zu tun,
die irren sich und sind bequem,
dass man auf ihre Weise sie nähm'.
Wer Preise erkennt und Preise stellt,
der will am End' auch, dass man ihm gefällt.
Eu'r Lied, das hat ihnen bang gemacht;
und das mit Recht:
denn wohlbedacht,
mit solchem Dicht'- und Liebesfeuer
verführt man wohl Töchter zum Abenteuer;
doch für liebseligen Ehestand
man andre Wort' und Weisen fand.

\Waltherspeaks

\direct{lächelnd}

Die kenn' ich nun auch seit dieser Nacht:
es hat viel Lärm auf der Gasse gemacht.

\Sachsspeaks

\direct{lachend}

Ja, ja! Schon gut! Den Takt dazu
hörtet Ihr auch! Doch, lasst dem Ruh'
und folgt meinem Rate, kurz und gut,
fasst zu einem Meisterliede Mut.

\Waltherspeaks
Ein schönes Lied, ein Meisterlied,
wie fass ich da den Unterschied?

\Sachsspeaks

\direct{zart}

Mein Freund! In holder Jugendzeit,
wenn uns von mächt'gen Trieben
zum sel'gen ersten Lieben
die Brust sich schwellet hoch und weit,
ein schönes Lied zu singen
mocht' vielen da gelingen:
der Lenz, der sang für sie.
Kam Sommer, Herbst und Winterzeit,
viel Not und Sorg' im Leben,
manch ehlich Glück daneben,
Kindtauf', Geschäfte, Zwist und Streit:
denen's dann noch will gelingen,
ein schönes Lied zu singen,
seht, Meister nennt man die.

\Waltherspeaks
Ich lieb' ein Weib und will es frein,
mein dauernd Ehgemahl zu sein.

\Sachsspeaks
Die Meisterregeln lernt beizeiten,
dass sie getreulich Euch geleiten
und helfen wohl bewahren,
was in der Jugend Jahren
mit holdem Triebe Lenz und Liebe
Euch unbewusst ins Herz gelegt,
dass Ihr das unverloren hegt.

\Waltherspeaks
Stehn sie nun in so hohem Ruf,
wer war es, der die Regeln schuf?

\Sachsspeaks
Das waren hochbedürft'ge Meister,
von Lebensmüh' bedrängte Geister;
in ihrer Nöten Wildnis
sie schufen sich ein Bildnis,
dass ihnen bliebe der Jugendliebe
ein Angedenken klar und fest,
dran sich der Lenz erkennen lässt.

\Waltherspeaks
Doch, wem der Lenz schon lang entronnen,
wie wird er dem im Bild gewonnen?

\Sachsspeaks
Er frischt es an, so oft er kann!
Drum möcht' ich, als bedürft'ger Mann,
will ich die Regeln Euch lehren,
sollt Ihr sie mir neu erklären.
Seht, hier ist Tinte, Feder, Papier:
ich schreib's Euch auf, diktiert Ihr mir!

\Waltherspeaks
Wie ich's begänne, wüsst' ich kaum.

\Sachsspeaks
Erzählt mir Euren Morgentraum!

\Waltherspeaks
Durch Eurer Regeln gute Lehr'
ist mir's, als ob verwischt er wär'.

\Sachsspeaks
Grad' nehmt die Dichtkunst jetzt zur Hand;
mancher durch sie das Verlorene fand.

\Waltherspeaks
So wär's nicht Traum, doch Dichterei?

\Sachsspeaks
's sind Freunde beid', steh'n gern sich bei.

\Waltherspeaks
Wie fang' ich nach der Regel an?

\Sachsspeaks
Ihr stellt sie selbst und folgt ihr dann.
Gedenkt des schönen Traums am Morgen;
fürs and're lasst Hans Sachs nur sorgen!

\Waltherspeaks
\direct{hat sich zu Sachs am Werktisch gesetzt, wo dieser das Gedicht Walthers nachschreibt. Er beginnt sehr leise, wie heimlich}

``Morgenlich leuchtend in rosigem Schein,
von Blüt' und Duft geschwellt die Luft,
voll aller Wonnen, nie ersonnen,
ein Garten lud mich ein, Gast ihm zu sein.''

\Sachsspeaks
Das war ein Stollen:
nun achtet wohl,
dass ganz ein gleicher ihm folgen soll.

\Waltherspeaks
Warum ganz gleich?

\Sachsspeaks
Damit man seh',
Ihr wähltet Euch gleich ein Weib zur Eh'.

\Waltherspeaks
``Wonnig entragend dem seligen Raum
bot goldner Frucht heilsaft'ge Wucht
mit holdem Prangen dem Verlangen
an duft'ger Zweige Saum herrlich ein Baum.''

\Sachsspeaks
Ihr schlosset nicht im gleichen Ton.
Das macht den Meistern Pein;
doch nimmt Hans Sachs die Lehr' davon,
im Lenz wohl müss' es so sein.
Nun stellt mir einen Abgesang.

\Waltherspeaks
Was soll nun der?

\Sachsspeaks
Ob Euch gelang,
ein rechtes Paar zu finden,
das zeigt sich jetzt an den Kinden.
Den Stollen ähnlich, doch nicht gleich,
an eig'nen Reim' und Tönen reich;
dass man's recht schlank und selbstig find',
das freut die Eltern an dem Kind,
und Euren Stollen gibt's den Schluss,
dass nichts davon abfallen muss.

\Waltherspeaks
``Sei Euch vertraut,
welch hehres Wunder mir gescheh'n:
an meiner Seite stand ein Weib,
so hold und schön ich nie geseh'n;
gleich einer Braut
umfasste sie sanft meinen Leib;
mit Augen winkend,
die Hand wies blinkend,
was ich verlangend begehrt,
die Frucht so hold und wert
vom Lebensbaum.''

\Sachsspeaks

\direct{gerührt}

Das nenn' ich mir einen Abgesang!
Seht, wie der ganze Bar gelang.
Nur mit der Melodei seid Ihr ein wenig frei;
doch sag' ich nicht, dass das ein Fehler sei;
nur ist's nicht leicht zu behalten,
und das ärgert uns're Alten!
Jetzt richtet mir noch einen zweiten Bar,
damit man merk', welch' der erste war.
Auch weiss ich noch nicht, so gut Ihr's gereimt,
was Ihr gedichtet, was Ihr geträumt.

\Waltherspeaks
``Abendlich glühend in himmlischer Pracht
verschied der Tag, wie dort ich lag;
aus ihren Augen Wonne zu saugen,
Verlangen einz'ger Macht in mir nur wacht'.
Nächtlich umdämmert der Blick mir sich bricht!
Wie weit so nah' beschienen da
zwei lichte Sterne aus der Ferne
durch schlanker Zweige Licht hehr mein Gesicht.
Lieblich ein Quell
auf stiller Höhe dort mir rauscht;
jetzt schwellt er an sein hold' Getön',
so stark und süss ich's nie erlauscht:
leuchtend und hell, wie strahlten die Sterne da schön;
zu Tanz und Reigen in Laub und Zweigen
der gold'nen sammeln sich mehr,
statt Frucht ein Sternenheer
im Lorbeerbaum.''

\Sachsspeaks

\direct{sehr gerührt}

Freund!
Euer Traumbild wies Euch wahr;
gelungen ist auch der zweite Bar.
Wolltet Ihr noch einen dritten dichten?
Des Traumes Deutung würd' er berichten.

\Waltherspeaks

\direct{steht schnell auf}

Wo fänd' ich die? Genug der Wort'!

\Sachsspeaks

\direct{erhebt sich gleichfalls und tritt mit freundlicher Entschiedenheit zu Walther}

Dann Tat und Wort am rechten Ort!
Drum bitt' ich, merkt mir wohl die Weise:
gar lieblich drin sich's dichten lässt:
und singt Ihr sie im weit'ren Kreise,
so haltet mir auch das Traumbild fest.

\Waltherspeaks
Was habt Ihr vor?

\Sachsspeaks
Eu'r treuer Knecht
fand sich mit Sack und Tasch' zurecht;
die Kleider, drin am Hochzeitfest
daheim Ihr wolltet prangen,
die liess er her zu mir gelangen.
Ein Täubchen zeigt' ihm wohl das Nest,
darin sein Junker träumt!
Drum folgt mir jetzt ins Kämmerlein!
Mit Kleiden, wohlgesäumt,
sollen beide wir gezieret sein,
wenn's Stattliches zu wagen gilt.
Drum kommt, seid Ihr gleich mir gesinnt.

\StageDir{Walther schlägt in Sachsens Hand ein; so geleitet ihn dieser ruhig festen Schrittes zur Kammer, deren Tür er ihm ehrerbietig öffnet und dann ihm folgt.}


\scene

\StageDir{\Beckmesser. \Sachs. Man gewahrt \Beckmesser, welcher draussen vor dem Laden erscheint, in grosser Aufregung hereinlugt und, da er die Werkstatt leer findet, hastig eintritt Er ist reich aufgeputzt, aber in sehr leidendem Zustande. Er blickt sich erst unter der Tür nochmals genau in der Werkstatt um, dann hinkt er vorwärts, zuckt aber zusammen und streicht sich den Rücken. Er macht wieder einige Schritte, knickt aber mit den Knien und streicht nun diese. Er setzt sich auf den Schusterschemel, fährt aber schnell schmerzhaft wieder auf. Er betrachtet sich den Schemel und gerät dabei in immer aufgeregteres Nachsinnen. Er wird von den verdriesslichsten Erinnerungen und Vorstellungen gepeinigt; immer unruhiger beginnt er sich den Schweiss von der Stirne zu wischen. Er hinkt immer lebhafter umher und starrt dabei vor sich hin. Als ob er von allen Seiten verfolgt wäre, taumelt er fliehend hin und her. Wie um nicht umzusinken, hält er sich an dem Werktisch, zu dem er hin geschwankt war, an und starrt vor sich hin. Matt und verzweiflungsvoll sieht er um sich; sein Blick fällt endlich durch das Fenster auf Pogners Haus; er hinkt mühsam an dasselbe heran, und, nach dem gegenüberliegenden Fenster ausspähend, versucht er, sich in die Brust zu werfen, als ihm sogleich der Ritter Walther einfällt. Ärgerliche Gedanken entstehen dadurch, gegen die er mit schmeichelndem Selbstgefühl anzukämpfen sucht. Die Eifersucht übermannt ihn; er schlägt sich vor den Kopf. Er glaubt die Verhöhnung der Weiber und Buben auf der Gasse zu vernehmen, wendet sich wütend ab und schmeisst das Fenster zu. Sehr verstört wendet er sich mechanisch wieder dem Werktische zu, indem er vor sich hinbrütend nach einer neuen Weise zu suchen scheint. Sein Blick fällt auf das von \Sachs zuvor beschriebene Papier; er nimmt es neugierig auf, überfliegt es mit wachsender Aufregung und bricht endlich wütend aus.}

\Beckmesserspeaks
Ein Werbelied! Von Sachs! Ist's wahr?
Ha! Jetzt wird mir alles klar!

\direct{Da er die Kammertür gehen hört, fährt er zusammen und steckt das Papier eilig in die Tasche}

\Sachsspeaks

\direct{im Festgewande, tritt ein, kommt vor und hält an, als er \Beckmesser gewahrt}

Sieh da, Herr Schreiber! Auch am Morgen?
Euch machen die Schuh' doch nicht mehr Sorgen?

\Beckmesserspeaks
Zum Teufel! So dünn war ich noch nie beschuht!
Fühl' durch die Sohl' den kleinsten Kies!

\Sachsspeaks
Mein Merkersprüchlein wirkte dies,
trieb sie mit Merkerzeichen so weich.

\Beckmesserspeaks
Schon gut der Witz! Und genug der Streich'!
Glaubt mir, Freund Sachs, jetzt kenn' ich Euch!
Der Spass von dieser Nacht, der wird Euch noch gedacht.
Dass ich Euch nur nicht im Wege sei,
schuft Ihr gar Aufruhr und Meuterei!

\Sachsspeaks
's war Polterabend, lasst Euch bedeuten;
Eure Hochzeit spukte unter den Leuten:
je toller es da hergeh',
je besser bekommt's der Eh'.

\Beckmesserspeaks

\direct{wütend}

O Schuster, voll von Ränken
und pöbelhaften Schwänken,
du warst mein Feind von je:
nun hör, ob hell ich seh'!
Die ich mir auserkoren,
die ganz für mich geboren,
zu aller Witwer Schmach,
der Jungfer stellst du nach.
Dass sich Herr Sachs erwerbe
des Goldschmieds reiches Erbe,
im Meisterrat zur Hand
auf Klauseln er bestand,
ein Mägdlein zu betören,
das nur auf ihn sollt' hören
und, andern abgewandt,
zu ihm allein sich fand.
Darum! Darum!
Wär' ich so dumm?
Mit Schreien und mit Klopfen
wollt' er mein Lied zustopfen,
dass nicht dem Kind werd' kund,
wie auch ein and'rer bestund!
Jaja! Haha! Hab ich dich da?
Aus seiner Schusterstuben
hetzt' endlich er den Buben
mit Knüppeln auf mich her,
dass meiner los er wär'!
Au au! Au au! Wohl grün und blau,
zum Spott der allerliebsten Frau,
zerschlagen und zerprügelt,
dass kein Schneider mich aufbügelt!
Gar auf mein Leben war's angegeben!
Doch kam ich noch so davon,
dass ich die Tat Euch lohn'!
Zieht heut' nur aus zum Singen,
merkt auf, wie's mag gelingen;
bin ich gezwackt auch und zerhackt,
Euch bring' ich doch sicher aus dem Takt!

\Sachsspeaks
Gut Freund, Ihr seid in argem Wahn!
Glaubt, was Ihr wollt, dass ich getan,
gebt Eure Eifersucht nur hin;
zu werben kommt mir nicht in Sinn.

\Beckmesserspeaks
Lug und Trug! Ich kenn' es besser.

\Sachsspeaks
Was fällt Euch nur ein, Meister Beckmesser?
Was ich sonst im Sinn, geht Euch nichts an.
Doch glaubt, ob der Werbung seid Ihr im Wahn.

\Beckmesserspeaks
Ihr sängt heut nicht?

\Sachsspeaks
Nicht zur Wette.

\Beckmesserspeaks
Kein Werbelied?

\Sachsspeaks
Gewisslich, nein!

\Beckmesserspeaks
Wenn ich aber drob ein Zeugnis hätte?

\direct{Er greift in die Tasche}


\Sachsspeaks

\direct{blickt auf den Werktisch}

Das Gedicht? Hier liess ich's. Stecktet Ihr's ein?

\Beckmesserspeaks

\direct{das Blatt hervorziehend}

Ist das Eure Hand?

\Sachsspeaks
Ja. War es das?

\Beckmesserspeaks
Ganz frisch noch die Schrift?

\Sachsspeaks
Und die Tinte noch nass!

\Beckmesserspeaks
's wär' wohl gar ein biblisches Lied?

\Sachsspeaks
Der fehlte wohl, wer darauf riet.

\Beckmesserspeaks
Nun denn?

\Sachsspeaks
Wie doch?

\Beckmesserspeaks
Ihr fragt?

\Sachsspeaks
Was noch?

\Beckmesserspeaks
Dass Ihr mit aller Biederkeit
der ärgste aller Spitzbuben seid!

\Sachsspeaks
Mag sein! Doch hab ich noch nie entwandt,
was ich auf fremden Tischen fand;
und dass man von Euch auch nicht Übles denkt,
behaltet das Blatt, es sei Euch geschenkt.

\Beckmesserspeaks
in freudigem Schreck aufspringend
Herrgott! 
Ein Gedicht? Ein Gedicht von Sachs!
Doch halt, dass kein neuer Schad' mir erwachs'!
Ihr habt's wohl schon recht gut memoriert?

\Sachsspeaks
Seid meinethalb doch nur unbeirrt!

\Beckmesserspeaks
Ihr lasst mir das Blatt?

\Sachsspeaks
Damit Ihr kein Dieb.

\Beckmesserspeaks
Und mach' ich Gebrauch?

\Sachsspeaks
Wie's Euch belieb'.

\Beckmesserspeaks
Doch sing' ich das Lied?

\Sachsspeaks
Wenn's nicht zu schwer!

\Beckmesserspeaks
Und wenn ich gefiel'?

\Sachsspeaks
Das wunderte mich sehr!

\Beckmesserspeaks

\direct{ganz zutraulich}

Da seid Ihr nun wieder zu bescheiden:
ein Lied von Sachs,
gleichsam pfeifend
das will was bedeuten!
Und seht nur, wie mir's ergeht,
wie's mit mir Ärmsten steht!
Erseh' ich doch mit Schmerzen,
das Lied, das nachts ich sang---
dank Euren lust'gen Scherzen!---
es machte der Pognerin bang'.
Wie schaff' ich mir nun zur Stelle
ein neues Lied herzu?
Ich armer, zerschlag'ner Geselle,
wie fänd' ich heut dazu Ruh'?
Werbung und ehlich Leben,
ob das mir Gott beschied,
muss ich nun grad aufgeben,
hab ich kein neues Lied.
Ein Lied von Euch, des bin ich gewiss,
mit dem besieg' ich jed' Hindernis!
Soll ich das heute haben,
vergessen, begraben
sei Zwist, Hader und Streit
und was uns je entzweit.

\direct{Er blickt seitwärts in das Blatt: plötzlich runzelt sich seine Stirn}

Und doch! Wenn's nur eine Falle wär'?
Noch gestern wart Ihr mein Feind:
Wie käm's, dass nach so grosser Beschwer'
Ihr's freundlich heut' mit mir meint?

\Sachsspeaks
Ich macht' Euch Schuh' in später Nacht:
hat man je so einen Feind bedacht?

\Beckmesserspeaks
Ja ja! Recht gut! Doch eines schwört:
wo und wie Ihr das Lied auch hört,
dass nie Ihr Euch beikommen lasst,
zu sagen, das Lied sei von Euch verfasst.

\Sachsspeaks
Das schwör' ich und gelob' es Euch,
nie mich zu rühmen, das Lied sei von mir.

\Beckmesserspeaks

\direct{sich vergnügt die Hände reibend}

Was will ich mehr? Ich bin geborgen!
Jetzt braucht sich Beckmesser nicht mehr zu sorgen!

\Sachsspeaks
Doch, Freund, ich führ's Euch zu Gemüte
und rat' es Euch in aller Güte:
studiert mir recht das Lied!
Sein Vortrag ist nicht leicht:
ob Euch die Weise geriet
und Ihr den Ton erreicht!

\Beckmesserspeaks
Freund Sachs, Ihr seid ein guter Poet;
doch was Ton und Weise betrifft,
gesteht, da tut mir's keiner vor!
Drum spitzt nur fein das Ohr.
Und:
``Beckmesser, keiner besser!''
darauf macht Euch gefasst,
wenn Ihr mich ruhig singen lasst.
Doch nun memorieren,
schnell nach Haus;
ohne Zeit zu verlieren
richt' ich das aus.
Hans Sachs, mein Teurer!
ich hab Euch verkannt;
durch den Abenteurer
war ich verrannt:

\direct{sehr zutraulich}

So einer fehlte uns bloss!
Den wurden wir Meister doch los!
Doch mein Besinnen
läuft mir von hinnen.
Bin ich verwirrt
und ganz verirrt?
Die Silben, die Reime,
die Worte, die Verse:
ich kleb' wie am Leime,
und brennt doch die Ferse.
Ade, ich muss fort!
An andrem Ort
dank' ich Euch inniglich,
weil Ihr so minniglich;
für Euch nun stimme ich,
kauf' Eure Werke gleich,
mache zum Merker Euch:
doch fein mit Kreide weich,
nicht mit dem Hammerstreich!
Merker! Merker! Merker Hans Sachs!
Dass Nürnberg schusterlich blüh' und wachs'!

\StageDir{\Beckmesser nimmt tanzend von \Sachs Abschied, taumelt und poltert der Ladentür zu; plötzlich glaubt er das Gedicht in seiner Tasche vergessen zu haben, läuft wieder vor, sucht ängstlich auf dem Werktische, bis er es in der eigenen Hand gewahr wird; darüber scherzhaft erfreut, umarmt er \Sachs nochmals voll feurigen Dankes und stürzt dann, hinkend und strauchelnd, geräuschvoll durch die Ladentür ab.}

\Sachsspeaks

\direct{sieht Beckmesser gedankenvoll lächelnd nach}

So ganz boshaft doch keinen ich fand;
er hält's auf die Länge nicht aus:
vergeudet mancher oft viel Verstand,
doch hält er auch damit Haus;
die schwache Stunde kommt für jeden,
da wird er dumm und lässt mit sich reden.
Dass hier Herr Beckmesser ward zum Dieb,
ist mir für meinen Plan sehr lieb.

\direct{\Eva nähert sich auf der Strasse der Ladentür. Sachs wendet sich um und gewahrt \Eva}

Sieh, Evchen! Dacht' ich doch, wo sie blieb'!


\scene

\StageDir{\Eva, reich geschmückt, in glänzender weisser Kleidung, etwas leidend und blass, tritt zum Laden herein und schreitet langsam vor.}

\Sachsspeaks
Grüss Gott, mein Evchen! Ei, wie herrlich
und stolz du's heute meinst!
Du machst wohl alt und jung begehrlich,
wenn du so schön erscheinst.

\Evaspeaks
Meister! 's ist nicht so gefährlich:
und ist's dem Schneider geglückt,
wer sieht dann, wo's mir beschwerlich,
wo still der Schuh mich drückt?

\Sachsspeaks
Der böse Schuh! 's war deine Laun',
dass du ihn gestern nicht probiert.

\Evaspeaks
Merk' wohl, ich hatt' zu viel Vertrau'n;
im Meister hatt' ich mich geirrt.

\Sachsspeaks
Ei, 's tut mir leid!
Zeig' her, mein Kind,
dass ich dir helfe gleich geschwind.

\Evaspeaks
Sobald ich stehe,
will es geh'n;
doch will ich geh'n,
zwingt's mich zu steh'n.

\Sachsspeaks
Hier auf den Schemel streck den Fuss:
der üblen Not ich wehren muss.

\direct{Sie streckt einen Fuss auf dem Schemel am Werktisch aus.}

Was ist's mit dem?

\Evaspeaks
Ihr seht, zu weit!

\Sachsspeaks
Kind, das ist pure Eitelkeit,
der Schuh ist knapp.

\Evaspeaks
Das sagt' ich ja:
drum drückt er mich an den Zehen da.

\Sachsspeaks
Hier links?

\Evaspeaks
Nein, rechts.

\Sachsspeaks
Wohl mehr am Spann?

\Evaspeaks
Hier, mehr am Hacken.

\Sachsspeaks
Kommt der auch dran?

\Evaspeaks
Ach Meister! Wüsstet Ihr besser als ich,
wo der Schuh mich drückt?

\Sachsspeaks
Ei, 's wundert mich,
dass er zu weit und doch drückt überall?

\StageDir{Walther, in glänzender Rittertracht, tritt unter die Tür der Kammer. Eva stösst einen Schrei aus und bleibt, unverwandt auf Walther blickend, in ihrer Stellung, mit dem Fusse auf dem Schemel. Sachs, der vor ihr niedergebückt steht, bleibt mit dem Rücken der Tür zugekehrt, ohne Walthers Eintritt zu beachten. Walther, durch den Anblick Evas festgebannt, bleibt ebenfalls unbeweglich unter der Tür stehen.}

Aha! Hier sitzt's! Nun begreif' ich den Fall!
Kind, du hast recht:
's stak in der Naht.
Nun warte, dem Übel schaff' ich Rat.
Bleib nur so steh'n; ich nehm' dir den Schuh
eine Weil' auf den Leisten:
dann lässt er dir Ruh'!

\direct{Er hat ihr sanft den Schuh vom Fusse gezogen; während sie in ihrer Stellung verbleibt, macht er sich am Werktisch mit dem Schuh zu schaffen und tut, als beachte er nichts anderes.}


\Sachsspeaks

\direct{bei der Arbeit}

Immer schustern, das ist nun mein Los;
des Nachts, des Tags komm' nicht davon los!
Kind, hör' zu! Ich hab mir's überdacht,
was meinem Schustern ein Ende macht:
am besten, ich werbe doch noch um dich;
da gewänn' ich doch was als Poet für mich!
Du hörst nicht drauf? - So sprich doch jetzt!
Hast mir's ja selbst in den Kopf gesetzt.
Schon gut! Ich merk':
``Mach deinen Schuh!''
Säng' mir nur wenigstens einer dazu!
Hörte heut' gar ein schönes Lied:
wem dazu wohl ein dritter Vers geriet?

\Waltherspeaks

\direct{den Blick unverwandt auf Eva geheftet}

``Weilten die Sterne im lieblichen Tanz?
So licht und klar im Lockenhaar,
vor allen Frauen hehr zu schauen,
lag ihr mit zartem Glanz ein Sternenkranz.''

\Sachsspeaks

\direct{immerfort arbeitend}

Lausch, Kind, das ist ein Meisterlied!

\Waltherspeaks
``Wunder ob Wunder nun bieten sich dar:
zwiefachen Tag ich grüssen mag;
denn gleich zwei'n Sonnen reinster Wonnen
der hehrsten Augen Paar nahm ich da wahr.''

\Sachsspeaks

\direct{beiseite zu \Eva}

Derlei hörst du jetzt bei mir singen.

\Waltherspeaks
``Huldreichstes Bild,
dem ich zu nahen mich erkühnt:
den Kranz, von zweier Sonnen Strahl
zugleich geblichen und ergrünt,
minnig und mild,
sie flocht ihn um das Haupt dem Gemahl.''

\Sachsspeaks

\direct{hat den Schuh zurückgebracht und ist jetzt darüber, ihn \Eva wieder anzuziehen}

Nun schau, ob dazu mein Schuh geriet?

\Waltherspeaks
``Dort Huld-geboren, nun Ruhm-erkoren,''

\Sachsspeaks
Mein' endlich doch,
es tät' mir gelingen?

\Waltherspeaks
``giesst paradiesische Lust sie in des Dichters Brust,''

\Sachsspeaks
Versuch's! Tritt auf! Sag, drückt er dich noch?

\Waltherspeaks
``im Liebestraum.''

\StageDir{\Eva, die wie bezaubert regungslos gestanden, gesehen und gehört hat, bricht jetzt in heftiges Weinen aus, sinkt Sachs an die Brust und drückt ihn schluchzend an sich. \Walther ist zu ihnen getreten; er drückt begeistert \Sachs die Hand. \Sachs tut sich endlich Gewalt an, reisst sich wie unmutig los und lässt dadurch \Eva unwillkürlich an Walthers Schulter sich anlehnen.}

\Sachsspeaks
Hat man mit dem Schuhwerk nicht seine Not!
Wär' ich nicht noch Poet dazu,
ich machte länger keine Schuh'!
Das ist eine Müh', ein Aufgebot!
Zu weit dem einen, dem andern zu eng;
von allen Seiten Lauf und Gedräng':
da klappt's, da schlappt's,
hier drückt's, da zwickt's!
Der Schuster soll auch alles wissen,
flicken, was nur immer zerrissen
und ist er nun gar Poet dazu,
da lässt man am End' ihm auch da keine Ruh';
und ist er erst noch Witwer gar,
zum Narren hält man ihn fürwahr.
Die jüngsten Mädchen, ist Not am Mann,
begehren. er hielte um sie an.
Versteht er sie, versteht er sie nicht,
all eins, ob ja, ob nein er spricht:
am End' riecht er doch nach Pech
und gilt für dumm, tückisch und frech!
Ei, 's ist mir nur um den Lehrbuben leid;
der verliert mir allen Respekt;
die Lene macht' ihn schon nicht recht gescheit,
dass aus Töpf' und Tellern er leckt!
Wo Teufel er jetzt nur wieder steckt?

\direct{Er stellt sich, als wolle er nach David sehen}


\Evaspeaks

\direct{indem sie Sachs zurückhält und von neuem an sich zieht}

O Sachs, mein Freund! Du teurer Mann!
Wie ich dir Edlem lohnen kann?
Was ohne deine Liebe, was wär' ich ohne dich,
ob je auch Kind ich bliebe,
erwecktest du mich nicht?
Durch dich gewann ich,
was man preist,
durch dich ersann ich,
was ein Geist!
Durch dich erwacht',
durch dich nur dacht'
ich edel, frei und kühn,
du liessest mich erblüh'n!
Ja, lieber Meister, schilt mich nur!
Ich war doch auf der rechten Spur:
denn, hatte ich die Wahl,
nur dich erwählt' ich mir:
du warest mein Gemahl.
Den Preis reicht' ich nur dir!
Doch nun hat's mich gewählt
zu nie gekannter Qual:
und werd' ich heut' vermählt,
so war's ohn' alle Wahl!
Das war ein Müssen, war ein Zwang!
Euch selbst, mein Meister, wurde bang'.

\Sachsspeaks
Mein Kind, von Tristan und Isolde
kenn' ich ein traurig Stück:
Hans Sachs war klug und wollte
nichts von Herrn Markes Glück.
's war Zeit, dass ich den Rechten fand,
wär' sonst am End' doch hineingerannt.
Aha! Da streicht die Lene schon ums Haus:
Nur herein! He, David! Kommst nicht heraus?

\direct{\Magdalene, in festlichem Staate, tritt durch die Ladentür herein; David ebenfalls im Festkleid, mit Blumen und Bändern sehr reich und zierlich aufgeputzt, kommt zugleich aus der Kammer}

Die Zeugen sind da, Gevatter zur Hand;
jetzt schnell zur Taufe, nehmt euren Stand.

\direct{Alle blicken ihn verwundert an}

Ein Kind ward hier geboren;
jetzt sei ihm ein Nam' erkoren!
So ist's nach Meisterweis' und Art,
wenn eine Meister-Weise geschaffen ward:
dass die einen guten Namen trag',
dran jeder sie erkennen mag.
Vernehmt, respektable Gesellschaft,
was euch hier zur Stell' schafft!
Eine Meisterweise ist gelungen,
von Junker Walther gedichtet und gesungen;
der jungen Weise lebender Vater
lud mich und die Pognerin zu Gevatter.
Weil wir die Weise wohl vernommen,
sind wir zur Taufe hierher gekommen.
Auch dass wir zur Handlung Zeugen haben,
ruf' ich Jungfer Lene und meinen Knaben.
Doch da's zum Zeugen kein Lehrbube tut
und heut' auch den Spruch er gesungen gut,
so mach' ich den Burschen gleich zum Gesell;
knie nieder, David, und nimm diese Schell'!

\direct{\David ist niedergekniet: \Sachs gibt ihm eine starke Ohrfeige}

Steh' auf, Gesell', und denk' an den Streich;
du merkst dir dabei die Taufe zugleich!
Fehlt sonst noch was, uns keiner schilt:
wer weiss, ob's nicht gar einer Nottaufe gilt.
Dass die Weise Kraft behalte zum Leben,
will ich nur gleich den Namen ihr geben:
%% START HERE: KNAPPERTSBUSCH CD4
``Die selige Morgentraumdeut-Weise''
sei sie genannt zu des Meisters Preise.
Nun wachse sie gross, ohn' Schad' und Bruch.
Die jüngste Gevatterin spricht den Spruch.

\direct{Er tritt aus der Mitte des Halbkreises, der von den übrigen um ihn gebildet war, auf die Seite, so dass nun Eva in die Mitte zu stehen kommt}

\Evaspeaks
Selig, wie die Sonne
meines Glückes lacht,
Morgen voller Wonne
selig mir erwacht!
Traum der höchsten Hulden,
himmlisch' Morgenglüh'n!
Deutung euch zu schulden,
selig süss Bemüh'n!
Einer Weise mild und hehr
sollt' es hold gelingen,
meines Herzens süss Beschwer'
deutend zu bezwingen.

\Sachsspeaks
Vor dem Kinde lieblich hold
möcht' ich gern wohl singen;
doch des Herzens süss' Beschwer
galt es zu bezwingen.

\Waltherspeaks
Deine Liebe liess mir es gelingen,
meines Herzens süss' Beschwer' deutend zu bezwingen.

MAGDALENE und DAVID
Wach' oder träum' ich schon so früh?
Das zu erklären macht mir Müh':

\Evaspeaks
Ob es nur ein Morgentraum?

\Waltherspeaks
Ob es noch der Morgentraum?

\Sachsspeaks
's war ein schöner Morgen-Traum:

EVA und WALTHER
Selig deut' ich mir es kaum.
Doch die Weise, was sie leise
mir/dir vertraut

\Waltherspeaks
im stillen Raum,

Beide
hell und laut,
in der Meister vollem Kreis

\Waltherspeaks
werbe sie um den höchsten Preis!

\Evaspeaks
deute sie auf den höchsten Preis!

\Sachsspeaks
dran zu deuten wag' ich kaum.
Diese Weise, was sie leise
mir anvertraut' im stillen Raum,
sagt mir laut:
auch der Jugend ew'ges Reis
grünt nur durch des Dichters Preis.

MAGDALENE und DAVID
's ist wohl nur ein Morgentraum?
Was ich seh', begreif' ich kaum!

\Davidspeaks
Ward zur Stelle gleich Geselle?
Lene Braut?
Im Kirchenraum wir gar getraut?
's geht der Kopf mir wie im Kreis,
dass Meister gar bald ich heiss'!

\Magdalenespeaks
Er zur Stelle gleich Geselle?
Ich die Braut?
Im Kirchenraum wir gar getraut?
Ja, wahrhaftig! 's geht:
wer weiss,
dass ich die Meist'rin bald heiss'!

\Sachsspeaks
zu den übrigen sich wendend
Jetzt all' am Fleck!
zu Eva
Den Vater grüss'!
Auf nach der Wies', schnell auf die Füss'!
Eva und Magdalene gehen
zu Walther
Nun, Junker, kommt! Habt frohen Mut! -
David, Gesell! Schliess' den Laden gut!

Als Sachs und Walther ebenfalls auf die Strasse gehen und David über das Schliessen der Ladentür sich hermacht, wird ein Vorhang von beiden Seiten zusammengezogen, so dass im Proszenium er die Szene gänzlich verschliesst


\scene

Die Vorhänge sind nach der Höhe aufgezogen worden; die Bühne ist verwandelt. Diese stellt einen freien Wiesenplan, im ferneren Hintergrunde die Stadt Nürnberg. Die Pegnitz schlängelt sich durch den Plan, der schmale Fluss ist an den nächsten Punkten praktikabel gehalten. Buntbeflaggte Kähne setzen die ankommenden, festlich gekleideten Bürger der Zünfte mit Frauen und Kindern, an das Ufer der Festwiese über. Eine erhöhte Bühne mit Bänken und Sitzen darauf ist rechts zur Seite aufgeschlagen; bereits ist sie mit den Fahnen der angekommenen Zünfte geschmückt; im Verlaufe stecken die Fahnenträger der noch ankommenden Zünfte ihre Fahnen ebenfalls um die Sängerbühne auf so dass diese schliesslich nach drei Seiten hin ganz davon eingefasst ist. Zelte mit Getränken und Erfrischungen aller Art begrenzen im übrigen die Seiten des vorderen Hauptraumes. Vor den Zelten geht es bereits lustig her: Bürger mit Frauen, Kindern und Gesellen sitzen und lagern daselbst. Die Lehrbuben der Meistersinger, festlich gekleidet, mit Blumen und Bändern reich und anmutig geschmückt, üben mit schlanken Stäben, die ebenfalls mit Blumen und Bändern geziert sind, in lustiger Weise das Amt von Herolden und Marschällen aus. Sie empfangen die am Ufer Aussteigenden, ordnen die Züge der Zünfte und geleiten diese nach der Sängerbühne, von wo aus, nachdem der Bannerträger die Fahne aufgepflanzt, die Zunftbürger und Gesellen sich unter den Zelten zerstreuen. Soeben werden die Schuster am Ufer empfangen und nach dem Vordergrunde geleitet

DIE SCHUSTER
mit fliegender Fahne aufziehend
Sankt Krispin, lobet ihn!
War gar ein heilig' Mann,
zeigt', was ein Schuster kann.
Die Armen hatten gute Zeit,
macht' ihnen warme Schuh';
und wenn ihm keiner 's Leder leiht,
so stahl er sich's dazu.
Der Schuster hat ein weit Gewissen,
macht Schuhe selbst mit Hindernissen;
und ist vom Gerber das Fell erst weg,
dann streck, streck, streck!
Leder taugt nur am rechten Fleck.

Die Stadtwächter und Heerhornbläser mit Trompeten und Trommeln sowie die Stadtpfeifer, Lautenmacher usw. ziehen, auf ihren Instrumenten spielend, auf. Ihnen folgen Gesellen mit Kinderinstrumenten

DIE SCHNEIDER
mit fliegender Fahne aufziehend
Als Nürnberg belagert war
und Hungersnot sich fand,
wär' Stadt und Volk verdorben gar,
war nicht ein Schneider zur Hand,
der viel Mut hatt' und Verstand.
Hat sich in ein Bocksfell eingenäht,
auf dem Stadtwall da spazierengeht
und macht wohl seine Sprünge
gar lustig guter Dinge.
Der Feind, der sieht's und zieht vom Fleck:
der Teufel hol' die Stadt sich weg,
hat's drin noch so lustige Meck-meck-meck!
Meck! Meck! Meck!
Wer glaubt's, dass ein Schneider im Bocke steck'!

DIE BÄCKER
ziehen mit fliegender Fahne auf
Hungersnot! Hungersnot!
Das ist ein greulich Leiden!
Gäb' euch der Bäcker nicht täglich Brot,
müsst' alle Welt verscheiden.
Beck! Beck! Beck!
Täglich auf dem Fleck!
Nimm uns den Hunger weg!

DIE SCHUSTER
welche ihre Fahne aufgesteckt, begegnen beim Herabschreiten von der Sängerbühne den Bäckern
Streck! Streck! Streck!
Leder taugt nur am rechten Fleck.

DIE SCHNEIDER
nachdem die Fahne aufgesteckt, herabschreitend
Meck! Meck! Meck!
Wer meint, dass ein Schneider im Bocke steck'!

Ein bunter Kahn mit jungen Mädchen in reicher bäuerischer Tracht kommt an

LEHRBUBEN
laufen nach dem Gestade
Herrje! Herrje! Mädel von Fürth!
Stadtpfeifer, spielt, dass's lustig wird!

Sie heben die Mädchen aus dem Kahn. Das Charakteristische des Tanzes, mit welchem die Lehrbuben und Mädchen zunächst nach dem Vordergrund kommen, besteht darin, dass die Lehrbuben die Mädchen scheinbar nur an den Platz bringen wollen; sowie die Gesellen zugreifen wollen, ziehen die Buben die Mädchen aber immer wieder zurück, als ob sie sie anderswo unterbringen wollten, wobei sie den ganzen Kreis, wie wählend, ausmessen und somit die scheinbare Absicht anmutig und lustig verzögern

\Davidspeaks
kommt vom Landungsplatz vor und sieht missbilligend dem Tanze zu
Ihr tanzt? Was werden die Meister sagen?
Die Lehrbuben drehen ihm Nasen
Hört nicht? - Lass ich mir's auch behagen!
Er nimmt sich ein junges, schönes Mädchen und gerät im Tanze mit ihr schnell in grosses Feuer. Die Zuschauer freuen sich und lachen

EINIGE LEHRBUBEN
winken David
David! David! Die Lene sieht zu!

\Davidspeaks
lässt das Mädchen erschrocken fahren, um das die Lehrbuben sogleich tanzend einen Kreis schliessen. Da er Lene nirgends gewahrt, merkt David, dass er nur geneckt worden, durchbricht den Kreis, erfasst sein Mädchen wieder und tanzt noch feuriger weiter
Ach, lasst mich mit euren Possen in Ruh'!
Die Buben suchen ihm das Mädchen zu entreissen, er wendet sich mit ihr jedesmal glücklich ab, so dass nun ein ähnliches Spiel entsteht wie zuvor, als die Gesellen nach den Mädchen fassten

GESELLEN
vom Ufer her
Die Meistersinger!

LEHRBUBEN
Die Meistersinger!
Sie unterbrechen schnell den Tanz und eilen zum Ufer

\Davidspeaks
Herrgott! Ade, ihr hübschen Dinger!
Er gibt dem Mädchen einen feurigen Kuss und reisst sich los

Die Lehrbuben reihen sich zum Empfang der Meistersinger. Das Volk macht ihnen willig Platz. Die Meistersinger ordnen sich am Landungsplatze zum festlichen Aufzuge. Wenn Kothner im Vordergrunde ankommt, wird die geschwungene Fahne, auf welcher König David mit der Harfe abgebildet ist, von allem Volk mit Hutschwenken begrüsst. Der Zug der Meistersinger ist nun auf der Singerbühne angelangt, wo Kothner die Fahne aufpflanzt. Pogner, Eva an der Hand führend, diese von festlich geschmückten, reich gekleideten jungen Mädchen, unter denen auch Magdalene, begleitet, voran. Als Eva, von den Mädchen umgeben, den mit Blumen geschmückten Ehrenplatz eingenommen und alle übrigen, die Meister auf den Bänken, die Gesellen hinter ihnen stehend, ebenfalls Platz genommen, treten die Lehrbuben, dem Volke zugewendet, feierlich vor die Bühne in Reih und Glied

LEHRBUBEN
Silentium! Silentium!
Sachs erhebt sich und tritt vor. Bei seinem Anblick stösst sich alles an; Hüte und Mützen werden abgezogen. Alle deuten auf ihn
Macht kein Reden und kein Gesumm'.

EINIGE IM VOLK
Ha! Sachs! 's ist Sachs!
Seht Meister Sachs!

MEHRERE
Stimmt an! Stimmt an!
Alle Sitzenden erheben sich; die Männer bleiben mit entblösstem Haupte. Beckmesser bleibt, mit dem Memorieren des Gedichtes beschäftigt, hinter den anderen Meistern versteckt, so dass er bei dieser Gelegenheit der Beachtung des Publikums entzogen wird

ALLE
ausser Sachs
Wach' auf, es nahet gen den Tag,
ich hör' singen im grünen Hag
ein' wonnigliche Nachtigal,
ihr' Stimm' durchdringet Berg und Tal;
die Nacht neigt sich zum Okzident,
der Tag geht auf von Orient,
die rotbrünstige Morgenröt'
her durch die trüben Wolken geht.«

DAS VOLK
nimmt wieder eine jubelnd bewegte Haltung an und singt nun allein. Die Meister auf der Bühne sowie die anderen Teilnehmer am Gesange geben sich dem Schauspiele des Volksjubels hin
Heil Sachs! Heil dir, Sachs!
Heil Nürnbergs teurem Sachs! Heil! Heil!

Sachs, der unbeweglich, wie geistesabwesend, über die Menge hinweg geblickt hatte, richtet endlich seine Blicke vertrauter auf sie und beginnt mit ergriffener, schnell sich festigender Stimme

\Sachsspeaks
Euch macht Ihr's leicht, mir macht Ihr's schwer,
gebt Ihr mir Armen zuviel Ehr'.
Soll vor der Ehr' ich besteh'n,
sei's, mich von Euch geliebt zu seh'n!
Schon grosse Ehr' ward mir erkannt,
ward heut' ich zum Spruchsprecher ernannt.
Und was mein Spruch Euch künden soll,
glaubt, das ist hoher Ehren voll!
Wenn Ihr die Kunst so hoch schon ehrt,
da galt es zu beweisen,
dass, wer ihr selbst gar angehört,
sie schätzt ob allen Preisen.
Ein Meister, reich und hochgemut,
der will heut' Euch das zeigen:
sein Töchterlein, sein höchstes Gut,
mit allem Hab und Eigen,
dem Singer, der im Kunstgesang
vor allem Volk den Preis errang,
als höchsten Preises Kron'
er bietet das zum Lohn.
Darum so hört und stimmt mir bei:
die Werbung steh' dem Dichter frei.
Ihr Meister, die Ihr's Euch getraut,
Euch ruf' ich's vor dem Volke laut:
erwägt der Werbung seltnen Preis,
und wem sie soll gelingen,
dass der sich rein und edel weiss
im Werben wie im Singen,
will er das Reis erringen,
das nie bei Neuen noch bei Alten
ward je so herrlich hoch gehalten
als von der lieblich Reinen,
die niemals soll beweinen,
dass Nürenberg mit höchstem Wert
die Kunst und ihre Meister ehrt.

Grosse Bewegung unter allen. Sachs geht auf Pogner zu, der ihm gerührt die Hand drückt

\Pognerspeaks
O Sachs! Mein Freund! Wie dankenswert!
Wie wisst Ihr, was mein Herz beschwert!

\Sachsspeaks
zu Pogner
's war viel gewagt! Jetzt habt nur Mut!
Er wendet sich zu Beckmesser, der fortwährend eifrig das Blatt mit dem Gedicht herausgezogen, memoriert, genau zu lesen versucht und oft verzweiflungsvoll sich den Schweiss getrocknet hat
Herr Merker! Sagt, wie steht es? Gut?

\Beckmesserspeaks
O dieses Lied! Werd' nicht draus klug
und hab' doch dran studiert genug!

\Sachsspeaks
Mein Freund, 's ist Euch nicht aufgezwungen.

\Beckmesserspeaks
Was hilft's? - Mit dem meinen ist doch versungen!
's war Eure Schuld! Jetzt seid hübsch für mich!
's wär' schändlich, liesst Ihr mich im Stich!

\Sachsspeaks
Ich dächt', Ihr gäbt's auf.

\Beckmesserspeaks
Warum nicht gar?
Die and'ren sing' ich alle zu Paar', wenn Ihr nur nicht singt!

\Sachsspeaks
So seht, wie's geht!

\Beckmesserspeaks
Das Lied! - bin's sicher - zwar niemand versteht;
doch bau' ich auf Eure Popularität.

\Sachsspeaks
Nun denn, wenn's Meistern und Volk beliebt,
zum Wettgesang man den Anfang gibt.

\Kothnerspeaks
tritt vor
Ihr ledig' Meister, macht Euch bereit!
Der Ältest' sich zuerst anlässt:
Herr Beckmesser, Ihr fangt an, 's ist Zeit!

Die Lehrbuben führen Beckmesser zu einem kleinen Rasenhügel vor der Singerbühne, welchen sie zuvor festgerammt und reich mit Blumen überdeckt haben

\Beckmesserspeaks
strauchelt darauf, tritt unsicher und schwankt
Zum Teufel! Wie wackelig! Macht das hübsch fest!

Die Buben lachen unter sich und stopfen lustig am Rasen

DAS VOLK
stösst sich gegenseitig lustig an
Wie, der? Der wirbt? Scheint mir nicht der Rechte!
An der Tochter Stell' ich den nicht möchte.
Seid still! 's ist gar ein tücht'ger Meister!
Still! Macht keinen Witz;
der hat im Rate Stimm' und Sitz.
Ach, der kann ja nicht mal steh'n.
Wie soll es mit dem geh'n?
Er fällt fast um! Gott, ist der dumm!
Stadtschreiber ist er:
Beckmesser heisst er.
Gott, ist der dumm!
Still! Macht keinen Witz!
Er fällt fast um!
Der hat im Rate Stimm und Sitz!
Viele lachen

DIE LEHRBUBEN
in Aufstellung
Silentium! Silentium!
Macht kein Reden und kein Gesumm!

\Kothnerspeaks
Fanget an!

\Beckmesserspeaks
der sich endlich mit Mühe auf dem Rasenhügel festgestellt hat, macht eine erste Verbeugung gegen die Meister, eine zweite gegen das Volk, dann gegen Eva, auf welche er, da sie sich abwendet, nochmals verlegen hinblinzelt. Grosse Beklommenheit erfasst ihn; er sucht sich durch das Vorspiel auf der Laute zu ermutigen
"Morgen ich leuchte in rosigem Schein,
von Blut und Duft geht schnell die Luft; -
wohl bald gewonnen wie zerronnen -
im Garten lud ich ein - garstig und fein.«

Er versucht, besser auf den Füssen zu stehen. Die Meistersinger leise unter sich

DIE MEISTER
Mein! Was ist das?
Ist er von Sinnen?
Was ist das?
Ist er von Sinnen?
Höchst merkwürd'ger Fall! Was kommt ihm bei?
Woher mocht' er solche Gedanken gewinnen?

VOLK
leise unter sich
Sonderbar! Hört ihr's? Wen lud er ein?
Verstand man recht? Wie kann das sein?

\Beckmesserspeaks
zieht das Blatt verstohlen hervor und lugt eifrig hinein; dann steckt er es ängstlich wieder ein
Wohn' ich erträglich im selbigen Raum,
hol' Gold und Frucht - Bleisaft und Wucht.
Er lugt in das Blatt
Mich holt am Pranger - der Verlanger -
auf luft'ger Steige kaum - häng' ich am Baum.«
Er wackelt wieder sehr; sucht im Blatt zu lesen, vermag es nicht,' ihm schwindelt, Angstschweiss bricht aus

DAS VOLK
Schöner Werber! Der find't wohl seinen Lohn:
bald hängt er am Galgen; man sieht ihn schon.

DIE MEISTER
Was soll das heissen?
Ist er nur toll?
Sein Lied ist ganz von Unsinn voll!

\Beckmesserspeaks
rafft sich verzweiflungsvoll und ingrimmig auf
"Heimlich mir graut,
weil hier es munter will hergeh'n:
an meiner Leiter stand ein Weib,
sie schämt' und wollt' mich nicht beseh'n.
Bleich wie ein Kraut
umfasset mir Hanf meinen Leib; -
mit Augen zwinkend - der Hund blies winkend -
was ich vor langem verzehrt -
wie Frucht, so Holz und Pferd -
vom Leberbaum.«
Alles bricht in ein dröhnendes Gelächter aus

\Beckmesserspeaks
verlässt wütend den Hügel und stürzt auf Sachs zu
Verdammter Schuster, das dank' ich dir!
Das Lied, es ist gar nicht von mir.
Von Sachs, der hier so hoch verehrt,
von Eurem Sachs ward mir's beschert!
Mich hat der Schändliche bedrängt,
sein schlechtes Lied mir aufgehängt.
Er stürzt wütend fort und verliert sich unter dem Volke

VOLK
Mein! Was soll das sein? Jetzt wird's immer bunter!
Von Sachs das Lied? Das nähm' uns doch wunder!

\Kothnerspeaks
Erklärt doch, Sachs!

\Nachtigallspeaks
Welch ein Skandal!

\Vogelgesangspeaks
Von Euch das Lied?

ORTEL und FOLTZ
Welch eig'ner Fall!

\Sachsspeaks
hat ruhig das Blatt, welches ihm Beckmesser hingeworfen, aufgenommen
Das Lied fürwahr ist nicht von mir.
Herr Beckmesser irrt wie dort so hier!
Wie er dazu kam, mag selbst er sagen;
doch möcht' ich nie mich zu rühmen wagen,
ein Lied, so schön wie dies erdacht,
sei von mir, Hans Sachs, gemacht.

MEISTERSINGER
Wie? Schön? Dieser Unsinnswust!

VOLK
Hört, Sachs macht Spass! Er sagt es nur zur Lust.

\Sachsspeaks
Ich sag' Euch Herrn, das Lied ist schön:
nur ist's auf den ersten Blick zu ersehn,
dass Freund Beckmesser es entstellt.
Doch schwör' ich, dass es Euch gefällt,
wenn richtig Wort' und Weise
hier einer säng' im Kreise.
Und wer dies verstünd', zugleich bewies',
dass er des Liedes Dichter
und gar mit Rechte Meister hiess',
fänd' er gerechte Richter.
Ich bin verklagt und muss besteh'n:
drum lasst mich meinen Zeugen auserseh'n!
Ist jemand hier, der Recht mir weiss,
der tret' als Zeug' in diesen Kreis!
Walther tritt aus dem Volke hervor und begrüsst Sachs, sodann Meister und Volk mit ritterlicher Freundlichkeit. Es entsteht sogleich eine angenehme Bewegung. Alles weilt einen Augenblick schweigend in seiner Betrachtung
So zeuget, das Lied sei nicht von mir,
und zeuget auch, dass, was ich hier
vom Lied hab' gesagt, zuviel nicht sei gewagt.

DIE MEISTER
Wie fein ist Sachs! Ei Sachs, Ihr seid gar fein!
Doch mag es heut' geschehen sein!

\Sachsspeaks
Der Regel Güte daraus man erwägt,
dass sie auch mal 'ne Ausnahm' verträgt.

DAS VOLK
Ein guter Zeuge, stolz und kühn!
Mich dünkt, dem kann wohl was Gut's erblühn.

\Sachsspeaks
Meister und Volk sind gewillt
zu vernehmen, was mein Zeuge gilt.
Herr Walther von Stolzing, singt das Lied!
Ihr Meister lest, ob's ihm geriet.
Er übergibt Kothner das Blatt zum Nachlesen

DIE LEHRBUBEN
in Aufstellung
Alles gespannt! 's gibt kein Gesumm.
Da rufen wir auch nicht Silentium!

\Waltherspeaks
\direct{beschreitet festen Schrittes den kleinen Blumenhügel}

``Morgenlich leuchtend in rosigem Schein,
von Blüt' und Duft geschwellt die Luft,
voll aller Wonnen, nie ersonnen,
ein Garten lud mich ein -
Kothner lässt das Blatt, in welchem er mit den anderen Meistern eifrig nachzulesen begonnen, vor Ergriffenheit unwillkürlich fallen; er und die übrigen hören nur noch teilnahmsvoll zu
Wie entrückt.
dort unter einem Wunderbaum,
von Früchten reich behangen,
zu schaun in sel'gem Liebestraum,
was höchstem Lustverlangen
Erfüllung kühn verhiess -
das schönste Weib, Eva im Paradies.''

\speaker{Das Volk}
\direct{leise flüsternd}

Das ist was andres! Wer hätt's gedacht?
Was doch recht Wort und Vortrag macht!

\speaker{Die Meistersinger}
\direct{ohne Foltz und Schwarz, leise flüsternd}

Jawohl! Ich merk'! 's ist ein ander Ding,

\Sachsspeaks
Zeuge am Ort, fahret fort!

\Waltherspeaks
``Abendlich dämmernd umschloss mich die Nacht;
auf steilem Pfad war ich genaht
zu einer Quelle reiner Welle,
die lockend mir gelacht:
dort unter einem Lorbeerbaum,
von Sternen hell durchschienen,
ich schaut' im wachen Dichtertraum
von heilig holden Mienen,
mich netzend mit dem edlen Nass,
das hehrste Weib,
die Muse des Parnass.''

\speaker{Das Volk}
\direct{immer leiser, für sich}

Wie so hold und traut, wie fern es schwebt,
doch ist es grad', als ob man selber alles miterlebt!

\speaker{Die Meistersinger}

's ist kühn und seltsam, das ist wahr;
doch wohlgereimt und singebar.

\Sachsspeaks
Zeuge wohl erkiest, fahret fort und schliesst!

\Waltherspeaks
\direct{sehr feurig}
``Huldreichster Tag,
dem ich aus Dichters Traum erwacht!
Das ich erträumt, das Paradies,
in himmlisch neu verklärter Pracht
hell vor mir lag,
dahin lachend nun der Quell den Pfad mir wies:
die dort geboren, mein Herz erkoren,
der Erde lieblichstes Bild,
als Muse mir geweiht,
so heilig ernst als mild,
ward kühn von mir gefreit,
am lichten Tag der Sonnen
durch Sanges Sieg gewonnen
Parnass und Paradies!''

\speaker{Volk}
Gewiegt wie in den schönsten Traum,
hör' ich es wohl, doch fass es kaum.
zu Eva
Reich ihm das Reis! Sein sei der Preis!
Keiner wie er zu werben weiss!

\speaker{Die Meister}
\direct{sich erhebend}

Ja, holder Sänger!
Nimm das Reis!
Dein Sang erwarb dir Meisterpreis!
Keiner so wie nur er zu werben weiss!

\Pognerspeaks
\direct{mit grosser Ergriffenheit zu Sachs sich wendend}

O Sachs! Dir dank' ich Glück und Ehr'!
Vorüber nun all Herzbeschwer!

\direct{Walther ist auf die Stufen der Singerbühne geleitet worden und lässt sich vor Eva auf ein Knie nieder}

\Evaspeaks
\direct{zu Walther, indem sie ihn mit einem Kranz aus Lorbeer und Myrten bekränzt, sich hinabneigend}

Keiner wie du so hold zu werben weiss!

\Sachsspeaks
\direct{zum Volk gewandt, auf Walther und Eva deutend}

Den Zeugen, denk es, wählt' ich gut:
tragt Ihr Hans Sachs drum üblen Mut?

\speaker{Volk}
\direct{bricht schnell und heftig in jubelnde Bewegung aus}

Hans Sachs! Nein! Das war schön erdacht!
Das habt Ihr einmal wieder gut gemacht!

\speaker{Meistersinger}
\direct{sich feierlich zu Pogner wendend}

Auf, Meister Pogner! Euch zum Ruhm
meldet dem Junker sein Meistertum.

\Pognerspeaks
\direct{mit einer goldnen Kette, daran drei grosse Denkmünzen, zu Walther}

Geschmückt mit König Davids Bild,
nehm' ich Euch auf in der Meister Gild'.

\Waltherspeaks
\direct{mit schmerzlicher Heftigkeit abweisend}

Nicht Meister! Nein!
Er blickt zärtlich auf Eva
Will ohne Meister selig sein!

\direct{Alles blickt in grosser Betroffenheit auf Sachs}

\Sachsspeaks
\direct{schreitet auf Walther zu und fasst ihn bedeutungsvoll bei der Hand}

Verachtet mir die Meister nicht
und ehrt mir ihre Kunst!
Was ihnen hoch zum Lobe spricht,
fiel reichlich Euch zur Gunst!
Nicht Euren Ahnen, noch so wert,
nicht Eurem Wappen, Speer noch Schwert,
dass Ihr ein Dichter seid,
ein Meister Euch gefreit,
dem dankt Ihr heut' Eu'r höchstes Glück.
Drum, denkt mit Dank Ihr d'ran zurück,
wie kann die Kunst wohl unwert sein,
die solche Preise schliesset ein?
Dass uns're Meister sie gepflegt,
grad' recht nach ihrer Art,
nach ihrem Sinne treu gehegt,
das hat sie echt bewahrt.
Blieb sie nicht adlig wie zur Zeit,
wo Höf' und Fürsten sie geweiht,
im Drang der schlimmen Jahr'
blieb sie doch deutsch und wahr;
und wär' sie anders nicht geglückt,
als wie, wo alles drängt und drückt,
Ihr seht, wie hoch sie blieb in Ehr'!
Was wollt Ihr von den Meistern mehr?
Habt acht! Uns dräuen üble Streich'!
Zerfällt erst deutsches Volk und Reich,
in falscher welscher Majestät
kein Fürst bald mehr sein Volk versteht;
und welschen Dunst mit welschem Tand
sie pflanzen uns in deutsches Land.
Was deutsch und echt, wüsst' keiner mehr,
lebt's nicht in deutscher Meister Ehr'.
Drum sag' ich Euch:
ehrt Eure deutschen Meister,
dann bannt Ihr gute Geister!
Und gebt Ihr ihrem Wirken Gunst,
zerging' in Dunst
das Heil'ge Röm'sche Reich,
uns bliebe gleich
die heil'ge deutsche Kunst!

\StageDir{
Während des Schlussgesangs nimmt Eva den Kranz von Walthers Stirn und drückt ihn Sachs auf; dieser nimmt die Kette aus Pogners Hand und hängt sie Walther um. Nachdem Sachs das Paar umarmt, bleiben Walther und Eva zu beiden Seiten an Sachs' Schultern gestützt; Pogner lässt sich, wie huldigend, auf ein Knie vor Sachs nieder. Die Meistersinger deuten auf Sachs als auf ihr Haupt}


\speaker{Alle}

Ehrt Eure deutschen Meister,
dann bannt Ihr gute Geister!
Und gebt Ihr ihrem Wirken Gunst,
zerging' in Dunst
das Heil'ge Röm'sche Reich,
uns bliebe gleich
die heil'ge deutsche Kunst!

\StageDir{Das Volk schwenkt begeistert Hüte und Tücher; die Lehrbuben tanzen und schlagen jauchzend in die Hände}


\speaker{Volk}

Heil Sachs! Nürnbergs teurem Sachs!

\end{drama}

%%% Local Variables:
%%% mode: latex
%%% TeX-master: "../lib_rom"
%%% End:
