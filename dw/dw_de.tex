\begin{drama}
\act

\scene

\StageDir{In der Mitte steht der Stamm einer mächtigen Esche, dessen stark erhabene Wurzeln sich weithin in den Erdboden verlieren; von seinem Wipfel ist der Baum durch ein gezimmertes Dach geschieden, welches so durchschnitten ist, daß der Stamm und die nach allen Seiten hin sich ausstreckenden Äste durch genau entsprechende Öffnungen hindurchgehen; von dem belaubten Wipfel wird angenommen, daß er sich über dieses Dach ausbreite. Um den Eschenstamm, als Mittelpunkt, ist nun ein Saal gezimmert; die Wände sind aus roh behauenem Holzwerk, hier und da mit geflochtenen und gewebten Decken behangen. Rechts im Vordergrunde steht der Herd, dessen Rauchfang seitwärts zum Dache hinausführt: hinter dem Herde befindet sich ein innerer Raum, gleich einem Vorratsspeicher, zu dem man auf einigen hölzernen Stufen hinaufsteigt: davor hängt, halb zurückgeschlagen, eine geflochtene Decke. Im Hintergrunde eine Eingangstür mit schlichtem Holzriegel. Links die Tür zu einem inneren Gemache, zu dem gleichfalls Stufen hinaufführen; weiter vornen auf derselben Seite ein Tisch mit einer breiten, an der Wand angezimmerten Bank dahinter und hölzernen Schemeln davor.

\direct{Ein kurzes Orchestervorspiel von heftiger, stürmischer Bewegung leitet ein. Als der Vorhang aufgeht, öffnet Siegmund von außen hastig die Eingangstür und tritt ein: es ist gegen Abend, starkes Gewitter, im Begriff, sich zu legen. Siegmund hält einen Augenblick den Riegel in der Hand und überblickt den Wohnraum: er scheint von übermäßiger Anstrengung erschöpft; sein Gewand und Aussehen zeigen, daß er sich auf der Flucht befinde. Da er niemand gewahrt, schließt er die Tür hinter sich, schreitet auf den Herd zu und wirft sich dort ermattet auf eine Decke von Bärenfell.}}

\Siegmundspeaks
Wes Herd dies auch sei, hier muß ich rasten.
 


\direct{Er sinkt zurück und bleibt einige Zeit regungslos ausgestreckt. Sieglinde tritt aus der Tür des inneren Gemaches; sie glaubte ihren Mann heimgekehrt: ihre ernste Miene zeigt sich dann verwundert, als sie einen Fremden am Herde ausgestreckt sieht}


\Sieglindespeaks

\direct{noch im Hintergrunde}

Ein fremder Mann? Ihn muß ich fragen.

\direct{Sie tritt ruhig einige Schritte näher}

Wer kam ins Haus und liegt dort am Herd?

\direct{Da Siegmund sich nicht regt, tritt sie noch etwas näher und betrachtet ihn}

Müde liegt er, von Weges Müh'n.
Schwanden die Sinne ihm? Wäre er siech?

\direct{Sie neigt sich zu ihm herab und lauscht}

Noch schwillt ihm der Atem; das Auge nur schloß er. -
Mutig dünkt mich der Mann, sank er müd' auch hin.
 

\Siegmundspeaks

\direct{fährt jäh mit dem Haupt in die Höhe}

Ein Quell! Ein Quell!
 

\Sieglindespeaks
Erquickung schaff' ich.

\direct{Sie nimmt schnell ein Trinkhorn und geht damit aus dem Hause. Sie kommt zurück und reicht das gefüllte Trinkhorn Siegmund}

Labung biet' ich dem lechzenden Gaumen:
Wasser, wie du gewollt.
 


\direct{Siegmund trinkt und reicht ihr das Horn zurück. Als er ihr mit dem Haupte Dank zuwinkt, haftet sein Blick mit steigender Teilnahme an ihren Mienen.}


\Siegmundspeaks
Kühlende Labung gab mir der Quell,
des Müden Last machte er leicht:
erfrischt ist der Mut,
das Aug' erfreut des Sehens selige Lust.
Wer ist's, der so mir es labt?
 

\Sieglindespeaks
Dies Haus und dies Weib sind Hundings Eigen;
gastlich gönn' er dir Rast: harre, bis heim er kehrt!
 

\Siegmundspeaks
Waffenlos bin ich:
dem wunden Gast wird dein Gatte nicht wehren.
 

\Sieglindespeaks

\direct{mit besorgter Hast}

Die Wunden weise mir schnell!
 

\Siegmundspeaks

\direct{schüttelt sich und springt lebhaft vom Lager zum Sitz auf}

Gering sind sie, der Rede nicht wert;
noch fügen des Leibes Glieder sich fest.
Hätten halb so stark wie mein Arm
Schild und Speer mir gehalten,
nimmer floh ich dem Feind,
doch zerschellten mir Speer und Schild.
Der Feinde Meute hetzte mich müd',
Gewitterbrunst brach meinen Leib;
doch schneller, als ich der Meute,
schwand die Müdigkeit mir:
sank auf die Lider mir Nacht;
die Sonne lacht mir nun neu.
 

\Sieglindespeaks

\direct{geht nach dem Speicher, füllt ein Horn mit Met und reicht es Siegmund mit freundlicher Bewegtheit}

Des seimigen Metes süßen Trank
mög'st du mir nicht verschmähn.
 

\Siegmundspeaks
Schmecktest du mir ihn zu?
 


\direct{Sieglinde nippt am Horne und reicht es ihm wieder. Siegmund tut einen langen Zug, indem er den Blick mit wachsender Wärme auf sie heftet. Er setzt so das Horn ab und läßt es langsam sinken, während der Ausdruck seiner Miene in starke Ergriffenheit übergeht. Er seufzt tief auf und senkt den Blick düster zu Boden.}


\Siegmundspeaks

\direct{mit bebender Stimme}

Einen Unseligen labtest du:
Unheil wende der Wunsch von dir!
 


\direct{Er bricht schnell auf, um fortzugehen}

Gerastet hab' ich und süß geruht.
Weiter wend' ich den Schritt.
 


\direct{er geht nach hinten}


\Sieglindespeaks

\direct{lebhaft sich umwendend}

Wer verfolgt dich, daß du schon fliehst?
 

\Siegmundspeaks

\direct{von ihrem Rufe gefesselt, wendet sich wieder; langsam und düster}

Mißwende folgt mir, wohin ich fliehe;
Mißwende naht mir, wo ich mich neige. -
Dir, Frau, doch bleibe sie fern!
Fort wend' ich Fuß und Blick.
 


\direct{Er schreitet schnell bis zur Tür und hebt den Riegel}


\Sieglindespeaks

\direct{in heftigem Selbstvergessen ihm nachrufend}

So bleibe hier!
Nicht bringst du Unheil dahin,
wo Unheil im Hause wohnt!
 


\direct{Siegmund bleibt tief erschüttert stehen; er forscht in Sieglindes Mienen; diese schlägt verschämt und traurig die Augen nieder. Langes Schweigen}


\Siegmundspeaks

\direct{kehrt zurück}

Wehwalt hieß ich mich selbst:
Hunding will ich erwarten.
 


\direct{Er lehnt sich an den Herd; sein Blick haftet mit ruhiger und entschlossener Teilnahme an Sieglinde; diese hebt langsam das Auge wieder zu ihm auf. Beide blicken sich in langem Schweigen mit dem Ausdruck tiefster Ergriffenheit in die Augen}

\scene

\StageDir{Sieglinde fährt plötzlich auf, lauscht und hört Hunding, der sein Roß außen zum Stall führt. Sie geht hastig zur Tür und öffnet; Hunding, gewaffnet mit Schild und Speer, tritt ein und hält unter der Tür, als er Siegmund gewahrt. Hunding wendet sich mit einem ernst fragenden Blick an Sieglinde}


\Sieglindespeaks

\direct{dem Blicke Hundings entgegnend}

Müd am Herd fand ich den Mann:
Not führt' ihn ins Haus.
 

\Hundingspeaks
Du labtest ihn?
 

\Sieglindespeaks
Den Gaumen letzt' ich ihm, gastlich sorgt' ich sein!
 

\Siegmundspeaks

\direct{der ruhig und fest Hunding beobachtet}

Dach und Trank dank' ich ihr:
willst du dein Weib drum schelten?
 

\Hundingspeaks
Heilig ist mein Herd: -
heilig sei dir mein Haus!
 


\direct{er legt seine Waffen ab und übergibt sie Sieglinde. Zu Sieglinde}

Rüst' uns Männern das Mahl!
 


\direct{Sieglinde hängt die Waffen an Ästen des Eschenstammes auf, dann holt sie Speise und Trank aus dem Speicher und rüstet auf dem Tische das Nachtmahl. Unwillkürlich heftet sie wieder den Blick auf Siegmund}


\direct{Hunding mißt scharf und verwundert Siegmunds Züge, die er mit denen seiner Frau vergleicht; für sich}

Wie gleicht er dem Weibe!
Der gleißende Wurm glänzt auch ihm aus dem Auge.
 


\direct{er birgt sein Befremden und wendet sich wie unbefangen zu Siegmund}

Weit her, traun, kamst du des Wegs;
ein Roß nicht ritt, der Rast hier fand:
welch schlimme Pfade schufen dir Pein?
 

\Siegmundspeaks
Durch Wald und Wiese, Heide und Hain,
jagte mich Sturm und starke Not:
nicht kenn' ich den Weg, den ich kam.
Wohin ich irrte, weiß ich noch minder:
Kunde gewänn' ich des gern.
 

\Hundingspeaks

\direct{am Tische und Siegmund den Sitz bietend}

Des Dach dich deckt, des Haus dich hegt,
Hunding heißt der Wirt;
wendest von hier du nach West den Schritt,
in Höfen reich hausen dort Sippen,
die Hundings Ehre behüten.
Gönnt mir Ehre mein Gast,
wird sein Name nun mir gennant.
 


\direct{Siegmund, der sich am Tisch niedergesetzt, blickt nachdenklich vor sich hin. Sieglinde, die sich neben Hunding, Siegmund gegenüber, gesetzt, heftet ihr Auge mit auffallender Teilnahme und Spannung auf diesen.}


\Hundingspeaks

\direct{der beide beobachtet}

Trägst du Sorge, mir zu vertraun,
der Frau hier gib doch Kunde:
sieh, wie gierig sie dich frägt!
 

\Sieglindespeaks

\direct{unbefangen und teilnahmsvoll}

Gast, wer du bist, wüßt' ich gern.
 

\Siegmundspeaks

\direct{blickt auf, sieht ihr in das Auge und beginnt ernst}

Friedmund darf ich nicht heißen;
Frohwalt möcht' ich wohl sein:
doch Wehwalt mußt ich mich nennen.
Wolfe, der war mein Vater;
zu zwei kam ich zur Welt,
eine Zwillingsschwester und ich.
Früh schwanden mir Mutter und Maid.
Die mich gebar und die mit mir sie barg,
kaum hab' ich je sie gekannt.
Wehrlich und stark war Wolfe;
der Feinde wuchsen ihm viel.
Zum Jagen zog mit dem Jungen der Alte:
Von Hetze und Harst einst kehrten wir heim:
da lag das Wolfsnest leer.
Zu Schutt gebrannt der prangende Saal,
zum Stumpf der Eiche blühender Stamm;
erschlagen der Mutter mutiger Leib,
verschwunden in Gluten der Schwester Spur:
uns schuf die herbe Not
der Neidinge harte Schar.
Geächtet floh der Alte mit mir;
lange Jahre lebte der Junge
mit Wolfe im wilden Wald:
manche Jagd ward auf sie gemacht;
doch mutig wehrte das Wolfspaar sich.
 


\direct{zu Hunding gewandt}

Ein Wölfing kündet dir das,
den als ``Wölfing'' mancher wohl kennt.
 

\Hundingspeaks
Wunder und wilde Märe kündest du, kühner Gast,
Wehwalt---der Wölfing!
Mich dünkt, von dem wehrlichen Paar
vernahm ich dunkle Sage,
kannt' ich auch Wolfe und Wölfing nicht.
 

\Sieglindespeaks
Doch weiter künde, Fremder:
wo weilt dein Vater jetzt?
 

\Siegmundspeaks
Ein starkes Jagen auf uns stellten die Neidinge an:
der Jäger viele fielen den Wölfen,
in Flucht durch den Wald
trieb sie das Wild.
Wie Spreu zerstob uns der Feind.
Doch ward ich vom Vater versprengt;
seine Spur verlor ich, je länger ich forschte:
eines Wolfes Fell nur
traf ich im Forst;
leer lag das vor mir, den Vater fand ich nicht.
Aus dem Wald trieb es mich fort;
mich drängt' es zu Männern und Frauen.
Wieviel ich traf, wo ich sie fand,
ob ich um Freund', um Frauen warb,
immer doch war ich geächtet:
Unheil lag auf mir.
Was Rechtes je ich riet, andern dünkte es arg,
was schlimm immer mir schien,
andre gaben ihm Gunst.
In Fehde fiel ich, wo ich mich fand,
Zorn traf mich, wohin ich zog;
gehrt' ich nach Wonne, weckt' ich nur Weh':
drum mußt' ich mich Wehwalt nennen;
des Wehes waltet' ich nur.
 


\direct{Er sieht zu Sieglinde auf und gewahrt ihren teilnehmenden Blick}


\Hundingspeaks
Die so leidig Los dir beschied,
nicht liebte dich die Norn':
froh nicht grüßt dich der Mann,
dem fremd als Gast du nahst.
 

\Sieglindespeaks
Feige nur fürchten den, der waffenlos einsam fährt! -
Künde noch, Gast,
wie du im Kampf zuletzt die Waffe verlorst!
 

\Siegmundspeaks

\direct{immer lebhafter}

Ein trauriges Kind rief mich zum Trutz:
vermählen wollte der Magen Sippe
dem Mann ohne Minne die Maid.
Wider den Zwang zog ich zum Schutz,
der Dränger Troß traf ich im Kampf:
dem Sieger sank der Feind.
Erschlagen lagen die Brüder:
die Leichen umschlang da die Maid,
den Grimm verjagt' ihr der Gram.
Mit wilder Tränen Flut betroff sie weinend die Wal:
um des Mordes der eignen Brüder
klagte die unsel'ge Braut.
Der Erschlagnen Sippen stürmten daher;
übermächtig ächzten nach Rache sie;
rings um die Stätte ragten mir Feinde.
Doch von der Wal wich nicht die Maid;
mit Schild und Speer schirmt' ich sie lang',
bis Speer und Schild im Harst mir zerhaun.
Wund und waffenlos stand ich -
sterben sah ich die Maid:
mich hetzte das wütende Heer -
auf den Leichen lag sie tot.
 


\direct{mit einem Blicke voll schmerzlichen Feuers auf Sieglinde}

Nun weißt du, fragende Frau,
warum ich Friedmund nicht heiße!
 


\direct{Er steht auf und schreitet auf den Herd zu. Sieglinde blickt erbleichend und tief erschüttert zu Boden}


\Hundingspeaks

\direct{erhebt sich, sehr finster}

Ich weiß ein wildes Geschlecht,
nicht heilig ist ihm, was andern hehr:
verhaßt ist es allen und mir.
Zur Rache ward ich gerufen,
Sühne zu nehmen für Sippenblut:
zu spät kam ich, und kehrte nun heim,
des flücht'gen Frevlers Spur im eignen Haus zu erspähn. -
 


\direct{Er geht herab}

Mein Haus hütet, Wölfing, dich heut';
für die Nacht nahm ich dich auf;
mit starker Waffe doch wehre dich morgen;
zum Kampfe kies' ich den Tag:
für Tote zahlst du mir Zoll.
 


\direct{Sieglinde schreitet mit besorgter Gebärde zwischen die beiden Männer vor}


\Hundingspeaks

\direct{barsch}

Fort aus dem Saal! Säume hier nicht!
Den Nachttrunk rüste mir drin und harre mein' zur Ruh'.
 


\direct{Sieglinde steht eine Weile unentschieden und sinnend. Sie wendet sich langsam und zögernden Schrittes nach dem Speicher. Dort hält sie wieder an und bleibt, in Sinnen verloren, mit halb abgewandtem Gesicht stehen. Mit ruhigem Entschluß öffnet sie den Schrein, füllt ein Trinkhorn und schüttet aus einer Büchse Würze hinein. Dann wendet sie das Auge auf Siegmund, um seinem Blicke zu begegnen, den dieser fortwährend auf sie heftet. Sie gewahrt Hundings Spähen und wendet sich sogleich zum Schlafgemach. Auf den Stufen kehrt sie sich noch einmal um, heftet das Auge sehnsuchtsvoll auf Siegmund und deutet mit dem Blicke andauernd und mit sprechender Bestimmtheit auf eine Stelle am Eschenstamme. Hunding fährt auf und treibt sie mit einer heftigen Gebärde zum Fortgehen an. Mit einem letzten Blick auf Siegmund geht sie in das Schlafgemach und schließt hinter sich die Türe.}

 

\Hundingspeaks

\direct{nimmt seine Waffen vom Stamme herab}

Mit Waffen wehrt sich der Mann.
 


\direct{Im Abgehen sich zu Siegmund wendend}

Dich Wölfing treffe ich morgen;
mein Wort hörtest du, hüte dich wohl!
 


\direct{Er geht mit den Waffen in das Gemach; man hört ihn von innen den Riegel schließen}

\scene

\StageDir{Siegmund allein. Es ist vollständig Nacht geworden; der Saal ist nur noch von einem schwachen Feuer im Herde erhellt. Siegmund läßt sich, nah beim Feuer, auf dem Lager nieder und brütet in großer innerer Aufregung eine Zeitlang schweigend vor sich hin}


\Siegmundspeaks
Ein Schwert verhieß mir der Vater,
ich fänd' es in höchster Not.
Waffenlos fiel ich in Feindes Haus;
seiner Rache Pfand, raste ich hier:
ein Weib sah ich, wonnig und hehr:
entzückend Bangen zehrt mein Herz.
Zu der mich nun Sehnsucht zieht,
die mit süßem Zauber mich sehrt,
im Zwange hält sie der Mann,
der mich Wehrlosen höhnt!
Wälse! Wälse! Wo ist dein Schwert?
Das starke Schwert,
das im Sturm ich schwänge,
bricht mir hervor aus der Brust,
was wütend das Herz noch hegt?
 


\direct{Das Feuer bricht zusammen; es fällt aus der aufsprühenden Glut plötzlich ein greller Schein auf die Stelle des Eschenstammes, welche Sieglindes Blick bezeichnet hatte und an der man jetzt deutlich einen Schwertgriff haften sieht}

Was gleißt dort hell im Glimmerschein?
Welch ein Strahl bricht aus der Esche Stamm?
Des Blinden Auge leuchtet ein Blitz:
lustig lacht da der Blick.
Wie der Schein so hehr das Herz mir sengt!
Ist es der Blick der blühenden Frau,
den dort haftend sie hinter sich ließ,
als aus dem Saal sie schied?
 


\direct{Von hier an verglimmt das Herdfeuer allmählich}

Nächtiges Dunkel deckte mein Aug',
ihres Blickes Strahl streifte mich da:
Wärme gewann ich und Tag.
Selig schien mir der Sonne Licht;
den Scheitel umgliß mir ihr wonniger Glanz -
bis hinter Bergen sie sank.
 


\direct{Ein neuer schwacher Aufschein des Feuers}

Noch einmal, da sie schied,
traf mich abends ihr Schein;
selbst der alten Esche Stamm
erglänzte in goldner Glut:
da bleicht die Blüte, das Licht verlischt;
nächtiges Dunkel deckt mir das Auge:
tief in des Busens Berge glimmt nur noch lichtlose Glut.
 


\direct{Das Feuer ist gänzlich verloschen: volle Nacht.  Das Seitengemach öffnet sich leise: Sieglinde, in weißem Gewande, tritt heraus und schreitet leise, doch rasch, auf den Herd zu}

\Sieglindespeaks
Schläfst du, Gast?
 

\Siegmundspeaks

\direct{freudig überrascht aufspringend}

Wer schleicht daher?
 

\Sieglindespeaks

\direct{mit geheimnisvoller Hast}

Ich bin's: höre mich an!
In tiefem Schlaf liegt Hunding;
ich würzt' ihm betäubenden Trank:
nütze die Nacht dir zum Heil!
 

\Siegmundspeaks

\direct{hitzig unterbrechend}

Heil macht mich dein Nah'n!
 

\Sieglindespeaks
Eine Waffe laß mich dir weisen: o wenn du sie gewännst!
Den hehrsten Helden dürft' ich dich heißen:
dem Stärksten allein ward sie bestimmt.
O merke wohl, was ich dir melde!
Der Männer Sippe saß hier im Saal,
von Hunding zur Hochzeit geladen:
er freite ein Weib,
das ungefragt Schächer ihm schenkten zur Frau.
Traurig saß ich, während sie tranken;
ein Fremder trat da herein:
ein Greis in blauem Gewand;
tief hing ihm der Hut,
der deckt' ihm der Augen eines;
doch des andren Strahl, Angst schuf es allen,
traf die Männer sein mächtiges Dräu'n:
mir allein weckte das Auge
süß sehnenden Harm,
Tränen und Trost zugleich.
Auf mich blickt' er und blitzte auf jene,
als ein Schwert in Händen er schwang;
das stieß er nun in der Esche Stamm,
bis zum Heft haftet' es drin:
dem sollte der Stahl geziemen,
der aus dem Stamm es zög'.
Der Männer alle, so kühn sie sich mühten,
die Wehr sich keiner gewann;
Gäste kamen und Gäste gingen,
die stärksten zogen am Stahl -
keinen Zoll entwich er dem Stamm:
dort haftet schweigend das Schwert. -
Da wußt' ich, wer der war,
der mich Gramvolle gegrüßt; ich weiß auch,
wem allein im Stamm das Schwert er bestimmt.
O fänd' ich ihn hier und heut', den Freund;
käm' er aus Fremden zur ärmsten Frau.
Was je ich gelitten in grimmigem Leid,
was je mich geschmerzt in Schande und Schmach, -
süßeste Rache sühnte dann alles!
Erjagt hätt' ich, was je ich verlor,
was je ich beweint, wär' mir gewonnen,
fänd' ich den heiligen Freund,
umfing' den Helden mein Arm!
 

\Siegmundspeaks

\direct{mit Glut Sieglinde umfassend}

Dich selige Frau hält nun der Freund,
dem Waffe und Weib bestimmt!
Heiß in der Brust brennt mir der Eid,
der mich dir Edlen vermählt.
Was je ich ersehnt, ersah ich in dir;
in dir fand ich, was je mir gefehlt!
Littest du Schmach,
und schmerzte mich Leid;
war ich geächtet, und warst du entehrt:
freudige Rache lacht nun den Frohen!
Auf lach' ich in heiliger Lust,
halt' ich dich Hehre umfangen,
fühl' ich dein schlagendes Herz!
 


\direct{Die große Türe springt auf}


\Sieglindespeaks

\direct{fährt erschrocken zusammen und reißt sich los.}

Ha, wer ging? Wer kam herein?
 


\direct{Die Tür bleibt weit geöffnet: außen herrliche Frühlingsnacht; der Vollmond leuchtet herein und wirft sein helles Licht auf das Paar, das so sich plötzlich in voller Deutlichkeit wahrnehmen kann}


\Siegmundspeaks

\direct{in leiser Entzückung}

Keiner ging---doch einer kam:
siehe, der Lenz lacht in den Saal!
 


\direct{Siegmund zieht Sieglinde mit sanfter Gewalt zu sich auf das Lager, so daß sie neben ihm zu sitzen kommt

Wachsende Helligkeit des Mondscheines}

Winterstürme wichen
dem Wonnemond,
in mildem Lichte leuchtet der Lenz;
auf linden Lüften leicht und lieblich,
Wunder webend er sich wiegt;
durch Wald und Auen weht sein Atem,
weit geöffnet lacht sein Aug': -
aus sel'ger Vöglein Sange süß er tönt,
holde Düfte haucht er aus;
seinem warmen Blut entblühen wonnige Blumen,
Keim und Sproß entspringt seiner Kraft.
Mit zarter Waffen Zier bezwingt er die Welt;
Winter und Sturm wichen der starken Wehr:
wohl mußte den tapfern Streichen
die strenge Türe auch weichen,
die trotzig und starr uns trennte von ihm. -
Zu seiner Schwester schwang er sich her;
die Liebe lockte den Lenz:
in unsrem Busen barg sie sich tief;
nun lacht sie selig dem Licht.
Die bräutliche Schwester befreite der Bruder;
zertrümmert liegt, was je sie getrennt:
jauchzend grüßt sich das junge Paar:
vereint sind Liebe und Lenz!
 

\Sieglindespeaks
Du bist der Lenz, nach dem ich verlangte
in frostigen Winters Frist.
Dich grüßte mein Herz mit heiligem Grau'n,
als dein Blick zuerst mir erblühte.
Fremdes nur sah ich von je,
freudlos war mir das Nahe.
Als hätt' ich nie es gekannt, war, was immer mir kam.
Doch dich kannt' ich deutlich und klar:
als mein Auge dich sah,
warst du mein Eigen;
was im Busen ich barg, was ich bin,
hell wie der Tag taucht' es mir auf,
o wie tönender Schall schlug's an mein Ohr,
als in frostig öder Fremde
zuerst ich den Freund ersah.
 


\direct{Sie hängt sich entzückt an seinen Hals und blickt ihm nahe ins Gesicht}


\Siegmundspeaks

\direct{mit Hingerissenheit}

O süßeste Wonne!
O seligstes Weib!
 

\Sieglindespeaks

\direct{dicht an seinen Augen}

O laß in Nähe zu dir mich neigen,
daß hell ich schaue den hehren Schein,
der dir aus Aug' und Antlitz bricht
und so süß die Sinne mir zwingt.
 

\Siegmundspeaks
Im Lenzesmond leuchtest du hell;
hehr umwebt dich das Wellenhaar:
was mich berückt, errat' ich nun leicht,
denn wonnig weidet mein Blick.
 

\Sieglindespeaks

\direct{schlägt ihm die Locken von der Stirn zurück und betrachtet ihn staunend}

Wie dir die Stirn so offen steht,
der Adern Geäst in den Schläfen sich schlingt!
Mir zagt es vor der Wonne, die mich entzückt!
Ein Wunder will mich gemahnen:
den heut' zuerst ich erschaut,
mein Auge sah dich schon!
 

\Siegmundspeaks
Ein Minnetraum gemahnt auch mich:
in heißem Sehnen sah ich dich schon!
 

\Sieglindespeaks
Im Bach erblickt' ich mein eigen Bild -
und jetzt gewahr' ich es wieder:
wie einst dem Teich es enttaucht,
bietest mein Bild mir nun du!
 

\Siegmundspeaks
Du bist das Bild,
das ich in mir barg.
 

\Sieglindespeaks

\direct{den Blick schnell abwendend}

O still! Laß mich der Stimme lauschen:
 

mich dünkt, ihren Klang
hört' ich als Kind.
 


\direct{aufgeregt}

Doch nein! Ich hörte sie neulich,
als meiner Stimme Schall
mir widerhallte der Wald.
 

\Siegmundspeaks
O lieblichste Laute,
denen ich lausche!
 

\Sieglindespeaks

\direct{ihm wieder in die Augen spähend}

Deines Auges Glut erglänzte mir schon:
so blickte der Greis grüßend auf mich,
als der Traurigen Trost er gab.
An dem Blick erkannt' ihn sein Kind -
schon wollt' ich beim Namen ihn nennen!
 


\direct{Sie hält inne und fährt dann leise fort}

Wehwalt heißt du fürwahr?
 

\Siegmundspeaks
Nicht heiß' ich so, seit du mich liebst:
nun walt' ich der hehrsten Wonnen!
 

\Sieglindespeaks
Und Friedmund darfst du
froh dich nicht nennen?
 

\Siegmundspeaks
Nenne mich du, wie du liebst, daß ich heiße:
den Namen nehm' ich von dir!
 

\Sieglindespeaks
Doch nanntest du Wolfe den Vater?
 

\Siegmundspeaks
Ein Wolf war er feigen Füchsen!
Doch dem so stolz strahlte das Auge,
wie, Herrliche, hehr dir es strahlt,
der war: Wälse genannt.
 

\Sieglindespeaks

\direct{außer sich}

War Wälse dein Vater, und bist du ein Wälsung,
stieß er für dich sein Schwert in den Stamm,
so laß mich dich heißen, wie ich dich liebe:
Siegmund---so nenn' ich dich!
 

\Siegmundspeaks

\direct{springt auf den Stamm zu und faßt den Schwertgriff}

Siegmund heiß' ich und Siegmund bin ich!
Bezeug' es dies Schwert, das zaglos ich halte!
Wälse verhieß mir, in höchster Not
fänd' ich es einst: ich faß' es nun!
Heiligster Minne höchste Not,
sehnender Liebe sehrende Not
brennt mir hell in der Brust,
drängt zu Tat und Tod:
Notung! Notung! So nenn' ich dich, Schwert -
Notung! Notung! Neidlicher Stahl!
Zeig' deiner Schärfe schneidenden Zahn:
heraus aus der Scheide zu mir!
 


\direct{Er zieht mit einem gewaltigen Zuck das Schwert aus dem Stamme und zeigt es der von Staunen und Entzücken erfaßten Sieglinde}


Siegmund, den Wälsung, siehst du, Weib!
Als Brautgabe bringt er dies Schwert:
so freit er sich
die seligste Frau;
dem Feindeshaus entführt er dich so.
Fern von hier folge mir nun,
fort in des Lenzes lachendes Haus:
dort schützt dich Notung, das Schwert,
wenn Siegmund dir liebend erlag!
 


\direct{Er hat sie umfaßt, um sie mit sich fortzuziehen}


\Sieglindespeaks

\direct{reißt sich in höchster Trunkenheit von ihm los und stellt sich ihm gegenüber}

Bist du Siegmund, den ich hier sehe,
Sieglinde bin ich, die dich ersehnt:
die eigne Schwester
gewannst du zu eins mit dem Schwert!
 

\Siegmundspeaks
Braut und Schwester bist du dem Bruder -
so blühe denn, Wälsungen-Blut!
 


\direct{Er zieht sie mit wütender Glut an sich; sie sinkt mit einem Schrei an seine Brust. Der Vorhang fällt schnell}


\act

\scene

\StageDir{Wildes Felsengebirge

Im Hintergrund zieht sich von unten her eine Schlucht herauf, die auf ein erhöhtes Felsjoch mündet; von diesem senkt sich der Boden dem Vordergrunde zu wieder abwärts.


\direct{Wotan, kriegerisch gewaffnet, mit dem Speer; vor ihm Brünnhilde, als Walküre, ebenfalls in voller Waffenrüstung}}


\Wotanspeaks
Nun zäume dein Roß, reisige Maid!
Bald entbrennt brünstiger Streit:
Brünnhilde stürme zum Kampf,
dem Wälsung kiese sie Sieg!
Hunding wähle sich, wem er gehört;
nach Walhall taugt er mir nicht.
Drum rüstig und rasch, reite zur Wal!
 

\Brunnhildespeaks

\direct{jauchzend von Fels zu Fels die Höhe rechts hinaufspringend}

Hojotoho! Hojotoho!
Heiaha! Heiaha! Hojotoho! Heiaha!
 


\direct{Sie hält auf einer hohen Felsspitze an, blickt in die hintere Schlucht hinab und ruft zu Wotan zurück}

Dir rat' ich, Vater, rüste dich selbst;
harten Sturm sollst du bestehn.
Fricka naht, deine Frau,
im Wagen mit dem Widdergespann.
Hei! Wie die goldne Geißel sie schwingt!
Die armen Tiere ächzen vor Angst;
wild rasseln die Räder;
zornig fährt sie zum Zank!
In solchem Strauße streit' ich nicht gern,
lieb' ich auch mutiger Männer Schlacht!
Drum sieh, wie den Sturm du bestehst:
ich Lustige laß' dich im Stich!
Hojotoho! Hojotoho!
Heiaha! Heiaha!
Heiahaha!
 


\direct{Brünnhilde verschwindet hinter der Gebirgshöhe zur Seite

In einem mit zwei Widdern bespannten Wagen langt Fricka aus der Schlucht auf dem Felsjoche an, dort hält sie rasch an und steigt aus. Sie schreitet heftig in den Vordergrund auf Wotan zu}

\Wotanspeaks

\direct{Fricka auf sich zuschreiten sehend, für sich}

Der alte Sturm, die alte Müh'!
Doch stand muß ich hier halten!
 

\Frickaspeaks

\direct{je näher sie kommt, desto mehr mäßigt sie den Schritt und stellt sich mit Würde vor Wotan hin}

Wo in den Bergen du dich birgst,
der Gattin Blick zu entgehn,
einsam hier such' ich dich auf,
daß Hilfe du mir verhießest.
 

\Wotanspeaks
Was Fricka kümmert, künde sie frei.
 

\Frickaspeaks
Ich vernahm Hundings Not,
um Rache rief er mich an:
der Ehe Hüterin hörte ihn,
verhieß streng zu strafen die Tat
des frech frevelnden Paars,
das kühn den Gatten gekränkt.
 

\Wotanspeaks
Was so Schlimmes schuf das Paar,
das liebend einte der Lenz?
Der Minne Zauber entzückte sie:
wer büßt mir der Minne Macht?
 

\Frickaspeaks
Wie töricht und taub du dich stellst,
als wüßtest fürwahr du nicht,
daß um der Ehe heiligen Eid,
den hart gekränkten, ich klage!
 

\Wotanspeaks
Unheilig acht' ich den Eid,
der Unliebende eint;
und mir wahrlich mute nicht zu,
daß mit Zwang ich halte, was dir nicht haftet:
denn wo kühn Kräfte sich regen,
da rat' ich offen zum Krieg.
 

\Frickaspeaks
Achtest du rühmlich der Ehe Bruch,
so prahle nun weiter und preis' es heilig,
daß Blutschande entblüht
dem Bund eines Zwillingspaars!
Mir schaudert das Herz, es schwindelt mein Hirn:
bräutlich umfing die Schwester der Bruder!
Wann ward es erlebt,
daß leiblich Geschwister sich liebten?
 

\Wotanspeaks
Heut' hast du's erlebt!
Erfahre so, was von selbst sich fügt,
sei zuvor auch noch nie es geschehn.
Daß jene sich lieben, leuchtet dir hell;
drum höre redlichen Rat:
Soll süße Lust deinen Segen dir lohnen,
so segne, lachend der Liebe,
Siegmunds und Sieglindes Bund!
 

\Frickaspeaks

\direct{in höchste Entrüstung ausbrechend}

So ist es denn aus mit den ewigen Göttern,
seit du die wilden Wälsungen zeugtest?
Heraus sagt' ich's; traf ich den Sinn?
Nichts gilt dir der Hehren heilige Sippe;
hin wirfst du alles, was einst du geachtet;
zerreißest die Bande, die selbst du gebunden,
lösest lachend des Himmels Haft: -
daß nach Lust und Laune nur walte
dies frevelnde Zwillingspaar,
deiner Untreue zuchtlose Frucht!
O, was klag' ich um Ehe und Eid,
da zuerst du selbst sie versehrt!
Die treue Gattin trogest du stets;
wo eine Tiefe, wo eine Höhe,
dahin lugte lüstern dein Blick,
wie des Wechsels Lust du gewännest
und höhnend kränktest mein Herz.
Trauernden Sinnes mußt' ich's ertragen,
zogst du zur Schlacht mit den schlimmen Mädchen,
die wilder Minne Bund dir gebar:
denn dein Weib noch scheutest du so,
daß der Walküren Schar
und Brünnhilde selbst, deines Wunsches Braut,
in Gehorsam der Herrin du gabst.
Doch jetzt, da dir neue
Namen gefielen,
als ``Wälse'' wölfisch im Walde du schweiftest;
jetzt, da zu niedrigster
Schmach du dich neigtest,
gemeiner Menschen ein Paar zu erzeugen:
jetzt dem Wurfe der Wölfin
wirfst du zu Füßen dein Weib!
So führ' es denn aus! Fülle das Maß!
Die Betrogne laß auch zertreten!
 

\Wotanspeaks

\direct{ruhig}

Nichts lerntest du, wollt' ich dich lehren,
was nie du erkennen kannst,
eh' nicht ertagte die Tat.
Stets Gewohntes nur magst du verstehn:
doch was noch nie sich traf,
danach trachtet mein Sinn.
Eines höre! Not tut ein Held,
der, ledig göttlichen Schutzes,
sich löse vom Göttergesetz.
So nur taugt er zu wirken die Tat,
die, wie not sie den Göttern,
dem Gott doch zu wirken verwehrt.
 

\Frickaspeaks
Mit tiefem Sinne willst du mich täuschen:
was Hehres sollten Helden je wirken,
das ihren Göttern wäre verwehrt,
deren Gunst in ihnen nur wirkt?
 

\Wotanspeaks
lhres eignen Mutes achtest du nicht?
 

\Frickaspeaks
Wer hauchte Menschen ihn ein?
Wer hellte den Blöden den Blick?
In deinem Schutz scheinen sie stark,
durch deinen Stachel streben sie auf:
du reizest sie einzig,
die so mir Ew'gen du rühmst,
Mit neuer List willst du mich belügen,
durch neue Ränke
mir jetzt entrinnen;
doch diesen Wälsung gewinnst du dir nicht:
in ihm treff' ich nur dich,
denn durch dich trotzt er allein.
 

\Wotanspeaks

\direct{ergriffen}

In wildem Leiden erwuchs er sich selbst:
mein Schutz schirmte ihn nie.
 

\Frickaspeaks
So schütz' auch heut' ihn nicht!
Nimm ihm das Schwert, das du ihm geschenkt!
 

\Wotanspeaks
Das Schwert?
 

\Frickaspeaks
Ja, das Schwert,
das zauberstark zuckende Schwert,
das du Gott dem Sohne gabst.
 

\Wotanspeaks

\direct{heftig}

Siegmund gewann es sich

\direct{mit unterdrücktem Beben}

selbst in der Not.
 


\direct{Wotan drückt in seiner ganzen Haltung von hier an einen immer wachsenden unheimlichen, tiefen Unmut aus}


\Frickaspeaks

\direct{eifrig fortfahrend}

Du schufst ihm die Not,
wie das neidliche Schwert.
Willst du mich täuschen,
die Tag und Nacht auf den Fersen dir folgt?
Für ihn stießest du das Schwert in den Stamm,
du verhießest ihm die hehre Wehr:
willst du es leugnen,
daß nur deine List
ihn lockte, wo er es fänd'?
 


\direct{Wotan fährt mit einer grimmigen Gebärde auf}


\Frickaspeaks

\direct{immer sicherer, da sie den Eindruck gewahrt, den sie auf Wotan hervorgebracht hat}

Mit Unfreien streitet kein Edler,
den Frevler straft nur der Freie.
Wider deine Kraft
führt' ich wohl Krieg:
doch Siegmund verfiel mir als Knecht!
 


\direct{Neue heftige Gebärde Wotans, dann Versinken in das Gefühl seiner Ohnmacht}

Der dir als Herren hörig und eigen,
gehorchen soll ihm dein ewig Gemahl?
Soll mich in Schmach der Niedrigste schmähen,
dem Frechen zum Sporn,
dem Freien zum Spott?
Das kann mein Gatte nicht wollen,
die Göttin entweiht er nicht so!
 

\Wotanspeaks

\direct{finster}

Was verlangst du?
 

\Frickaspeaks
Laß von dem Wälsung!
 

\Wotanspeaks

\direct{mit gedämpfter Stimme}

Er geh' seines Wegs.
 

\Frickaspeaks
Doch du schütze ihn nicht,
wenn zur Schlacht ihn der Rächer ruft!
 

\Wotanspeaks
Ich schütze ihn nicht.
 

\Frickaspeaks
Sieh mir ins Auge, sinne nicht Trug:
die Walküre wend' auch von ihm!
 

\Wotanspeaks
Die Walküre walte frei.
 

\Frickaspeaks
Nicht doch; deinen Willen vollbringt sie allein:
verbiete ihr Siegmunds Sieg!
 

\Wotanspeaks

\direct{in heftigen inneren Kampf ausbrechend}

Ich kann ihn nicht fällen: er fand mein Schwert!
 

\Frickaspeaks
Entzieh' dem den Zauber, zerknick' es dem Knecht!
Schutzlos schau' ihn der Feind!
 

\Brunnhildespeaks

\direct{noch unsichtbar von der Höhe her}

Heiaha! Heiaha! Hojotoho!
 

\Frickaspeaks
Dort kommt deine kühne Maid;
jauchzend jagt sie daher.
 

\Brunnhildespeaks

\direct{wie oben}

Heiaha! Heiaha! Heiohotojo! Hotojoha!
 

\Wotanspeaks

\direct{dumpf für sich}

Ich rief sie für Siegmund zu Roß!
 


\direct{Brünnhilde erscheint mit ihrem Roß auf dem Felsenpfade rechts. Als sie Fricka gewahrt, bricht sie schnell ab und geleitet ihr Roß still und langsam während des Folgenden den Felsweg herab: dort birgt sie es dann in einer Höhle}


\Frickaspeaks
Deiner ew'gen Gattin heilige Ehre
beschirme heut' ihr Schild!
Von Menschen verlacht, verlustig der Macht,
gingen wir Götter zugrund:
würde heut' nicht hehr und herrlich mein Recht
gerächt von der mutigen Maid.
Der Wälsung fällt meiner Ehre:
Empfah' ich von Wotan den Eid?
 

\Wotanspeaks

\direct{in furchtbarem Unmut und innerem Grimm auf einen Felsensitz sich werfend}

Nimm den Eid!
 


\direct{Fricka schreitet dem Hintergrunde zu: dort begegnet sie Brünnhilde und hält einen Augenblick vor ihr an}


\Frickaspeaks

\direct{zu Brünnhilde}

Heervater harret dein:
lass' ihn dir künden, wie das Los er gekiest!
 


\direct{Sie besteigt den Wagen und fährt schnell davon}

\direct{Brünnhilde tritt mit besorgter Miene verwundert vor Wotan, der, auf dem Felssitz zurückgelehnt, das Haupt auf die Hand gestützt, in finstres Brüten versunken ist}

\scene

\Brunnhildespeaks
Schlimm, fürcht' ich, schloß der Streit,
lachte Fricka dem Lose.
Vater, was soll dein Kind erfahren?
Trübe scheinst du und traurig!

\Wotanspeaks

\direct{läßt den Arm machtlos sinken und den Kopf in den Nacken fallen}

In eigner Fessel fing ich mich:
ich Unfreiester aller!
 

\Brunnhildespeaks
So sah ich dich nie!
Was nagt dir das Herz?
 

\Wotanspeaks

\direct{von hier an steigert sich Wotans Ausdruck und Gebärde bis zum furchtbarsten Ausbruch}

O heilige Schmach! O schmählicher Harm!
Götternot! Götternot!
Endloser Grimm! Ewiger Gram!
Der Traurigste bin ich von allen!
 

\Brunnhildespeaks

\direct{wirft erschrocken Schild, Speer und Helm von sich und läßt sich mit besorgter Zutraulichkeit zu Wotans Füßen nieder}

Vater! Vater! Sage, was ist dir?
Wie erschreckst du mit Sorge dein Kind?
Vertraue mir! Ich bin dir treu:
sieh, Brünnhilde bittet!
 


\direct{Sie legt traulich und ängstlich Haupt und Hände ihm auf Knie und Schoß}


\Wotanspeaks

\direct{blickt ihr lange ins Auge; dann streichelt er ihr mit unwillkürlicher Zärtlichkeit die Locken. Wie aus tiefem Sinnen zu sich kommend, beginnt er endlich sehr leise}

Lass' ich's verlauten,
lös' ich dann nicht meines Willens haltenden Haft?
 

\Brunnhildespeaks

\direct{ihm ebenso erwidernd}

Zu Wotans Willen sprichst du,
sagst du mir, was du willst;
wer bin ich, wär' ich dein Wille nicht?
 

\Wotanspeaks

\direct{sehr leise}

Was keinem in Worten ich künde,
unausgesprochen bleib' es denn ewig:
mit mir nur rat' ich, red' ich zu dir. -
 


\direct{mit noch gedämpfterer, schauerlicher Stimme, während er Brünnhilde unverwandt in das Auge blickt}

Als junger Liebe Lust mir verblich,
verlangte nach Macht mein Mut:
von jäher Wünsche Wüten gejagt,
gewann ich mir die Welt.
Unwissend trugvoll, Untreue übt' ich,
band durch Verträge, was Unheil barg:
listig verlockte mich Loge,
der schweifend nun verschwand.
Von der Liebe doch mocht' ich nicht lassen,
in der Macht verlangt' ich nach Minne.
Den Nacht gebar, der bange Nibelung,
Alberich, brach ihren Bund;
er fluchte der Lieb' und gewann durch den Fluch
des Rheines glänzendes Gold
und mit ihm maßlose Macht.
Den Ring, den er schuf,
entriß ich ihm listig;
doch nicht dem Rhein gab ich ihn zurück:
mit ihm bezahlt' ich Walhalls Zinnen,
der Burg, die Riesen mir bauten,
aus der ich der Welt nun gebot.
Die alles weiß, was einstens war,
Erda, die weihlich weiseste Wala,
riet mir ab von dem Ring,
warnte vor ewigem Ende.
Von dem Ende wollt' ich mehr noch wissen;
doch schweigend entschwand mir das Weib...
Da verlor ich den leichten Mut,
zu wissen begehrt' es den Gott:
in den Schoß der Welt schwang ich mich hinab,
mit Liebeszauber zwang ich die Wala,
stört' ihres Wissens Stolz, daß sie Rede nun mir stand.
Kunde empfing ich von ihr;
von mir doch barg sie ein Pfand:
der Welt weisestes Weib
gebar mir, Brünnhilde, dich.
Mit acht Schwestern zog ich dich auf;
durch euch Walküren wollt' ich wenden,
was mir die Wala zu fürchten schuf:
ein schmähliches Ende der Ew'gen.
Daß stark zum Streit uns fände der Feind,
hieß ich euch Helden mir schaffen:
die herrisch wir sonst
in Gesetzen hielten,
die Männer, denen den Mut wir gewehrt,
die durch trüber Verträge trügende Bande
zu blindem Gehorsam wir uns gebunden,
die solltet zu Sturm
und Streit ihr nun stacheln,
ihre Kraft reizen zu rauhem Krieg,
daß kühner Kämpfer Scharen
ich sammle in Walhalls Saal!
 

\Brunnhildespeaks
Deinen Saal füllten wir weidlich:
viele schon führt' ich dir zu.
Was macht dir nun Sorge, da nie wir gesäumt?
 

\Wotanspeaks

\direct{wieder gedämpfter}

Ein andres ist's:
achte es wohl, wes mich die Wala gewarnt!
Durch Alberichs Heer
droht uns das Ende:
mit neidischem Grimm grollt mir der Niblung:
doch scheu' ich nun nicht seine nächtigen Scharen,
meine Helden schüfen mir Sieg.
Nur wenn je den Ring
zurück er gewänne,
dann wäre Walhall verloren:
der der Liebe fluchte, er allein
nützte neidisch des Ringes Runen
zu aller Edlen endloser Schmach:
der Helden Mut entwendet' er mir;
die Kühnen selber
zwäng' er zum Kampf;
mit ihrer Kraft bekriegte er mich.
Sorgend sann ich nun selbst,
den Ring dem Feind zu entreißen.
Der Riesen einer, denen ich einst
mit verfluchtem Gold den Fleiß vergalt:
Fafner hütet den Hort,
um den er den Bruder gefällt.
Ihm müßt' ich den Reif entringen,
den selbst als Zoll ich ihm zahlte.
Doch mit dem ich vertrug,
ihn darf ich nicht treffen;
machtlos vor ihm erläge mein Mut: -
das sind die Bande, die mich binden:
der durch Verträge ich Herr,
den Verträgen bin ich nun Knecht.
Nur einer könnte, was ich nicht darf:
ein Held, dem helfend nie ich mich neigte;
der fremd dem Gotte, frei seiner Gunst,
unbewußt, ohne Geheiß,
aus eigner Not, mit der eignen Wehr
schüfe die Tat, die ich scheuen muß,
die nie mein Rat ihm riet,
wünscht sie auch einzig mein Wunsch!
Der, entgegen dem Gott, für mich föchte,
den freundlichen Feind, wie fände ich ihn?
Wie schüf' ich den Freien, den nie ich schirmte,
der im eignen Trotze der Trauteste mir?
Wie macht' ich den andren, der nicht mehr ich,
und aus sich wirkte, was ich nur will?
O göttliche Not! Gräßliche Schmach!
Zum Ekel find' ich ewig nur mich
in allem, was ich erwirke!
Das andre, das ich ersehne,
das andre erseh' ich nie:
denn selbst muß der Freie sich schaffen:
Knechte erknet' ich mir nur!
 

\Brunnhildespeaks
Doch der Wälsung, Siegmund, wirkt er nicht selbst?
 

\Wotanspeaks
Wild durchschweift' ich mit ihm die Wälder;
gegen der Götter Rat reizte kühn ich ihn auf:
gegen der Götter Rache
schützt ihn nun einzig das Schwert,
 


\direct{gedehnt und bitter}

das eines Gottes Gunst ihm beschied.
Wie wollt' ich listig selbst mich belügen?
So leicht ja entfrug mir Fricka den Trug:
zu tiefster Scham durchschaute sie mich!
Ihrem Willen muß ich gewähren.
 

\Brunnhildespeaks
So nimmst du von Siegmund den Sieg?
 

\Wotanspeaks
Ich berührte Alberichs Ring,
gierig hielt ich das Gold!
Der Fluch, den ich floh,
nicht flieht er nun mich:
Was ich liebe, muß ich verlassen,
morden, wen je ich minne,
trügend verraten, wer mir traut!
 


\direct{Wotans Gebärde geht aus dem Ausdruck des furchtbarsten Schmerzes zu dem der Verzweiflung über}


Fahre denn hin, herrische Pracht,
göttlichen Prunkes prahlende Schmach!
Zusammenbreche, was ich gebaut!
Auf geb' ich mein Werk; nur eines will ich noch:
das Ende,
das Ende! -
 


\direct{Er hält sinnend ein}

Und für das Ende sorgt Alberich!
Jetzt versteh' ich den stummen Sinn
des wilden Wortes der Wala:
``Wenn der Liebe finstrer Feind
zürnend zeugt einen Sohn,
der Sel'gen Ende säumt dann nicht!''
Vom Niblung jüngst vernahm ich die Mär',
daß ein Weib der Zwerg bewältigt,
des' Gunst Gold ihm erzwang:
Des Hasses Frucht hegt eine Frau,
des Neides Kraft kreißt ihr im Schoß:
das Wunder gelang dem Liebelosen;
doch der in Lieb' ich freite,
den Freien erlang' ich mir nicht.
 


\direct{mit bitterem Grimm sich aufrichtend}

So nimm meinen Segen, Niblungen-Sohn!
Was tief mich ekelt, dir geb' ich's zum Erbe,
der Gottheit nichtigen Glanz:
zernage ihn gierig dein Neid!
 

\Brunnhildespeaks

\direct{erschrocken}

O sag', künde, was soll nun dein Kind?
 

\Wotanspeaks

\direct{bitter}

Fromm streite für Fricka; hüte ihr Eh' und Eid!

\direct{trocken}

Was sie erkor, das kiese auch ich:
was frommte mir eigner Wille?
Einen Freien kann ich nicht wollen:
für Frickas Knechte kämpfe nun du!
 

\Brunnhildespeaks
Weh'! Nimm reuig zurück das Wort!
Du liebst Siegmund;
dir zulieb', ich weiß es, schütz' ich den Wälsung.
 

\Wotanspeaks
Fällen sollst du Siegmund,
für Hunding erfechten den Sieg!
Hüte dich wohl und halte dich stark,
all deiner Kühnheit entbiete im Kampf:
ein Siegschwert schwingt Siegmund; -
schwerlich fällt er dir feig!
 

\Brunnhildespeaks
Den du zu lieben stets mich gelehrt,

\direct{sehr warm}

der in hehrer Tugend dem Herzen dir teuer,
gegen ihn zwingt mich nimmer dein zwiespältig Wort!
 

\Wotanspeaks
Ha, Freche du! Frevelst du mir?
Wer bist du, als meines Willens
blind wählende Kür?
Da mit dir ich tagte, sank ich so tief,
daß zum Schimpf der eignen
Geschöpfe ich ward?
Kennst du, Kind, meinen Zorn?
Verzage dein Mut,
wenn je zermalmend
auf dich stürzte sein Strahl!
In meinem Busen berg' ich den Grimm,
der in Grau'n und Wust wirft eine Welt,
die einst zur Lust mir gelacht:
wehe dem, den er trifft!
Trauer schüf' ihm sein Trotz!
Drum rat' ich dir, reize mich nicht!
Besorge, was ich befahl:
Siegmund falle -
Dies sei der Walküre Werk!
 


\direct{er stürmt fort und verschwindet schnell links in Gebirge}


\Brunnhildespeaks

\direct{steht lange erschrocken und betäubt}

So sah ich Siegvater nie,

\direct{sie starrt wild vor sich hin}

erzürnt' ihn sonst wohl auch ein Zank!

\direct{Sie neigt sich betrübt und nimmt ihre Waffen auf, mit denen sie sich wieder rüstet}

Schwer wiegt mir der Waffen Wucht: -
wenn nach Lust ich focht,
wie waren sie leicht!
Zu böser Schlacht schleich' ich heut' so bang.

\direct{Sie sinnt vor sich hin und seufzt dann auf}

Weh', mein Wälsung!
Im höchsten Leid
muß dich treulos die Treue verlassen!

\direct{Sie wendet sich langsam dem Hintergrunde zu}

\scene

\StageDir{Auf dem Bergjoch angelangt, gewahrt Brünnhilde, in die Schlucht hinabblickend, Siegmund und Sieglinde; sie betrachtet die Nahenden einen Augenblick und wendet sich dann in die Höhle zu ihrem Roß, so daß sie dem Zuschauer gänzlich verschwindet. Siegmund und Sieglinde erscheinen auf dem Bergjoche. Sieglinde schreitet hastig voraus; Siegmund sucht sie aufzuhalten}


\Siegmundspeaks
Raste nun hier; gönne dir Ruh'!

\Sieglindespeaks
Weiter! Weiter!
 

\Siegmundspeaks

\direct{umfaßt sie mit sanfter Gewalt}

Nicht weiter nun!
 


\direct{Er schließt sie fest an sich}

Verweile, süßestes Weib!
Aus Wonne-Entzücken zucktest du auf,
mit jäher Hast jagtest du fort:
kaum folgt' ich der wilden Flucht;
durch Wald und Flur, über Fels und Stein,
sprachlos, schweigend sprangst du dahin,
kein Ruf hielt dich zur Rast!
 

Ruhe nun aus: rede zu mir!
Ende des Schweigens Angst!
Sieh, dein Bruder hält seine Braut:
Siegmund ist dir Gesell'!
 


\direct{Er hat sie unvermerkt nach dem Steinsitze geleitet}


\Sieglindespeaks

\direct{blickt Siegmund mit wachsendem Entzücken in die Augen, dann umschlingt sie leidenschaftlich seinen Hals und verweilt so; dann fährt sie mit jähem Schreck auf}

Hinweg! Hinweg! Flieh' die Entweihte!
Unheilig umfängt dich ihr Arm;
entehrt, geschändet schwand dieser Leib:
flieh' die Leiche, lasse sie los!
Der Wind mag sie verwehn,
die ehrlos dem Edlen sich gab!
Da er sie liebend umfing,
da seligste Lust sie fand,
da ganz sie minnte der Mann,
der ganz ihre Minne geweckt:
vor der süßesten Wonne heiligster Weihe,
die ganz ihr Sinn und Seele durchdrang,
Grauen und Schauder ob gräßlichster Schande
mußte mit Schreck die Schmähliche fassen,
die je dem Manne gehorcht,
der ohne Minne sie hielt!
Laß die Verfluchte, laß sie dich fliehn!
Verworfen bin ich, der Würde bar!
Dir reinstem Manne muß ich entrinnen,
dir Herrlichem darf ich nimmer gehören.
Schande bring' ich dem Bruder,
Schmach dem freienden Freund!
 

\Siegmundspeaks
Was je Schande dir schuf,
das büßt nun des Frevlers Blut!
Drum fliehe nicht weiter; harre des Feindes;
hier soll er mir fallen:
wenn Notung ihm das Herz zernagt,
Rache dann hast du erreicht!
 

\Sieglindespeaks

\direct{schrickt auf und lauscht}

Horch! Die Hörner, hörst du den Ruf?
Ringsher tönt wütend Getös':
aus Wald und Gau gellt es herauf.
Hunding erwachte aus hartem Schlaf!
Sippen und Hunde ruft er zusammen;
mutig gehetzt heult die Meute,
wild bellt sie zum Himmel
um der Ehe gebrochenen Eid!
 


\direct{Sieglinde starrt wie wahnsinnig vor sich hin}

Wo bist du, Siegmund? Seh' ich dich noch,
brünstig geliebter, leuchtender Bruder?
Deines Auges Stern laß noch einmal mir strahlen:
wehre dem Kuß des verworfnen Weibes nicht! -
 


\direct{Sie hat sich ihm schluchzend an die Brust geworfen: dann schrickt sie ängstlich wieder auf}

Horch! O horch! Das ist Hundings Horn!
Seine Meute naht mit mächt'ger Wehr:
kein Schwert frommt
vor der Hunde Schwall:
wirf es fort, Siegmund! Siegmund! Wo bist du?
Ha dort! Ich sehe dich! Schrecklich Gesicht!
Rüden fletschen die Zähne nach Fleisch;
sie achten nicht deines edlen Blicks;
bei den Füßen packt dich das feste Gebiß -
du fällst in Stücken zerstaucht das Schwert:
die Esche stürzt, es bricht der Stamm!
Bruder! Mein Bruder! Siegmund! Ha!
 


\direct{Sie sinkt ohnmächtig in Siegmunds Arme}


\Siegmundspeaks
Schwester! Geliebte!
 


\StageDir{Er lauscht ihrem Atem und überzeugt sich, daß sie noch lebe. Er läßt sie an sich herabgleiten, so daß sie, als er sich selbst zum Sitze niederläßt, mit ihrem Haupt auf seinem Schoß zu ruhen kommt. In dieser Stellung verbleiben beide bis zum Schlusse des folgenden Auftrittes.

Langes Schweigen, währenddessen Siegmund mit zärtlicher Sorge über Sieglinde sich hinneigt und mit einem langen Kusse ihr die Stirne küßt.

Brünnhilde, ihr Roß am Zaume geleitend, tritt aus der Höhle und schreitet langsam und feierlich nach vorne. Sie hält an und betrachtet Siegmund von fern. Sie schreitet wieder langsam vor. Sie hält in größerer Nähe an. Sie trägt Schild und Speer in der einen Hand, lehnt sich mit der andern an den Hals des Rosses und betrachtet so mit ernster Miene Siegmund}


\Brunnhildespeaks
Siegmund! Sieh auf mich!
Ich bin's, der bald du folgst.

\Siegmundspeaks

\direct{richtet den Blick zu ihr auf}

Wer bist du, sag',
die so schön und ernst mir erscheint?
 

\Brunnhildespeaks
Nur Todgeweihten taugt mein Anblick;
wer mich erschaut, der scheidet vom Lebenslicht.
Auf der Walstatt allein erschein' ich Edlen:
wer mich gewahrt, zur Wal kor ich ihn mir!
 

\Siegmundspeaks

\direct{blickt ihr lange forschend und fest in das Auge, senkt dann sinnend das Haupt und wendet sich endlich mit feierlichem Ernste wieder zu ihr}

Der dir nun folgt, wohin führst du den Helden?
 

\Brunnhildespeaks
Zu Walvater, der dich gewählt,
führ' ich dich: nach Walhall folgst du mir.
 

\Siegmundspeaks
In Walhalls Saal Walvater find' ich allein?
 

\Brunnhildespeaks
Gefallner Helden hehre Schar
umfängt dich hold mit hoch-heiligem Gruß.
 

\Siegmundspeaks
Fänd' ich in Walhall Wälse, den eignen Vater?
 

\Brunnhildespeaks
Den Vater findet der Wälsung dort.
 

\Siegmundspeaks
Grüßt mich in Walhall froh eine Frau?
 

\Brunnhildespeaks
Wunschmädchen walten dort hehr:
Wotans Tochter reicht dir traulich den Trank!
 

\Siegmundspeaks
Hehr bist du,
und heilig gewahr' ich das Wotanskind:
doch eines sag' mir, du Ew'ge!
Begleitet den Bruder die bräutliche Schwester?
Umfängt Siegmund Sieglinde dort?
 

\Brunnhildespeaks
Erdenluft muß sie noch atmen:
Sieglinde sieht Siegmund dort nicht!
 

\Siegmundspeaks

\direct{neigt sich sanft über Sieglinde, küßt sie leise auf die Stirn und wendet sich ruhig wieder zu Brünnhilde}

So grüße mir Walhall, grüße mir Wotan,
grüße mir Wälse und alle Helden,
grüß' auch die holden Wunschesmädchen: -
 


\direct{sehr bestimmt}

zu ihnen folg' ich dir nicht.
 

\Brunnhildespeaks
Du sahest der Walküre sehrenden Blick:
mit ihr mußt du nun ziehn!
 

\Siegmundspeaks
Wo Sieglinde lebt in Lust und Leid,
da will Siegmund auch säumen:
noch machte dein Blick nicht mich erbleichen:
vom Bleiben zwingt er mich nie.
 

\Brunnhildespeaks
Solang du lebst, zwäng' dich wohl nichts:
doch zwingt dich Toren der Tod:
ihn dir zu künden kam ich her.
 

\Siegmundspeaks
Wo wäre der Held, dem heut' ich fiel?
 

\Brunnhildespeaks
Hunding fällt dich im Streit.
 

\Siegmundspeaks
Mit Stärkrem drohe,
als Hundings Streichen!
Lauerst du hier lüstern auf Wal,
jenen kiese zum Fang:
ich denk ihn zu fällen im Kampf!
 

\Brunnhildespeaks

\direct{den Kopf schüttelnd}

Dir, Wälsung, höre mich wohl:
dir ward das Los gekiest.
 

\Siegmundspeaks
Kennst du dies Schwert?
Der mir es schuf, beschied mir Sieg:
deinem Drohen trotz' ich mit ihm!
 

\Brunnhildespeaks

\direct{mit stark erhobener Stimme}

Der dir es schuf, beschied dir jetzt Tod:
seine Tugend nimmt er dem Schwert!
 

\Siegmundspeaks

\direct{heftig}

Schweig, und schrecke die Schlummernde nicht!
 


\direct{Er beugt sich mit hervorbrechendem Schmerze zärtlich über Sieglinde}

Weh! Weh! Süßestes Weib!
Du traurigste aller Getreuen!
Gegen dich wütet in Waffen die Welt:
und ich, dem du einzig vertraut,
für den du ihr einzig getrotzt,
mit meinem Schutz nicht soll ich dich schirmen,
die Kühne verraten im Kampf?
Ha, Schande ihm, der das Schwert mir schuf,
beschied er mir Schimpf für Sieg!
Muß ich denn fallen, nicht fahr' ich nach Walhall:
Hella halte mich fest!
 


\direct{Er neigt sich tief zu Sieglinde}


\Brunnhildespeaks

\direct{erschüttert}

So wenig achtest du ewige Wonne?
 


\direct{zögernd und zurückhaltend}

Alles wär' dir das arme Weib,
das müd' und harmvoll matt von dem Schoße dir hängt?
Nichts sonst hieltest du hehr?
 

\Siegmundspeaks

\direct{bitter zu ihr aufblickend}

So jung und schön erschimmerst du mir:
doch wie kalt und hart erkennt dich mein Herz!
Kannst du nur höhnen, so hebe dich fort,
du arge, fühllose Maid!
Doch mußt du dich weiden an meinem Weh',
mein Leiden letze dich denn;
meine Not labe dein neidvolles Herz:
nur von Walhalls spröden Wonnen
sprich du wahrlich mir nicht!
 

\Brunnhildespeaks
Ich sehe die Not, die das Herz dir zernagt,
ich fühle des Helden heiligen Harm -
Siegmund, befiehl mir dein Weib:
mein Schutz umfange sie fest!
 

\Siegmundspeaks
Kein andrer als ich soll die Reine lebend berühren:
verfiel ich dem Tod, die Betäubte töt' ich zuvor!
 

\Brunnhildespeaks

\direct{in wachsender Ergriffenheit}

Wälsung! Rasender! Hör' meinen Rat:
befiehl mir dein Weib um des Pfandes willen,
das wonnig von dir es empfing!
 

\Siegmundspeaks

\direct{sein Schwert ziehend}

Dies Schwert, das dem Treuen ein Trugvoller schuf;
dies Schwert, das feig vor dem Feind mich verrät:
frommt es nicht gegen den Feind,
so fromm' es denn wider den Freund! -
 


\direct{Er zückt das Schwert auf Sieglinde}

Zwei Leben lachen dir hier:
nimm sie, Notung, neidischer Stahl!
Nimm sie mit einem Streich!
 

\Brunnhildespeaks

\direct{im heftigsten Sturme des Mitgefühls}

Halt' ein Wälsung! Höre mein Wort!
Sieglinde lebe, und Siegmund lebe mit ihr!
Beschlossen ist's; das Schlachtlos wend' ich:
dir, Siegmund, schaff' ich Segen und Sieg!
 


\direct{Man hört aus dem fernen Hintergrunde Hornrufe erschallen}

Hörst du den Ruf? Nun rüste dich, Held!
Traue dem Schwert und schwing' es getrost:
treu hält dir die Wehr,
wie die Walküre treu dich schützt!
Leb' wohl, Siegmund, seligster Held!
Auf der Walstatt seh' ich dich wieder!
 


\direct{Sie stürmt fort und verschwindet mit dem Rosse rechts in einer Seitenschlucht. Siegmund blickt ihr freudig und erhoben nach. Die Bühne hat sich allmählich verfinstert; schwere Gewitterwolken senken sich auf den Hintergrund herab und hüllen die Gebirgswände, die Schlucht und das erhöhte Bergjoch nach und nach gänzlich ein}


\Siegmundspeaks

\direct{neigt sich wieder über Sieglinde, dem Atem lauschend}

Zauberfest bezähmt ein Schlaf
der Holden Schmerz und Harm.
Da die Walküre zu mir trat,
schuf sie ihr den wonnigen Trost?
Sollte die grimmige Wal
nicht schrecken ein gramvolles Weib?
Leblos scheint sie, die dennoch lebt:
der Traurigen kost ein lächelnder Traum. -


\direct{Neue Hornrufe}

So schlummre nun fort,
bis die Schlacht gekämpft,
und Friede dich erfreu'!
 


\direct{Er legt sie sanft auf den Steinsitz und küßt ihr zum Abschied die Stirne. Siegmund vernimmt Hundings Hornruf und bricht entschlossen auf}

Der dort mich ruft, rüste sich nun;
was ihm gebührt, biet' ich ihm:
Notung zahl' ihm den Zoll!
 


\direct{Er zieht das Schwert, eilt dem Hintergrunde zu und verschwindet, auf dem Joche angekommen, sogleich in finstrem Gewittergewölk, aus welchem alsbald Wetterleuchten aufblitzt}


\Sieglindespeaks

\direct{beginnt sich träumend unruhiger zu bewegen}

Kehrte der Vater nur heim!
Mit dem Knaben noch weilt er im Wald.
Mutter! Mutter! Mir bangt der Mut:
nicht freund und friedlich scheinen die Fremden!
Schwarze Dämpfe---schwüles Gedünst---
feurige Lohe leckt schon nach uns---
es brennt das Haus---zu Hilfe, Bruder!
Siegmund! Siegmund!
 


\direct{Sie springt auf. Starker Blitz und Donner}

Siegmund---Ha!
 


\direct{Sie starrt in Angst um sich her: fast die ganze Bühne ist in schwarze Gewitterwolken gehüllt, fortwährender Blitz und Donner. Der Hornruf Hundings ertönt in der Nähe}


\speaker{\Hunding (Stimme)}

\direct{im Hintergrunde vom Bergjoche her}

Wehwalt! Wehwalt!
Steh' mir zum Streit, sollen dich Hunde nicht halten!
 

\speaker{\Siegmund (Stimme)}

\direct{von weiter hinten her aus der Schlucht}

Wo birgst du dich, daß ich vorbei dir schoß?
Steh', daß ich dich stelle!
 

\Sieglindespeaks

\direct{in furchtbarer Aufregung lauschend}

Hunding! Siegmund!
Könnt' ich sie sehen!
 

\Hundingspeaks
Hieher, du frevelnder Freier!
Fricka fälle dich hier!
 

\Siegmundspeaks

\direct{nun ebenfalls vom Joche her}

Noch wähnst du mich waffenlos, feiger Wicht?
Drohst du mit Frauen, so ficht nun selber,
sonst läßt dich Fricka im Stich!
Denn sieh: deines Hauses heimischem Stamm
entzog ich zaglos das Schwert;
seine Schneide schmecke jetzt du!
 


\direct{Ein Blitz erhellt für einen Augenblick das Bergjoch, auf welchem jetzt Hunding und Siegmund kämpfend gewahrt werden}


\Sieglindespeaks

\direct{mit höchster Kraft}

Haltet ein, ihr Männer!
Mordet erst mich!
 


\direct{Sie stürzt auf das Bergjoch zu, ein von rechts her über den Kämpfern ausbrechender, heller Schein blendet sie aber plötzlich so heftig, daß sie, wie erblindet, zur Seite schwankt. In dem Lichtglanze erscheint Brünnhilde über Siegmund schwebend und diesen mit dem Schilde deckend}


\Brunnhildespeaks
Triff ihn, Siegmund!
traue dem Schwert!
 


\direct{Als Siegmund soeben zu einem tödlichen Streiche gegen Hunding ausholt, bricht von links her ein glühend rötlicher Schein durch das Gewölk aus, in welchem Wotan erscheint, über Hunding stehend und seinen Speer Siegmund quer entgegenhaltend}


\Wotanspeaks
Zurück vor dem Speer!
In Stücken das Schwert!
 


\direct{Brünnhilde weicht erschrocken vor Wotan mit dem Schilde zurück; Siegmunds Schwert zerspringt an dem vorgehaltenen Speere. Dem Unbewehrten stößt Hunding seinen Speer in die Brust. Siegmund stürzt tot zu Boden. Sieglinde, die seinen Todesseufzer gehört, sinkt mit einem Schrei wie leblos zusammen. Mit Siegmunds Fall ist zugleich von beiden Seiten der glänzende Schein verschwunden; dichte Finsternis ruht im Gewölk bis nach vorn: in ihm wird Brünnhilde undeutlich sichtbar, wie sie in jäher Hast sich Sieglinden zuwendet.}


\Brunnhildespeaks
Zu Roß, daß ich dich rette!
 


\direct{Sie hebt Sieglinde schnell zu sich auf ihr der Seitenschlucht nahestehendes Roß und verschwindet sogleich mit ihr. Alsbald zerteilt sich das Gewölk in der Mitte, so daß man deutlich Hunding gewahrt, der soeben seinen Speer dem gefallenen Siegmund aus der Brust zieht. Wotan, von Gewölk umgeben, steht dahinter auf einem Felsen, an seinen Speer gelehnt und schmerzlich auf Siegmunds Leiche blickend}


\Wotanspeaks

\direct{zu Hunding}

Geh' hin, Knecht! Kniee vor Fricka:
meld' ihr, daß Wotans Speer
gerächt, was Spott ihr schuf.
Geh'! Geh'!
 


\direct{Vor seinem verächtlichen Handwink sinkt Hunding tot zu Boden}


\Wotanspeaks

\direct{plötzlich in furchtbarer Wut auffahrend}

Doch Brünnhilde! Weh' der Verbrecherin!
Furchtbar sei die Freche gestraft,
erreicht mein Roß ihre Flucht!
 


\direct{Er verschwindet mit Blitz und Donner. Der Vorhang fällt schnell}

   

\act
 
\scene

\StageDir{Auf dem Gipfel eines Felsenberges.

Rechts begrenzt ein Tannenwald die Szene. Links der Eingang einer Felshöhle, die einen natürlichen Saal bildet: darüber steigt der Fels zu seiner höchsten Spitze auf. Nach hinten ist die Aussicht gänzlich frei; höhere und niedere Felssteine bilden den Rand vor dem Abhange, der---wie anzunehmen ist---nach dem Hintergrund zu steil hinabführt. Einzelne Wolkenzüge jagen, wie vom Sturm getrieben, am Felsensaume vorbei.
 


\direct{Gerhilde, Ortlinde, Waltraute und Schwertleite haben sich auf der Felsspitze, an und über der Höhle, gelagert, sie sind in voller Waffenrüstung.}}


\Gerhildespeaks

\direct{zuhöchst gelagert und dem Hintergrunde zurufend, wo ein starkes Gewölk herzieht}

Hojotoho! Hojotoho! Heiaha! Heiaha!
Helmwige! Hier! Hieher mit dem Roß!
 

\speaker{\Helmwige (Stimme)}

\direct{im Hintergrunde}

Hojotoho! Hojotoho! Heiaha!
 


\direct{In dem Gewölk bricht Blitzesglanz aus; eine Walküre zu Roß wird in ihm sichtbar: über ihrem Sattel hängt ein erschlagener Krieger. Die Erscheinung zieht, immer näher, am Felsensaume von links nach rechts vorbei}


\speaker{\Gerhilde, \Waltraute, und \Schwertleite}

\direct{der Ankommenden entgegenrufend}

Heiaha! Heiaha!
 


\direct{Die Wolke mit der Erscheinung ist rechts hinter dem Tann verschwunden}


\Ortlindespeaks

\direct{in den Tann hineinrufend}

Zu Ortlindes Stute stell deinen Hengst:
mit meiner Grauen grast gern dein Brauner!
 

\Waltrautespeaks

\direct{hineinrufend}

Wer hängt dir im Sattel?
 

\Helmwigespeaks

\direct{aus dem Tann auftretend}

Sintolt, der Hegeling!
 

\Schwertleitespeaks
Führ' deinen Brauen fort von der Grauen:
Ortlindes Mähre trägt Wittig, den Irming!
 

\Gerhildespeaks

\direct{ist etwas näher herabgestiegen}

Als Feinde nur sah ich Sintolt und Wittig!
 

\Ortlindespeaks

\direct{springt auf}

Heiaha! Die Stute stößt mir der Hengst!
 


\direct{Sie läuft in den Tann}


\direct{Schwertleite, Gerhilde und Helmwige lachen laut auf}


\Gerhildespeaks
Der Recken Zwist entzweit noch die Rosse!
 

\Helmwigespeaks

\direct{in den Tann zurückrufend}

Ruhig, Brauner!
Brich nicht den Frieden!
 

\Waltrautespeaks

\direct{auf der Höhe, wo sie für Gerhilde die Wacht übernommen, nach rechts in den Hintergrund rufend}

Hoioho! Hoioho!
Siegrune, hier! Wo säumst du so lang?
 


\direct{Sie lauscht nach rechts}


\speaker{\Siegrune (Stimme)}

\direct{von der rechten Seite des Hintergrundes her}

Arbeit gab's!
Sind die andren schon da?
 

\speaker{\Schwertleite und \Waltraute}

\direct{nach rechts in den Hintergrund rufend}

Hojotoho! Hojotoho!
Heiaha!
 

\Gerhildespeaks
Heiaha!
 


\direct{Ihre Gebärden sowie ein heller Glanz hinter dem Tann zeigen an, daß soeben Siegrune dort angelangt ist. Aus der Tiefe hört man zwei Stimmen zugleich}


\speaker{\Grimgerde und \Rossweisse}

\direct{links im Hintergrunde}

Hojotoho! Hojotoho!
Heiaha!
 

\Waltrautespeaks

\direct{nach links}

Grimgerd' und Roßweiße!
 

\Gerhildespeaks

\direct{ebenso}

Sie reiten zu zwei.
 


\direct{In einem blitzerglänzenden Wolkenzuge, der von links her vorbeizieht, erscheinen Grimgerde und Roßweiße, ebenfalls auf Rossen, jede einen Erschlagenen im Sattel führend. Helmwige, Ortlinde und Siegrune sind aus dem Tann getreten und winken vom Felsensaume den Ankommenden zu}


\speaker{\Helmwige, \Ortlinde, und \Siegrune}
Gegrüßt, ihr Reisige!
Roßweiß' und Grimgerde!
 

\speaker{\Rossweisse und \Grimgerde (Stimmen)}
Hojotoho! Hojotoho!
Heiaha!
 


\direct{Die Erscheinung verschwindet hinter dem Tann}


\speaker{Die sechs anderen Walküren}
Hojotoho! Hojotoho! Heiaha! Heiaha!
 

\Gerhildespeaks

\direct{in den Tann rufend}

In Wald mit den Rossen zu Weid' und Rast!
 

\Ortlindespeaks

\direct{ebenfalls in den Tann rufend}

Führet die Mähren fern von einander,
bis unsrer Helden Haß sich gelegt!
 


\direct{Die Walküren lachen}


\Helmwigespeaks

\direct{während die anderen lachen}

Der Helden Grimm büßte schon die Graue!
 


\direct{Die Walküren lachen}


\speaker{\Rossweisse und \Grimgerde}

\direct{aus dem Tann tretend}

Hojotoho! Hojotoho!
 

\speaker{Die sechs anderen Walküren}
Willkommen! Willkommen!
 

\Schwertleitespeaks
Wart ihr Kühnen zu zwei?
 

\Grimgerdespeaks
Getrennt ritten wir und trafen uns heut'.
 

\Rossweissespeaks
Sind wir alle versammelt, so säumt nicht lange:
nach Walhall brechen wir auf,
Wotan zu bringen die Wal.
 

\Helmwigespeaks
Acht sind wir erst: eine noch fehlt.
 

\Gerhildespeaks
Bei dem braunen Wälsung
weilt wohl noch Brünnhilde.
 

\Waltrautespeaks
Auf sie noch harren müssen wir hier:
Walvater gäb' uns grimmigen Gruß,
säh' ohne sie er uns nahn!
 

\Siegrunespeaks

\direct{auf der Felswarte, von wo sie hinausspäht}

Hojotoho! Hojotoho!
 


\direct{in den Hintergrund rufend}

Hieher! Hieher!
 


\direct{zu den anderen}

In brünstigem Ritt
jagt Brünnhilde her.
 

\speaker{Die acht Walküren}

\direct{alle eilen auf die Warte}

Hojotoho! Hojotoho!
Brünnhilde! Hei!
 


\direct{Sie spähen mit wachsender Verwunderung}


\Waltrautespeaks
Nach dem Tann lenkt sie das taumelnde Roß.
 

\Grimgerdespeaks
Wie schnaubt Grane vom schnellen Ritt!
 

\Rossweissespeaks
So jach sah ich nie Walküren jagen!
 

\Ortlindespeaks
Was hält sie im Sattel?
 

\Helmwigespeaks
Das ist kein Held!
 

\Siegrunespeaks
Eine Frau führt sie!
 

\Gerhildespeaks
Wie fand sie die Frau?
 

\Schwertleitespeaks
Mit keinem Gruß grüßt sie die Schwestern!
 

\Waltrautespeaks

\direct{hinabrufend}

Heiaha! Brünnhilde! Hörst du uns nicht?
 

\Ortlindespeaks
Helft der Schwester
vom Roß sich schwingen!
 


\direct{Gerhilde und Helmwige stürzen in den Tann}


\direct{Siegrune und Roßweiße laufen ihnen nach}


\speaker{\Helmwige, \Gerhilde, \Siegrune, und \Rossweisse}
Hojotoho! Hojotoho!
 

\speaker{\Ortlinde, \Waltraute, \Grimgerde, und \Schwertleite}
Heiaha!
 

\Waltrautespeaks

\direct{in den Tann blickend}

Zu Grunde stürzt Grane, der Starke!
 

\Grimgerdespeaks
Aus dem Sattel hebt sie hastig das Weib!
 

\speaker{\Ortlinde, \Waltraute, \Grimgerde, und \Schwertleite}

\direct{alle in den Tann laufend}

Schwester! Schwester! Was ist geschehn?
 


\direct{Alle Walküren kehren auf die Bühne zurück; mit ihnen kommt Brünnhilde, Sieglinde unterstützend und hereingeleitend}


\Brunnhildespeaks

\direct{atemlos}

Schützt mich und helft in höchster Not!
 

\speaker{Die acht Walküren}
Wo rittest du her in rasender Hast?
So fliegt nur, wer auf der Flucht!
 

\Brunnhildespeaks
Zum erstenmal flieh' ich und bin verfolgt:
Heervater hetzt mir nach!
 

\speaker{Die acht Walküren}

\direct{heftig erschreckend}

Bist du von Sinnen? Sprich! Sage uns! Wie?
Verfolgt dich Heervater?
Fliehst du vor ihm?
 

\Brunnhildespeaks

\direct{wendet sich ängstlich, um zu spähen, und kehrt wieder zurück}

O Schwestern, späht von des Felsens Spitze!
Schaut nach Norden, ob Walvater naht!
 


\direct{Ortlinde und Waltraute springen auf die Felsenspitze zur Warte}

Schnell! Seht ihr ihn schon?
 

\Ortlindespeaks
Gewittersturm naht von Norden.
 

\Waltrautespeaks
Starkes Gewölk staut sich dort auf!
 

\speaker{Die Weiteren sechs Walküren}
Heervater reitet sein heiliges Roß!
 

\Brunnhildespeaks
Der wilde Jäger, der wütend mich jagt,
er naht, er naht von Norden!
Schützt mich, Schwestern! Wahret dies Weib!
 

\speaker{Sechs Walküren}
Was ist mit dem Weibe?
 

\Brunnhildespeaks
Hört mich in Eile:
Sieglinde ist es, Siegmunds Schwester und Braut:
gegen die Wälsungen
wütet Wotan in Grimm;
dem Bruder sollte Brünnhilde heut'
entziehen den Sieg;
doch Siegmund schützt' ich mit meinem Schild,
trotzend dem Gott!
Der traf ihn da selbst mit dem Speer:
Siegmund fiel;
doch ich floh fern mit der Frau;
sie zu retten, eilt' ich zu euch---
ob mich Bange auch
 


\direct{kleinmütig}

ihr berget vor dem strafenden Streich!
 

\speaker{Sechs Walküren}

\direct{in größter Bestürzung}

Betörte Schwester, was tatest du?
Wehe! Brünnhilde, wehe!
Brach ungehorsam
Brünnhilde Heervaters heilig Gebot?
 

\Waltrautespeaks

\direct{von der Warte}

Nächtig zieht es von Norden heran.
 

\Ortlindespeaks

\direct{ebenso}

Wütend steuert hieher der Sturm.
 

\speaker{\Rossweisse, \Grimgerde, und \Schwertleite}

\direct{dem Hintergrunde zugewendet}

Wild wiehert Walvaters Roß.
 

\speaker{\Helmwige, \Gerhilde, und \Schwertleite}
Schrecklich schnaubt es daher!
 

\Brunnhildespeaks
Wehe der Armen, wenn Wotan sie trifft:
den Wälsungen allen droht er Verderben!
Wer leiht mir von euch das leichteste Roß,
das flink die Frau ihm entführ'?
 

\Siegrunespeaks
Auch uns rätst du rasenden Trotz?
 

\Brunnhildespeaks
Roßweiße, Schwester,
leih' mir deinen Renner!
 

\Rossweissespeaks
Vor Walvater floh der fliegende nie.
 

\Brunnhildespeaks
Helmwige, höre!
 

\Helmwigespeaks
Dem Vater gehorch' ich.
 

\Brunnhildespeaks
Grimgerde! Gerhilde! Gönnt mir eu'r Roß!
Schwertleite! Siegrune! Seht meine Angst!
Seid mir treu, wie traut ich euch war:
rettet dies traurige Weib!
 

\Sieglindespeaks

\direct{die bisher finster und kalt vor sich hingestarrt, fährt, als Brünnhilde sie lebhaft---wie zum Schutze---umfaßt, mit einer abwehrenden Gebärde auf}

Nicht sehre dich Sorge um mich:
einzig taugt mir der Tod!
Wer hieß dich Maid,
dem Harst mich entführen?
Im Sturm dort hätt' ich den Streich empfah'n
von derselben Waffe, der Siegmund fiel:
das Ende fand ich
vereint mit ihm!
Fern von Siegmund---Siegmund, von dir!---
O deckte mich Tod, daß ich's denke!
Soll um die Flucht
dir, Maid, ich nicht fluchen,
so erhöre heilig mein Flehen:
stoße dein Schwert mir ins Herz!
 

\Brunnhildespeaks
Lebe, o Weib, um der Liebe willen!
Rette das Pfand, das von ihm du empfingst:
 


\direct{stark und drängend}

ein Wälsung wächst dir im Schoß!
 

\Sieglindespeaks

\direct{erschrickt zunächst heftig; sogleich strahlt aber ihr Gesicht in erhabener Freude auf}

Rette mich, Kühne! Rette mein Kind!
Schirmt mich, ihr Mädchen, mit mächtigstem Schutz!
 


\direct{Immer finstereres Gewitter steigt im Hintergrunde auf: nahender Donner}


\Waltrautespeaks

\direct{auf der Warte}

Der Sturm kommt heran.
 

\Ortlindespeaks

\direct{ebenso}

Flieh', wer ihn fürchtet!
 

\speaker{Die sechs anderen Walküren}
Fort mit dem Weibe, droht ihm Gefahr:
der Walküren keine wag' ihren Schutz!
 

\Sieglindespeaks

\direct{auf den Knien vor Brünnhilde}

Rette mich, Maid! Rette die Mutter!
 

\Brunnhildespeaks

\direct{mit lebhaftem Entschluß hebt sie Sieglinde auf}

So fliehe denn eilig und fliehe allein!
Ich bleibe zurück, biete mich Wotans Rache:
an mir zögr' ich den Zürnenden hier,
während du seinem Rasen entrinnst.
 

\Sieglindespeaks
Wohin soll ich mich wenden?
 

\Brunnhildespeaks
Wer von euch Schwestern schweifte nach Osten?
 

\Siegrunespeaks
Nach Osten weithin dehnt sich ein Wald:
der Niblungen Hort entführte Fafner dorthin.
 

\Schwertleitespeaks
Wurmesgestalt schuf sich der Wilde:
in einer Höhle hütet er Alberichs Reif!
 

\Grimgerdespeaks
Nicht geheu'r ist's dort für ein hilflos' Weib.
 

\Brunnhildespeaks
Und doch vor Wotans Wut schützt sie sicher der Wald:
ihn scheut der Mächt'ge und meidet den Ort.
 

\Waltrautespeaks

\direct{auf der Warte}

Furchtbar fährt
dort Wotan zum Fels.
 

\speaker{Sechs Walküren}
Brünnhilde, hör' seines Nahens Gebraus'!
 

\Brunnhildespeaks

\direct{Sieglinde die Richtung weisend}

Fort denn eile, nach Osten gewandt!
Mutigen Trotzes ertrag' alle Müh'n,
Hunger und Durst, Dorn und Gestein;
lache, ob Not, ob Leiden dich nagt!
Denn eines wiss' und wahr' es immer:
den hehrsten Helden der Welt
hegst du, o Weib, im schirmenden Schoß!
 


\direct{Sie zieht die Stücken von Siegmunds Schwert unter ihrem Panzer hervor und überreicht sie Sieglinde}

Verwahr' ihm die starken Schwertesstücken;
seines Vaters Walstatt entführt' ich sie glücklich:
der neugefügt das Schwert einst schwingt,
den Namen nehm' er von mir---
``Siegfried'' erfreu' sich des Siegs!
 

\Sieglindespeaks

\direct{in größter Rührung}

O hehrstes Wunder! Herrlichste Maid!
Dir Treuen dank' ich heiligen Trost!
Für ihn, den wir liebten, rett' ich das Liebste:
meines Dankes Lohn lache dir einst!
Lebe wohl! Dich segnet Sieglindes Weh'!
 


\direct{Sie eilt rechts im Vordergrunde von dannen. Die Felsenhöhe ist von schwarzen Gewitterwolken umlagert; furchtbarer Sturm braust aus dem Hintergrunde daher, wachsender Feuerschein rechts daselbst}


\speaker{\Wotan (Stimme)}
Steh'! Brünnhild'!
 


\direct{Brünnhilde, nachdem sie eine Weile Sieglinde nachgesehen, wendet sich in den Hintergrund, blickt in den Tann und kommt angstvoll wieder vor}


\speaker{\Ortlinde und \Waltraute}

\direct{von der Warte herabsteigend}

Den Fels erreichten Roß und Reiter!
 

\speaker{Alle acht Walküren}
Weh', Brünnhild'! Rache entbrennt!
 

\Brunnhildespeaks
Ach, Schwestern, helft! Mir schwankt das Herz!
Sein Zorn zerschellt mich,
wenn euer Schutz ihn nicht zähmt.
 

\speaker{Die acht Walküren}

\direct{flüchten ängstlich nach der Felsenspitze hinauf; Brünnhilde läßt sich von ihnen nachziehen}

Hieher, Verlor'ne! Laß dich nicht sehn!
Schmiege dich an uns und schweige dem Ruf!
 


\direct{Sie verbergen Brünnhilde unter sich und blicken ängstlich nach dem Tann, der jetzt von grellem Feuerschein erhellt wird, während der Hintergrund ganz finster geworden ist}

Weh'! Wütend schwingt sich Wotan vom Roß!
Hieher rast sein rächender Schritt!

\scene

\StageDir{Wotan tritt in höchster zorniger Aufgeregtheit aus dem Tann auf und schreitet vor der Gruppe der Walküren auf der Höhe, nach Brünnhilde spähend, heftig einher}


\Wotanspeaks
Wo ist Brünnhild', wo die Verbrecherin?
Wagt ihr, die Böse vor mir zu bergen?

\speaker{Die acht Walküren}
Schrecklich ertost dein Toben!
Was taten, Vater, die Töchter,
daß sie dich reizten zu rasender Wut?
 

\Wotanspeaks
Wollt ihr mich höhnen? Hütet euch, Freche!
Ich weiß: Brünnhilde bergt ihr vor mir.
Weichet von ihr, der ewig Verworfnen,
wie ihren Wert von sich sie warf!
 

\Rossweissespeaks
Zu uns floh die Verfolgte.
 

\speaker{Die acht Walküren}
Unsern Schutz flehte sie an!
Mit Furcht und Zagen faßt sie dein Zorn:
für die bange Schwester bitten wir nun,
daß den ersten Zorn du bezähmst.
Laß dich erweichen für sie, zähm deinen Zorn!
 

\Wotanspeaks
Weichherziges Weibergezücht!
So matten Mut gewannt ihr von mir?
Erzog ich euch, kühn zum Kampfe zu zieh'n,
schuf ich die Herzen
euch hart und scharf,
daß ihr Wilden nun weint und greint,
wenn mein Grimm eine Treulose straft?
So wißt denn, Winselnde, was sie verbrach,
um die euch Zagen die Zähre entbrennt:
Keine wie sie
kannte mein innerstes Sinnen;
keine wie sie
wußte den Quell meines Willens!
Sie selbst war
meines Wunsches schaffender Schoß:
und so nun brach sie den seligen Bund,
daß treulos sie meinem Willen getrotzt,
mein herrschend Gebot offen verhöhnt,
gegen mich die Waffe gewandt,
die mein Wunsch allein ihr schuf!
Hörst du's, Brünnhilde? Du, der ich Brünne,
Helm und Wehr, Wonne und Huld,
Namen und Leben verlieh?
Hörst du mich Klage erheben,
und birgst dich bang dem Kläger,
daß feig du der Straf' entflöhst?
 

\Brunnhildespeaks

\direct{tritt aus der Schar der Walküren hervor, schreitet demütigen, doch festen Schrittes von der Felsenspitze herab und tritt so in geringer Entfernung vor Wotan}

Hier bin ich, Vater: gebiete die Strafe!
 

\Wotanspeaks
Nicht straf' ich dich erst:
deine Strafe schufst du dir selbst.
Durch meinen Willen warst du allein:
gegen ihn doch hast du gewollt;
meinen Befehl nur führtest du aus:
gegen ihn doch hast du befohlen;
Wunschmaid warst du mir:
gegen mich doch hast du gewünscht;
Schildmaid warst du mir:
gegen mich doch hobst du den Schild;
Loskieserin warst du mir:
gegen mich doch kiestest du Lose;
Heldenreizerin warst du mir:
gegen mich doch reiztest du Helden.
Was sonst du warst, sagte dir Wotan:
was jetzt du bist, das sage dir selbst!
Wunschmaid bist du nicht mehr;
Walküre bist du gewesen:
nun sei fortan, was so du noch bist!
 

\Brunnhildespeaks

\direct{heftig erschreckend}

Du verstößest mich? Versteh' ich den Sinn?
 

\Wotanspeaks
Nicht send' ich dich mehr aus Walhall;
nicht weis' ich dir mehr Helden zur Wal;
nicht führst du mehr Sieger
in meinen Saal:
bei der Götter trautem Mahle
das Trinkhorn nicht reichst du traulich mir mehr;
nicht kos' ich dir mehr den kindischen Mund;
von göttlicher Schar bist du geschieden,
ausgestoßen aus der Ewigen Stamm;
gebrochen ist unser Bund;
aus meinem Angesicht bist du verbannt.
 

\speaker{Die acht Walküren}

\direct{verlassen, in aufgeregter Bewegung, ihre Stellung, indem sie sich etwas tiefer herabziehen}

Wehe! Weh'!
Schwester, ach Schwester!
 

\Brunnhildespeaks
Nimmst du mir alles, was einst du gabst?
 

\Wotanspeaks
Der dich zwingt, wird dir's entziehn!
Hieher auf den Berg banne ich dich;
in wehrlosen Schlaf schließ' ich dich fest:
der Mann dann fange die Maid,
der am Wege sie findet und weckt.
 

\speaker{Die acht Walküren}

\direct{kommen in höchster Aufregung von der Felsenspitze ganz herab und umgeben in ängstlichen Gruppen Brünnhilde, welche halb kniend vor Wotan liegt}

Halt' ein, o Vater! Halt' ein den Fluch!
Soll die Maid verblühn und verbleichen dem Mann?
Hör unser Fleh'n! Schrecklicher Gott,
wende von ihr die schreiende Schmach!
Wie die Schwester träfe uns selber der Schimpf!
 

\Wotanspeaks
Hörtet ihr nicht, was ich verhängt?
 

Aus eurer Schar ist die treulose Schwester geschieden;
mit euch zu Roß durch die Lüfte nicht reitet sie länger;
die magdliche Blume verblüht der Maid;
ein Gatte gewinnt ihre weibliche Gunst;
dem herrischen Manne gehorcht sie fortan;
am Herde sitzt sie und spinnt,
aller Spottenden Ziel und Spiel.
 


\direct{Brünnhilde sinkt mit einem Schrei zu Boden; die Walküren weichen entsetzt mit heftigem Geräusch von ihrer Seite}


Schreckt euch ihr Los? So flieht die Verlorne!
Weichet von ihr und haltet euch fern!
Wer von euch wagte bei ihr zu weilen,
wer mir zum Trotz
zu der Traurigen hielt'
die Törin teilte ihr Los:
das künd' ich der Kühnen an!
Fort jetzt von hier; meidet den Felsen!
Hurtig jagt mir von hinnen,
sonst erharrt Jammer euch hier!
 

\speaker{Die acht Walküren}

Weh! Weh!
 


\StageDir{Die Walküren fahren mit wildem Wehschrei auseinander und stürzen in hastiger Flucht in den Tann. Schwarzes Gewölk lagert sich dicht am Felsenrande: man hört wildes Geräusch im Tann. Ein greller Blitzesglanz bricht in dem Gewölk aus; in ihm erblickt man die Walküren mit verhängtem Zügel, in eine Schar zusammengedrängt, wild davonjagen. Bald legt sich der Sturm; die Gewitterwolken verziehen sich allmählich. In der folgenden Szene bricht, bei endlich ruhigem Wetter, Abenddämmerung ein, der am Schlusse Nacht folgt.}

\scene

\StageDir{Wotan und Brünnhilde, die noch zu seinen Füßen hingestreckt liegt, sind allein zurückgeblieben. Langes, feierliches Schweigen: unveränderte Stellung}


\Brunnhildespeaks

\direct{beginnt das Haupt langsam ein wenig zu erheben. Schüchtern beginnend und steigernd}

War es so schmählich, was ich verbrach,
daß mein Verbrechen so schmählich du bestrafst?
War es so niedrig, was ich dir tat,
daß du so tief mir Erniedrigung schaffst?
War es so ehrlos, was ich beging,
daß mein Vergehn nun die Ehre mir raubt?


\direct{Sie erhebt sich allmählich bis zur knienden Stellung}

O sag', Vater! Sieh mir ins Auge:
schweige den Zorn, zähme die Wut,
und deute mir hell die dunkle Schuld,
die mit starrem Trotze dich zwingt,
zu verstoßen dein trautestes Kind!
 

\Wotanspeaks

\direct{in unveränderter Stellung, ernst und düster}

Frag' deine Tat, sie deutet dir deine Schuld!
 

\Brunnhildespeaks
Deinen Befehl führte ich aus.
 

\Wotanspeaks
Befahl ich dir, für den Wälsung zu fechten?
 

\Brunnhildespeaks
So hießest du mich als Herrscher der Wal!
 

\Wotanspeaks
Doch meine Weisung nahm ich wieder zurück!
 

\Brunnhildespeaks
Als Fricka den eignen Sinn dir entfremdet;
da ihrem Sinn du dich fügtest,
warst du selber dir Feind.
 

\Wotanspeaks

\direct{leise und bitter}

Daß du mich verstanden, wähnt' ich,
und strafte den wissenden Trotz:
doch feig und dumm dachtest du mich!
So hätt' ich Verrat nicht zu rächen;
zu gering wärst du meinem Grimm?
 

\Brunnhildespeaks
Nicht weise bin ich, doch wußt' ich das eine,
daß den Wälsung du liebtest.
Ich wußte den Zwiespalt, der dich zwang,
dies eine ganz zu vergessen.
Das andre mußtest einzig du sehn,
was zu schaun so herb schmerzte dein Herz:
daß Siegmund Schutz du versagtest.
 

\Wotanspeaks
Du wußtest es so, und wagtest dennoch den Schutz?
 

\Brunnhildespeaks

\direct{leise beginnend}

Weil für dich im Auge das eine ich hielt,
dem, im Zwange des andren
schmerzlich entzweit,
ratlos den Rücken du wandtest!
Die im Kampfe Wotan den Rücken bewacht,
die sah nun das nur, was du nicht sahst:
Siegmund mußt' ich sehn.
Tod kündend trat ich vor ihn,
gewahrte sein Auge, hörte sein Wort;
ich vernahm des Helden heilige Not;
tönend erklang mir des Tapfersten Klage:
freiester Liebe furchtbares Leid,
traurigsten Mutes mächtigster Trotz!
Meinem Ohr erscholl, mein Aug' erschaute,
was tief im Busen das Herz
zu heilgem Beben mir traf.
Scheu und staunend stand ich in Scham.
Ihm nur zu dienen konnt' ich noch denken:
Sieg oder Tod mit Siegmund zu teilen:
dies nur erkannt' ich zu kiesen als Los!
Der diese Liebe mir ins Herz gehaucht,
dem Willen, der dem Wälsung mich gesellt,
ihm innig vertraut, trotzt' ich deinem Gebot.
 

\Wotanspeaks
So tatest du, was so gern zu tun ich begehrt,
doch was nicht zu tun die Not zwiefach mich zwang?
So leicht wähntest du Wonne des Herzens erworben,
wo brennend Weh' in das Herz mir brach,
wo gräßliche Not
den Grimm mir schuf,
einer Welt zuliebe der Liebe Quell
im gequälten Herzen zu hemmen?
Wo gegen mich selber
ich sehrend mich wandte,
aus Ohnmachtschmerzen
schäumend ich aufschoß,
wütender Sehnsucht sengender Wunsch
den schrecklichen Willen mir schuf,
in den Trümmern der eignen Welt
meine ew'ge Trauer zu enden:
da labte süß dich selige Lust;
wonniger Rührung üppigen Rausch
enttrankst du lachend der Liebe Trank,
als mir göttlicher Not nagende Galle gemischt?
Deinen leichten Sinn laß dich denn leiten:
von mir sagtest du dich los.
Dich muß ich meiden,
gemeinsam mit dir
nicht darf ich Rat mehr raunen;
getrennt, nicht dürfen
traut wir mehr schaffen:
so weit Leben und Luft
darf der Gott dir nicht mehr begegnen!
 

\Brunnhildespeaks
Wohl taugte dir nicht die tör'ge Maid,
die staunend im Rate
nicht dich verstand,
wie mein eigner Rat
nur das eine mir riet:
zu lieben, was du geliebt. -
Muß ich denn scheiden und scheu dich meiden,
mußt du spalten, was einst sich umspannt,
die eigne Hälfte fern von dir halten,
daß sonst sie ganz dir gehörte,
du Gott, vergiß das nicht!
Dein ewig Teil nicht wirst du entehren,
Schande nicht wollen, die dich beschimpft:
dich selbst ließest du sinken,
sähst du dem Spott mich zum Spiel!
 

\Wotanspeaks
Du folgtest selig der Liebe Macht:
folge nun dem, den du lieben mußt!
 

\Brunnhildespeaks
Soll ich aus Walhall scheiden,
nicht mehr mit dir schaffen und walten,
dem herrischen Manne gehorchen fortan:
dem feigen Prahler gib mich nicht preis!
Nicht wertlos sei er, der mich gewinnt.
 

\Wotanspeaks
Von Walvater schiedest du -
nicht wählen darf er für dich.
 

\Brunnhildespeaks

\direct{leise mit vertraulicher Heimlichkeit}

Du zeugtest ein edles Geschlecht;
kein Zager kann je ihm entschlagen:
der weihlichste Held---ich weiß es---
entblüht dem Wälsungenstamm.
 

\Wotanspeaks
Schweig' von dem Wälsungenstamm!
Von dir geschieden, schied ich von ihm:
vernichten mußt' ihn der Neid!
 

\Brunnhildespeaks
Die von dir sich riß, rettete ihn.
 


\direct{heimlich}

Sieglinde hegt die heiligste Frucht;
in Schmerz und Leid, wie kein Weib sie gelitten,
wird sie gebären,
was bang sie birgt.
 

\Wotanspeaks
Nie suche bei mir Schutz für die Frau,
noch für ihres Schoßes Frucht!
 

\Brunnhildespeaks

\direct{heimlich}

Sie wahret das Schwert, das du Siegmund schufest.
 

\Wotanspeaks

\direct{heftig}

Und das ich ihm in Stücken schlug!
Nicht streb', o Maid, den Mut mir zu stören;
erwarte dein Los, wie sich's dir wirft;
nicht kiesen kann ich es dir!
Doch fort muß ich jetzt, fern mich verziehn;
zuviel schon zögert' ich hier;
von der Abwendigen wend' ich mich ab;
nicht wissen darf ich, was sie sich wünscht:
die Strafe nur muß vollstreckt ich sehn!
 

\Brunnhildespeaks
Was hast du erdacht, daß ich erdulde?
 

\Wotanspeaks
In festen Schlaf verschließ' ich dich:
wer so die Wehrlose weckt,
dem ward, erwacht, sie zum Weib!
 

\Brunnhildespeaks

\direct{stürzt auf ihre Knie}

Soll fesselnder Schlaf fest mich binden,
dem feigsten Manne zur leichten Beute:
dies eine muß du erhören,
was heil'ge Angst zu dir fleht!
Die Schlafende schütze mit scheuchenden Schrecken,
daß nur ein furchtlos freiester Held
hier auf dem Felsen einst mich fänd'!
 

\Wotanspeaks
Zu viel begehrst du, zu viel der Gunst!
 

\Brunnhildespeaks

\direct{seine Knie umfassend}

Dies eine mußt du erhören!
Zerknicke dein Kind, das dein Knie umfaßt;
zertritt die Traute, zertrümmre die Maid,
ihres Leibes Spur zerstöre dein Speer:
doch gib, Grausamer, nicht
der gräßlichsten Schmach sie preis!

\direct{mit wilder Begeisterung}

Auf dein Gebot entbrenne ein Feuer;
den Felsen umglühe lodernde Glut;
es leck' ihre Zung', es fresse ihr Zahn
den Zagen, der frech sich wagte,
dem freislichen Felsen zu nahn!
 
\Wotanspeaks

\direct{überwältigt und tief ergriffen, wendet sich lebhhaft zu Brünnhilde, erhebt sie von den Knien und blickt ihr gerührt in das Auge}

Leb' wohl, du kühnes, herrliches Kind!
Du meines Herzens heiligster Stolz!
Leb' wohl! Leb' wohl! Leb' wohl!

\direct{sehr leidenschaftlich}

Muß ich dich meiden,
und darf nicht minnig
mein Gruß dich mehr grüßen;
sollst du nun nicht mehr neben mir reiten,
noch Met beim Mahl mir reichen;
muß ich verlieren dich, die ich liebe,
du lachende Lust meines Auges:
ein bräutliches Feuer soll dir nun brennen,
wie nie einer Braut es gebrannt!
Flammende Glut umglühe den Fels;
mit zehrenden Schrecken
scheuch' es den Zagen;
der Feige fliehe Brünnhildes Fels!
Denn einer nur freie die Braut,
der freier als ich, der Gott!
 


\direct{Brünnhilde sinkt, gerührt und begeistert, an Wotans Brust; er hält sie lange umfangen. Sie schlägt das Haupt wieder zurück und blickt, immer noch ihn umfassend, feierlich ergriffen Wotan in das Auge}

Der Augen leuchtendes Paar,
das oft ich lächelnd gekost,
wenn Kampfeslust ein Kuß dir lohnte,
wenn kindisch lallend der Helden Lob
von holden Lippen dir floß:
dieser Augen strahlendes Paar,
das oft im Sturm mir geglänzt,
wenn Hoffnungssehnen das Herz mir sengte,
nach Weltenwonne mein Wunsch verlangte
aus wild webendem Bangen:
zum letztenmal
letz' es mich heut'
mit des Lebewohles letztem Kuß!
Dem glücklichen Manne
glänze sein Stern:
dem unseligen Ew'gen
muß es scheidend sich schließen.
 


\direct{Er faßt ihr Haupt in beide Hände}

Denn so kehrt der Gott sich dir ab,
so küßt er die Gottheit von dir!
 


\direct{Er küßt sie lange auf die Augen. Sie sinkt mit geschlossenen Augen, sanft ermattend, in seinen Armen zurück. Er geleitet sie zart auf einen niedrigen Mooshügel zu liegen, über den sich eine breitästige Tanne ausstreckt. Er betrachtet sie und schließt ihr den Helm: sein Auge weilt dann auf der Gestalt der Schlafenden, die er mit dem großen Stahlschilde der Walküre ganz zudeckt. Langsam kehrt er sich ab, mit einem schmerzlichen Blicke wendet er sich noch einmal um. Dann schreitet er mit feierlichem Entschlusse in die Mitte der Bühne und kehrt seines Speeres Spitze gegen einen mächtigen Felsstein.}

Loge, hör'! Lausche hieher!
Wie zuerst ich dich fand, als feurige Glut,
wie dann einst du mir schwandest,
als schweifende Lohe;
wie ich dich band, bann ich dich heut'!
Herauf, wabernde Lohe,
umlodre mir feurig den Fels!
 


\direct{Er stößt mit dem Folgenden dreimal mit dem Speer auf den Stein}

Loge! Loge! Hieher!
 


\StageDir{Dem Stein entfährt ein Feuerstrahl, der zur allmählich immer helleren Flammenglut anschwillt. Lichte Flackerlohe bricht aus. Lichte Brunst umgibt Wotan mit wildem Flackern. Er weist mit dem Speere gebieterisch dem Feuermeere den Umkreis des Felsenrandes zur Strömung an; alsbald zieht es sich nach dem Hintergrunde, wo es nun fortwährend den Bergsaum umlodert.}

Wer meines Speeres Spitze fürchtet,
durchschreite das Feuer nie!
 

\StageDir{Er streckt den Speer wie zum Banne aus, dann blickt er schmerzlich auf Brünnhilde zurück, wendet sich langsam zum Gehen und blickt noch einmal zurück, ehe er durch das Feuer verschwindet. Der Vorhang fällt.}



\end{drama}
