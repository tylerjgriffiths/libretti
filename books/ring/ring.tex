\documentclass[a5paper]{memoir}
\usepackage{etex}

%% Use UTF-8 encoding for input, to allow for
%% use of äöüß instead of \"a\"o\"u\ss.
\usepackage[utf8]{inputenc}

%% Set paper size.
\usepackage[a5paper]{geometry}

%% Use German versions of "Part", etc. in the
%% produced document.
\usepackage[german]{babel}

%% =dramatist= is used to set the text of the operas.
\usepackage{dramatist}

%% pgfornaments provides some nice vector motifs.
\usepackage{pgfornament}

%% Add hyperlinks for ToC.
\usepackage{hyperref}

%% Garamond as the main document font.
\usepackage{garamondx}

%% Good microtypography.
\usepackage{microtype}

%% Set title information.
\title{\HUGE \emph{der} \\ RING \\ \emph{des} \\ NIBELUNGEN}
\author{Richard Wagner}
\date{}

\nouppercaseheads

\providecommand{\saythescene}{Akt {\printactnum}, Szene {\printscenenum}}
\makeevenhead{headings}%
    {\thepage}{\textit{Der Ring des Nibelungen}}{}
\makeoddhead{headings}{}{\textit{\playname{} --- {\saythescene}}}{\thepage}
%\makeevenfoot{headings}{}{}{(Copyright notice)}
%\makeoddfoot{headings}{(Copyright notice)}{}{}
\renewcommand{\scenecontentsline}{\scenename~\Roman{scene}}
\renewcommand{\actcontentsline}{\actname~\Roman{act}}

\let\realscene\scene
\renewcommand\scene[1]{\vspace{1em}\realscene{#1}}

\begin{document}

\sloppy

\frontmatter

%% A bastard-title.
\thispagestyle{empty}
\begin{centering}
  \centering
  \vspace*{5cm}
  \hfill\begin{minipage}{5cm}
    \centering
    \textsc{Der Ring des Nibelungen}
    
  \vspace*{0.2cm}
  \pgfornament[width=2.5cm]{75}
  \end{minipage}
\end{centering}


\cleardoublepage

%% Print the full title page.
\thispagestyle{empty}
\maketitle

\cleardoublepage

%% Set some lengths for the ToC page.
%% We use the chapter level for acts, so we reduce
%% the spacing between Part (corresponding to Opera)
%% and act. We also set hspacing for Chapters and
%% Sections.
\setlength{\cftbeforechapterskip}{3pt}
\setlength{\cftchapterindent}{1em}
\setlength{\cftsectionindent}{1.6em}

%% Print table of contents
\tableofcontents

\mainmatter

%% This is German, after all.
\renewcommand{\actname}{Akt}
\renewcommand{\scenename}{Szene}

%% Don't print Act-number in scene titles.
\renewcommand{\printscenenum}{\scenenumfont \thescene}

\makeatletter
\renewcommand{\@direct}[1]{%
    \if@drverse
    \vskip2\normallineskip
    \parbox[b]{\dirwidth}{\dirdelimiter{{\textit{#1}}}}\@centercr
    \else
    \dirdelimiter{{\textit{#1}}}\unskip
    \fi
  }
  \makeatother
  
%% Declare the characters.
\Character{Alberich}{Alberich}
\Character{Br\"unnhilde}{Brunnhilde}
\Character{Die Dritte Norn}{DieDritteNorn}
\Character{Die Erste Norn}{DieErsteNorn}
\Character{Die Zweite Norn}{DieZweiteNorn}
\Character{Donner}{Donner}
\Character{Erda}{Erda}
\Character{Fafner}{Fafner}
\Character{Fasolt}{Fasolt}
\Character{Flosshilde}{Flosshilde}
\Character{Freia}{Freia}
\Character{Fricka}{Fricka}
\Character{Froh}{Froh}
\Character{Gerhilde}{Gerhilde}
\Character{Grimgerde}{Grimgerde}
\Character{Gunther}{Gunther}
\Character{Gutrune}{Gutrune}
\Character{Hagen}{Hagen}
\Character{Helmwige}{Helmwige}
\Character{Hunding}{Hunding}
\Character{Loge}{Loge}
\Character{Mime}{Mime}
\Character{Ortlinde}{Ortlinde}
\Character{Roßweiße}{Rossweisse}
\Character{Schwertleite}{Schwertleite}
\Character{Siegfried}{Siegfried}
\Character{Sieglinde}{Sieglinde}
\Character{Siegmund}{Siegmund}
\Character{Siegrune}{Siegrune}
\Character{Waltraute}{Waltraute}
\Character{Wanderer}{Wanderer}
\Character{Wellgunde}{Wellgunde}
\Character{Woglinde}{Woglinde}
\Character{Wotan}{Wotan}




%% Print the operas.
%% We need to reset the Act counters between each, as
%% =dramatist= is only really designed to handle _one_
%% play per document.

%% Der Vorabend...
\let\oldsaythescene\saythescene
\renewcommand{\saythescene}{Szene \printscenenum}
\providecommand{\playname}{Das Rheingold}
\part{Das Rheingold}
\begin{drama}
  \item
  \scene

\StageDir{Auf dem Grunde des Rheines. Grünliche Dämmerung, nach oben zu lichter, nach unten zu dunkler. Die Höhe ist von wogendem Gewässer erfüllt, das rastlos von rechts nach links zu strömt. Nach der Tiefe zu lösen die Fluten sich in einen immer feineren feuchten Nebel auf, so daß der Raum in Manneshöhe vom Boden auf gänzlich frei vom Wasser zu sein scheint, welches wie in Wolkenzügen über den nächtlichen Grund dahinfließt. Überall ragen schroffe Felsenriffe aus der Tiefe auf und grenzen den Raum der Bühne ab; der ganze Boden ist in ein wildes Zackengewirr zerspalten, so daß er nirgends vollkommen eben ist und nach allen Seiten hin in dichtester Finsternis tiefere Schlüfte annehmen läßt.
\\ Um ein Riff in der Mitte der Bühne, welches mit seiner schlanken Spitze bis in die dichtere, heller dämmernde Wasserflut hinaufragt, kreist in anmutig schwimmender Bewegung eine der Rheintöchter}

\Woglindespeaks
Weia! Waga! Woge, du Welle,
walle zur Wiege! Wagalaweia!
Wallala, weiala weia!
 

\Wellgundespeaks

\direct{Stimme von oben}

Woglinde, wachst du allein?
 

\Woglindespeaks
Mit Wellgunde wär' ich zu zwei.
 

\Wellgundespeaks

\direct{taucht aus der Flut zum Riff herab}

Laß sehn, wie du wachst!
 


\direct{sie sucht Woglinde zu erhaschen}


\Woglindespeaks

\direct{entweicht ihr schwimmend}

Sicher vor dir!
 


\direct{sie necken sich und suchen sich spielend zu fangen}



\Flosshildespeaks

\direct{Stimme von oben}

Heiaha weia! Wildes Geschwister!
 

\Wellgundespeaks
Flosshilde, schwimm'! Woglinde flieht:
hilf mir die Fließende fangen!
 

\Flosshildespeaks

\direct{taucht herab und fährt zwischen die Spielenden}

Des Goldes Schlaf hütet ihr schlecht!
Besser bewacht des schlummernden Bett,
sonst büßt ihr beide das Spiel!
 

\StageDir{Mit muntrem Gekreisch fahren die beiden auseinander. Flosshilde sucht bald die eine, bald die andere zu erhaschen; sie entschlüpfen ihr und vereinigen sich endlich, um gemeinschaftlich auf Flosshilde Jagd zu machen. So schnellen sie gleich Fischen von Riff zu Riff, scherzend und lachend.
  
Aus einer finstern Schluft ist währenddem Alberich, an einem Riffe klimmend, dem Abgrunde entstiegen. Er hält, noch vom Dunkel umgeben, an und schaut dem Spiele der Rheintöchter mit steigendem Wohlgefallen zu.}

\Alberichspeaks
Hehe! Ihr Nicker!
Wie seid ihr niedlich, neidliches Volk!
Aus Nibelheims Nacht naht' ich mich gern,
neigtet ihr euch zu mir!
 


\direct{die Mädchen halten, sobald sie Alberichs Stimme hören, mit dem Spiele ein}


\Woglindespeaks
Hei! Wer ist dort?


\Wellgundespeaks
Es dämmert und ruft!
 

\Flosshildespeaks
Lugt, wer uns lauscht!
 

\speaker{\Woglinde und \Wellgunde}

\direct{sie tauchen tiefer herab und erkennen den Nibelung}

Pfui! Der Garstige!
 

\Flosshildespeaks

\direct{schnell auftauchend}

Hütet das Gold!
Vater warnte vor solchem Feind.
 


\direct{Die beiden andern folgen ihr, und alle drei versammeln sich schnell um das mittlere Riff}


\Alberichspeaks
Ihr, da oben!
 

\speaker{Die drei Rheintöchter}
Was willst du dort unten?
 

\Alberichspeaks
Stör' ich eu'r Spiel,
wenn staunend ich still hier steh'?
Tauchtet ihr nieder, mit euch tollte
und neckte der Niblung sich gern!
 

\Woglindespeaks
Mit uns will er spielen?
 

\Wellgundespeaks
Ist ihm das Spott?
 

\Alberichspeaks
Wie scheint im Schimmer ihr hell und schön!
Wie gern umschlänge der Schlanken eine mein Arm,
schlüpfte hold sie herab!
 

\Flosshildespeaks
Nun lach' ich der Furcht: der Feind ist verliebt!
 


\direct{Sie lachen}


\Wellgundespeaks
Der lüsterne Kauz!
 

\Woglindespeaks
Laßt ihn uns kennen!
 


\direct{Sie läßt sich auf die Spitze des Riffes hinab, an dessen Fuße Alberich angelangt ist}


\Alberichspeaks
Die neigt sich herab.
 

\Woglindespeaks
Nun nahe dich mir!
 


\direct{Alberich klettert mit koboldartiger Behendigkeit, doch wiederholt aufgehalten, der Spitze des Riffes zu}


\Alberichspeaks
Garstig glatter glitschiger Glimmer!
Wie gleit' ich aus! Mit Händen und Füßen
nicht fasse noch halt' ich das schlecke Geschlüpfer!
 


\direct{er prustet}

Feuchtes Naß füllt mir die Nase:
verfluchtes Niesen!
 


\direct{er ist in Woglindes Nähe angelangt}


\Woglindespeaks

\direct{lachend}

Prustend naht meines Freiers Pracht!
 

\Alberichspeaks
Mein Friedel sei, du fräuliches Kind!
 


\direct{er sucht sie zu umfassen}


\Woglindespeaks

\direct{sich ihm entwindend}

Willst du mich frei'n, so freie mich hier!
 


\direct{sie taucht auf einem andern Riff auf, die Schwestern lachen}


\Alberichspeaks

\direct{kratzt sich den Kopf}

O weh! Du entweichst? Komm' doch wieder!
Schwer ward mir, was so leicht du erschwingst.
 

\Woglindespeaks

\direct{schwingt sich auf ein drittes Riff in größerer Tiefe}

Steig' nur zu Grund, da greifst du mich sicher!
 

\Alberichspeaks

\direct{hastig hinab kletternd}

Wohl besser da unten!
 

\Woglindespeaks

\direct{schnellt sich rasch aufwärts nach einem hohen Seitenriffe}

Nun aber nach oben!
 

\speaker{\Wellgunde und \Flosshilde}
Hahahahaha!
 

\Alberichspeaks
Wie fang' ich im Sprung den spröden Fisch?
Warte, du Falsche!
 


\direct{er will ihr eilig nachklettern}


\Wellgundespeaks

\direct{hat sich auf ein tieferes Riff auf der anderen Seite gesenkt}

Heia, du Holder! Hörst du mich nicht?
 

\Alberichspeaks

\direct{sich umwendend}

Rufst du nach mir?
 

\Wellgundespeaks
Ich rate dir wohl: zu mir wende dich,
Woglinde meide!
 

\Alberichspeaks

\direct{klettert hastig über den Bodengrund zu Wellgunde}

Viel schöner bist du als jene Scheue,
die minder gleißend und gar zu glatt.
Nur tiefer tauche, willst du mir taugen.
 

\Wellgundespeaks

\direct{noch etwas mehr sich zu ihm herabsenkend}

Bin nun ich dir nah?
 

\Alberichspeaks
Noch nicht genug!
Die schlanken Arme schlinge um mich,
daß ich den Nacken dir neckend betaste,
mit schmeichelnder Brunst
an die schwellende Brust mich dir schmiege.
 

\Wellgundespeaks
Bist du verliebt und lüstern nach Minne,
laß sehn, du Schöner, wie bist du zu schau'n?
Pfui! Du haariger, höckriger Geck!
Schwarzes, schwieliges Schwefelgezwerg!
Such' dir ein Friedel, dem du gefällst!
 

\Alberichspeaks

\direct{sucht sie mit Gewalt zu halten}

Gefall' ich dir nicht, dich fass' ich doch fest!
 

\Wellgundespeaks

\direct{schnell zum mittleren Riffe auftauchend}

Nur fest, sonst fließ ich dir fort!
 

\speaker{\Woglinde und \Flosshilde}
Hahahahaha!
 

\Alberichspeaks

\direct{Wellgunde erbost nachzankend}

Falsches Kind! Kalter, grätiger Fisch!
Schein' ich nicht schön dir,
niedlich und neckisch, glatt und glau---
hei, so buhle mit Aalen, ist dir eklig mein Balg!
 

\Flosshildespeaks
Was zankst du, Alp? Schon so verzagt?
Du freitest um zwei: frügst du die dritte,
süßen Trost schüfe die Traute dir!
 

\Alberichspeaks
Holder Sang singt zu mir her!
Wie gut, daß ihr eine nicht seid!
Von vielen gefall' ich wohl einer:
bei einer kieste mich keine!
Soll ich dir glauben, so gleite herab!
 

\Flosshildespeaks

\direct{taucht zu Alberich hinab}

Wie törig seid ihr, dumme Schwestern,
dünkt euch dieser nicht schön!
 

\Alberichspeaks

\direct{ihr nahend}

Für dumm und häßlich darf ich sie halten,
seit ich dich Holdeste seh'.
 

\Flosshildespeaks

\direct{schmeichelnd}

O singe fort so süß und fein,
wie hehr verführt es mein Ohr!
 

\Alberichspeaks

\direct{zutraulich sie berührend}

Mir zagt, zuckt und zehrt sich das Herz,
lacht mir so zierliches Lob.
 

\Flosshildespeaks

\direct{ihn sanft abwehrend}

Wie deine Anmut mein Aug' erfreut,
deines Lächelns Milde den Mut mir labt!
 


\direct{Sie zieht ihn selig an sich}

Seligster Mann!
 

\Alberichspeaks
Süßeste Maid!
 

\Flosshildespeaks
Wärst du mir hold!
 

\Alberichspeaks
Hielt dich immer!
 

\Flosshildespeaks

\direct{ihn ganz in ihren Armen haltend}

Deinen stechenden Blick, deinen struppigen Bart,
o säh ich ihn, faßt' ich ihn stets!
Deines stachligen Haares strammes Gelock,
umflöß es Flosshilde ewig!
Deine Krötengestalt, deiner Stimme Gekrächz,
o dürft' ich staunend und stumm
sie nur hören und sehn!
 

\speaker{\Woglinde und \Wellgunde}
Hahahahaha!
 

\Alberichspeaks

\direct{erschreckt aus Flosshildes Armen auffahrend}

Lacht ihr Bösen mich aus?
 

\Flosshildespeaks

\direct{sich plötzlich ihm entreissend}

Wie billig am Ende vom Lied!
 


\direct{sie taucht mit den Schwestern schnell auf}


\speaker{\Woglinde und \Wellgunde}
Hahahahaha!
 

\Alberichspeaks

\direct{mit kreischender Stimme}

Wehe! Ach wehe! O Schmerz! O Schmerz!
Die dritte, so traut, betrog sie mich auch?
Ihr schmählich schlaues, lüderlich schlechtes Gelichter!
Nährt ihr nur Trug, ihr treuloses Nickergezücht?
 

\speaker{Die drei Rheintöchter}
Wallala! Lalaleia! Leialalei!
Heia! Heia! Haha!
Schäme dich, Albe! Schilt nicht dort unten!
Höre, was wir dich heißen!
Warum, du Banger, bandest du nicht
das Mädchen, das du minnst?
Treu sind wir und ohne Trug
dem Freier, der uns fängt.
Greife nur zu, und grause dich nicht!
In der Flut entflieh'n wir nicht leicht!
Wallala! Lalaleia! Leialalei!
Heia! Heia! Haha!
 


\direct{Sie schwimmen auseinander, hierher und dorthin, bald tiefer, bald höher, um Alberich zur Jagd auf sie zu reizen}


\Alberichspeaks
Wie in den Gliedern brünstige Glut
mir brennt und glüht!
Wut und Minne, wild und mächtig,
wühlt mir den Mut auf!
Wie ihr auch lacht und lügt,
lüstern lechz' ich nach euch,
und eine muß mir erliegen!
 


\direct{Er macht sich mit verzweifelter Anstrengung zur Jagd auf: mit grauenhafter Behendigkeit erklimmt er Riff für Riff, springt von einem zum andern, sucht bald dieses, bald jenes der Mädchen zu erhaschen, die mit lustigem Gekreisch stets ihm entweichen. Er strauchelt, stürzt in den Abgrund hinab, klettert dann hastig wieder in die Höhe zu neuer Jagd. Sie neigen sich etwas herab. Fast erreicht er sie, stürzt abermals zurück und versucht es nochmals. Er hält endlich, vor Wut schäumend, atemlos an und streckt die geballte Faust nach den Mädchen hinauf.}


\Alberichspeaks

\direct{kaum seiner mächtig}

Fing' eine diese Faust!...
 


\direct{Er verbleibt in sprachloser Wut, den Blick aufwärts gerichtet, wo er dann plötzlich von dem folgenden Schauspiele angezogen und gefesselt wird. Durch die Flut ist von oben her ein immer lichterer Schein gedrungen, der sich an einer hohen Stelle des mittelsten Riffes allmählich zu einem blendend hell strahlenden Goldglanze entzündet: ein zauberisch goldenes Licht bricht von hier durch das Wasser}


\Woglindespeaks
Lugt, Schwestern!
Die Weckerin lacht in den Grund.
 

\Wellgundespeaks
Durch den grünen Schwall
den wonnigen Schläfer sie grüßt.
 

\Flosshildespeaks
Jetzt küßt sie sein Auge, daß er es öffne.
 

\Wellgundespeaks
Schaut, er lächelt in lichtem Schein.
 

\Woglindespeaks
Durch die Fluten hin fließt sein strahlender Stern!
 

\speaker{Die drei Rheintöchter}

\direct{zusammen das Riff anmutig umschwimmend}

Heiajaheia! Heiajaheia!
Wallalalalala leiajahei!
Rheingold! Rheingold!
Leuchtende Lust, wie lachst du so hell und hehr!
Glühender Glanz entgleißet dir weihlich im Wag'!
Heiajaheia! Heiajaheia!
Wache, Freund, Wache froh!
Wonnige Spiele spenden wir dir:
flimmert der Fluß, flammet die Flut,
umfließen wir tauchend, tanzend und singend
im seligem Bade dein Bett!
Rheingold! Rheingold!
Heiajaheia! Wallalalalala leiajahei!
 


\direct{Mit immer ausgelassenerer Lust umschwimmen die Mädchen das Riff. Die ganze Flut flimmert in hellem Goldglanze}


\Alberichspeaks

\direct{dessen Augen, mächtig vom Glanze angezogen, starr an dem Golde haften}

Was ist's, ihr Glatten, das dort so glänzt und gleißt?
 

\speaker{Die drei Rheintöchter}
Wo bist du Rauher denn heim,
daß vom Rheingold nie du gehört?
 

\Wellgundespeaks
Nichts weiß der Alp von des Goldes Auge,
das wechselnd wacht und schläft?
 

\Woglindespeaks
Von der Wassertiefe wonnigem Stern,
der hehr die Wogen durchhellt?
 

\speaker{Die drei Rheintöchter}
Sieh, wie selig im Glanze wir gleiten!
Willst du Banger in ihm dich baden,
so schwimm' und schwelge mit uns!
Wallalalala leialalai! Wallalalala leiajahei!
 

\Alberichspeaks
Eurem Taucherspiele nur taugte das Gold?
Mir gält' es dann wenig!
 

\Woglindespeaks
Des Goldes Schmuck schmähte er nicht,
wüßte er all seine Wunder!
 

\Wellgundespeaks
Der Welt Erbe gewänne zu eigen,
wer aus dem Rheingold schüfe den Ring,
der maßlose Macht ihm verlieh'.
 

\Flosshildespeaks
Der Vater sagt' es, und uns befahl er,
klug zu hüten den klaren Hort,
daß kein Falscher der Flut ihn entführe:
drum schweigt, ihr schwatzendes Heer!
 

\Wellgundespeaks
Du klügste Schwester, verklagst du uns wohl?
Weißt du denn nicht, wem nur allein
das Gold zu schmieden vergönnt?
 

\Woglindespeaks
Nur wer der Minne Macht entsagt,
nur wer der Liebe Lust verjagt,
nur der erzielt sich den Zauber,
zum Reif zu zwingen das Gold.
 

\Wellgundespeaks
Wohl sicher sind wir und sorgenfrei:
denn was nur lebt, will lieben,
meiden will keiner die Minne.
 

\Woglindespeaks
Am wenigsten er, der lüsterne Alp;
vor Liebesgier möcht' er vergehn!
 

\Flosshildespeaks
Nicht fürcht' ich den, wie ich ihn erfand:
seiner Minne Brunst brannte fast mich.
 

\Wellgundespeaks
Ein Schwefelbrand in der Wogen Schwall:
vor Zorn der Liebe zischt er laut!
 

\speaker{Die drei Rheintöchter}
Wallala! Wallaleialala!
Lieblichster Albe! Lachst du nicht auch?
In des Goldes Scheine wie leuchtest du schön!
O komm', Lieblicher, lache mit uns!
Heiajaheia! Heiajaheia! Wallalalala leiajahei!
 


\direct{Sie schwimmen lachend im Glanze auf und ab}


\Alberichspeaks

\direct{die Augen starr auf das Gold gerichtet, hat dem Geplauder der Schwestern wohl gelauscht}

Der Welt Erbe
gewänn' ich zu eigen durch dich?
Erzwäng' ich nicht Liebe,
doch listig erzwäng' ich mir Lust?
 


\direct{furchtbar laut}

Spottet nur zu!
Der Niblung naht eurem Spiel!
 


\direct{wütend springt er nach dem mittleren Riff hinüber und klettert in grausiger Hast nach dessen Spitze hinauf. Die Mädchen fahren kreischend auseinander und tauchen nach verschiedenen Seiten hin auf}


Die drei Rheintöchter
Heia! Heia! Heiajahei!
Rettet euch! Es raset der Alp:
in den Wassern sprüht's, wohin er springt:
die Minne macht ihn verrückt!
 


\direct{sie lachen im tollsten Übermut}


\Alberichspeaks

\direct{gelangt mit einem letzten Satze zur Spitze des Riffes}

Bangt euch noch nicht?
So buhlt nun im Finstern, feuchtes Gezücht!
 


\direct{er streckt die Hand nach dem Golde aus}

Das Licht lösch' ich euch aus, entreiße dem Riff das Gold,
schmiede den rächende Ring;
denn hör' es die Flut: so verfluch' ich die Liebe!
 


\direct{Er reißt mit furchtbarer Gewalt das Gold aus dem Riffe und stürzt damit hastig in die Tiefe, wo er schnell verschwindet. Dichte Nacht bricht plötzlich überall herein. Die Mädchen tauchen dem Räuber in die Tiefe nach}


\Flosshildespeaks
Haltet den Räuber!
 

\Wellgundespeaks
Rettet das Gold!
 

\speaker{\Woglinde und \Wellgunde}
Hilfe! Hilfe!
 

\speaker{Die drei Rheintöchter}
Weh! Weh!
 

\StageDir{Die Flut fällt mit ihnen nach der Tiefe hinab. Aus dem untersten Grunde hört man Alberichs gellendes Hohngelächter. In dichtester Finsternis verschwinden die Riffe; die ganze Bühne ist von der Höhe bis zur Tiefe von schwarzem Wassergewoge erfüllt, das eine Zeitlang immer noch abwärts zu sinken scheint.}

\scene

\StageDir{Allmählich sind die Wogen in Gewölke übergegangen, welches, als eine immer heller dämmernde Beleuchtung dahinter tritt, zu feinerem Nebel sich abklärt. Als der Nebel in zarten Wölkchen gänzlich sich in der Höhe verliert, wird im Tagesgrauen eine
freie Gegend auf Bergeshöhen
sichtbar. Der hervorbrechende Tag beleuchtet mit wachsendem Glanze eine Burg mit blinkenden Zinnen, die auf einem Felsgipfel im Hintergrunde steht; zwischen diesem burggekrönten Felsgipfel und dem Vordergrunde der Szene ist ein tiefes Tal, durch welches der Rhein fließt, anzunehmen. Zur Seite auf blumigem Grunde liegt Wotan, neben ihm Fricka, beide schlafend. Die Burg ist ganz sichtbar geworden.}

\Frickaspeaks

\direct{erwacht; ihr Blick fällt auf die Burg; sie staunt und erschrickt}

Wotan, Gemahl, erwache!
 

\Wotanspeaks

\direct{im Traume leise}

Der Wonne seligen Saal
bewachen mir Tür und Tor:
Mannes Ehre, ewige Macht,
ragen zu endlosem Ruhm!
 

\Frickaspeaks

\direct{rüttelt ihn}

Auf, aus der Träume wonnigem Trug!
Erwache, Mann, und erwäge!
 

\Wotanspeaks

\direct{erwacht und erhebt sich ein wenig, sein Auge wird sogleich vom Anblick der Burg gefesselt}

Vollendet das ewige Werk!
Auf Berges Gipfel die Götterburg;
prächtig prahlt der prangende Bau!
Wie im Traum ich ihn trug,
wie mein Wille ihn wies, stark und schön
steht er zur Schau; hehrer, herrlicher Bau!
 

\Frickaspeaks
Nur Wonne schafft dir, was mich erschreckt?
Dich freut die Burg, mir bangt es um Freia!
Achtloser, laß mich erinnern
des ausbedungenen Lohns!
Die Burg ist fertig, verfallen das Pfand:
vergaßest du, was du vergabst?
 

\Wotanspeaks
Wohl dünkt mich's, was sie bedangen,
die dort die Burg mir gebaut;
durch Vertrag zähmt' ich ihr trotzig Gezücht,
daß sie die hehre Halle mir schüfen;
die steht nun, dank den Starken:
um den Sold sorge dich nicht.
 

\Frickaspeaks
O lachend frevelnder Leichtsinn!
Liebelosester Frohmut!
Wußt' ich um euren Vertrag,
dem Truge hätt' ich gewehrt;
doch mutig entferntet ihr Männer die Frauen,
um taub und ruhig vor uns,
allein mit den Riesen zu tagen:
so ohne Scham verschenktet ihr Frechen
Freia, mein holdes Geschwister,
froh des Schächergewerbs!
Was ist euch Harten doch heilig und wert,
giert ihr Männer nach Macht!
 

\Wotanspeaks

\direct{ruhig}

Gleiche Gier war Fricka wohl fremd,
als selbst um den Bau sie mich bat?
 

\Frickaspeaks
Um des Gatten Treue besorgt,
muß traurig ich wohl sinnen,
wie an mich er zu fesseln,
zieht's in die Ferne ihn fort:
herrliche Wohnung, wonniger Hausrat
sollten dich binden zu säumender Rast.
Doch du bei dem Wohnbau sannst auf Wehr und Wall allein;
Herrschaft und Macht soll er dir mehren;
nur rastlosern Sturm zu erregen,
erstand dir die ragende Burg.
 

\Wotanspeaks

\direct{lächelnd}

Wolltest du Frau in der Feste mich fangen,
mir Gotte mußt du schon gönnen,
daß, in der Burg gebunden, ich mir
von außen gewinne die Welt.
Wandel und Wechsel liebt, wer lebt;
das Spiel drum kann ich nicht sparen!
 

\Frickaspeaks
Liebeloser, leidigster Mann!
Um der Macht und Herrschaft müßigen Tand
verspielst du in lästerndem Spott
Liebe und Weibes Wert?
 

\Wotanspeaks

\direct{ernst}

Um dich zum Weib zu gewinnen,
mein eines Auge setzt' ich werbend daran;
wie törig tadelst du jetzt!
Ehr' ich die Frauen doch mehr als dich freut;
und Freia, die gute, geb' ich nicht auf;
nie sann dies ernstlich mein Sinn.
 

\Frickaspeaks

\direct{mit ängstlicher Spannung in die Szene blickend}

So schirme sie jetzt: in schutzloser Angst
läuft sie nach Hilfe dort her!
 

\Freiaspeaks

\direct{tritt wie in hastiger Flucht auf}

Hilf mir, Schwester! Schütze mich, Schwäher!
Vom Felsen drüben drohte mir Fasolt,
mich Holde käm' er zu holen.
 

\Wotanspeaks
Laß ihn droh'n! Sahst du nicht Loge?
 

\Frickaspeaks
Daß am liebsten du immer dem Listigen traust!
Viel Schlimmes schuf er uns schon,
doch stets bestrickt er dich wieder.
 

\Wotanspeaks
Wo freier Mut frommt,
allein frag' ich nach keinem.
Doch des Feindes Neid zum Nutz sich fügen,
lehrt nur Schlauheit und List,
wie Loge verschlagen sie übt.
Der zum Vertrage mir riet,
versprach mir, Freia zu lösen:
auf ihn verlass' ich mich nun.
 

\Frickaspeaks
Und er läßt dich allein!
Dort schreiten rasch die Riesen heran:
wo harrt dein schlauer Gehilf'?
 

\Freiaspeaks
Wo harren meine Brüder, daß Hilfe sie brächten,
da mein Schwäher die Schwache verschenkt?
Zu Hilfe, Donner! Hieher, hieher!
Rette Freia, mein Froh!
 

\Frickaspeaks
Die in bösem Bund dich verrieten,
sie alle bergen sich nun!
 


\direct{Fasolt und Fafner, beide in riesiger Gestalt, mit starken Pfählen bewaffnet, treten auf}


\Fasoltspeaks
Sanft schloß Schlaf dein Aug';
wir beide bauten Schlummers bar die Burg.
Mächt'ger Müh' müde nie,
stauten starke Stein' wir auf;
steiler Turm, Tür und Tor,
deckt und schließt im schlanken Schloß den Saal.
 


\direct{Auf die Burg deutend}

Dort steht's, was wir stemmten,
schimmernd hell, bescheint's der Tag:
zieh nun ein, uns zahl' den Lohn!
 

\Wotanspeaks
Nennt, Leute, den Lohn:
was dünkt euch zu bedingen?
 

\Fasoltspeaks
Bedungen ist, was tauglich uns dünkt:
gemahnt es dich so matt?
Freia, die Holde, Holda, die Freie,
vertragen ist's, sie tragen wir heim.
 

\Wotanspeaks

\direct{schnell}

Seid ihr bei Trost mit eurem Vertrag?
Denkt auf andern Dank: Freia ist mir nicht feil.
 

\Fasoltspeaks

\direct{steht, in höchster Bestürzung, einen Augenblick sprachlos}

Was sagst du? Ha, sinnst du Verrat?
Verrat am Vertrag? Die dein Speer birgt,
sind sie dir Spiel, des berat'nen Bundes Runen?
 

\Fafnerspeaks

\direct{höhnisch}

Getreu'ster Bruder,
merkst du Tropf nun Betrug?
 

\Fasoltspeaks
Lichtsohn du, leicht gefügter!
Hör' und hüte dich: Verträgen halte Treu'!
Was du bist, bist du nur durch Verträge;
bedungen ist, wohl bedacht deine Macht.
Bist weiser du, als witzig wir sind,
bandest uns Freie zum Frieden du:
all deinem Wissen fluch' ich,
fliehe weit deinen Frieden,
weißt du nicht offen, ehrlich und frei
Verträgen zu wahren die Treu'!
Ein dummer Riese rät dir das:
Du Weiser, wiss' es von ihm.
 

\Wotanspeaks
Wie schlau für Ernst du achtest,
was wir zum Scherz nur beschlossen!
Die liebliche Göttin, licht und leicht,
was taugt euch Tölpeln ihr Reiz?
 

\Fasoltspeaks
Höhnst du uns? Ha, wie unrecht!
Die ihr durch Schönheit herrscht,
schimmernd hehres Geschlecht,
wir törig strebt ihr nach Türmen von Stein,
setzt um Burg und Saal
Weibes Wonne zum Pfand!
Wir Plumpen plagen uns
schwitzend mit schwieliger Hand,
ein Weib zu gewinnen, das wonnig und mild
bei uns Armen wohne;
und verkehrt nennst du den Kauf?
 

\Fafnerspeaks
Schweig' dein faules Schwatzen,
Gewinn werben wir nicht:
Freias Haft hilft wenig,
doch viel gilt's den Göttern sie zu entreißen.
 


\direct{leise}

Goldene Äpfel wachsen in ihrem Garten;
sie allein weiß die Äpfel zu pflegen!
Der Frucht Genuß frommt ihren Sippen
zu ewig nie alternder Jugend:
siech und bleich doch sinkt ihre Blüte,
alt und schwach schwinden sie hin,
müssen Freia sie missen.
 


\direct{grob}

Ihrer Mitte drum sei sie entführt!
 

\Wotanspeaks

\direct{für sich}

Loge säumt zu lang!
 

\Fasoltspeaks
Schlicht gib nun Bescheid!
 

\Wotanspeaks
Sinnt auf andern Sold!
 

\Fasoltspeaks
Kein andrer: Freia allein!
 

\Fafnerspeaks
Du da! Folg' uns fort!
 


\direct{Sie dringen auf Freia zu}


\Freiaspeaks

\direct{fliehend}

Helft! Helft, vor den Harten!
 

\Frohspeaks

\direct{Freia in seine Arme fassend}

Zu mir, Freia! Meide sie, Frecher!
Froh schützt die Schöne.
 

\Donnerspeaks

\direct{sich vor die beiden Riesen stellend}

Fasolt und Fafner,
fühltet ihr schon meines Hammers harten Schlag?
 

\Fafnerspeaks
Was soll das Drohn?
 

\Fasoltspeaks
Was dringst du her?
Kampf kiesten wir nicht,
verlangen nur unsern Lohn.
 

\Donnerspeaks
Schon oft zahlt' ich Riesen den Zoll.
Kommt her, des Lohnes Last
wäg' ich mit gutem Gewicht!
 


\direct{er schwingt den Hammer}


\Wotanspeaks

\direct{seinen Speer zwischen den Streitenden ausstreckend}

Halt, du Wilder! Nichts durch Gewalt!
Verträge schützt meines Speeres Schaft:
spar' deines Hammers Heft!
 

\Freiaspeaks
Wehe! Wehe! Wotan verläßt mich!
 

\Frickaspeaks
Begreif' ich dich noch, grausamer Mann?
 

\Wotanspeaks

\direct{wendet sich ab und sieht Loge kommen}

Endlich Loge! Eiltest du so,
den du geschlossen,
den schlimmen Handel zu schlichten?
 

\Logespeaks

\direct{ist im Hintergrunde aus dem Tale heraufgestiegen}

Wie? Welchen Handel hätt' ich geschlossen?
Wohl was mit den Riesen dort im Rate du dangst?
In Tiefen und Höhen treibt mich mein Hang;
Haus und Herd behagt mir nicht.
Donner und Froh,
die denken an Dach und Fach,
wollen sie frei'n,
ein Haus muß sie erfreu'n.
Ein stolzer Saal, ein starkes Schloß,
danach stand Wotans Wunsch.
Haus und Hof, Saal und Schloß,
die selige Burg, sie steht nun fest gebaut.
Das Prachtgemäuer prüft' ich selbst,
ob alles fest, forscht' ich genau:
Fasolt und Fafner fand ich bewährt:
kein Stein wankt in Gestemm'.
Nicht müßig war ich, wie mancher hier;
der lügt, wer lässig mich schilt!
 

\Wotanspeaks
Arglistig weichst du mir aus:
mich zu betrügen hüte in Treuen dich wohl!
Von allen Göttern dein einz'ger Freund,
nahm ich dich auf in der übel trauenden Troß.
Nun red' und rate klug!
Da einst die Bauer der Burg
zum Dank Freia bedangen,
du weißt, nicht anders willigt' ich ein,
als weil auf Pflicht du gelobtest,
zu lösen das hehre Pfand.
 

\Logespeaks
Mit höchster Sorge drauf zu sinnen,
wie es zu lösen, das hab' ich gelobt.
Doch, daß ich fände,
was nie sich fügt, was nie gelingt,
wie ließ sich das wohl geloben?
 

\Frickaspeaks

\direct{zu Wotan}

Sieh, welch trugvollem Schelm du getraut!
 

\Frohspeaks
Loge heißt du,
doch nenn' ich dich Lüge!
 

\Donnerspeaks
Verfluchte Lohe, dich lösch' ich aus!
 

\Logespeaks
Ihre Schmach zu decken,
schmähen mich Dumme!
 


\direct{Donner holt auf Loge aus}


\Wotanspeaks

\direct{tritt dazwischen}

In Frieden laßt mir den Freund!
Nicht kennt ihr Loges Kunst:
reicher wiegt seines Rates Wert,
zahlt er zögernd ihn aus.
 

\Fafnerspeaks
Nichts gezögert! Rasch gezahlt!
 

\Fasoltspeaks
Lang währt's mit dem Lohn!
 

\Wotanspeaks

\direct{wendet sich hart zu Loge, drängend}

Jetzt hör', Störrischer! Halte Stich!
Wo schweiftest du hin und her?
 

\Logespeaks
Immer ist Undank Loges Lohn!
Für dich nur besorgt, sah ich mich um,
durchstöbert' im Sturm alle Winkel der Welt,
Ersatz für Freia zu suchen,
wie er den Riesen wohl recht.
Umsonst sucht' ich, und sehe nun wohl:
in der Welten Ring nichts ist so reich,
als Ersatz zu muten dem Mann
für Weibes Wonne und Wert!
 


\direct{Alle geraten in Erstaunen und verschiedenartige Betroffenheit}

So weit Leben und Weben,
In Wasser, Erd' und Luft,
viel frug' ich, forschte bei allen,
wo Kraft nur sich rührt, und Keime sich regen:
was wohl dem Manne mächt'ger dünk',
als Weibes Wonne und Wert?
Doch so weit Leben und Weben,
verlacht nur ward meine fragende List:
in Wasser, Erd' und Luft,
lassen will nichts von Lieb' und Weib.
Nur einen sah' ich, der sagte der Liebe ab:
um rotes Gold entriet er des Weibes Gunst.
Des Rheines klare Kinder
klagten mir ihre Not:
der Nibelung, Nacht-Alberich,
buhlte vergebens um der Badenden Gunst;
das Rheingold da
raubte sich rächend der Dieb:
das dünkt ihn nun das teuerste Gut,
hehrer als Weibes Huld.
Um den gleißenden Tand,
der Tiefe entwandt,
erklang mir der Töchter Klage:
an dich, Wotan, wenden sie sich,
daß zu Recht du zögest den Räuber,
das Gold dem Wasser wieder gebest,
und ewig es bliebe ihr Eigen.
 


\direct{Hingebende Bewegung aller}

Dir's zu melden, gelobt' ich den Mädchen:
nun löste Loge sein Wort.
 

\Wotanspeaks
Törig bist du, wenn nicht gar tückisch!
Mich selbst siehst du in Not:
wie hülft' ich andern zum Heil?
 

\Fasoltspeaks

\direct{der aufmerksam zugehört, zu Fafner}

Nicht gönn' ich das Gold dem Alben;
viel Not schon schuf uns der Niblung,
doch schlau entschlüpfte unserm
Zwange immer der Zwerg.
 

\Fafnerspeaks
Neue Neidtat sinnt uns der Niblung,
gibt das Gold ihm Macht.
Du da, Loge! Sag' ohne Lug:
was Großes gilt denn das Gold,
daß dem Niblung es genügt?
 

\Logespeaks
Ein Tand ist's in des Wassers Tiefe,
lachenden Kindern zur Lust,
doch ward es zum runden Reife geschmiedet,
hilft es zur höchsten Macht,
gewinnt dem Manne die Welt.
 

\Wotanspeaks

\direct{sinnend}

Von des Rheines Gold hört' ich raunen:
Beute-Runen berge sein roter Glanz;
Macht und Schätze schüf ohne Maß ein Reif.
 

\Frickaspeaks

\direct{leise zu Loge}

Taugte wohl des goldnen Tandes
gleißend Geschmeid
auch Frauen zu schönem Schmuck?
 

\Logespeaks
Des Gatten Treu' ertrotzte die Frau,
trüge sie hold den hellen Schmuck,
den schimmernd Zwerge schmieden,
rührig im Zwange des Reifs.
 

\Frickaspeaks

\direct{schmeichelnd zu Wotan}

Gewänne mein Gatte sich wohl das Gold?
 

\Wotanspeaks

\direct{wie in einem Zustande wachsender Bezauberung}

Des Reifes zu walten,
rätlich will es mich dünken.
Doch wie, Loge, lernt' ich die Kunst?
Wie schüf' ich mir das Geschmeid'?
 

\Logespeaks
Ein Runenzauber zwingt das Gold zum Reif;
keiner kennt ihn;
doch einer übt ihn leicht,
der sel'ger Lieb' entsagt.
 


\direct{Wotan wendet sich unmutig ab}

Das sparst du wohl; zu spät auch kämst du:
Alberich zauderte nicht.
Zaglos gewann er des Zaubers Macht:
 


\direct{grell}

geraten ist ihm der Ring!
 

\Donnerspeaks

\direct{zu Wotan}

Zwang uns allen schüfe der Zwerg,
würd' ihm der Reif nicht entrissen.
 

\Wotanspeaks
Den Ring muß ich haben!
 

\Frohspeaks
Leicht erringt ohne Liebesfluch er sich jetzt.
 

\Logespeaks
Spottleicht, ohne Kunst, wie im Kinderspiel!
 

\Wotanspeaks

\direct{grell}

So rate, wie?
 

\Logespeaks
Durch Raub!
Was ein Dieb stahl, das stiehlst du dem Dieb;
ward leichter ein Eigen erlangt?
Doch mit arger Wehr wahrt sich Alberich;
klug und fein mußt du verfahren,
ziehst den Räuber du zu Recht,
um des Rheines Töchtern, den roten Tand,
 


\direct{mit Wärme}

das Gold wiederzugeben;
denn darum flehen sie dich.
 

\Wotanspeaks
Des Rheines Töchtern? Was taugt mir der Rat?
 

\Frickaspeaks
Von dem Wassergezücht mag ich nichts wissen:
schon manchen Mann---mir zum Leid---
verlockten sie buhlend im Bad.
 


\direct{Wotan steht stumm mit sich kämpfend; die übrigen Götter heften in schweigender Spannung die Blicke auf ihn. Währenddem hat Fafner beiseite mit Fasolt beraten}


\Fafnerspeaks

\direct{zu Fasolt}

Glaub' mir, mehr als Freia
frommt das gleißende Gold:
auch ew'ge Jugend erjagt,
wer durch Goldes Zauber sie zwingt.
 


\direct{Fasolts Gebärde deutet an, daß er sich wider Willen überredet fühlt. Fafner tritt mit Fasolt wieder an Wotan heran.}

Hör', Wotan, der Harrenden Wort!
Freia bleib' euch in Frieden;
leicht'ren Lohn fand ich zur Lösung:
uns rauhen Riesen genügt
des Niblungen rotes Gold.
 

\Wotanspeaks
Seid ihr bei Sinn?
Was nicht ich besitze,
soll ich euch Schamlosen schenken?
 

\Fafnerspeaks
Schwer baute dort sich die Burg;
leicht wird dir's mit list'ger Gewalt
was im Neidspiel nie uns gelang,
den Niblungen fest zu fahn.
 

\Wotanspeaks
Für euch müht' ich mich um den Alben?
Für euch fing' ich den Feind?
Unverschämt und überbegehrlich,
macht euch Dumme mein Dank!
 

\Fasoltspeaks

\direct{ergreift plötzlich Freia und führt sie mit Fafner zur Seite}

Hieher, Maid! In unsre Macht!
Als Pfand folgst du uns jetzt,
bis wir Lösung empfah'n!
 

\Freiaspeaks

\direct{wehklagend}

Wehe! Wehe! Wehe!
 


\direct{alle Götter sind in höchster Bestürzung}


\Fafnerspeaks
Fort von hier sei sie entführt!
Bis Abend---achtet's wohl---
pflegen wir sie als Pfand;
wir kehren wieder; doch kommen wir,
und bereit liegt nicht als Lösung
das Rheingold licht und rot---
 

\Fasoltspeaks
Zu End' ist die Frist dann,
Freia verfallen:
für immer folge sie uns!
 

\Freiaspeaks

\direct{schreiend}

Schwester! Brüder! Rettet! Helft!
 


\direct{sie wird von den hastig enteilenden Riesen fortgetragen}


\Frohspeaks
Auf, ihnen nach!
 

\Donnerspeaks
Breche denn alles!
 


\direct{Sie blicken Wotan fragend an}


\Freiaspeaks

\direct{aus weiter Ferne}

Rettet! Helft!
 

\Logespeaks

\direct{den Riesen nachsehend}

Über Stock und Stein zu Tal
stapfen sie hin:
durch des Rheines Wasserfurt
waten die Riesen.
Fröhlich nicht hängt Freia
den Rauhen über dem Rücken! -
Heia! Hei! Wie taumeln die Tölpel dahin!
Durch das Tal talpen sie schon.
Wohl an Riesenheims Mark
erst halten sie Rast. -
 


\direct{er wendet sich zu den Göttern}

Was sinnt nun Wotan so wild?
Den sel'gen Göttern wie geht's?
 


\direct{Ein fahler Nebel erfüllt mit wachsender Dichtheit die Bühne; in ihm erhalten die Götter ein zunehmend bleiches und ältliches Aussehen: alle stehen bang und erwartungsvoll auf Wotan blickend, der sinnend die Augen an den Boden heftet}


\Logespeaks
Trügt mich ein Nebel?
Neckt mich ein Traum?
Wie bang und bleich verblüht ihr so bald!
Euch erlischt der Wangen Licht;
der Blick eures Auges verblitzt!
Frisch, mein Froh, noch ist's ja früh!
Deiner Hand, Donner, entsinkt ja der Hammer!
Was ist's mit Fricka? Freut sie sich wenig
ob Wotans grämlichem Grau,
das schier zum Greisen ihn schafft?
 

\Frickaspeaks
Wehe! Wehe! Was ist geschehen?
 

\Donnerspeaks
Mir sinkt die Hand!
 

\Frohspeaks
Mir stockt das Herz!
 

\Logespeaks
Jetzt fand' ich's: hört, was euch fehlt!
Von Freias Frucht genosset ihr heute noch nicht.
Die goldnen Äpfel in ihrem Garten,
sie machten euch tüchtig und jung,
aßt ihr sie jeden Tag.
Des Gartens Pflegerin ist nun verpfändet;
an den Ästen darbt und dorrt das Obst,
bald fällt faul es herab. -
Mich kümmert's minder;
an mir ja kargte Freia von je
knausernd die köstliche Frucht:
denn halb so echt nur bin ich wie, Selige, ihr!
Doch ihr setztet alles auf das jüngende Obst:
das wußten die Riesen wohl;
auf eurer Leben legten sie's an:
nun sorgt, wie ihr das wahrt!
Ohne die Äpfel,
alt und grau, greis und grämlich,
welkend zum Spott aller Welt,
erstirbt der Götter Stamm.
 

\Frickaspeaks

\direct{bang}

Wotan, Gemahl, unsel'ger Mann!
Sieh, wie dein Leichtsinn lachend uns allen
Schimpf und Schmach erschuf!
 

\Wotanspeaks

\direct{mit plötzlichem Entschluß auffahrend}

Auf, Loge, hinab mit mir!
Nach Nibelheim fahren wir nieder:
gewinnen will ich das Gold.
 

\Logespeaks
Die Rheintöchter riefen dich an:
so dürfen Erhörung sie hoffen?
 

\Wotanspeaks

\direct{heftig}

Schweige, Schwätzer!
Freia, die Gute, Freia gilt es zu lösen!
 

\Logespeaks
Wie du befiehlst
führ' ich dich gern
steil hinab
steigen wir denn durch den Rhein?
 

\Wotanspeaks
Nicht durch den Rhein!
 

\Logespeaks
So schwingen wir uns durch die Schwefelkluft.
Dort schlüpfe mit mir hinein!
 


\direct{Er geht voran und verschwindet seitwärts in einer Kluft, aus der sogleich ein schwefliger Dampf hervorquillt}


\Wotanspeaks
Ihr andern harrt bis Abend hier:
verlorner Jugend erjag' ich erlösendes Gold!
 


\direct{Er steigt Loge nach in die Kluft hinab: der aus ihr dringende Schwefeldampf verbreitet sich über die ganze Bühne und erfüllt diese schnell mit dickem Gewölk. Bereits sind die Zurückbleibenden unsichtbar.}


\Donnerspeaks
Fahre wohl, Wotan!
 

\Frohspeaks
Glück auf! Glück auf!
 

\Frickaspeaks
O kehre bald zur bangenden Frau!
 

\StageDir{Der Schwefeldampf verdüstert sich bis zu ganz schwarzem Gewölk, welches von unten nach oben steigt; dann verwandelt sich dieses in festes, finsteres Steingeklüft, das sich immer aufwärts bewegt, so daß es den Anschein hat, als sänke die Szene immer tiefer in die Erde hinab. Wachsendes Geräusch wie von Schmiedenden wird überallher vernommen.}
 
\scene
 

\StageDir{Von verschiedenen Seiten her dämmert aus der Ferne dunkelroter Schein auf: eine unabsehbar weit sich dahinziehende unterirdische Kluft wird erkennbar, die nach allen Seiten hin in enge Schachte auszumünden scheint.

  Alberich zerrt den kreischenden Mime an den Ohren aus einer Seitenschlucht herbei.}

\Alberichspeaks
Hehe! Hehe!
Hieher! Hieher! Tückischer Zwerg!
Tapfer gezwickt sollst du mir sein,
schaffst du nicht fertig, wie ich's bestellt,
zur Stund' das feine Geschmeid'!
 

\Mimespeaks

\direct{heulend}

Ohe! Ohe! Au! Au!
Laß mich nur los!
Fertig ist's, wie du befahlst,
mit Fleiß und Schweiß ist es gefügt:
nimm nur
 


\direct{grell}

die Nägel vom Ohr!
 

\Alberichspeaks

\direct{loslassend}

Was zögerst du dann
und zeigst es nicht?
 

\Mimespeaks
Ich Armer zagte,
daß noch was fehle.
 

\Alberichspeaks
Was wär' noch nicht fertig?
 

\Mimespeaks

\direct{verlegen}

Hier - und da -
 

\Alberichspeaks
Was hier und da? Her das Geschmeid'!
 


\direct{Er will ihm wieder an das Ohr fahren; vor Schreck läßt Mime ein metallenes Gewirke, das er krampfhaft in den Händen hielt, sich entfallen. Alberich hebt es hastig auf und prüft es genau.}

Schau, du Schelm! Alles geschmiedet
und fertig gefügt, wie ich's befahl!
So wollte der Tropf schlau mich betrügen?
Für sich behalten das hehre Geschmeid',
das meine List ihn zu schmieden gelehrt?
Kenn' ich dich dummen Dieb?
 


\direct{Er setzt das Gewirk als ``Tarnhelm'' auf den Kopf}

Dem Haupt fügt sich der Helm:
ob sich der Zauber auch zeigt?
 


\direct{sehr leise}

``Nacht und Nebel - niemand gleich!''
 


\direct{seine Gestalt verschwindet; statt ihrer gewahrt man eine Nebelsäule}

Siehst du mich, Bruder?
 

\Mimespeaks

\direct{blickt sich verwundert um}

Wo bist du? Ich sehe dich nicht.
 

\Alberichspeaks

\direct{unsichtbar}

So fühle mich doch, du fauler Schuft!
Nimm das für dein Diebesgelüst!
 

\Mimespeaks

\direct{schreit und windet sich unter empfangenen Geißelhieben, deren Fall man vernimmt, ohne die Geißel selbst zu sehen}

Ohe, Ohe! Au! Au! Au!
 

\Alberichspeaks

\direct{lachend, unsichtbar}

Hahahahahaha!
Hab' Dank, du Dummer!
Dein Werk bewährt sich gut!
Hoho! Hoho!
Niblungen all', neigt euch nun Alberich!
Überall weilt er nun, euch zu bewachen;
Ruh' und Rast ist euch zerronnen;
ihm müßt ihr schaffen wo nicht ihr ihn schaut;
wo nicht ihr ihn gewahrt, seid seiner gewärtig!
Untertan seid ihr ihm immer!
 


\direct{grell}

Hoho! Hoho! Hört' ihn, er naht:
der Niblungen Herr!
 


\direct{Die Nebelsäule verschwindet dem Hintergrunde zu: man hört in immer weiterer Ferne Alberichs Toben und Zanken; Geheul und Geschrei antwortet ihm, das sich endlich in immer weiterer Ferne unhörbar verliert. Mime ist vor Schmerz zusammengesunken. Wotan und Loge lassen sich aus einer Schlucht von oben herab.}


\Logespeaks
Nibelheim hier:
Durch bleiche Nebel
was blitzen dort feurige Funken?
 

\Mimespeaks
Au! Au! Au!
 

\Wotanspeaks
Hier stöhnt es laut:
was liegt im Gestein?
 

\Logespeaks

\direct{neigt sich zu Mime}

Was Wunder wimmerst du hier?
 

\Mimespeaks
Ohe! Ohe! Au! Au!
 

\Logespeaks
Hei, Mime! Munt'rer Zwerg!
Was zwickt und zwackt dich denn so?
 

\Mimespeaks
Laß mich in Frieden!
 

\Logespeaks
Das will ich freilich,
und mehr noch, hör':
helfen will ich dir, Mime!
 


\direct{Er stellt ihn mühsam aufrecht}


\Mimespeaks
Wer hälfe mir?
Gehorchen muß ich dem leiblichen Bruder,
der mich in Bande gelegt.
 

\Logespeaks
Dich, Mime, zu binden,
was gab ihm die Macht?
 

\Mimespeaks
Mit arger List schuf sich Alberich
aus Rheines Gold einem gelben Reif:
seinem starken Zauber zittern wir staunend;
mit ihm zwingt er uns alle,
der Niblungen nächt'ges Heer.
Sorglose Schmiede, schufen wir sonst wohl
Schmuck unsern Weibern, wonnig Geschmeid',
niedlichen Niblungentand;
wir lachten lustig der Müh'.
Nun zwingt uns der Schlimme,
in Klüfte zu schlüpfen,
für ihn allein uns immer zu müh'n.
Durch des Ringes Gold errät seine Gier,
wo neuer Schimmer in Schachten sich birgt:
da müssen wir spähen, spüren und graben,
die Beute schmelzen und schmieden den Guß,
ohne Ruh' und Rast
dem Herrn zu häufen den Hort.
 

\Logespeaks
Dich Trägen so eben traf wohl sein Zorn?
 

\Mimespeaks
Mich Ärmsten, ach, mich zwang er zum Ärgsten:
ein Helmgeschmeid' hieß er mich schweißen;
genau befahl er, wie es zu fügen.
Wohl merkt' ich klug, welch mächtige Kraft
zu eigen dem Werk, das aus Erz ich wob;
für mich drum hüten wollt' ich dem Helm;
durch seinen Zauber
Alberichs Zwang mich entzieh'n:
vielleicht - ja vielleicht
den Lästigen selbst überlisten,
in meine Gewalt ihn zu werfen,
den Ring ihm zu entreißen,
daß, wie ich Knecht jetzt dem Kühnen,
 


\direct{grell}

mir Freien er selber dann frön'!
 

\Logespeaks
Warum, du Kluger, glückte dir's nicht?
 

\Mimespeaks
Ach, der das Werk ich wirkte,
den Zauber, der ihm entzuckt,
den Zauber erriet ich nicht recht!
Der das Werk mir riet und mir's entriß,
der lehrte mich nun,
- doch leider zu spät, -
welche List läg' in dem Helm:
Meinem Blick entschwand er,
doch Schwielen dem Blinden
schlug unschaubar sein Arm.
 


\direct{heulend und schluchzend}

Das schuf ich mir Dummen schön zu Dank!
 


\direct{er streicht sich den Rücken. Wotan und Loge lachen}


\Logespeaks

\direct{zu Wotan}

Gesteh', nicht leicht gelingt der Fang.
 

\Wotanspeaks
Doch erliegt der Feind, hilft deine List!
 

\Mimespeaks

\direct{von dem Lachen der Götter betroffen, betrachtet diese aufmerksamer}

Mit eurem Gefrage,
wer seid denn ihr Fremde?
 

\Logespeaks
Freunde dir; von ihrer Not
befrei'n wir der Niblungen Volk!
 

\Mimespeaks

\direct{schrickt zusammen, da er Alberich sich wieder nahen hört}

Nehmt euch in acht! Alberich naht.
 

\Wotanspeaks
Sein' harren wir hier.
 


\direct{Er setzt sich ruhig auf einen Stein; Loge lehnt ihm zur Seite. Alberich, der den Tarnhelm vom Haupte genommen und an den Gürtel gehängt hat, treibt mit geschwungener Geißel aus der unteren, tiefer gelegenen Schlucht aufwärts eine Schar Nibelungen vor sich her: diese sind mit goldenem und silbernem Geschmeide beladen, das sie, unter Alberichs steter Nötigung, all auf einen Haufen speichern und so zu einem Horte häufen.}


\Alberichspeaks
Hieher! Dorthin! Hehe! Hoho!
Träges Heer, dort zu Hauf schichtet den Hort!
Du da, hinauf! Willst du voran?
Schmähliches Volk, ab das Geschmeide!
Soll ich euch helfen? Alle hieher!
 


\direct{er gewahrt plötzlich Wotan und Loge}

He! Wer ist dort? Wer drang hier ein?
Mime, zu mir, schäbiger Schuft!
Schwatztest du gar mit dem schweifenden Paar?
Fort, du Fauler!
Willst du gleich schmieden und schaffen?
 


\direct{Er treibt Mime mit Geißelhieben unter den Haufen der Nibelungen hinein}

He! An die Arbeit!
Alle von hinnen! Hurtig hinab!
Aus den neuen Schachten schafft mir das Gold!
Euch grüßt die Geißel, grabt ihr nicht rasch!
Daß keiner mir müßig, bürge mir Mime,
sonst birgt er sich schwer meiner Geißel Schwunge!
Daß ich überall weile, wo keiner mich wähnt,
das weiß er, dünkt mich, genau!
Zögert ihr noch? Zaudert wohl gar?
 


\direct{Er zieht seinen Ring vom Finger, küßt ihn und streckt ihn drohend aus}

Zittre und zage, gezähmtes Heer!
Rasch gehorcht des Ringes Herrn!
 


\direct{Unter Geheul und Gekreisch stieben die Nibelungen, unter ihnen Mime, auseinander und schlüpfen in die Schächte hinab}


\Alberichspeaks

\direct{betrachtet lange und mißtrauisch Wotan und Loge}

Was wollt ihr hier?
 

\Wotanspeaks
Von Nibelheims nächt'gem Land
vernahmen wir neue Mär':
mächtige Wunder wirke hier Alberich;
daran uns zu weiden, trieb uns Gäste die Gier.
 

\Alberichspeaks
Nach Nibelheim führt euch der Neid:
so kühne Gäste, glaubt, kenn' ich gut!
 

\Logespeaks
Kennst du mich gut, kindischer Alp?
Nun sag', wer bin ich, daß du so bellst?
Im kalten Loch, da kauern du lagst,
wer gab dir Licht und wärmende Lohe,
wenn Loge nie dir gelacht?
Was hülf' dir dein Schmieden,
heizt' ich die Schmiede dir nicht?
Dir bin ich Vetter, und war dir Freund:
nicht fein drum dünkt mich dein Dank!
 

\Alberichspeaks
Den Lichtalben lacht jetzt Loge,
der list'ge Schelm:
bist du falscher ihr Freund,
wie mir Freund du einst warst:
haha! Mich freut's!
Von ihnen fürcht' ich dann nichts.
 

\Logespeaks
So denk' ich, kannst du mir traun?
 

\Alberichspeaks
Deiner Untreu trau' ich, nicht deiner Treu'!
 


\direct{eine herausfordernde Stellung einnehmend}

Doch getrost trotz' ich euch allen!
 

\Logespeaks
Hohen Mut verleiht deine Macht;
grimmig groß wuchs dir die Kraft!
 

\Alberichspeaks
Siehst du den Hort,
den mein Heer dort mir gehäuft?
 

\Logespeaks
So neidlichen sah ich noch nie.
 

\Alberichspeaks
Das ist für heut, ein kärglich Häufchen:
Kühn und mächtig soll er künftig sich mehren.
 

\Wotanspeaks
Zu was doch frommt dir der Hort,
da freudlos Nibelheim,
und nichts für Schätze hier feil?
 

\Alberichspeaks
Schätze zu schaffen und Schätze zu bergen,
nützt mir Nibelheims Nacht.
Doch mit dem Hort, in der Höhle gehäuft,
denk' ich dann Wunder zu wirken:
die ganze Welt gewinn' ich mit ihm mir zu eigen!
 

\Wotanspeaks
Wie beginnst du, Gütiger, das?
 

\Alberichspeaks
Die in linder Lüfte Weh'n da oben ihr lebt,
lacht und liebt: mit goldner Faust
euch Göttliche fang' ich mir alle!
Wie ich der Liebe abgesagt,
alles, was lebt, soll ihr entsagen!
Mit Golde gekirrt,
nach Gold nur sollt ihr noch gieren!
Auf wonnigen Höhn,
in seligem Weben wiegt ihr euch;
den Schwarzalben
verachtet ihr ewigen Schwelger!
Habt acht! Habt acht!
Denn dient ihr Männer erst meiner Macht,
eure schmucken Frau'n, die mein Frei'n verschmäht,
sie zwingt zur Lust sich der Zwerg,
lacht Liebe ihm nicht!
 


\direct{wild lachend}

Hahahaha! Habt ihr's gehört?
Habt acht vor dem nächtlichen Heer,
entsteigt des Niblungen Hort
aus stummer Tiefe zu Tag!
 

\Wotanspeaks

\direct{auffahrend}

Vergeh, frevelnder Gauch!
 

\Alberichspeaks
Was sagt der?
 

\Logespeaks

\direct{ist dazwischengetreten}

Sei doch bei Sinnen!
 


\direct{zu Alberich}

Wen doch faßte nicht Wunder,
erfährt er Alberichs Werk?
Gelingt deiner herrlichen List,
was mit dem Horte du heischest:
den Mächtigsten muß ich dich rühmen;
denn Mond und Stern', und die strahlende Sonne,
sie auch dürfen nicht anders,
dienen müssen sie dir.
Doch - wichtig acht' ich vor allem,
daß des Hortes Häufer, der Niblungen Heer,
neidlos dir geneigt.
Einen Reif rührtest du kühn;
dem zagte zitternd dein Volk: -
doch, wenn im Schlaf ein Dieb dich beschlich',
den Ring schlau dir entriss', -
wie wahrtest du, Weiser, dich dann?
 

\Alberichspeaks
Der Listigste dünkt sich Loge;
andre denkt er immer sich dumm:
daß sein' ich bedürfte zu Rat und Dienst,
um harten Dank,
das hörte der Dieb jetzt gern!
Den hehlenden Helm ersann ich mir selbst;
der sorglichste Schmied,
Mime, mußt' ihn mir schmieden:
schnell mich zu wandeln, nach meinem Wunsch
die Gestalt mir zu tauschen, taugt der Helm.
Niemand sieht mich, wenn er mich sucht;
doch überall bin ich, geborgen dem Blick.
So ohne Sorge
bin ich selbst sicher vor dir,
du fromm sorgender Freund!
 

\Logespeaks
Vieles sah ich, Seltsames fand ich,
doch solches Wunder gewahrt' ich nie.
Dem Werk ohnegleichen kann ich nicht glauben;
wäre das eine möglich,
deine Macht währte dann ewig!
 

\Alberichspeaks
Meinst du, ich lüg' und prahle wie Loge?
 

\Logespeaks
Bis ich's geprüft,
bezweifl' ich, Zwerg, dein Wort.
 

\Alberichspeaks
Vor Klugheit bläht sich
zum Platzen der Blöde!
Nun plage dich Neid!
Bestimm', in welcher Gestalt
soll ich jach vor dir stehn?
 

\Logespeaks
In welcher du willst;
nur mach' vor Staunen mich stumm.
 

\Alberichspeaks

\direct{hat den Helm aufgesetzt}

``Riesen-Wurm winde sich ringelnd!''
 


\direct{Sogleich verschwindet er: eine ungeheure Riesenschlange windet sich statt seiner am Boden; sie bäumt sich und streckt den aufgesperrten Rachen nach Wotan und Loge hin.}


\Logespeaks

\direct{stellt sich von Furcht ergriffen}

Ohe! Ohe!
Schreckliche Schlange, verschlinge mich nicht!
Schone Logen das Leben!
 

\Wotanspeaks
Hahaha! Gut, Alberich!
Gut, du Arger!
Wie wuchs so rasch
zum riesigen Wurme der Zwerg!
 

\direct{Die Schlange verschwindet; statt ihrer erscheint sogleich Alberich wieder in seiner wirklichen Gestalt.}

\Alberichspeaks
Hehe! Ihr Klugen, glaubt ihr mir nun?
 

\Logespeaks
Mein Zittern mag dir's bezeugen.
Zur großen Schlange schufst du dich schnell:
weil ich's gewahrt,
willig glaub' ich dem Wunder.
Doch, wie du wuchsest,
kannst du auch winzig
und klein dich schaffen?
Das Klügste schien' mir das,
Gefahren schlau zu entfliehn:
das aber dünkt mich zu schwer!
 

\Alberichspeaks
Zu schwer dir, weil du zu dumm!
Wie klein soll ich sein?
 

\Logespeaks
Daß die feinste Klinze dich fasse,
wo bang die Kröte sich birgt.
 

\Alberichspeaks
Pah! Nichts leichter! Luge du her!
 


\direct{Er setzt den Tarnhelm wieder auf}

``Krumm und grau krieche Kröte!''
 


\direct{Er verschwindet; die Götter gewahren im Gestein eine Kröte auf sich zukriechen.}


\Logespeaks
\direct{zu Wotan}
Dort, die Kröte, greife sie rasch!
 


\direct{Wotan setzt seinen Fuß auf die Kröte, Loge fährt ihr nach dem Kopfe und hält den Tarnhelm in der Hand. Alberich wird plötzlich in seiner wirklichen Gestalt sichtbar, wie er sich unter Wotans Fuße windet}


\Alberichspeaks
Ohe! Verflucht! Ich bin gefangen!
 

\Logespeaks
Halt' ihn fest, bis ich ihn band.
 


\direct{Er hat ein Bastseil hervorgeholt und bindet Alberich damit Hände und Beine; den Geknebelten, der sich wütend zu wehren sucht, fassen dann beide und schleppen ihn mit sich nach der Kluft, aus der sie herauskamen.}


\Logespeaks
Nun schnell hinauf: dort ist er unser!
 


\direct{Sie verschwinden, aufwärts steigend}

 
\scene

\StageDir{Die Szene verwandelt sich, nur in umgekehrter Weise, wie zuvor; die Verwandlung führt wieder an den Schmieden vorüber. Fortdauernde Verwandlung nach oben. Schließlich erscheint wieder die
freie Gegend auf Bergeshöhen
wie in der zweiten Szene; nur ist sie jetzt noch in fahle Nebel verhüllt, wie vor der zweiten Verwandlung nach Freias Abführung.

Wotan und Loge, den gebundenen Alberich mit sich führend, steigen aus der Kluft herauf.}

\Logespeaks
Da, Vetter, sitze du fest!
Luge Liebster, dort liegt die Welt,
die du Lungrer gewinnen dir willst:
welch Stellchen, sag',
bestimmst du drin mir zu Stall?
 


\direct{er schlägt ihm tanzend Schnippchen}


\Alberichspeaks
Schändlicher Schächer! Du Schalk! Du Schelm!
Löse den Bast, binde mich los,
den Frevel sonst büßest du Frecher!
 

\Wotanspeaks
Gefangen bist du, fest mir gefesselt,
wie du die Welt, was lebt und webt,
in deiner Gewalt schon wähntest,
in Banden liegst du vor mir,
du Banger kannst es nicht leugnen!
Zu ledigen dich, bedarf 's nun der Lösung.
 

\Alberichspeaks
O ich Tropf, ich träumender Tor!
Wie dumm traut' ich dem diebischen Trug!
Furchtbare Rache räche den Fehl!
 

\Logespeaks
Soll Rache dir frommen,
vor allem rate dich frei:
dem gebundnen Manne
büßt kein Freier den Frevel.
Drum, sinnst du auf Rache,
rasch ohne Säumen
sorg' um die Lösung zunächst!
 


\direct{er zeigt ihm, mit den Fingern schnalzend, die Art der Lösung an}


\Alberichspeaks

\direct{barsch}

So heischt, was ihr begehrt!
 

\Wotanspeaks
Den Hort und dein helles Gold.
 

\Alberichspeaks
Gieriges Gaunergezücht!
 


\direct{für sich}

Doch behalt' ich mir nur den Ring,
des Hortes entrat' ich dann leicht;
denn von neuem gewonnen
und wonnig genährt
ist er bald durch des Ringes Gebot:
eine Witzigung wär 's,
die weise mich macht;
zu teuer nicht zahl' ich,
lass' für die Lehre ich den Tand.
 

\Wotanspeaks
Erlegst du den Hort?
 

\Alberichspeaks
Löst mir die Hand, so ruf' ich ihn her.
 


\direct{Loge löst ihm die Schlinge an der rechten Hand. Alberich berührt den Ring mit den Lippen und murmelt heimlich einen Befehl.}

Wohlan, die Nibelungen rief ich mir nah'.
Ihrem Herrn gehorchend, hör' ich den Hort
aus der Tiefe sie führen zu Tag:
nun löst mich vom lästigen Band!
 

\Wotanspeaks
Nicht eh'r, bis alles gezahlt.
 


\direct{Die Nibelungen steigen aus der Kluft herauf, mit den Geschmeiden des Hortes beladen. Während des Folgenden schichten sie den Hort auf.}


\Alberichspeaks
O schändliche Schmach!
Daß die scheuen Knechte
geknebelt selbst mich ersch'aun!
 


\direct{zu den Nibelungen}

Dorthin geführt, wie ich's befehlt'!
All zu Hauf schichtet den Hort!
Helf' ich euch Lahmen?
Hieher nicht gelugt!
Rasch da, rasch!
Dann rührt euch von hinnen,
daß ihr mir schafft!
Fort in die Schachten!
Weh' euch, find' ich euch faul!
Auf den Fersen folg' ich euch nach!
 


\direct{er küßt seinen Ring und streckt ihn gebieterisch aus. Wie von einem Schlage getroffen, drängen sich die Nibelungen scheu und ängstlich der Kluft zu, in die sie schnell hinabschlüpfen.}

Gezahlt hab' ich;
nun laßt mich zieh'n:
und das Helmgeschmeid',
das Loge dort hält,
das gebt mir nun gütlich zurück!
 

\Logespeaks

\direct{den Tarnhelm zum Horte werfend}

Zur Buße gehört auch die Beute.
 

\Alberichspeaks
Verfluchter Dieb!
 


\direct{leise}

Doch nur Geduld!
Der den alten mir schuf, schafft einen andern:
noch halt' ich die Macht, der Mime gehorcht.
Schlimm zwar ist's, dem schlauen Feind
zu lassen die listige Wehr!
Nun denn! Alberich ließ euch alles:
jetzt löst, ihr Bösen, das Band.
 

\Logespeaks

\direct{zu Wotan}

Bist du befriedigt? Lass' ich ihn frei?
 

\Wotanspeaks
Ein goldner Ring ragt dir am Finger;
hörst du, Alp?
Der, acht' ich, gehört mit zum Hort.
 

\Alberichspeaks

\direct{entsetzt}

Der Ring?
 

\Wotanspeaks
Zu deiner Lösung mußt du ihn lassen.
 

\Alberichspeaks

\direct{bebend}

Das Leben, doch nicht den Ring!
 

\Wotanspeaks

\direct{heftiger}

Den Reif' verlang' ich,
mit dem Leben mach', was du willst!
 

\Alberichspeaks
Lös' ich mir Leib und Leben,
den Ring auch muß ich mir lösen;
Hand und Haupt, Aug' und Ohr
sind nicht mehr mein Eigen,
als hier dieser rote Ring!
 

\Wotanspeaks
Dein Eigen nennst du den Ring?
Rasest du, schamloser Albe?
Nüchtern sag',
wem entnahmst du das Gold,
daraus du den schimmernden schufst?
War's dein Eigen, was du Arger
der Wassertiefe entwandt?
Bei des Rheines Töchtern hole dir Rat,
ob ihr Gold sie zu eigen dir gaben,
das du zum Ring dir geraubt!
 

\Alberichspeaks
Schmähliche Tücke! Schändlicher Trug!
Wirfst du Schächer die Schuld mir vor,
die dir so wonnig erwünscht?
Wie gern raubtest
du selbst dem Rheine das Gold,
war nur so leicht
die Kunst, es zu schmieden, erlangt?
Wie glückt es nun dir Gleißner zum Heil,
daß der Niblung, ich, aus schmählicher Not,
in des Zornes Zwange,
den schrecklichen Zauber gewann,
dess' Werk nun lustig dir lacht?
Des Unseligen, Angstversehrten
fluchfertige, furchtbare Tat,
zu fürstlichem Tand soll sie fröhlich dir taugen,
zur Freude dir frommen mein Fluch?
Hüte dich, herrischer Gott!
Frevelte ich, so frevelt' ich frei an mir:
doch an allem, was war,
ist und wird,
frevelst, Ewiger, du,
entreißest du frech mir den Ring!
 

\Wotanspeaks
Her der Ring!
Kein Recht an ihm
schwörst du schwatzend dir zu.
 


\direct{er ergreift Alberich und entzieht seinem Finger mit heftiger Gewalt den Ring.}


\Alberichspeaks
\direct{gräßlich aufschreiend}
Ha! Zertrümmert! Zerknickt!
Der Traurigen traurigster Knecht!
 

\Wotanspeaks

\direct{den Ring betrachtend}

Nun halt' ich, was mich erhebt,
der Mächtigen mächtigsten Herrn!
 


\direct{er steckt den Ring an}


\Logespeaks
Ist er gelöst?
 

\Wotanspeaks
Bind' ihn los!
 

\Logespeaks

\direct{löst Alberich vollends die Bande}

Schlüpfe denn heim!
Keine Schlinge hält dich:
frei fahre dahin!
 

\Alberichspeaks

\direct{sich vom Boden erhebend}

Bin ich nun frei?
 


\direct{mit wütendem Lachen}

Wirklich frei?
So grüß' euch denn
meiner Freiheit erster Gruß! -
Wie durch Fluch er mir geriet,
verflucht sei dieser Ring!
Gab sein Gold mir Macht ohne Maß,
nun zeug' sein Zauber Tod dem, der ihn trägt!
Kein Froher soll seiner sich freun,
keinem Glücklichen lache sein lichter Glanz!
Wer ihn besitzt, den sehre die Sorge,
und wer ihn nicht hat, den nage der Neid!
Jeder giere nach seinem Gut,
doch keiner genieße mit Nutzen sein!
Ohne Wucher hüt' ihn sein Herr;
doch den Würger zieh' er ihm zu!
Dem Tode verfallen, feßle den Feigen die Furcht:
solang er lebt, sterb' er lechzend dahin,
des Ringes Herr als des Ringes Knecht:
bis in meiner Hand den geraubten wieder ich halte! -
So segnet in höchster Not
der Nibelung seinen Ring!
Behalt' ihn nun,
 


\direct{lachend}

hüte ihn wohl:
 


\direct{grimmig}

meinem Fluch fliehest du nicht!
 


\direct{er verschwindet schnell in der Kluft. Der dichte Nebelduft des Vordergrundes klärt sich allmählich auf}


\Logespeaks
Lauschtest du seinem Liebesgruß?
 

\Wotanspeaks

\direct{in den Anblick des Ringes an seiner Hand versunken}

Gönn' ihm die geifernde Lust!
 


\direct{es wird immer heller}


\Logespeaks

\direct{nach rechts in die Szene blickend}

Fasolt und Fafner nahen von fern:
Freia führen sie her.
 


\direct{Aus dem sich immer mehr zerteilenden Nebel erscheinen Donner, Froh und Fricka und eilen dem Vordergrunde zu.}


\Frohspeaks
Sie kehren zurück!
 

\Donnerspeaks
Willkommen, Bruder!
 

\Frickaspeaks

\direct{besorgt zu Wotan}

Bringst du gute Kunde?
 

\Logespeaks

\direct{auf den Hort deutend}

Mit List und Gewalt gelang das Werk:
dort liegt, was Freia löst.
 

\Donnerspeaks
Aus der Riesen Haft naht dort die Holde.
 

\Frohspeaks
Wie liebliche Luft wieder uns weht,
wonnig' Gefühl die Sinne erfüllt!
Traurig ging es uns allen,
getrennt für immer von ihr,
die leidlos ewiger Jugend
jubelnde Lust uns verleiht.
 


\direct{Der Vordergrund ist wieder hell geworden; das Aussehen der Götter gewinnt wieder die erste Frische: über dem Hintergrunde haftet jedoch noch der Nebelschleier, so daß die Burg unsichtbar bleibt. Fasolt und Fafner treten auf, Freia zwischen sich führend.}


\Frickaspeaks
\direct{eilt freudig auf die Schwester zu, um sie zu umarmen}
Lieblichste Schwester, süßeste Lust!
Bist du mir wieder gewonnen?
 

\Fasoltspeaks

\direct{ihr wehrend}

Halt! Nicht sie berührt!
Noch gehört sie uns.
Auf Riesenheims ragender Mark
rasteten wir; mit treuem Mut
des Vertrages Pfand pflegten wir.
So sehr mich's reut, zurück doch bring' ich's,
erlegt uns Brüdern die Lösung ihr.
 

\Wotanspeaks
Bereit liegt die Lösung:
des Goldes Maß sei nun gütlich gemessen.
 

\Fasoltspeaks
Das Weib zu missen, wisse, gemutet mich weh:
soll aus dem Sinn sie mir schwinden
des Geschmeides Hort häufet denn so,
daß meinem Blick die Blühende ganz er verdeck'!
 

\Wotanspeaks
So stellt das Maß nach Freias Gestalt!
 


\direct{Freia wird von den beiden Riesen in die Mitte gestellt. Darauf stoßen sie ihre Pfähle zu Freias beiden Seiten so in den Boden, daß sie gleiche Höhe und Breite mit ihrer Gestalt messen.}


\Fafnerspeaks
Gepflanzt sind die Pfähle nach Pfandes Maß;
Gehäuft nun füll' es der Hort!
 

\Wotanspeaks
Eilt mit dem Werk: widerlich ist mir's!
 

\Logespeaks
Hilf mir, Froh!
 

\Frohspeaks
Freias Schmach eil' ich zu enden.
 


\direct{Loge und Froh häufen hastig zwischen den Pfählen die Geschmeide}


\Fafnerspeaks
Nicht so leicht und locker gefügt!
 


\direct{er drückt mit roher Kraft die Geschmeide dicht zusammen}

Fest und dicht füll' er das Maß.
 


\direct{er beugt sich, um nach Lücken zu spähen}

Hier lug' ich noch durch:
verstopft mir die Lücken!
 

\Logespeaks
Zurück, du Grober!
 

\Fafnerspeaks
Hierher!
 

\Logespeaks
Greif' mir nichts an!
 

\Fafnerspeaks
Hierher! Die Klinze verklemmt!
 

\Wotanspeaks

\direct{unmutig sich abwendend}

Tief in der Brust brennt mir die Schmach!
 

\Frickaspeaks

\direct{den Blick auf Freia geheftet}

Sieh, wie in Scham schmählich die Edle steht:
um Erlösung fleht stumm der leidende Blick.
Böser Mann! Der Minnigen botest du das!
 

\Fafnerspeaks
Noch mehr! Noch mehr hierher!
 

\Donnerspeaks
Kaum halt' ich mich: schäumende Wut
weckt mir der schamlose Wicht!
Hierher, du Hund! Willst du messen,
so miß dich selber mit mir!
 

\Fafnerspeaks
Ruhig, Donner! Rolle, wo's taugt:
hier nützt dein Rasseln dir nichts!
 

\Donnerspeaks

\direct{holt aus}

Nicht dich Schmähl'chen zu zerschmettern?
 

\Wotanspeaks
Friede doch!
Schon dünkt mich Freia verdeckt.
 

\Logespeaks
Der Hort ging auf.
 

\Fafnerspeaks

\direct{mißt den Hort genau mit dem Blick und späht nach Lücken}

Noch schimmert mir Holdas Haar:
dort das Gewirk wirf auf den Hort!
 

\Logespeaks
Wie? Auch den Helm?
 

\Fafnerspeaks
Hurtig, her mit ihm!
 

\Wotanspeaks
Laß ihn denn fahren!
 

\Logespeaks

\direct{wirft den Tarnhelm auf den Hort}

So sind wir denn fertig!
Seid ihr zufrieden?
 

\Fasoltspeaks
Freia, die Schöne, schau' ich nicht mehr:
so ist sie gelöst? Muß ich sie lassen?
 


\direct{er tritt nahe hinzu und späht durch den Hort}

Weh! Noch blitzt ihr Blick zu mir her;
des Auges Stern strahlt mich noch an:
durch eine Spalte muß ich's erspäh'n.
 


\direct{außer sich}

Seh' ich dies wonnige Auge,
von dem Weibe lass' ich nicht ab!
 

\Fafnerspeaks
He! Euch rat' ich,
verstopft mir die Ritze!
 

\Logespeaks
Nimmersatte! Seht ihr denn nicht,
ganz schwand uns der Hort?
 

\Fafnerspeaks
Mitnichten, Freund! An Wotans Finger
glänzt von Gold noch ein Ring:
den gebt, die Ritze zu füllen!
 

\Wotanspeaks
Wie! Diesen Ring?
 

\Logespeaks
Laßt euch raten!
Den Rheintöchtern gehört dies Gold;
ihnen gibt Wotan es wieder.
 

\Wotanspeaks
Was schwatztest du da?
Was schwer ich mir erbeutet,
ohne Bangen wahr' ich's für mich!
 

\Logespeaks
Schlimm dann steht's um mein Versprechen,
das ich den Klagenden gab!
 

\Wotanspeaks
Dein Versprechen bindet mich nicht;
als Beute bleibt mir der Reif.
 

\Fafnerspeaks
Doch hier zur Lösung mußt du ihn legen.
 

\Wotanspeaks
Fordert frech, was ihr wollt,
alles gewähr' ich;
um alle Welt,
doch nicht fahren lass' ich den Ring!
 

\Fasoltspeaks

\direct{zieht wütend Freia hinter dem Horte hervor}

Aus denn ist's, beim Alten bleibt's;
nun folgt uns Freia für immer!
 

\Freiaspeaks
Hilfe! Hilfe!
 

\Frickaspeaks
Harter Gott, gib ihnen nach!
 

\Frohspeaks
Spare das Gold nicht!
 

\Donnerspeaks
Spende den Ring doch!
 


\direct{Fafner hält den fortdrängenden Fasolt noch auf; alle stehen bestürzt}


\Wotanspeaks
Laßt mich in Ruh'! Den Reif geb' ich nicht!
 


\direct{Wotan wendet sich zürnend zur Seite. Die Bühne hat sich von neuem verfinstert; aus der Felskluft zur Seite bricht ein bläulicher Schein hervor: in ihm wird plötzlich Erda sichtbar, die bis zu halber Leibeshöhe aus der Tiefe aufsteigt; sie ist von edler Gestalt, weithin von schwarzem Haar umwallt.}


\Erdaspeaks

\direct{die Hand mahnend gegen Wotan ausstreckend}

Weiche, Wotan! Weiche!
Flieh' des Ringes Fluch!
Rettungslos dunklem Verderben
weiht dich sein Gewinn.
 

\Wotanspeaks
Wer bist du, mahnendes Weib?
 

\Erdaspeaks
Wie alles war - weiß ich;
wie alles wird, wie alles sein wird,
seh' ich auch, -
der ew'gen Welt Ur-Wala,
Erda, mahnt deinen Mut. Drei der Töchter,
ur-erschaff'ne, gebar mein Schoß;
was ich sehe, sagen dir nächtlich die Nornen.
Doch höchste Gefahr führt mich heut'
selbst zu dir her.
Höre! Höre! Höre!
Alles was ist, endet.
Ein düst'rer Tag dämmert den Göttern:
dir rat' ich, meide den Ring!
 


\direct{sie versinkt langsam bis an die Brust, während der bläuliche Schein zu dunkeln beginnt}


\Wotanspeaks
Geheimnis-hehr
hallt mir dein Wort:
weile, daß mehr ich wisse!
 

\Erdaspeaks
\direct{im Versinken}
Ich warnte dich; du weißt genug:
sinn' in Sorg' und Furcht!
 


\direct{sie verschwindet gänzlich}


\Wotanspeaks
Soll ich sorgen und fürchten,
dich muß ich fassen, alles erfahren!
 


\direct{er will der Verschwindenden in die Kluft nach, um sie zu halten. Froh und Fricka werfen sich ihm entgegen und halten ihn zurück}


\Frickaspeaks
Was willst du, Wütender?
 

\Frohspeaks
Halt' ein, Wotan!
Scheue die Edle, achte ihr Wort!
 


\direct{Wotan starrt sinnend vor sich hin}


\Donnerspeaks

\direct{sich entschlossen zu den Riesen wendend}

Hört, ihr Riesen! Zurück, und harret:
das Gold wird euch gegeben.
 

\Freiaspeaks
Darf ich es hoffen?
Dünkt euch Holda wirklich der Lösung wert?
 

\direct{Alle blicken gespannt auf Wotan; dieser nach tiefem Sinnen zu sich kommend, erfaßt seinen Speer und schwenkt ihn wie zum Zeichen eines mutigen Entschlusses}

\Wotanspeaks
Zu mir, Freia! Du bist befreit.
Wieder gekauft kehr' uns die Jugend zurück!
Ihr Riesen, nehmt euren Ring!
 


\direct{er wirft den Ring auf den Hort}


\direct{Die Riesen lassen Freia los; sie eilt freudig auf die Götter zu, die sie abwechselnd längere Zeit in höchster Freude liebkosen. Fafner breitet sogleich einen ungeheuren Sack aus und macht sich über den Hort her, um ihn da hineinzuschichten}


\Fasoltspeaks

\direct{dem Bruder sich entgegenwerfend}

Halt, du Gieriger! Gönne mir auch was!
Redliche Teilung taugt uns beiden.
 

\Fafnerspeaks
Mehr an der Maid als am Gold
lag dir verliebtem Geck:
mit Müh' zum Tausch vermocht' ich dich Toren;
Ohne zu teilen, hättest du Freia gefreit:
teil' ich den Hort,
billig behalt' ich die größte Hälfte für mich.
 

\Fasoltspeaks
Schändlicher du! Mir diesen Schimpf?
 


\direct{zu den Göttern}

Euch ruf' ich zu Richtern:
teilet nach Recht uns redlich den Hort!
 


\direct{Wotan wendet sich verächtlich ab}


\Logespeaks
Den Hort laß ihn raffen;
halte du nur auf den Ring!
 

\Fasoltspeaks

\direct{stürzt sich auf Fafner, der immerzu eingesackt hat}

Zurück, du Frecher! Mein ist der Ring;
mir blieb er für Freias Blick!
 


\direct{Er greift hastig nach dem Reif. Sie ringen.}


\Fafnerspeaks
Fort mit der Faust! Der Ring ist mein!
 


\direct{Fasolt entreißt Fafner den Ring}


\Fasoltspeaks
Ich halt' ihn, mir gehört er!
 

\Fafnerspeaks

\direct{mit einem Pfahle nach Fasolt ausholend}

Halt' ihn fest, daß er nicht fall'!
 


\direct{Er streckt Fasolt mit einem Streiche zu Boden, dem Sterbenden entreißt er dann hastig den Ring}


\Fafnerspeaks
Nun blinzle nach Freias Blick!
An den Reif rührst du nicht mehr!
 


\direct{Er steckt den Ring in den Sack und rafft dann gemächlich den Hort vollends ein. Alle Götter stehen entsetzt. Langes, feierliches Schweigen}


\Wotanspeaks
Furchtbar nun erfind' ich des Fluches Kraft!
 

\Logespeaks
Was gleicht, Wotan, wohl deinem Glücke?
Viel erwarb dir des Ringes Gewinn;
daß er nun dir genommen, nützt dir noch mehr:
deine Feinde - sieh - fällen sich selbst
um das Gold, das du vergabst.
 

\Wotanspeaks

\direct{tief erschüttert}

Wie doch Bangen mich bindet!
Sorg' und Furcht fesseln den Sinn:
wie sie zu enden, lehre mich Erda:
zu ihr muß ich hinab!
 

\Frickaspeaks

\direct{schmeichelnd sich an ihn schmiegend}

Wo weilst du, Wotan?
Winkt dir nicht hold die hehre Burg,
die des Gebieters gastlich bergend nun harrt?
 

\Wotanspeaks

\direct{düster}

Mit bösem Zoll zahlt' ich den Bau.
 

\Donnerspeaks

\direct{auf den Hintergrund deutend, der noch in Nebel gehüllt ist}

Schwüles Gedünst schwebt in der Luft;
lästig ist mir der trübe Druck!
Das bleiche Gewölk
samml' ich zu blitzendem Wetter,
das fegt den Himmel mir hell.
 


\direct{er besteigt einen hohen Felsstein am Talabhange und schwingt dort seinen Hammer; Nebel ziehen sich um ihn zusammen}

He da! He da! He do!
Zu mir, du Gedüft! Ihr Dünste, zu mir!
Donner, der Herr, ruft euch zu Heer!
 


\direct{er schwingt den Hammer}

Auf des Hammers Schwung schwebet herbei!
Dunstig Gedämpf! Schwebend Gedüft!
Donner, der Herr, ruft euch zu Heer!
He da! He da! He do!
 


\direct{er verschwindet völlig in einer immer finsterer sich ballenden Gewitterwolke. Man hört Donners Hammerschlag schwer auf den Felsstein fallen: ein starker Blitz entfährt der Wolke; ein heftiger Donnerschlag folgt. Froh ist mit dem Gewölk verschwunden.}


\Donnerspeaks

\direct{unsichtbar}

Bruder, hieher! Weise der Brücke den Weg!
 


\direct{Plötzlich verzieht sich die Wolke; Donner und Froh werden sichtbar: von ihren Füßen aus zieht sich, mit blendendem Leuchten, eine Regenbogenbrücke über das Tal hinüber bis zur Burg, die jetzt, von der Abendsonne beschienen, im hellsten Glanze erstrahlt. Fafner, der neben der Leiche seines Bruders endlich den ganzen Hort eingerafft, hat, den ungeheuren Sack auf dem Rücken, während Donners Gewitterzauber die Bühne verlassen.}


\Frohspeaks

\direct{der der Brücke mit der ausgestreckten Hand den Weg über das Tal angewiesen, zu den Göttern}

Zur Burg führt die Brücke,
leicht, doch fest eurem Fuß:
beschreitet kühn ihren schrecklosen Pfad!
 


\direct{Wotan und die anderen Götter sind sprachlos in den prächtigen Anblick verloren}


\Wotanspeaks
Abendlich strahlt der Sonne Auge;
in prächtiger Glut prangt glänzend die Burg.
In des Morgens Scheine mutig erschimmernd,
lag sie herrenlos, hehr verlockend vor mir.
Von Morgen bis Abend, in Müh' und Angst,
nicht wonnig ward sie gewonnen!
Es naht die Nacht: vor ihrem Neid
biete sie Bergung nun.
 


\direct{Wie von einem großen Gedanken ergriffen, sehr entschlossen}

So grüß' ich die Burg,
sicher vor Bang' und Grau'n!
 


\direct{er wendet sich feierlich zu Fricka}

Folge mir, Frau:
in Walhall wohne mit mir!
 

\Frickaspeaks
Was deutet der Name?
Nie, dünkt mich, hört' ich ihn nennen.
 

\Wotanspeaks
Was, mächtig der Furcht,
mein Mut mir erfand,
wenn siegend es lebt,
leg' es den Sinn dir dar!
 


\direct{er faßt Fricka an der Hand und schreitet mit ihr langsam der Brücke zu; Froh, Freia und Donner folgen}


\Logespeaks

\direct{im Vordergrunde verharrend und den Göttern nachblickend}

Ihrem Ende eilen sie zu,
die so stark in Bestehen sich wähnen.
Fast schäm' ich mich, mit ihnen zu schaffen;
zur leckenden Lohe mich wieder zu wandeln,
spür' ich lockende Lust:
sie aufzuzehren, die einst mich gezähmt,
statt mit den Blinden blöd zu vergehn,
und wären es göttlichste Götter!
Nicht dumm dünkte mich das!
Bedenken will ich's: wer weiß, was ich tu'!
 


\direct{er geht, um sich den Göttern in nachlässiger Haltung anzuschließen. Aus der Tiefe hört man den Gesang der Rheintöchter heraufschallen}


Die drei Rheintöchter
\direct{in der Tiefe des Tales, unsichtbar}
Rheingold! Rheingold! Reines Gold!
Wie lauter und hell leuchtest hold du uns!
Um dich, du klares, wir nun klagen:
gebt uns das Gold!
O gebt uns das reine zurück!
 

\Wotanspeaks

\direct{im Begriff, den Fuß auf die Brücke zu setzen, hält an und wendet sich um}

Welch' Klagen klingt zu mir her?
 

\Logespeaks

\direct{späht in das Tal hinab}

Des Rheines Kinder beklagen des Goldes Raub!
 

\Wotanspeaks
Verwünschte Nicker!
 


\direct{zu Loge}

Wehre ihrem Geneck!
 

\Logespeaks

\direct{in das Tal hinabrufend}

Ihr da im Wasser, was weint ihr herauf?
Hört, was Wotan euch wünscht!
Glänzt nicht mehr euch Mädchen das Gold,
in der Götter neuem Glanze
sonnt euch selig fortan!
 


\direct{Die Götter lachen und beschreiten dann die Brücke}


\speaker{Die drei Rheintöchter}

\direct{aus der Tiefe}

Rheingold! Rheingold! Reines Gold!
O leuchtete noch in der Tiefe dein laut'rer Tand!
Traulich und treu ist's nur in der Tiefe:
falsch und feig ist, was dort oben sich freut!
 


\direct{während die Götter auf der Brücke der Burg zuschreiten, fällt der Vorhang}


\end{drama}

\cleardoublepage

\part{Die Walk\"ure}
\let\saythescene\oldsaythescene
\renewcommand{\playname}{Die Walküre}
\setcounter{act}{0}
\begin{drama}
\act

\scene

\StageDir{In der Mitte steht der Stamm einer mächtigen Esche, dessen stark erhabene Wurzeln sich weithin in den Erdboden verlieren; von seinem Wipfel ist der Baum durch ein gezimmertes Dach geschieden, welches so durchschnitten ist, daß der Stamm und die nach allen Seiten hin sich ausstreckenden Äste durch genau entsprechende Öffnungen hindurchgehen; von dem belaubten Wipfel wird angenommen, daß er sich über dieses Dach ausbreite. Um den Eschenstamm, als Mittelpunkt, ist nun ein Saal gezimmert; die Wände sind aus roh behauenem Holzwerk, hier und da mit geflochtenen und gewebten Decken behangen. Rechts im Vordergrunde steht der Herd, dessen Rauchfang seitwärts zum Dache hinausführt: hinter dem Herde befindet sich ein innerer Raum, gleich einem Vorratsspeicher, zu dem man auf einigen hölzernen Stufen hinaufsteigt: davor hängt, halb zurückgeschlagen, eine geflochtene Decke. Im Hintergrunde eine Eingangstür mit schlichtem Holzriegel. Links die Tür zu einem inneren Gemache, zu dem gleichfalls Stufen hinaufführen; weiter vornen auf derselben Seite ein Tisch mit einer breiten, an der Wand angezimmerten Bank dahinter und hölzernen Schemeln davor.

\direct{Ein kurzes Orchestervorspiel von heftiger, stürmischer Bewegung leitet ein. Als der Vorhang aufgeht, öffnet Siegmund von außen hastig die Eingangstür und tritt ein: es ist gegen Abend, starkes Gewitter, im Begriff, sich zu legen. Siegmund hält einen Augenblick den Riegel in der Hand und überblickt den Wohnraum: er scheint von übermäßiger Anstrengung erschöpft; sein Gewand und Aussehen zeigen, daß er sich auf der Flucht befinde. Da er niemand gewahrt, schließt er die Tür hinter sich, schreitet auf den Herd zu und wirft sich dort ermattet auf eine Decke von Bärenfell.}}

\Siegmundspeaks
Wes Herd dies auch sei, hier muß ich rasten.
 


\direct{Er sinkt zurück und bleibt einige Zeit regungslos ausgestreckt. Sieglinde tritt aus der Tür des inneren Gemaches; sie glaubte ihren Mann heimgekehrt: ihre ernste Miene zeigt sich dann verwundert, als sie einen Fremden am Herde ausgestreckt sieht}


\Sieglindespeaks

\direct{noch im Hintergrunde}

Ein fremder Mann? Ihn muß ich fragen.

\direct{Sie tritt ruhig einige Schritte näher}

Wer kam ins Haus und liegt dort am Herd?

\direct{Da Siegmund sich nicht regt, tritt sie noch etwas näher und betrachtet ihn}

Müde liegt er, von Weges Müh'n.
Schwanden die Sinne ihm? Wäre er siech?

\direct{Sie neigt sich zu ihm herab und lauscht}

Noch schwillt ihm der Atem; das Auge nur schloß er. -
Mutig dünkt mich der Mann, sank er müd' auch hin.
 

\Siegmundspeaks

\direct{fährt jäh mit dem Haupt in die Höhe}

Ein Quell! Ein Quell!
 

\Sieglindespeaks
Erquickung schaff' ich.

\direct{Sie nimmt schnell ein Trinkhorn und geht damit aus dem Hause. Sie kommt zurück und reicht das gefüllte Trinkhorn Siegmund}

Labung biet' ich dem lechzenden Gaumen:
Wasser, wie du gewollt.
 


\direct{Siegmund trinkt und reicht ihr das Horn zurück. Als er ihr mit dem Haupte Dank zuwinkt, haftet sein Blick mit steigender Teilnahme an ihren Mienen.}


\Siegmundspeaks
Kühlende Labung gab mir der Quell,
des Müden Last machte er leicht:
erfrischt ist der Mut,
das Aug' erfreut des Sehens selige Lust.
Wer ist's, der so mir es labt?
 

\Sieglindespeaks
Dies Haus und dies Weib sind Hundings Eigen;
gastlich gönn' er dir Rast: harre, bis heim er kehrt!
 

\Siegmundspeaks
Waffenlos bin ich:
dem wunden Gast wird dein Gatte nicht wehren.
 

\Sieglindespeaks

\direct{mit besorgter Hast}

Die Wunden weise mir schnell!
 

\Siegmundspeaks

\direct{schüttelt sich und springt lebhaft vom Lager zum Sitz auf}

Gering sind sie, der Rede nicht wert;
noch fügen des Leibes Glieder sich fest.
Hätten halb so stark wie mein Arm
Schild und Speer mir gehalten,
nimmer floh ich dem Feind,
doch zerschellten mir Speer und Schild.
Der Feinde Meute hetzte mich müd',
Gewitterbrunst brach meinen Leib;
doch schneller, als ich der Meute,
schwand die Müdigkeit mir:
sank auf die Lider mir Nacht;
die Sonne lacht mir nun neu.
 

\Sieglindespeaks

\direct{geht nach dem Speicher, füllt ein Horn mit Met und reicht es Siegmund mit freundlicher Bewegtheit}

Des seimigen Metes süßen Trank
mög'st du mir nicht verschmähn.
 

\Siegmundspeaks
Schmecktest du mir ihn zu?
 


\direct{Sieglinde nippt am Horne und reicht es ihm wieder. Siegmund tut einen langen Zug, indem er den Blick mit wachsender Wärme auf sie heftet. Er setzt so das Horn ab und läßt es langsam sinken, während der Ausdruck seiner Miene in starke Ergriffenheit übergeht. Er seufzt tief auf und senkt den Blick düster zu Boden.}


\Siegmundspeaks

\direct{mit bebender Stimme}

Einen Unseligen labtest du:
Unheil wende der Wunsch von dir!
 


\direct{Er bricht schnell auf, um fortzugehen}

Gerastet hab' ich und süß geruht.
Weiter wend' ich den Schritt.
 


\direct{er geht nach hinten}


\Sieglindespeaks

\direct{lebhaft sich umwendend}

Wer verfolgt dich, daß du schon fliehst?
 

\Siegmundspeaks

\direct{von ihrem Rufe gefesselt, wendet sich wieder; langsam und düster}

Mißwende folgt mir, wohin ich fliehe;
Mißwende naht mir, wo ich mich neige. -
Dir, Frau, doch bleibe sie fern!
Fort wend' ich Fuß und Blick.
 


\direct{Er schreitet schnell bis zur Tür und hebt den Riegel}


\Sieglindespeaks

\direct{in heftigem Selbstvergessen ihm nachrufend}

So bleibe hier!
Nicht bringst du Unheil dahin,
wo Unheil im Hause wohnt!
 


\direct{Siegmund bleibt tief erschüttert stehen; er forscht in Sieglindes Mienen; diese schlägt verschämt und traurig die Augen nieder. Langes Schweigen}


\Siegmundspeaks

\direct{kehrt zurück}

Wehwalt hieß ich mich selbst:
Hunding will ich erwarten.
 


\direct{Er lehnt sich an den Herd; sein Blick haftet mit ruhiger und entschlossener Teilnahme an Sieglinde; diese hebt langsam das Auge wieder zu ihm auf. Beide blicken sich in langem Schweigen mit dem Ausdruck tiefster Ergriffenheit in die Augen}

\scene

\StageDir{Sieglinde fährt plötzlich auf, lauscht und hört Hunding, der sein Roß außen zum Stall führt. Sie geht hastig zur Tür und öffnet; Hunding, gewaffnet mit Schild und Speer, tritt ein und hält unter der Tür, als er Siegmund gewahrt. Hunding wendet sich mit einem ernst fragenden Blick an Sieglinde}


\Sieglindespeaks

\direct{dem Blicke Hundings entgegnend}

Müd am Herd fand ich den Mann:
Not führt' ihn ins Haus.
 

\Hundingspeaks
Du labtest ihn?
 

\Sieglindespeaks
Den Gaumen letzt' ich ihm, gastlich sorgt' ich sein!
 

\Siegmundspeaks

\direct{der ruhig und fest Hunding beobachtet}

Dach und Trank dank' ich ihr:
willst du dein Weib drum schelten?
 

\Hundingspeaks
Heilig ist mein Herd: -
heilig sei dir mein Haus!
 


\direct{er legt seine Waffen ab und übergibt sie Sieglinde. Zu Sieglinde}

Rüst' uns Männern das Mahl!
 


\direct{Sieglinde hängt die Waffen an Ästen des Eschenstammes auf, dann holt sie Speise und Trank aus dem Speicher und rüstet auf dem Tische das Nachtmahl. Unwillkürlich heftet sie wieder den Blick auf Siegmund}


\direct{Hunding mißt scharf und verwundert Siegmunds Züge, die er mit denen seiner Frau vergleicht; für sich}

Wie gleicht er dem Weibe!
Der gleißende Wurm glänzt auch ihm aus dem Auge.
 


\direct{er birgt sein Befremden und wendet sich wie unbefangen zu Siegmund}

Weit her, traun, kamst du des Wegs;
ein Roß nicht ritt, der Rast hier fand:
welch schlimme Pfade schufen dir Pein?
 

\Siegmundspeaks
Durch Wald und Wiese, Heide und Hain,
jagte mich Sturm und starke Not:
nicht kenn' ich den Weg, den ich kam.
Wohin ich irrte, weiß ich noch minder:
Kunde gewänn' ich des gern.
 

\Hundingspeaks

\direct{am Tische und Siegmund den Sitz bietend}

Des Dach dich deckt, des Haus dich hegt,
Hunding heißt der Wirt;
wendest von hier du nach West den Schritt,
in Höfen reich hausen dort Sippen,
die Hundings Ehre behüten.
Gönnt mir Ehre mein Gast,
wird sein Name nun mir gennant.
 


\direct{Siegmund, der sich am Tisch niedergesetzt, blickt nachdenklich vor sich hin. Sieglinde, die sich neben Hunding, Siegmund gegenüber, gesetzt, heftet ihr Auge mit auffallender Teilnahme und Spannung auf diesen.}


\Hundingspeaks

\direct{der beide beobachtet}

Trägst du Sorge, mir zu vertraun,
der Frau hier gib doch Kunde:
sieh, wie gierig sie dich frägt!
 

\Sieglindespeaks

\direct{unbefangen und teilnahmsvoll}

Gast, wer du bist, wüßt' ich gern.
 

\Siegmundspeaks

\direct{blickt auf, sieht ihr in das Auge und beginnt ernst}

Friedmund darf ich nicht heißen;
Frohwalt möcht' ich wohl sein:
doch Wehwalt mußt ich mich nennen.
Wolfe, der war mein Vater;
zu zwei kam ich zur Welt,
eine Zwillingsschwester und ich.
Früh schwanden mir Mutter und Maid.
Die mich gebar und die mit mir sie barg,
kaum hab' ich je sie gekannt.
Wehrlich und stark war Wolfe;
der Feinde wuchsen ihm viel.
Zum Jagen zog mit dem Jungen der Alte:
Von Hetze und Harst einst kehrten wir heim:
da lag das Wolfsnest leer.
Zu Schutt gebrannt der prangende Saal,
zum Stumpf der Eiche blühender Stamm;
erschlagen der Mutter mutiger Leib,
verschwunden in Gluten der Schwester Spur:
uns schuf die herbe Not
der Neidinge harte Schar.
Geächtet floh der Alte mit mir;
lange Jahre lebte der Junge
mit Wolfe im wilden Wald:
manche Jagd ward auf sie gemacht;
doch mutig wehrte das Wolfspaar sich.
 


\direct{zu Hunding gewandt}

Ein Wölfing kündet dir das,
den als ``Wölfing'' mancher wohl kennt.
 

\Hundingspeaks
Wunder und wilde Märe kündest du, kühner Gast,
Wehwalt---der Wölfing!
Mich dünkt, von dem wehrlichen Paar
vernahm ich dunkle Sage,
kannt' ich auch Wolfe und Wölfing nicht.
 

\Sieglindespeaks
Doch weiter künde, Fremder:
wo weilt dein Vater jetzt?
 

\Siegmundspeaks
Ein starkes Jagen auf uns stellten die Neidinge an:
der Jäger viele fielen den Wölfen,
in Flucht durch den Wald
trieb sie das Wild.
Wie Spreu zerstob uns der Feind.
Doch ward ich vom Vater versprengt;
seine Spur verlor ich, je länger ich forschte:
eines Wolfes Fell nur
traf ich im Forst;
leer lag das vor mir, den Vater fand ich nicht.
Aus dem Wald trieb es mich fort;
mich drängt' es zu Männern und Frauen.
Wieviel ich traf, wo ich sie fand,
ob ich um Freund', um Frauen warb,
immer doch war ich geächtet:
Unheil lag auf mir.
Was Rechtes je ich riet, andern dünkte es arg,
was schlimm immer mir schien,
andre gaben ihm Gunst.
In Fehde fiel ich, wo ich mich fand,
Zorn traf mich, wohin ich zog;
gehrt' ich nach Wonne, weckt' ich nur Weh':
drum mußt' ich mich Wehwalt nennen;
des Wehes waltet' ich nur.
 


\direct{Er sieht zu Sieglinde auf und gewahrt ihren teilnehmenden Blick}


\Hundingspeaks
Die so leidig Los dir beschied,
nicht liebte dich die Norn':
froh nicht grüßt dich der Mann,
dem fremd als Gast du nahst.
 

\Sieglindespeaks
Feige nur fürchten den, der waffenlos einsam fährt! -
Künde noch, Gast,
wie du im Kampf zuletzt die Waffe verlorst!
 

\Siegmundspeaks

\direct{immer lebhafter}

Ein trauriges Kind rief mich zum Trutz:
vermählen wollte der Magen Sippe
dem Mann ohne Minne die Maid.
Wider den Zwang zog ich zum Schutz,
der Dränger Troß traf ich im Kampf:
dem Sieger sank der Feind.
Erschlagen lagen die Brüder:
die Leichen umschlang da die Maid,
den Grimm verjagt' ihr der Gram.
Mit wilder Tränen Flut betroff sie weinend die Wal:
um des Mordes der eignen Brüder
klagte die unsel'ge Braut.
Der Erschlagnen Sippen stürmten daher;
übermächtig ächzten nach Rache sie;
rings um die Stätte ragten mir Feinde.
Doch von der Wal wich nicht die Maid;
mit Schild und Speer schirmt' ich sie lang',
bis Speer und Schild im Harst mir zerhaun.
Wund und waffenlos stand ich -
sterben sah ich die Maid:
mich hetzte das wütende Heer -
auf den Leichen lag sie tot.
 


\direct{mit einem Blicke voll schmerzlichen Feuers auf Sieglinde}

Nun weißt du, fragende Frau,
warum ich Friedmund nicht heiße!
 


\direct{Er steht auf und schreitet auf den Herd zu. Sieglinde blickt erbleichend und tief erschüttert zu Boden}


\Hundingspeaks

\direct{erhebt sich, sehr finster}

Ich weiß ein wildes Geschlecht,
nicht heilig ist ihm, was andern hehr:
verhaßt ist es allen und mir.
Zur Rache ward ich gerufen,
Sühne zu nehmen für Sippenblut:
zu spät kam ich, und kehrte nun heim,
des flücht'gen Frevlers Spur im eignen Haus zu erspähn. -
 


\direct{Er geht herab}

Mein Haus hütet, Wölfing, dich heut';
für die Nacht nahm ich dich auf;
mit starker Waffe doch wehre dich morgen;
zum Kampfe kies' ich den Tag:
für Tote zahlst du mir Zoll.
 


\direct{Sieglinde schreitet mit besorgter Gebärde zwischen die beiden Männer vor}


\Hundingspeaks

\direct{barsch}

Fort aus dem Saal! Säume hier nicht!
Den Nachttrunk rüste mir drin und harre mein' zur Ruh'.
 


\direct{Sieglinde steht eine Weile unentschieden und sinnend. Sie wendet sich langsam und zögernden Schrittes nach dem Speicher. Dort hält sie wieder an und bleibt, in Sinnen verloren, mit halb abgewandtem Gesicht stehen. Mit ruhigem Entschluß öffnet sie den Schrein, füllt ein Trinkhorn und schüttet aus einer Büchse Würze hinein. Dann wendet sie das Auge auf Siegmund, um seinem Blicke zu begegnen, den dieser fortwährend auf sie heftet. Sie gewahrt Hundings Spähen und wendet sich sogleich zum Schlafgemach. Auf den Stufen kehrt sie sich noch einmal um, heftet das Auge sehnsuchtsvoll auf Siegmund und deutet mit dem Blicke andauernd und mit sprechender Bestimmtheit auf eine Stelle am Eschenstamme. Hunding fährt auf und treibt sie mit einer heftigen Gebärde zum Fortgehen an. Mit einem letzten Blick auf Siegmund geht sie in das Schlafgemach und schließt hinter sich die Türe.}

 

\Hundingspeaks

\direct{nimmt seine Waffen vom Stamme herab}

Mit Waffen wehrt sich der Mann.
 


\direct{Im Abgehen sich zu Siegmund wendend}

Dich Wölfing treffe ich morgen;
mein Wort hörtest du, hüte dich wohl!
 


\direct{Er geht mit den Waffen in das Gemach; man hört ihn von innen den Riegel schließen}

\scene

\StageDir{Siegmund allein. Es ist vollständig Nacht geworden; der Saal ist nur noch von einem schwachen Feuer im Herde erhellt. Siegmund läßt sich, nah beim Feuer, auf dem Lager nieder und brütet in großer innerer Aufregung eine Zeitlang schweigend vor sich hin}


\Siegmundspeaks
Ein Schwert verhieß mir der Vater,
ich fänd' es in höchster Not.
Waffenlos fiel ich in Feindes Haus;
seiner Rache Pfand, raste ich hier:
ein Weib sah ich, wonnig und hehr:
entzückend Bangen zehrt mein Herz.
Zu der mich nun Sehnsucht zieht,
die mit süßem Zauber mich sehrt,
im Zwange hält sie der Mann,
der mich Wehrlosen höhnt!
Wälse! Wälse! Wo ist dein Schwert?
Das starke Schwert,
das im Sturm ich schwänge,
bricht mir hervor aus der Brust,
was wütend das Herz noch hegt?
 


\direct{Das Feuer bricht zusammen; es fällt aus der aufsprühenden Glut plötzlich ein greller Schein auf die Stelle des Eschenstammes, welche Sieglindes Blick bezeichnet hatte und an der man jetzt deutlich einen Schwertgriff haften sieht}

Was gleißt dort hell im Glimmerschein?
Welch ein Strahl bricht aus der Esche Stamm?
Des Blinden Auge leuchtet ein Blitz:
lustig lacht da der Blick.
Wie der Schein so hehr das Herz mir sengt!
Ist es der Blick der blühenden Frau,
den dort haftend sie hinter sich ließ,
als aus dem Saal sie schied?
 


\direct{Von hier an verglimmt das Herdfeuer allmählich}

Nächtiges Dunkel deckte mein Aug',
ihres Blickes Strahl streifte mich da:
Wärme gewann ich und Tag.
Selig schien mir der Sonne Licht;
den Scheitel umgliß mir ihr wonniger Glanz -
bis hinter Bergen sie sank.
 


\direct{Ein neuer schwacher Aufschein des Feuers}

Noch einmal, da sie schied,
traf mich abends ihr Schein;
selbst der alten Esche Stamm
erglänzte in goldner Glut:
da bleicht die Blüte, das Licht verlischt;
nächtiges Dunkel deckt mir das Auge:
tief in des Busens Berge glimmt nur noch lichtlose Glut.
 


\direct{Das Feuer ist gänzlich verloschen: volle Nacht.  Das Seitengemach öffnet sich leise: Sieglinde, in weißem Gewande, tritt heraus und schreitet leise, doch rasch, auf den Herd zu}

\Sieglindespeaks
Schläfst du, Gast?
 

\Siegmundspeaks

\direct{freudig überrascht aufspringend}

Wer schleicht daher?
 

\Sieglindespeaks

\direct{mit geheimnisvoller Hast}

Ich bin's: höre mich an!
In tiefem Schlaf liegt Hunding;
ich würzt' ihm betäubenden Trank:
nütze die Nacht dir zum Heil!
 

\Siegmundspeaks

\direct{hitzig unterbrechend}

Heil macht mich dein Nah'n!
 

\Sieglindespeaks
Eine Waffe laß mich dir weisen: o wenn du sie gewännst!
Den hehrsten Helden dürft' ich dich heißen:
dem Stärksten allein ward sie bestimmt.
O merke wohl, was ich dir melde!
Der Männer Sippe saß hier im Saal,
von Hunding zur Hochzeit geladen:
er freite ein Weib,
das ungefragt Schächer ihm schenkten zur Frau.
Traurig saß ich, während sie tranken;
ein Fremder trat da herein:
ein Greis in blauem Gewand;
tief hing ihm der Hut,
der deckt' ihm der Augen eines;
doch des andren Strahl, Angst schuf es allen,
traf die Männer sein mächtiges Dräu'n:
mir allein weckte das Auge
süß sehnenden Harm,
Tränen und Trost zugleich.
Auf mich blickt' er und blitzte auf jene,
als ein Schwert in Händen er schwang;
das stieß er nun in der Esche Stamm,
bis zum Heft haftet' es drin:
dem sollte der Stahl geziemen,
der aus dem Stamm es zög'.
Der Männer alle, so kühn sie sich mühten,
die Wehr sich keiner gewann;
Gäste kamen und Gäste gingen,
die stärksten zogen am Stahl -
keinen Zoll entwich er dem Stamm:
dort haftet schweigend das Schwert. -
Da wußt' ich, wer der war,
der mich Gramvolle gegrüßt; ich weiß auch,
wem allein im Stamm das Schwert er bestimmt.
O fänd' ich ihn hier und heut', den Freund;
käm' er aus Fremden zur ärmsten Frau.
Was je ich gelitten in grimmigem Leid,
was je mich geschmerzt in Schande und Schmach, -
süßeste Rache sühnte dann alles!
Erjagt hätt' ich, was je ich verlor,
was je ich beweint, wär' mir gewonnen,
fänd' ich den heiligen Freund,
umfing' den Helden mein Arm!
 

\Siegmundspeaks

\direct{mit Glut Sieglinde umfassend}

Dich selige Frau hält nun der Freund,
dem Waffe und Weib bestimmt!
Heiß in der Brust brennt mir der Eid,
der mich dir Edlen vermählt.
Was je ich ersehnt, ersah ich in dir;
in dir fand ich, was je mir gefehlt!
Littest du Schmach,
und schmerzte mich Leid;
war ich geächtet, und warst du entehrt:
freudige Rache lacht nun den Frohen!
Auf lach' ich in heiliger Lust,
halt' ich dich Hehre umfangen,
fühl' ich dein schlagendes Herz!
 


\direct{Die große Türe springt auf}


\Sieglindespeaks

\direct{fährt erschrocken zusammen und reißt sich los.}

Ha, wer ging? Wer kam herein?
 


\direct{Die Tür bleibt weit geöffnet: außen herrliche Frühlingsnacht; der Vollmond leuchtet herein und wirft sein helles Licht auf das Paar, das so sich plötzlich in voller Deutlichkeit wahrnehmen kann}


\Siegmundspeaks

\direct{in leiser Entzückung}

Keiner ging---doch einer kam:
siehe, der Lenz lacht in den Saal!
 


\direct{Siegmund zieht Sieglinde mit sanfter Gewalt zu sich auf das Lager, so daß sie neben ihm zu sitzen kommt

Wachsende Helligkeit des Mondscheines}

Winterstürme wichen
dem Wonnemond,
in mildem Lichte leuchtet der Lenz;
auf linden Lüften leicht und lieblich,
Wunder webend er sich wiegt;
durch Wald und Auen weht sein Atem,
weit geöffnet lacht sein Aug': -
aus sel'ger Vöglein Sange süß er tönt,
holde Düfte haucht er aus;
seinem warmen Blut entblühen wonnige Blumen,
Keim und Sproß entspringt seiner Kraft.
Mit zarter Waffen Zier bezwingt er die Welt;
Winter und Sturm wichen der starken Wehr:
wohl mußte den tapfern Streichen
die strenge Türe auch weichen,
die trotzig und starr uns trennte von ihm. -
Zu seiner Schwester schwang er sich her;
die Liebe lockte den Lenz:
in unsrem Busen barg sie sich tief;
nun lacht sie selig dem Licht.
Die bräutliche Schwester befreite der Bruder;
zertrümmert liegt, was je sie getrennt:
jauchzend grüßt sich das junge Paar:
vereint sind Liebe und Lenz!
 

\Sieglindespeaks
Du bist der Lenz, nach dem ich verlangte
in frostigen Winters Frist.
Dich grüßte mein Herz mit heiligem Grau'n,
als dein Blick zuerst mir erblühte.
Fremdes nur sah ich von je,
freudlos war mir das Nahe.
Als hätt' ich nie es gekannt, war, was immer mir kam.
Doch dich kannt' ich deutlich und klar:
als mein Auge dich sah,
warst du mein Eigen;
was im Busen ich barg, was ich bin,
hell wie der Tag taucht' es mir auf,
o wie tönender Schall schlug's an mein Ohr,
als in frostig öder Fremde
zuerst ich den Freund ersah.
 


\direct{Sie hängt sich entzückt an seinen Hals und blickt ihm nahe ins Gesicht}


\Siegmundspeaks

\direct{mit Hingerissenheit}

O süßeste Wonne!
O seligstes Weib!
 

\Sieglindespeaks

\direct{dicht an seinen Augen}

O laß in Nähe zu dir mich neigen,
daß hell ich schaue den hehren Schein,
der dir aus Aug' und Antlitz bricht
und so süß die Sinne mir zwingt.
 

\Siegmundspeaks
Im Lenzesmond leuchtest du hell;
hehr umwebt dich das Wellenhaar:
was mich berückt, errat' ich nun leicht,
denn wonnig weidet mein Blick.
 

\Sieglindespeaks

\direct{schlägt ihm die Locken von der Stirn zurück und betrachtet ihn staunend}

Wie dir die Stirn so offen steht,
der Adern Geäst in den Schläfen sich schlingt!
Mir zagt es vor der Wonne, die mich entzückt!
Ein Wunder will mich gemahnen:
den heut' zuerst ich erschaut,
mein Auge sah dich schon!
 

\Siegmundspeaks
Ein Minnetraum gemahnt auch mich:
in heißem Sehnen sah ich dich schon!
 

\Sieglindespeaks
Im Bach erblickt' ich mein eigen Bild -
und jetzt gewahr' ich es wieder:
wie einst dem Teich es enttaucht,
bietest mein Bild mir nun du!
 

\Siegmundspeaks
Du bist das Bild,
das ich in mir barg.
 

\Sieglindespeaks

\direct{den Blick schnell abwendend}

O still! Laß mich der Stimme lauschen:
 

mich dünkt, ihren Klang
hört' ich als Kind.
 


\direct{aufgeregt}

Doch nein! Ich hörte sie neulich,
als meiner Stimme Schall
mir widerhallte der Wald.
 

\Siegmundspeaks
O lieblichste Laute,
denen ich lausche!
 

\Sieglindespeaks

\direct{ihm wieder in die Augen spähend}

Deines Auges Glut erglänzte mir schon:
so blickte der Greis grüßend auf mich,
als der Traurigen Trost er gab.
An dem Blick erkannt' ihn sein Kind -
schon wollt' ich beim Namen ihn nennen!
 


\direct{Sie hält inne und fährt dann leise fort}

Wehwalt heißt du fürwahr?
 

\Siegmundspeaks
Nicht heiß' ich so, seit du mich liebst:
nun walt' ich der hehrsten Wonnen!
 

\Sieglindespeaks
Und Friedmund darfst du
froh dich nicht nennen?
 

\Siegmundspeaks
Nenne mich du, wie du liebst, daß ich heiße:
den Namen nehm' ich von dir!
 

\Sieglindespeaks
Doch nanntest du Wolfe den Vater?
 

\Siegmundspeaks
Ein Wolf war er feigen Füchsen!
Doch dem so stolz strahlte das Auge,
wie, Herrliche, hehr dir es strahlt,
der war: Wälse genannt.
 

\Sieglindespeaks

\direct{außer sich}

War Wälse dein Vater, und bist du ein Wälsung,
stieß er für dich sein Schwert in den Stamm,
so laß mich dich heißen, wie ich dich liebe:
Siegmund---so nenn' ich dich!
 

\Siegmundspeaks

\direct{springt auf den Stamm zu und faßt den Schwertgriff}

Siegmund heiß' ich und Siegmund bin ich!
Bezeug' es dies Schwert, das zaglos ich halte!
Wälse verhieß mir, in höchster Not
fänd' ich es einst: ich faß' es nun!
Heiligster Minne höchste Not,
sehnender Liebe sehrende Not
brennt mir hell in der Brust,
drängt zu Tat und Tod:
Notung! Notung! So nenn' ich dich, Schwert -
Notung! Notung! Neidlicher Stahl!
Zeig' deiner Schärfe schneidenden Zahn:
heraus aus der Scheide zu mir!
 


\direct{Er zieht mit einem gewaltigen Zuck das Schwert aus dem Stamme und zeigt es der von Staunen und Entzücken erfaßten Sieglinde}


Siegmund, den Wälsung, siehst du, Weib!
Als Brautgabe bringt er dies Schwert:
so freit er sich
die seligste Frau;
dem Feindeshaus entführt er dich so.
Fern von hier folge mir nun,
fort in des Lenzes lachendes Haus:
dort schützt dich Notung, das Schwert,
wenn Siegmund dir liebend erlag!
 


\direct{Er hat sie umfaßt, um sie mit sich fortzuziehen}


\Sieglindespeaks

\direct{reißt sich in höchster Trunkenheit von ihm los und stellt sich ihm gegenüber}

Bist du Siegmund, den ich hier sehe,
Sieglinde bin ich, die dich ersehnt:
die eigne Schwester
gewannst du zu eins mit dem Schwert!
 

\Siegmundspeaks
Braut und Schwester bist du dem Bruder -
so blühe denn, Wälsungen-Blut!
 


\direct{Er zieht sie mit wütender Glut an sich; sie sinkt mit einem Schrei an seine Brust. Der Vorhang fällt schnell}


\act

\scene

\StageDir{Wildes Felsengebirge

Im Hintergrund zieht sich von unten her eine Schlucht herauf, die auf ein erhöhtes Felsjoch mündet; von diesem senkt sich der Boden dem Vordergrunde zu wieder abwärts.


\direct{Wotan, kriegerisch gewaffnet, mit dem Speer; vor ihm Brünnhilde, als Walküre, ebenfalls in voller Waffenrüstung}}


\Wotanspeaks
Nun zäume dein Roß, reisige Maid!
Bald entbrennt brünstiger Streit:
Brünnhilde stürme zum Kampf,
dem Wälsung kiese sie Sieg!
Hunding wähle sich, wem er gehört;
nach Walhall taugt er mir nicht.
Drum rüstig und rasch, reite zur Wal!
 

\Brunnhildespeaks

\direct{jauchzend von Fels zu Fels die Höhe rechts hinaufspringend}

Hojotoho! Hojotoho!
Heiaha! Heiaha! Hojotoho! Heiaha!
 


\direct{Sie hält auf einer hohen Felsspitze an, blickt in die hintere Schlucht hinab und ruft zu Wotan zurück}

Dir rat' ich, Vater, rüste dich selbst;
harten Sturm sollst du bestehn.
Fricka naht, deine Frau,
im Wagen mit dem Widdergespann.
Hei! Wie die goldne Geißel sie schwingt!
Die armen Tiere ächzen vor Angst;
wild rasseln die Räder;
zornig fährt sie zum Zank!
In solchem Strauße streit' ich nicht gern,
lieb' ich auch mutiger Männer Schlacht!
Drum sieh, wie den Sturm du bestehst:
ich Lustige laß' dich im Stich!
Hojotoho! Hojotoho!
Heiaha! Heiaha!
Heiahaha!
 


\direct{Brünnhilde verschwindet hinter der Gebirgshöhe zur Seite

In einem mit zwei Widdern bespannten Wagen langt Fricka aus der Schlucht auf dem Felsjoche an, dort hält sie rasch an und steigt aus. Sie schreitet heftig in den Vordergrund auf Wotan zu}

\Wotanspeaks

\direct{Fricka auf sich zuschreiten sehend, für sich}

Der alte Sturm, die alte Müh'!
Doch stand muß ich hier halten!
 

\Frickaspeaks

\direct{je näher sie kommt, desto mehr mäßigt sie den Schritt und stellt sich mit Würde vor Wotan hin}

Wo in den Bergen du dich birgst,
der Gattin Blick zu entgehn,
einsam hier such' ich dich auf,
daß Hilfe du mir verhießest.
 

\Wotanspeaks
Was Fricka kümmert, künde sie frei.
 

\Frickaspeaks
Ich vernahm Hundings Not,
um Rache rief er mich an:
der Ehe Hüterin hörte ihn,
verhieß streng zu strafen die Tat
des frech frevelnden Paars,
das kühn den Gatten gekränkt.
 

\Wotanspeaks
Was so Schlimmes schuf das Paar,
das liebend einte der Lenz?
Der Minne Zauber entzückte sie:
wer büßt mir der Minne Macht?
 

\Frickaspeaks
Wie töricht und taub du dich stellst,
als wüßtest fürwahr du nicht,
daß um der Ehe heiligen Eid,
den hart gekränkten, ich klage!
 

\Wotanspeaks
Unheilig acht' ich den Eid,
der Unliebende eint;
und mir wahrlich mute nicht zu,
daß mit Zwang ich halte, was dir nicht haftet:
denn wo kühn Kräfte sich regen,
da rat' ich offen zum Krieg.
 

\Frickaspeaks
Achtest du rühmlich der Ehe Bruch,
so prahle nun weiter und preis' es heilig,
daß Blutschande entblüht
dem Bund eines Zwillingspaars!
Mir schaudert das Herz, es schwindelt mein Hirn:
bräutlich umfing die Schwester der Bruder!
Wann ward es erlebt,
daß leiblich Geschwister sich liebten?
 

\Wotanspeaks
Heut' hast du's erlebt!
Erfahre so, was von selbst sich fügt,
sei zuvor auch noch nie es geschehn.
Daß jene sich lieben, leuchtet dir hell;
drum höre redlichen Rat:
Soll süße Lust deinen Segen dir lohnen,
so segne, lachend der Liebe,
Siegmunds und Sieglindes Bund!
 

\Frickaspeaks

\direct{in höchste Entrüstung ausbrechend}

So ist es denn aus mit den ewigen Göttern,
seit du die wilden Wälsungen zeugtest?
Heraus sagt' ich's; traf ich den Sinn?
Nichts gilt dir der Hehren heilige Sippe;
hin wirfst du alles, was einst du geachtet;
zerreißest die Bande, die selbst du gebunden,
lösest lachend des Himmels Haft: -
daß nach Lust und Laune nur walte
dies frevelnde Zwillingspaar,
deiner Untreue zuchtlose Frucht!
O, was klag' ich um Ehe und Eid,
da zuerst du selbst sie versehrt!
Die treue Gattin trogest du stets;
wo eine Tiefe, wo eine Höhe,
dahin lugte lüstern dein Blick,
wie des Wechsels Lust du gewännest
und höhnend kränktest mein Herz.
Trauernden Sinnes mußt' ich's ertragen,
zogst du zur Schlacht mit den schlimmen Mädchen,
die wilder Minne Bund dir gebar:
denn dein Weib noch scheutest du so,
daß der Walküren Schar
und Brünnhilde selbst, deines Wunsches Braut,
in Gehorsam der Herrin du gabst.
Doch jetzt, da dir neue
Namen gefielen,
als ``Wälse'' wölfisch im Walde du schweiftest;
jetzt, da zu niedrigster
Schmach du dich neigtest,
gemeiner Menschen ein Paar zu erzeugen:
jetzt dem Wurfe der Wölfin
wirfst du zu Füßen dein Weib!
So führ' es denn aus! Fülle das Maß!
Die Betrogne laß auch zertreten!
 

\Wotanspeaks

\direct{ruhig}

Nichts lerntest du, wollt' ich dich lehren,
was nie du erkennen kannst,
eh' nicht ertagte die Tat.
Stets Gewohntes nur magst du verstehn:
doch was noch nie sich traf,
danach trachtet mein Sinn.
Eines höre! Not tut ein Held,
der, ledig göttlichen Schutzes,
sich löse vom Göttergesetz.
So nur taugt er zu wirken die Tat,
die, wie not sie den Göttern,
dem Gott doch zu wirken verwehrt.
 

\Frickaspeaks
Mit tiefem Sinne willst du mich täuschen:
was Hehres sollten Helden je wirken,
das ihren Göttern wäre verwehrt,
deren Gunst in ihnen nur wirkt?
 

\Wotanspeaks
lhres eignen Mutes achtest du nicht?
 

\Frickaspeaks
Wer hauchte Menschen ihn ein?
Wer hellte den Blöden den Blick?
In deinem Schutz scheinen sie stark,
durch deinen Stachel streben sie auf:
du reizest sie einzig,
die so mir Ew'gen du rühmst,
Mit neuer List willst du mich belügen,
durch neue Ränke
mir jetzt entrinnen;
doch diesen Wälsung gewinnst du dir nicht:
in ihm treff' ich nur dich,
denn durch dich trotzt er allein.
 

\Wotanspeaks

\direct{ergriffen}

In wildem Leiden erwuchs er sich selbst:
mein Schutz schirmte ihn nie.
 

\Frickaspeaks
So schütz' auch heut' ihn nicht!
Nimm ihm das Schwert, das du ihm geschenkt!
 

\Wotanspeaks
Das Schwert?
 

\Frickaspeaks
Ja, das Schwert,
das zauberstark zuckende Schwert,
das du Gott dem Sohne gabst.
 

\Wotanspeaks

\direct{heftig}

Siegmund gewann es sich

\direct{mit unterdrücktem Beben}

selbst in der Not.
 


\direct{Wotan drückt in seiner ganzen Haltung von hier an einen immer wachsenden unheimlichen, tiefen Unmut aus}


\Frickaspeaks

\direct{eifrig fortfahrend}

Du schufst ihm die Not,
wie das neidliche Schwert.
Willst du mich täuschen,
die Tag und Nacht auf den Fersen dir folgt?
Für ihn stießest du das Schwert in den Stamm,
du verhießest ihm die hehre Wehr:
willst du es leugnen,
daß nur deine List
ihn lockte, wo er es fänd'?
 


\direct{Wotan fährt mit einer grimmigen Gebärde auf}


\Frickaspeaks

\direct{immer sicherer, da sie den Eindruck gewahrt, den sie auf Wotan hervorgebracht hat}

Mit Unfreien streitet kein Edler,
den Frevler straft nur der Freie.
Wider deine Kraft
führt' ich wohl Krieg:
doch Siegmund verfiel mir als Knecht!
 


\direct{Neue heftige Gebärde Wotans, dann Versinken in das Gefühl seiner Ohnmacht}

Der dir als Herren hörig und eigen,
gehorchen soll ihm dein ewig Gemahl?
Soll mich in Schmach der Niedrigste schmähen,
dem Frechen zum Sporn,
dem Freien zum Spott?
Das kann mein Gatte nicht wollen,
die Göttin entweiht er nicht so!
 

\Wotanspeaks

\direct{finster}

Was verlangst du?
 

\Frickaspeaks
Laß von dem Wälsung!
 

\Wotanspeaks

\direct{mit gedämpfter Stimme}

Er geh' seines Wegs.
 

\Frickaspeaks
Doch du schütze ihn nicht,
wenn zur Schlacht ihn der Rächer ruft!
 

\Wotanspeaks
Ich schütze ihn nicht.
 

\Frickaspeaks
Sieh mir ins Auge, sinne nicht Trug:
die Walküre wend' auch von ihm!
 

\Wotanspeaks
Die Walküre walte frei.
 

\Frickaspeaks
Nicht doch; deinen Willen vollbringt sie allein:
verbiete ihr Siegmunds Sieg!
 

\Wotanspeaks

\direct{in heftigen inneren Kampf ausbrechend}

Ich kann ihn nicht fällen: er fand mein Schwert!
 

\Frickaspeaks
Entzieh' dem den Zauber, zerknick' es dem Knecht!
Schutzlos schau' ihn der Feind!
 

\Brunnhildespeaks

\direct{noch unsichtbar von der Höhe her}

Heiaha! Heiaha! Hojotoho!
 

\Frickaspeaks
Dort kommt deine kühne Maid;
jauchzend jagt sie daher.
 

\Brunnhildespeaks

\direct{wie oben}

Heiaha! Heiaha! Heiohotojo! Hotojoha!
 

\Wotanspeaks

\direct{dumpf für sich}

Ich rief sie für Siegmund zu Roß!
 


\direct{Brünnhilde erscheint mit ihrem Roß auf dem Felsenpfade rechts. Als sie Fricka gewahrt, bricht sie schnell ab und geleitet ihr Roß still und langsam während des Folgenden den Felsweg herab: dort birgt sie es dann in einer Höhle}


\Frickaspeaks
Deiner ew'gen Gattin heilige Ehre
beschirme heut' ihr Schild!
Von Menschen verlacht, verlustig der Macht,
gingen wir Götter zugrund:
würde heut' nicht hehr und herrlich mein Recht
gerächt von der mutigen Maid.
Der Wälsung fällt meiner Ehre:
Empfah' ich von Wotan den Eid?
 

\Wotanspeaks

\direct{in furchtbarem Unmut und innerem Grimm auf einen Felsensitz sich werfend}

Nimm den Eid!
 


\direct{Fricka schreitet dem Hintergrunde zu: dort begegnet sie Brünnhilde und hält einen Augenblick vor ihr an}


\Frickaspeaks

\direct{zu Brünnhilde}

Heervater harret dein:
lass' ihn dir künden, wie das Los er gekiest!
 


\direct{Sie besteigt den Wagen und fährt schnell davon}

\direct{Brünnhilde tritt mit besorgter Miene verwundert vor Wotan, der, auf dem Felssitz zurückgelehnt, das Haupt auf die Hand gestützt, in finstres Brüten versunken ist}

\scene

\Brunnhildespeaks
Schlimm, fürcht' ich, schloß der Streit,
lachte Fricka dem Lose.
Vater, was soll dein Kind erfahren?
Trübe scheinst du und traurig!

\Wotanspeaks

\direct{läßt den Arm machtlos sinken und den Kopf in den Nacken fallen}

In eigner Fessel fing ich mich:
ich Unfreiester aller!
 

\Brunnhildespeaks
So sah ich dich nie!
Was nagt dir das Herz?
 

\Wotanspeaks

\direct{von hier an steigert sich Wotans Ausdruck und Gebärde bis zum furchtbarsten Ausbruch}

O heilige Schmach! O schmählicher Harm!
Götternot! Götternot!
Endloser Grimm! Ewiger Gram!
Der Traurigste bin ich von allen!
 

\Brunnhildespeaks

\direct{wirft erschrocken Schild, Speer und Helm von sich und läßt sich mit besorgter Zutraulichkeit zu Wotans Füßen nieder}

Vater! Vater! Sage, was ist dir?
Wie erschreckst du mit Sorge dein Kind?
Vertraue mir! Ich bin dir treu:
sieh, Brünnhilde bittet!
 


\direct{Sie legt traulich und ängstlich Haupt und Hände ihm auf Knie und Schoß}


\Wotanspeaks

\direct{blickt ihr lange ins Auge; dann streichelt er ihr mit unwillkürlicher Zärtlichkeit die Locken. Wie aus tiefem Sinnen zu sich kommend, beginnt er endlich sehr leise}

Lass' ich's verlauten,
lös' ich dann nicht meines Willens haltenden Haft?
 

\Brunnhildespeaks

\direct{ihm ebenso erwidernd}

Zu Wotans Willen sprichst du,
sagst du mir, was du willst;
wer bin ich, wär' ich dein Wille nicht?
 

\Wotanspeaks

\direct{sehr leise}

Was keinem in Worten ich künde,
unausgesprochen bleib' es denn ewig:
mit mir nur rat' ich, red' ich zu dir. -
 


\direct{mit noch gedämpfterer, schauerlicher Stimme, während er Brünnhilde unverwandt in das Auge blickt}

Als junger Liebe Lust mir verblich,
verlangte nach Macht mein Mut:
von jäher Wünsche Wüten gejagt,
gewann ich mir die Welt.
Unwissend trugvoll, Untreue übt' ich,
band durch Verträge, was Unheil barg:
listig verlockte mich Loge,
der schweifend nun verschwand.
Von der Liebe doch mocht' ich nicht lassen,
in der Macht verlangt' ich nach Minne.
Den Nacht gebar, der bange Nibelung,
Alberich, brach ihren Bund;
er fluchte der Lieb' und gewann durch den Fluch
des Rheines glänzendes Gold
und mit ihm maßlose Macht.
Den Ring, den er schuf,
entriß ich ihm listig;
doch nicht dem Rhein gab ich ihn zurück:
mit ihm bezahlt' ich Walhalls Zinnen,
der Burg, die Riesen mir bauten,
aus der ich der Welt nun gebot.
Die alles weiß, was einstens war,
Erda, die weihlich weiseste Wala,
riet mir ab von dem Ring,
warnte vor ewigem Ende.
Von dem Ende wollt' ich mehr noch wissen;
doch schweigend entschwand mir das Weib...
Da verlor ich den leichten Mut,
zu wissen begehrt' es den Gott:
in den Schoß der Welt schwang ich mich hinab,
mit Liebeszauber zwang ich die Wala,
stört' ihres Wissens Stolz, daß sie Rede nun mir stand.
Kunde empfing ich von ihr;
von mir doch barg sie ein Pfand:
der Welt weisestes Weib
gebar mir, Brünnhilde, dich.
Mit acht Schwestern zog ich dich auf;
durch euch Walküren wollt' ich wenden,
was mir die Wala zu fürchten schuf:
ein schmähliches Ende der Ew'gen.
Daß stark zum Streit uns fände der Feind,
hieß ich euch Helden mir schaffen:
die herrisch wir sonst
in Gesetzen hielten,
die Männer, denen den Mut wir gewehrt,
die durch trüber Verträge trügende Bande
zu blindem Gehorsam wir uns gebunden,
die solltet zu Sturm
und Streit ihr nun stacheln,
ihre Kraft reizen zu rauhem Krieg,
daß kühner Kämpfer Scharen
ich sammle in Walhalls Saal!
 

\Brunnhildespeaks
Deinen Saal füllten wir weidlich:
viele schon führt' ich dir zu.
Was macht dir nun Sorge, da nie wir gesäumt?
 

\Wotanspeaks

\direct{wieder gedämpfter}

Ein andres ist's:
achte es wohl, wes mich die Wala gewarnt!
Durch Alberichs Heer
droht uns das Ende:
mit neidischem Grimm grollt mir der Niblung:
doch scheu' ich nun nicht seine nächtigen Scharen,
meine Helden schüfen mir Sieg.
Nur wenn je den Ring
zurück er gewänne,
dann wäre Walhall verloren:
der der Liebe fluchte, er allein
nützte neidisch des Ringes Runen
zu aller Edlen endloser Schmach:
der Helden Mut entwendet' er mir;
die Kühnen selber
zwäng' er zum Kampf;
mit ihrer Kraft bekriegte er mich.
Sorgend sann ich nun selbst,
den Ring dem Feind zu entreißen.
Der Riesen einer, denen ich einst
mit verfluchtem Gold den Fleiß vergalt:
Fafner hütet den Hort,
um den er den Bruder gefällt.
Ihm müßt' ich den Reif entringen,
den selbst als Zoll ich ihm zahlte.
Doch mit dem ich vertrug,
ihn darf ich nicht treffen;
machtlos vor ihm erläge mein Mut: -
das sind die Bande, die mich binden:
der durch Verträge ich Herr,
den Verträgen bin ich nun Knecht.
Nur einer könnte, was ich nicht darf:
ein Held, dem helfend nie ich mich neigte;
der fremd dem Gotte, frei seiner Gunst,
unbewußt, ohne Geheiß,
aus eigner Not, mit der eignen Wehr
schüfe die Tat, die ich scheuen muß,
die nie mein Rat ihm riet,
wünscht sie auch einzig mein Wunsch!
Der, entgegen dem Gott, für mich föchte,
den freundlichen Feind, wie fände ich ihn?
Wie schüf' ich den Freien, den nie ich schirmte,
der im eignen Trotze der Trauteste mir?
Wie macht' ich den andren, der nicht mehr ich,
und aus sich wirkte, was ich nur will?
O göttliche Not! Gräßliche Schmach!
Zum Ekel find' ich ewig nur mich
in allem, was ich erwirke!
Das andre, das ich ersehne,
das andre erseh' ich nie:
denn selbst muß der Freie sich schaffen:
Knechte erknet' ich mir nur!
 

\Brunnhildespeaks
Doch der Wälsung, Siegmund, wirkt er nicht selbst?
 

\Wotanspeaks
Wild durchschweift' ich mit ihm die Wälder;
gegen der Götter Rat reizte kühn ich ihn auf:
gegen der Götter Rache
schützt ihn nun einzig das Schwert,
 


\direct{gedehnt und bitter}

das eines Gottes Gunst ihm beschied.
Wie wollt' ich listig selbst mich belügen?
So leicht ja entfrug mir Fricka den Trug:
zu tiefster Scham durchschaute sie mich!
Ihrem Willen muß ich gewähren.
 

\Brunnhildespeaks
So nimmst du von Siegmund den Sieg?
 

\Wotanspeaks
Ich berührte Alberichs Ring,
gierig hielt ich das Gold!
Der Fluch, den ich floh,
nicht flieht er nun mich:
Was ich liebe, muß ich verlassen,
morden, wen je ich minne,
trügend verraten, wer mir traut!
 


\direct{Wotans Gebärde geht aus dem Ausdruck des furchtbarsten Schmerzes zu dem der Verzweiflung über}


Fahre denn hin, herrische Pracht,
göttlichen Prunkes prahlende Schmach!
Zusammenbreche, was ich gebaut!
Auf geb' ich mein Werk; nur eines will ich noch:
das Ende,
das Ende! -
 


\direct{Er hält sinnend ein}

Und für das Ende sorgt Alberich!
Jetzt versteh' ich den stummen Sinn
des wilden Wortes der Wala:
``Wenn der Liebe finstrer Feind
zürnend zeugt einen Sohn,
der Sel'gen Ende säumt dann nicht!''
Vom Niblung jüngst vernahm ich die Mär',
daß ein Weib der Zwerg bewältigt,
des' Gunst Gold ihm erzwang:
Des Hasses Frucht hegt eine Frau,
des Neides Kraft kreißt ihr im Schoß:
das Wunder gelang dem Liebelosen;
doch der in Lieb' ich freite,
den Freien erlang' ich mir nicht.
 


\direct{mit bitterem Grimm sich aufrichtend}

So nimm meinen Segen, Niblungen-Sohn!
Was tief mich ekelt, dir geb' ich's zum Erbe,
der Gottheit nichtigen Glanz:
zernage ihn gierig dein Neid!
 

\Brunnhildespeaks

\direct{erschrocken}

O sag', künde, was soll nun dein Kind?
 

\Wotanspeaks

\direct{bitter}

Fromm streite für Fricka; hüte ihr Eh' und Eid!

\direct{trocken}

Was sie erkor, das kiese auch ich:
was frommte mir eigner Wille?
Einen Freien kann ich nicht wollen:
für Frickas Knechte kämpfe nun du!
 

\Brunnhildespeaks
Weh'! Nimm reuig zurück das Wort!
Du liebst Siegmund;
dir zulieb', ich weiß es, schütz' ich den Wälsung.
 

\Wotanspeaks
Fällen sollst du Siegmund,
für Hunding erfechten den Sieg!
Hüte dich wohl und halte dich stark,
all deiner Kühnheit entbiete im Kampf:
ein Siegschwert schwingt Siegmund; -
schwerlich fällt er dir feig!
 

\Brunnhildespeaks
Den du zu lieben stets mich gelehrt,

\direct{sehr warm}

der in hehrer Tugend dem Herzen dir teuer,
gegen ihn zwingt mich nimmer dein zwiespältig Wort!
 

\Wotanspeaks
Ha, Freche du! Frevelst du mir?
Wer bist du, als meines Willens
blind wählende Kür?
Da mit dir ich tagte, sank ich so tief,
daß zum Schimpf der eignen
Geschöpfe ich ward?
Kennst du, Kind, meinen Zorn?
Verzage dein Mut,
wenn je zermalmend
auf dich stürzte sein Strahl!
In meinem Busen berg' ich den Grimm,
der in Grau'n und Wust wirft eine Welt,
die einst zur Lust mir gelacht:
wehe dem, den er trifft!
Trauer schüf' ihm sein Trotz!
Drum rat' ich dir, reize mich nicht!
Besorge, was ich befahl:
Siegmund falle -
Dies sei der Walküre Werk!
 


\direct{er stürmt fort und verschwindet schnell links in Gebirge}


\Brunnhildespeaks

\direct{steht lange erschrocken und betäubt}

So sah ich Siegvater nie,

\direct{sie starrt wild vor sich hin}

erzürnt' ihn sonst wohl auch ein Zank!

\direct{Sie neigt sich betrübt und nimmt ihre Waffen auf, mit denen sie sich wieder rüstet}

Schwer wiegt mir der Waffen Wucht: -
wenn nach Lust ich focht,
wie waren sie leicht!
Zu böser Schlacht schleich' ich heut' so bang.

\direct{Sie sinnt vor sich hin und seufzt dann auf}

Weh', mein Wälsung!
Im höchsten Leid
muß dich treulos die Treue verlassen!

\direct{Sie wendet sich langsam dem Hintergrunde zu}

\scene

\StageDir{Auf dem Bergjoch angelangt, gewahrt Brünnhilde, in die Schlucht hinabblickend, Siegmund und Sieglinde; sie betrachtet die Nahenden einen Augenblick und wendet sich dann in die Höhle zu ihrem Roß, so daß sie dem Zuschauer gänzlich verschwindet. Siegmund und Sieglinde erscheinen auf dem Bergjoche. Sieglinde schreitet hastig voraus; Siegmund sucht sie aufzuhalten}


\Siegmundspeaks
Raste nun hier; gönne dir Ruh'!

\Sieglindespeaks
Weiter! Weiter!
 

\Siegmundspeaks

\direct{umfaßt sie mit sanfter Gewalt}

Nicht weiter nun!
 


\direct{Er schließt sie fest an sich}

Verweile, süßestes Weib!
Aus Wonne-Entzücken zucktest du auf,
mit jäher Hast jagtest du fort:
kaum folgt' ich der wilden Flucht;
durch Wald und Flur, über Fels und Stein,
sprachlos, schweigend sprangst du dahin,
kein Ruf hielt dich zur Rast!
 

Ruhe nun aus: rede zu mir!
Ende des Schweigens Angst!
Sieh, dein Bruder hält seine Braut:
Siegmund ist dir Gesell'!
 


\direct{Er hat sie unvermerkt nach dem Steinsitze geleitet}


\Sieglindespeaks

\direct{blickt Siegmund mit wachsendem Entzücken in die Augen, dann umschlingt sie leidenschaftlich seinen Hals und verweilt so; dann fährt sie mit jähem Schreck auf}

Hinweg! Hinweg! Flieh' die Entweihte!
Unheilig umfängt dich ihr Arm;
entehrt, geschändet schwand dieser Leib:
flieh' die Leiche, lasse sie los!
Der Wind mag sie verwehn,
die ehrlos dem Edlen sich gab!
Da er sie liebend umfing,
da seligste Lust sie fand,
da ganz sie minnte der Mann,
der ganz ihre Minne geweckt:
vor der süßesten Wonne heiligster Weihe,
die ganz ihr Sinn und Seele durchdrang,
Grauen und Schauder ob gräßlichster Schande
mußte mit Schreck die Schmähliche fassen,
die je dem Manne gehorcht,
der ohne Minne sie hielt!
Laß die Verfluchte, laß sie dich fliehn!
Verworfen bin ich, der Würde bar!
Dir reinstem Manne muß ich entrinnen,
dir Herrlichem darf ich nimmer gehören.
Schande bring' ich dem Bruder,
Schmach dem freienden Freund!
 

\Siegmundspeaks
Was je Schande dir schuf,
das büßt nun des Frevlers Blut!
Drum fliehe nicht weiter; harre des Feindes;
hier soll er mir fallen:
wenn Notung ihm das Herz zernagt,
Rache dann hast du erreicht!
 

\Sieglindespeaks

\direct{schrickt auf und lauscht}

Horch! Die Hörner, hörst du den Ruf?
Ringsher tönt wütend Getös':
aus Wald und Gau gellt es herauf.
Hunding erwachte aus hartem Schlaf!
Sippen und Hunde ruft er zusammen;
mutig gehetzt heult die Meute,
wild bellt sie zum Himmel
um der Ehe gebrochenen Eid!
 


\direct{Sieglinde starrt wie wahnsinnig vor sich hin}

Wo bist du, Siegmund? Seh' ich dich noch,
brünstig geliebter, leuchtender Bruder?
Deines Auges Stern laß noch einmal mir strahlen:
wehre dem Kuß des verworfnen Weibes nicht! -
 


\direct{Sie hat sich ihm schluchzend an die Brust geworfen: dann schrickt sie ängstlich wieder auf}

Horch! O horch! Das ist Hundings Horn!
Seine Meute naht mit mächt'ger Wehr:
kein Schwert frommt
vor der Hunde Schwall:
wirf es fort, Siegmund! Siegmund! Wo bist du?
Ha dort! Ich sehe dich! Schrecklich Gesicht!
Rüden fletschen die Zähne nach Fleisch;
sie achten nicht deines edlen Blicks;
bei den Füßen packt dich das feste Gebiß -
du fällst in Stücken zerstaucht das Schwert:
die Esche stürzt, es bricht der Stamm!
Bruder! Mein Bruder! Siegmund! Ha!
 


\direct{Sie sinkt ohnmächtig in Siegmunds Arme}


\Siegmundspeaks
Schwester! Geliebte!
 


\StageDir{Er lauscht ihrem Atem und überzeugt sich, daß sie noch lebe. Er läßt sie an sich herabgleiten, so daß sie, als er sich selbst zum Sitze niederläßt, mit ihrem Haupt auf seinem Schoß zu ruhen kommt. In dieser Stellung verbleiben beide bis zum Schlusse des folgenden Auftrittes.

Langes Schweigen, währenddessen Siegmund mit zärtlicher Sorge über Sieglinde sich hinneigt und mit einem langen Kusse ihr die Stirne küßt.

Brünnhilde, ihr Roß am Zaume geleitend, tritt aus der Höhle und schreitet langsam und feierlich nach vorne. Sie hält an und betrachtet Siegmund von fern. Sie schreitet wieder langsam vor. Sie hält in größerer Nähe an. Sie trägt Schild und Speer in der einen Hand, lehnt sich mit der andern an den Hals des Rosses und betrachtet so mit ernster Miene Siegmund}


\Brunnhildespeaks
Siegmund! Sieh auf mich!
Ich bin's, der bald du folgst.

\Siegmundspeaks

\direct{richtet den Blick zu ihr auf}

Wer bist du, sag',
die so schön und ernst mir erscheint?
 

\Brunnhildespeaks
Nur Todgeweihten taugt mein Anblick;
wer mich erschaut, der scheidet vom Lebenslicht.
Auf der Walstatt allein erschein' ich Edlen:
wer mich gewahrt, zur Wal kor ich ihn mir!
 

\Siegmundspeaks

\direct{blickt ihr lange forschend und fest in das Auge, senkt dann sinnend das Haupt und wendet sich endlich mit feierlichem Ernste wieder zu ihr}

Der dir nun folgt, wohin führst du den Helden?
 

\Brunnhildespeaks
Zu Walvater, der dich gewählt,
führ' ich dich: nach Walhall folgst du mir.
 

\Siegmundspeaks
In Walhalls Saal Walvater find' ich allein?
 

\Brunnhildespeaks
Gefallner Helden hehre Schar
umfängt dich hold mit hoch-heiligem Gruß.
 

\Siegmundspeaks
Fänd' ich in Walhall Wälse, den eignen Vater?
 

\Brunnhildespeaks
Den Vater findet der Wälsung dort.
 

\Siegmundspeaks
Grüßt mich in Walhall froh eine Frau?
 

\Brunnhildespeaks
Wunschmädchen walten dort hehr:
Wotans Tochter reicht dir traulich den Trank!
 

\Siegmundspeaks
Hehr bist du,
und heilig gewahr' ich das Wotanskind:
doch eines sag' mir, du Ew'ge!
Begleitet den Bruder die bräutliche Schwester?
Umfängt Siegmund Sieglinde dort?
 

\Brunnhildespeaks
Erdenluft muß sie noch atmen:
Sieglinde sieht Siegmund dort nicht!
 

\Siegmundspeaks

\direct{neigt sich sanft über Sieglinde, küßt sie leise auf die Stirn und wendet sich ruhig wieder zu Brünnhilde}

So grüße mir Walhall, grüße mir Wotan,
grüße mir Wälse und alle Helden,
grüß' auch die holden Wunschesmädchen: -
 


\direct{sehr bestimmt}

zu ihnen folg' ich dir nicht.
 

\Brunnhildespeaks
Du sahest der Walküre sehrenden Blick:
mit ihr mußt du nun ziehn!
 

\Siegmundspeaks
Wo Sieglinde lebt in Lust und Leid,
da will Siegmund auch säumen:
noch machte dein Blick nicht mich erbleichen:
vom Bleiben zwingt er mich nie.
 

\Brunnhildespeaks
Solang du lebst, zwäng' dich wohl nichts:
doch zwingt dich Toren der Tod:
ihn dir zu künden kam ich her.
 

\Siegmundspeaks
Wo wäre der Held, dem heut' ich fiel?
 

\Brunnhildespeaks
Hunding fällt dich im Streit.
 

\Siegmundspeaks
Mit Stärkrem drohe,
als Hundings Streichen!
Lauerst du hier lüstern auf Wal,
jenen kiese zum Fang:
ich denk ihn zu fällen im Kampf!
 

\Brunnhildespeaks

\direct{den Kopf schüttelnd}

Dir, Wälsung, höre mich wohl:
dir ward das Los gekiest.
 

\Siegmundspeaks
Kennst du dies Schwert?
Der mir es schuf, beschied mir Sieg:
deinem Drohen trotz' ich mit ihm!
 

\Brunnhildespeaks

\direct{mit stark erhobener Stimme}

Der dir es schuf, beschied dir jetzt Tod:
seine Tugend nimmt er dem Schwert!
 

\Siegmundspeaks

\direct{heftig}

Schweig, und schrecke die Schlummernde nicht!
 


\direct{Er beugt sich mit hervorbrechendem Schmerze zärtlich über Sieglinde}

Weh! Weh! Süßestes Weib!
Du traurigste aller Getreuen!
Gegen dich wütet in Waffen die Welt:
und ich, dem du einzig vertraut,
für den du ihr einzig getrotzt,
mit meinem Schutz nicht soll ich dich schirmen,
die Kühne verraten im Kampf?
Ha, Schande ihm, der das Schwert mir schuf,
beschied er mir Schimpf für Sieg!
Muß ich denn fallen, nicht fahr' ich nach Walhall:
Hella halte mich fest!
 


\direct{Er neigt sich tief zu Sieglinde}


\Brunnhildespeaks

\direct{erschüttert}

So wenig achtest du ewige Wonne?
 


\direct{zögernd und zurückhaltend}

Alles wär' dir das arme Weib,
das müd' und harmvoll matt von dem Schoße dir hängt?
Nichts sonst hieltest du hehr?
 

\Siegmundspeaks

\direct{bitter zu ihr aufblickend}

So jung und schön erschimmerst du mir:
doch wie kalt und hart erkennt dich mein Herz!
Kannst du nur höhnen, so hebe dich fort,
du arge, fühllose Maid!
Doch mußt du dich weiden an meinem Weh',
mein Leiden letze dich denn;
meine Not labe dein neidvolles Herz:
nur von Walhalls spröden Wonnen
sprich du wahrlich mir nicht!
 

\Brunnhildespeaks
Ich sehe die Not, die das Herz dir zernagt,
ich fühle des Helden heiligen Harm -
Siegmund, befiehl mir dein Weib:
mein Schutz umfange sie fest!
 

\Siegmundspeaks
Kein andrer als ich soll die Reine lebend berühren:
verfiel ich dem Tod, die Betäubte töt' ich zuvor!
 

\Brunnhildespeaks

\direct{in wachsender Ergriffenheit}

Wälsung! Rasender! Hör' meinen Rat:
befiehl mir dein Weib um des Pfandes willen,
das wonnig von dir es empfing!
 

\Siegmundspeaks

\direct{sein Schwert ziehend}

Dies Schwert, das dem Treuen ein Trugvoller schuf;
dies Schwert, das feig vor dem Feind mich verrät:
frommt es nicht gegen den Feind,
so fromm' es denn wider den Freund! -
 


\direct{Er zückt das Schwert auf Sieglinde}

Zwei Leben lachen dir hier:
nimm sie, Notung, neidischer Stahl!
Nimm sie mit einem Streich!
 

\Brunnhildespeaks

\direct{im heftigsten Sturme des Mitgefühls}

Halt' ein Wälsung! Höre mein Wort!
Sieglinde lebe, und Siegmund lebe mit ihr!
Beschlossen ist's; das Schlachtlos wend' ich:
dir, Siegmund, schaff' ich Segen und Sieg!
 


\direct{Man hört aus dem fernen Hintergrunde Hornrufe erschallen}

Hörst du den Ruf? Nun rüste dich, Held!
Traue dem Schwert und schwing' es getrost:
treu hält dir die Wehr,
wie die Walküre treu dich schützt!
Leb' wohl, Siegmund, seligster Held!
Auf der Walstatt seh' ich dich wieder!
 


\direct{Sie stürmt fort und verschwindet mit dem Rosse rechts in einer Seitenschlucht. Siegmund blickt ihr freudig und erhoben nach. Die Bühne hat sich allmählich verfinstert; schwere Gewitterwolken senken sich auf den Hintergrund herab und hüllen die Gebirgswände, die Schlucht und das erhöhte Bergjoch nach und nach gänzlich ein}


\Siegmundspeaks

\direct{neigt sich wieder über Sieglinde, dem Atem lauschend}

Zauberfest bezähmt ein Schlaf
der Holden Schmerz und Harm.
Da die Walküre zu mir trat,
schuf sie ihr den wonnigen Trost?
Sollte die grimmige Wal
nicht schrecken ein gramvolles Weib?
Leblos scheint sie, die dennoch lebt:
der Traurigen kost ein lächelnder Traum. -


\direct{Neue Hornrufe}

So schlummre nun fort,
bis die Schlacht gekämpft,
und Friede dich erfreu'!
 


\direct{Er legt sie sanft auf den Steinsitz und küßt ihr zum Abschied die Stirne. Siegmund vernimmt Hundings Hornruf und bricht entschlossen auf}

Der dort mich ruft, rüste sich nun;
was ihm gebührt, biet' ich ihm:
Notung zahl' ihm den Zoll!
 


\direct{Er zieht das Schwert, eilt dem Hintergrunde zu und verschwindet, auf dem Joche angekommen, sogleich in finstrem Gewittergewölk, aus welchem alsbald Wetterleuchten aufblitzt}


\Sieglindespeaks

\direct{beginnt sich träumend unruhiger zu bewegen}

Kehrte der Vater nur heim!
Mit dem Knaben noch weilt er im Wald.
Mutter! Mutter! Mir bangt der Mut:
nicht freund und friedlich scheinen die Fremden!
Schwarze Dämpfe---schwüles Gedünst---
feurige Lohe leckt schon nach uns---
es brennt das Haus---zu Hilfe, Bruder!
Siegmund! Siegmund!
 


\direct{Sie springt auf. Starker Blitz und Donner}

Siegmund---Ha!
 


\direct{Sie starrt in Angst um sich her: fast die ganze Bühne ist in schwarze Gewitterwolken gehüllt, fortwährender Blitz und Donner. Der Hornruf Hundings ertönt in der Nähe}


\speaker{\Hunding (Stimme)}

\direct{im Hintergrunde vom Bergjoche her}

Wehwalt! Wehwalt!
Steh' mir zum Streit, sollen dich Hunde nicht halten!
 

\speaker{\Siegmund (Stimme)}

\direct{von weiter hinten her aus der Schlucht}

Wo birgst du dich, daß ich vorbei dir schoß?
Steh', daß ich dich stelle!
 

\Sieglindespeaks

\direct{in furchtbarer Aufregung lauschend}

Hunding! Siegmund!
Könnt' ich sie sehen!
 

\Hundingspeaks
Hieher, du frevelnder Freier!
Fricka fälle dich hier!
 

\Siegmundspeaks

\direct{nun ebenfalls vom Joche her}

Noch wähnst du mich waffenlos, feiger Wicht?
Drohst du mit Frauen, so ficht nun selber,
sonst läßt dich Fricka im Stich!
Denn sieh: deines Hauses heimischem Stamm
entzog ich zaglos das Schwert;
seine Schneide schmecke jetzt du!
 


\direct{Ein Blitz erhellt für einen Augenblick das Bergjoch, auf welchem jetzt Hunding und Siegmund kämpfend gewahrt werden}


\Sieglindespeaks

\direct{mit höchster Kraft}

Haltet ein, ihr Männer!
Mordet erst mich!
 


\direct{Sie stürzt auf das Bergjoch zu, ein von rechts her über den Kämpfern ausbrechender, heller Schein blendet sie aber plötzlich so heftig, daß sie, wie erblindet, zur Seite schwankt. In dem Lichtglanze erscheint Brünnhilde über Siegmund schwebend und diesen mit dem Schilde deckend}


\Brunnhildespeaks
Triff ihn, Siegmund!
traue dem Schwert!
 


\direct{Als Siegmund soeben zu einem tödlichen Streiche gegen Hunding ausholt, bricht von links her ein glühend rötlicher Schein durch das Gewölk aus, in welchem Wotan erscheint, über Hunding stehend und seinen Speer Siegmund quer entgegenhaltend}


\Wotanspeaks
Zurück vor dem Speer!
In Stücken das Schwert!
 


\direct{Brünnhilde weicht erschrocken vor Wotan mit dem Schilde zurück; Siegmunds Schwert zerspringt an dem vorgehaltenen Speere. Dem Unbewehrten stößt Hunding seinen Speer in die Brust. Siegmund stürzt tot zu Boden. Sieglinde, die seinen Todesseufzer gehört, sinkt mit einem Schrei wie leblos zusammen. Mit Siegmunds Fall ist zugleich von beiden Seiten der glänzende Schein verschwunden; dichte Finsternis ruht im Gewölk bis nach vorn: in ihm wird Brünnhilde undeutlich sichtbar, wie sie in jäher Hast sich Sieglinden zuwendet.}


\Brunnhildespeaks
Zu Roß, daß ich dich rette!
 


\direct{Sie hebt Sieglinde schnell zu sich auf ihr der Seitenschlucht nahestehendes Roß und verschwindet sogleich mit ihr. Alsbald zerteilt sich das Gewölk in der Mitte, so daß man deutlich Hunding gewahrt, der soeben seinen Speer dem gefallenen Siegmund aus der Brust zieht. Wotan, von Gewölk umgeben, steht dahinter auf einem Felsen, an seinen Speer gelehnt und schmerzlich auf Siegmunds Leiche blickend}


\Wotanspeaks

\direct{zu Hunding}

Geh' hin, Knecht! Kniee vor Fricka:
meld' ihr, daß Wotans Speer
gerächt, was Spott ihr schuf.
Geh'! Geh'!
 


\direct{Vor seinem verächtlichen Handwink sinkt Hunding tot zu Boden}


\Wotanspeaks

\direct{plötzlich in furchtbarer Wut auffahrend}

Doch Brünnhilde! Weh' der Verbrecherin!
Furchtbar sei die Freche gestraft,
erreicht mein Roß ihre Flucht!
 


\direct{Er verschwindet mit Blitz und Donner. Der Vorhang fällt schnell}

   

\act
 
\scene

\StageDir{Auf dem Gipfel eines Felsenberges.

Rechts begrenzt ein Tannenwald die Szene. Links der Eingang einer Felshöhle, die einen natürlichen Saal bildet: darüber steigt der Fels zu seiner höchsten Spitze auf. Nach hinten ist die Aussicht gänzlich frei; höhere und niedere Felssteine bilden den Rand vor dem Abhange, der---wie anzunehmen ist---nach dem Hintergrund zu steil hinabführt. Einzelne Wolkenzüge jagen, wie vom Sturm getrieben, am Felsensaume vorbei.
 


\direct{Gerhilde, Ortlinde, Waltraute und Schwertleite haben sich auf der Felsspitze, an und über der Höhle, gelagert, sie sind in voller Waffenrüstung.}}


\Gerhildespeaks

\direct{zuhöchst gelagert und dem Hintergrunde zurufend, wo ein starkes Gewölk herzieht}

Hojotoho! Hojotoho! Heiaha! Heiaha!
Helmwige! Hier! Hieher mit dem Roß!
 

\speaker{\Helmwige (Stimme)}

\direct{im Hintergrunde}

Hojotoho! Hojotoho! Heiaha!
 


\direct{In dem Gewölk bricht Blitzesglanz aus; eine Walküre zu Roß wird in ihm sichtbar: über ihrem Sattel hängt ein erschlagener Krieger. Die Erscheinung zieht, immer näher, am Felsensaume von links nach rechts vorbei}


\speaker{\Gerhilde, \Waltraute, und \Schwertleite}

\direct{der Ankommenden entgegenrufend}

Heiaha! Heiaha!
 


\direct{Die Wolke mit der Erscheinung ist rechts hinter dem Tann verschwunden}


\Ortlindespeaks

\direct{in den Tann hineinrufend}

Zu Ortlindes Stute stell deinen Hengst:
mit meiner Grauen grast gern dein Brauner!
 

\Waltrautespeaks

\direct{hineinrufend}

Wer hängt dir im Sattel?
 

\Helmwigespeaks

\direct{aus dem Tann auftretend}

Sintolt, der Hegeling!
 

\Schwertleitespeaks
Führ' deinen Brauen fort von der Grauen:
Ortlindes Mähre trägt Wittig, den Irming!
 

\Gerhildespeaks

\direct{ist etwas näher herabgestiegen}

Als Feinde nur sah ich Sintolt und Wittig!
 

\Ortlindespeaks

\direct{springt auf}

Heiaha! Die Stute stößt mir der Hengst!
 


\direct{Sie läuft in den Tann}


\direct{Schwertleite, Gerhilde und Helmwige lachen laut auf}


\Gerhildespeaks
Der Recken Zwist entzweit noch die Rosse!
 

\Helmwigespeaks

\direct{in den Tann zurückrufend}

Ruhig, Brauner!
Brich nicht den Frieden!
 

\Waltrautespeaks

\direct{auf der Höhe, wo sie für Gerhilde die Wacht übernommen, nach rechts in den Hintergrund rufend}

Hoioho! Hoioho!
Siegrune, hier! Wo säumst du so lang?
 


\direct{Sie lauscht nach rechts}


\speaker{\Siegrune (Stimme)}

\direct{von der rechten Seite des Hintergrundes her}

Arbeit gab's!
Sind die andren schon da?
 

\speaker{\Schwertleite und \Waltraute}

\direct{nach rechts in den Hintergrund rufend}

Hojotoho! Hojotoho!
Heiaha!
 

\Gerhildespeaks
Heiaha!
 


\direct{Ihre Gebärden sowie ein heller Glanz hinter dem Tann zeigen an, daß soeben Siegrune dort angelangt ist. Aus der Tiefe hört man zwei Stimmen zugleich}


\speaker{\Grimgerde und \Rossweisse}

\direct{links im Hintergrunde}

Hojotoho! Hojotoho!
Heiaha!
 

\Waltrautespeaks

\direct{nach links}

Grimgerd' und Roßweiße!
 

\Gerhildespeaks

\direct{ebenso}

Sie reiten zu zwei.
 


\direct{In einem blitzerglänzenden Wolkenzuge, der von links her vorbeizieht, erscheinen Grimgerde und Roßweiße, ebenfalls auf Rossen, jede einen Erschlagenen im Sattel führend. Helmwige, Ortlinde und Siegrune sind aus dem Tann getreten und winken vom Felsensaume den Ankommenden zu}


\speaker{\Helmwige, \Ortlinde, und \Siegrune}
Gegrüßt, ihr Reisige!
Roßweiß' und Grimgerde!
 

\speaker{\Rossweisse und \Grimgerde (Stimmen)}
Hojotoho! Hojotoho!
Heiaha!
 


\direct{Die Erscheinung verschwindet hinter dem Tann}


\speaker{Die sechs anderen Walküren}
Hojotoho! Hojotoho! Heiaha! Heiaha!
 

\Gerhildespeaks

\direct{in den Tann rufend}

In Wald mit den Rossen zu Weid' und Rast!
 

\Ortlindespeaks

\direct{ebenfalls in den Tann rufend}

Führet die Mähren fern von einander,
bis unsrer Helden Haß sich gelegt!
 


\direct{Die Walküren lachen}


\Helmwigespeaks

\direct{während die anderen lachen}

Der Helden Grimm büßte schon die Graue!
 


\direct{Die Walküren lachen}


\speaker{\Rossweisse und \Grimgerde}

\direct{aus dem Tann tretend}

Hojotoho! Hojotoho!
 

\speaker{Die sechs anderen Walküren}
Willkommen! Willkommen!
 

\Schwertleitespeaks
Wart ihr Kühnen zu zwei?
 

\Grimgerdespeaks
Getrennt ritten wir und trafen uns heut'.
 

\Rossweissespeaks
Sind wir alle versammelt, so säumt nicht lange:
nach Walhall brechen wir auf,
Wotan zu bringen die Wal.
 

\Helmwigespeaks
Acht sind wir erst: eine noch fehlt.
 

\Gerhildespeaks
Bei dem braunen Wälsung
weilt wohl noch Brünnhilde.
 

\Waltrautespeaks
Auf sie noch harren müssen wir hier:
Walvater gäb' uns grimmigen Gruß,
säh' ohne sie er uns nahn!
 

\Siegrunespeaks

\direct{auf der Felswarte, von wo sie hinausspäht}

Hojotoho! Hojotoho!
 


\direct{in den Hintergrund rufend}

Hieher! Hieher!
 


\direct{zu den anderen}

In brünstigem Ritt
jagt Brünnhilde her.
 

\speaker{Die acht Walküren}

\direct{alle eilen auf die Warte}

Hojotoho! Hojotoho!
Brünnhilde! Hei!
 


\direct{Sie spähen mit wachsender Verwunderung}


\Waltrautespeaks
Nach dem Tann lenkt sie das taumelnde Roß.
 

\Grimgerdespeaks
Wie schnaubt Grane vom schnellen Ritt!
 

\Rossweissespeaks
So jach sah ich nie Walküren jagen!
 

\Ortlindespeaks
Was hält sie im Sattel?
 

\Helmwigespeaks
Das ist kein Held!
 

\Siegrunespeaks
Eine Frau führt sie!
 

\Gerhildespeaks
Wie fand sie die Frau?
 

\Schwertleitespeaks
Mit keinem Gruß grüßt sie die Schwestern!
 

\Waltrautespeaks

\direct{hinabrufend}

Heiaha! Brünnhilde! Hörst du uns nicht?
 

\Ortlindespeaks
Helft der Schwester
vom Roß sich schwingen!
 


\direct{Gerhilde und Helmwige stürzen in den Tann}


\direct{Siegrune und Roßweiße laufen ihnen nach}


\speaker{\Helmwige, \Gerhilde, \Siegrune, und \Rossweisse}
Hojotoho! Hojotoho!
 

\speaker{\Ortlinde, \Waltraute, \Grimgerde, und \Schwertleite}
Heiaha!
 

\Waltrautespeaks

\direct{in den Tann blickend}

Zu Grunde stürzt Grane, der Starke!
 

\Grimgerdespeaks
Aus dem Sattel hebt sie hastig das Weib!
 

\speaker{\Ortlinde, \Waltraute, \Grimgerde, und \Schwertleite}

\direct{alle in den Tann laufend}

Schwester! Schwester! Was ist geschehn?
 


\direct{Alle Walküren kehren auf die Bühne zurück; mit ihnen kommt Brünnhilde, Sieglinde unterstützend und hereingeleitend}


\Brunnhildespeaks

\direct{atemlos}

Schützt mich und helft in höchster Not!
 

\speaker{Die acht Walküren}
Wo rittest du her in rasender Hast?
So fliegt nur, wer auf der Flucht!
 

\Brunnhildespeaks
Zum erstenmal flieh' ich und bin verfolgt:
Heervater hetzt mir nach!
 

\speaker{Die acht Walküren}

\direct{heftig erschreckend}

Bist du von Sinnen? Sprich! Sage uns! Wie?
Verfolgt dich Heervater?
Fliehst du vor ihm?
 

\Brunnhildespeaks

\direct{wendet sich ängstlich, um zu spähen, und kehrt wieder zurück}

O Schwestern, späht von des Felsens Spitze!
Schaut nach Norden, ob Walvater naht!
 


\direct{Ortlinde und Waltraute springen auf die Felsenspitze zur Warte}

Schnell! Seht ihr ihn schon?
 

\Ortlindespeaks
Gewittersturm naht von Norden.
 

\Waltrautespeaks
Starkes Gewölk staut sich dort auf!
 

\speaker{Die Weiteren sechs Walküren}
Heervater reitet sein heiliges Roß!
 

\Brunnhildespeaks
Der wilde Jäger, der wütend mich jagt,
er naht, er naht von Norden!
Schützt mich, Schwestern! Wahret dies Weib!
 

\speaker{Sechs Walküren}
Was ist mit dem Weibe?
 

\Brunnhildespeaks
Hört mich in Eile:
Sieglinde ist es, Siegmunds Schwester und Braut:
gegen die Wälsungen
wütet Wotan in Grimm;
dem Bruder sollte Brünnhilde heut'
entziehen den Sieg;
doch Siegmund schützt' ich mit meinem Schild,
trotzend dem Gott!
Der traf ihn da selbst mit dem Speer:
Siegmund fiel;
doch ich floh fern mit der Frau;
sie zu retten, eilt' ich zu euch---
ob mich Bange auch
 


\direct{kleinmütig}

ihr berget vor dem strafenden Streich!
 

\speaker{Sechs Walküren}

\direct{in größter Bestürzung}

Betörte Schwester, was tatest du?
Wehe! Brünnhilde, wehe!
Brach ungehorsam
Brünnhilde Heervaters heilig Gebot?
 

\Waltrautespeaks

\direct{von der Warte}

Nächtig zieht es von Norden heran.
 

\Ortlindespeaks

\direct{ebenso}

Wütend steuert hieher der Sturm.
 

\speaker{\Rossweisse, \Grimgerde, und \Schwertleite}

\direct{dem Hintergrunde zugewendet}

Wild wiehert Walvaters Roß.
 

\speaker{\Helmwige, \Gerhilde, und \Schwertleite}
Schrecklich schnaubt es daher!
 

\Brunnhildespeaks
Wehe der Armen, wenn Wotan sie trifft:
den Wälsungen allen droht er Verderben!
Wer leiht mir von euch das leichteste Roß,
das flink die Frau ihm entführ'?
 

\Siegrunespeaks
Auch uns rätst du rasenden Trotz?
 

\Brunnhildespeaks
Roßweiße, Schwester,
leih' mir deinen Renner!
 

\Rossweissespeaks
Vor Walvater floh der fliegende nie.
 

\Brunnhildespeaks
Helmwige, höre!
 

\Helmwigespeaks
Dem Vater gehorch' ich.
 

\Brunnhildespeaks
Grimgerde! Gerhilde! Gönnt mir eu'r Roß!
Schwertleite! Siegrune! Seht meine Angst!
Seid mir treu, wie traut ich euch war:
rettet dies traurige Weib!
 

\Sieglindespeaks

\direct{die bisher finster und kalt vor sich hingestarrt, fährt, als Brünnhilde sie lebhaft---wie zum Schutze---umfaßt, mit einer abwehrenden Gebärde auf}

Nicht sehre dich Sorge um mich:
einzig taugt mir der Tod!
Wer hieß dich Maid,
dem Harst mich entführen?
Im Sturm dort hätt' ich den Streich empfah'n
von derselben Waffe, der Siegmund fiel:
das Ende fand ich
vereint mit ihm!
Fern von Siegmund---Siegmund, von dir!---
O deckte mich Tod, daß ich's denke!
Soll um die Flucht
dir, Maid, ich nicht fluchen,
so erhöre heilig mein Flehen:
stoße dein Schwert mir ins Herz!
 

\Brunnhildespeaks
Lebe, o Weib, um der Liebe willen!
Rette das Pfand, das von ihm du empfingst:
 


\direct{stark und drängend}

ein Wälsung wächst dir im Schoß!
 

\Sieglindespeaks

\direct{erschrickt zunächst heftig; sogleich strahlt aber ihr Gesicht in erhabener Freude auf}

Rette mich, Kühne! Rette mein Kind!
Schirmt mich, ihr Mädchen, mit mächtigstem Schutz!
 


\direct{Immer finstereres Gewitter steigt im Hintergrunde auf: nahender Donner}


\Waltrautespeaks

\direct{auf der Warte}

Der Sturm kommt heran.
 

\Ortlindespeaks

\direct{ebenso}

Flieh', wer ihn fürchtet!
 

\speaker{Die sechs anderen Walküren}
Fort mit dem Weibe, droht ihm Gefahr:
der Walküren keine wag' ihren Schutz!
 

\Sieglindespeaks

\direct{auf den Knien vor Brünnhilde}

Rette mich, Maid! Rette die Mutter!
 

\Brunnhildespeaks

\direct{mit lebhaftem Entschluß hebt sie Sieglinde auf}

So fliehe denn eilig und fliehe allein!
Ich bleibe zurück, biete mich Wotans Rache:
an mir zögr' ich den Zürnenden hier,
während du seinem Rasen entrinnst.
 

\Sieglindespeaks
Wohin soll ich mich wenden?
 

\Brunnhildespeaks
Wer von euch Schwestern schweifte nach Osten?
 

\Siegrunespeaks
Nach Osten weithin dehnt sich ein Wald:
der Niblungen Hort entführte Fafner dorthin.
 

\Schwertleitespeaks
Wurmesgestalt schuf sich der Wilde:
in einer Höhle hütet er Alberichs Reif!
 

\Grimgerdespeaks
Nicht geheu'r ist's dort für ein hilflos' Weib.
 

\Brunnhildespeaks
Und doch vor Wotans Wut schützt sie sicher der Wald:
ihn scheut der Mächt'ge und meidet den Ort.
 

\Waltrautespeaks

\direct{auf der Warte}

Furchtbar fährt
dort Wotan zum Fels.
 

\speaker{Sechs Walküren}
Brünnhilde, hör' seines Nahens Gebraus'!
 

\Brunnhildespeaks

\direct{Sieglinde die Richtung weisend}

Fort denn eile, nach Osten gewandt!
Mutigen Trotzes ertrag' alle Müh'n,
Hunger und Durst, Dorn und Gestein;
lache, ob Not, ob Leiden dich nagt!
Denn eines wiss' und wahr' es immer:
den hehrsten Helden der Welt
hegst du, o Weib, im schirmenden Schoß!
 


\direct{Sie zieht die Stücken von Siegmunds Schwert unter ihrem Panzer hervor und überreicht sie Sieglinde}

Verwahr' ihm die starken Schwertesstücken;
seines Vaters Walstatt entführt' ich sie glücklich:
der neugefügt das Schwert einst schwingt,
den Namen nehm' er von mir---
``Siegfried'' erfreu' sich des Siegs!
 

\Sieglindespeaks

\direct{in größter Rührung}

O hehrstes Wunder! Herrlichste Maid!
Dir Treuen dank' ich heiligen Trost!
Für ihn, den wir liebten, rett' ich das Liebste:
meines Dankes Lohn lache dir einst!
Lebe wohl! Dich segnet Sieglindes Weh'!
 


\direct{Sie eilt rechts im Vordergrunde von dannen. Die Felsenhöhe ist von schwarzen Gewitterwolken umlagert; furchtbarer Sturm braust aus dem Hintergrunde daher, wachsender Feuerschein rechts daselbst}


\speaker{\Wotan (Stimme)}
Steh'! Brünnhild'!
 


\direct{Brünnhilde, nachdem sie eine Weile Sieglinde nachgesehen, wendet sich in den Hintergrund, blickt in den Tann und kommt angstvoll wieder vor}


\speaker{\Ortlinde und \Waltraute}

\direct{von der Warte herabsteigend}

Den Fels erreichten Roß und Reiter!
 

\speaker{Alle acht Walküren}
Weh', Brünnhild'! Rache entbrennt!
 

\Brunnhildespeaks
Ach, Schwestern, helft! Mir schwankt das Herz!
Sein Zorn zerschellt mich,
wenn euer Schutz ihn nicht zähmt.
 

\speaker{Die acht Walküren}

\direct{flüchten ängstlich nach der Felsenspitze hinauf; Brünnhilde läßt sich von ihnen nachziehen}

Hieher, Verlor'ne! Laß dich nicht sehn!
Schmiege dich an uns und schweige dem Ruf!
 


\direct{Sie verbergen Brünnhilde unter sich und blicken ängstlich nach dem Tann, der jetzt von grellem Feuerschein erhellt wird, während der Hintergrund ganz finster geworden ist}

Weh'! Wütend schwingt sich Wotan vom Roß!
Hieher rast sein rächender Schritt!

\scene

\StageDir{Wotan tritt in höchster zorniger Aufgeregtheit aus dem Tann auf und schreitet vor der Gruppe der Walküren auf der Höhe, nach Brünnhilde spähend, heftig einher}


\Wotanspeaks
Wo ist Brünnhild', wo die Verbrecherin?
Wagt ihr, die Böse vor mir zu bergen?

\speaker{Die acht Walküren}
Schrecklich ertost dein Toben!
Was taten, Vater, die Töchter,
daß sie dich reizten zu rasender Wut?
 

\Wotanspeaks
Wollt ihr mich höhnen? Hütet euch, Freche!
Ich weiß: Brünnhilde bergt ihr vor mir.
Weichet von ihr, der ewig Verworfnen,
wie ihren Wert von sich sie warf!
 

\Rossweissespeaks
Zu uns floh die Verfolgte.
 

\speaker{Die acht Walküren}
Unsern Schutz flehte sie an!
Mit Furcht und Zagen faßt sie dein Zorn:
für die bange Schwester bitten wir nun,
daß den ersten Zorn du bezähmst.
Laß dich erweichen für sie, zähm deinen Zorn!
 

\Wotanspeaks
Weichherziges Weibergezücht!
So matten Mut gewannt ihr von mir?
Erzog ich euch, kühn zum Kampfe zu zieh'n,
schuf ich die Herzen
euch hart und scharf,
daß ihr Wilden nun weint und greint,
wenn mein Grimm eine Treulose straft?
So wißt denn, Winselnde, was sie verbrach,
um die euch Zagen die Zähre entbrennt:
Keine wie sie
kannte mein innerstes Sinnen;
keine wie sie
wußte den Quell meines Willens!
Sie selbst war
meines Wunsches schaffender Schoß:
und so nun brach sie den seligen Bund,
daß treulos sie meinem Willen getrotzt,
mein herrschend Gebot offen verhöhnt,
gegen mich die Waffe gewandt,
die mein Wunsch allein ihr schuf!
Hörst du's, Brünnhilde? Du, der ich Brünne,
Helm und Wehr, Wonne und Huld,
Namen und Leben verlieh?
Hörst du mich Klage erheben,
und birgst dich bang dem Kläger,
daß feig du der Straf' entflöhst?
 

\Brunnhildespeaks

\direct{tritt aus der Schar der Walküren hervor, schreitet demütigen, doch festen Schrittes von der Felsenspitze herab und tritt so in geringer Entfernung vor Wotan}

Hier bin ich, Vater: gebiete die Strafe!
 

\Wotanspeaks
Nicht straf' ich dich erst:
deine Strafe schufst du dir selbst.
Durch meinen Willen warst du allein:
gegen ihn doch hast du gewollt;
meinen Befehl nur führtest du aus:
gegen ihn doch hast du befohlen;
Wunschmaid warst du mir:
gegen mich doch hast du gewünscht;
Schildmaid warst du mir:
gegen mich doch hobst du den Schild;
Loskieserin warst du mir:
gegen mich doch kiestest du Lose;
Heldenreizerin warst du mir:
gegen mich doch reiztest du Helden.
Was sonst du warst, sagte dir Wotan:
was jetzt du bist, das sage dir selbst!
Wunschmaid bist du nicht mehr;
Walküre bist du gewesen:
nun sei fortan, was so du noch bist!
 

\Brunnhildespeaks

\direct{heftig erschreckend}

Du verstößest mich? Versteh' ich den Sinn?
 

\Wotanspeaks
Nicht send' ich dich mehr aus Walhall;
nicht weis' ich dir mehr Helden zur Wal;
nicht führst du mehr Sieger
in meinen Saal:
bei der Götter trautem Mahle
das Trinkhorn nicht reichst du traulich mir mehr;
nicht kos' ich dir mehr den kindischen Mund;
von göttlicher Schar bist du geschieden,
ausgestoßen aus der Ewigen Stamm;
gebrochen ist unser Bund;
aus meinem Angesicht bist du verbannt.
 

\speaker{Die acht Walküren}

\direct{verlassen, in aufgeregter Bewegung, ihre Stellung, indem sie sich etwas tiefer herabziehen}

Wehe! Weh'!
Schwester, ach Schwester!
 

\Brunnhildespeaks
Nimmst du mir alles, was einst du gabst?
 

\Wotanspeaks
Der dich zwingt, wird dir's entziehn!
Hieher auf den Berg banne ich dich;
in wehrlosen Schlaf schließ' ich dich fest:
der Mann dann fange die Maid,
der am Wege sie findet und weckt.
 

\speaker{Die acht Walküren}

\direct{kommen in höchster Aufregung von der Felsenspitze ganz herab und umgeben in ängstlichen Gruppen Brünnhilde, welche halb kniend vor Wotan liegt}

Halt' ein, o Vater! Halt' ein den Fluch!
Soll die Maid verblühn und verbleichen dem Mann?
Hör unser Fleh'n! Schrecklicher Gott,
wende von ihr die schreiende Schmach!
Wie die Schwester träfe uns selber der Schimpf!
 

\Wotanspeaks
Hörtet ihr nicht, was ich verhängt?
 

Aus eurer Schar ist die treulose Schwester geschieden;
mit euch zu Roß durch die Lüfte nicht reitet sie länger;
die magdliche Blume verblüht der Maid;
ein Gatte gewinnt ihre weibliche Gunst;
dem herrischen Manne gehorcht sie fortan;
am Herde sitzt sie und spinnt,
aller Spottenden Ziel und Spiel.
 


\direct{Brünnhilde sinkt mit einem Schrei zu Boden; die Walküren weichen entsetzt mit heftigem Geräusch von ihrer Seite}


Schreckt euch ihr Los? So flieht die Verlorne!
Weichet von ihr und haltet euch fern!
Wer von euch wagte bei ihr zu weilen,
wer mir zum Trotz
zu der Traurigen hielt'
die Törin teilte ihr Los:
das künd' ich der Kühnen an!
Fort jetzt von hier; meidet den Felsen!
Hurtig jagt mir von hinnen,
sonst erharrt Jammer euch hier!
 

\speaker{Die acht Walküren}

Weh! Weh!
 


\StageDir{Die Walküren fahren mit wildem Wehschrei auseinander und stürzen in hastiger Flucht in den Tann. Schwarzes Gewölk lagert sich dicht am Felsenrande: man hört wildes Geräusch im Tann. Ein greller Blitzesglanz bricht in dem Gewölk aus; in ihm erblickt man die Walküren mit verhängtem Zügel, in eine Schar zusammengedrängt, wild davonjagen. Bald legt sich der Sturm; die Gewitterwolken verziehen sich allmählich. In der folgenden Szene bricht, bei endlich ruhigem Wetter, Abenddämmerung ein, der am Schlusse Nacht folgt.}

\scene

\StageDir{Wotan und Brünnhilde, die noch zu seinen Füßen hingestreckt liegt, sind allein zurückgeblieben. Langes, feierliches Schweigen: unveränderte Stellung}


\Brunnhildespeaks

\direct{beginnt das Haupt langsam ein wenig zu erheben. Schüchtern beginnend und steigernd}

War es so schmählich, was ich verbrach,
daß mein Verbrechen so schmählich du bestrafst?
War es so niedrig, was ich dir tat,
daß du so tief mir Erniedrigung schaffst?
War es so ehrlos, was ich beging,
daß mein Vergehn nun die Ehre mir raubt?


\direct{Sie erhebt sich allmählich bis zur knienden Stellung}

O sag', Vater! Sieh mir ins Auge:
schweige den Zorn, zähme die Wut,
und deute mir hell die dunkle Schuld,
die mit starrem Trotze dich zwingt,
zu verstoßen dein trautestes Kind!
 

\Wotanspeaks

\direct{in unveränderter Stellung, ernst und düster}

Frag' deine Tat, sie deutet dir deine Schuld!
 

\Brunnhildespeaks
Deinen Befehl führte ich aus.
 

\Wotanspeaks
Befahl ich dir, für den Wälsung zu fechten?
 

\Brunnhildespeaks
So hießest du mich als Herrscher der Wal!
 

\Wotanspeaks
Doch meine Weisung nahm ich wieder zurück!
 

\Brunnhildespeaks
Als Fricka den eignen Sinn dir entfremdet;
da ihrem Sinn du dich fügtest,
warst du selber dir Feind.
 

\Wotanspeaks

\direct{leise und bitter}

Daß du mich verstanden, wähnt' ich,
und strafte den wissenden Trotz:
doch feig und dumm dachtest du mich!
So hätt' ich Verrat nicht zu rächen;
zu gering wärst du meinem Grimm?
 

\Brunnhildespeaks
Nicht weise bin ich, doch wußt' ich das eine,
daß den Wälsung du liebtest.
Ich wußte den Zwiespalt, der dich zwang,
dies eine ganz zu vergessen.
Das andre mußtest einzig du sehn,
was zu schaun so herb schmerzte dein Herz:
daß Siegmund Schutz du versagtest.
 

\Wotanspeaks
Du wußtest es so, und wagtest dennoch den Schutz?
 

\Brunnhildespeaks

\direct{leise beginnend}

Weil für dich im Auge das eine ich hielt,
dem, im Zwange des andren
schmerzlich entzweit,
ratlos den Rücken du wandtest!
Die im Kampfe Wotan den Rücken bewacht,
die sah nun das nur, was du nicht sahst:
Siegmund mußt' ich sehn.
Tod kündend trat ich vor ihn,
gewahrte sein Auge, hörte sein Wort;
ich vernahm des Helden heilige Not;
tönend erklang mir des Tapfersten Klage:
freiester Liebe furchtbares Leid,
traurigsten Mutes mächtigster Trotz!
Meinem Ohr erscholl, mein Aug' erschaute,
was tief im Busen das Herz
zu heilgem Beben mir traf.
Scheu und staunend stand ich in Scham.
Ihm nur zu dienen konnt' ich noch denken:
Sieg oder Tod mit Siegmund zu teilen:
dies nur erkannt' ich zu kiesen als Los!
Der diese Liebe mir ins Herz gehaucht,
dem Willen, der dem Wälsung mich gesellt,
ihm innig vertraut, trotzt' ich deinem Gebot.
 

\Wotanspeaks
So tatest du, was so gern zu tun ich begehrt,
doch was nicht zu tun die Not zwiefach mich zwang?
So leicht wähntest du Wonne des Herzens erworben,
wo brennend Weh' in das Herz mir brach,
wo gräßliche Not
den Grimm mir schuf,
einer Welt zuliebe der Liebe Quell
im gequälten Herzen zu hemmen?
Wo gegen mich selber
ich sehrend mich wandte,
aus Ohnmachtschmerzen
schäumend ich aufschoß,
wütender Sehnsucht sengender Wunsch
den schrecklichen Willen mir schuf,
in den Trümmern der eignen Welt
meine ew'ge Trauer zu enden:
da labte süß dich selige Lust;
wonniger Rührung üppigen Rausch
enttrankst du lachend der Liebe Trank,
als mir göttlicher Not nagende Galle gemischt?
Deinen leichten Sinn laß dich denn leiten:
von mir sagtest du dich los.
Dich muß ich meiden,
gemeinsam mit dir
nicht darf ich Rat mehr raunen;
getrennt, nicht dürfen
traut wir mehr schaffen:
so weit Leben und Luft
darf der Gott dir nicht mehr begegnen!
 

\Brunnhildespeaks
Wohl taugte dir nicht die tör'ge Maid,
die staunend im Rate
nicht dich verstand,
wie mein eigner Rat
nur das eine mir riet:
zu lieben, was du geliebt. -
Muß ich denn scheiden und scheu dich meiden,
mußt du spalten, was einst sich umspannt,
die eigne Hälfte fern von dir halten,
daß sonst sie ganz dir gehörte,
du Gott, vergiß das nicht!
Dein ewig Teil nicht wirst du entehren,
Schande nicht wollen, die dich beschimpft:
dich selbst ließest du sinken,
sähst du dem Spott mich zum Spiel!
 

\Wotanspeaks
Du folgtest selig der Liebe Macht:
folge nun dem, den du lieben mußt!
 

\Brunnhildespeaks
Soll ich aus Walhall scheiden,
nicht mehr mit dir schaffen und walten,
dem herrischen Manne gehorchen fortan:
dem feigen Prahler gib mich nicht preis!
Nicht wertlos sei er, der mich gewinnt.
 

\Wotanspeaks
Von Walvater schiedest du -
nicht wählen darf er für dich.
 

\Brunnhildespeaks

\direct{leise mit vertraulicher Heimlichkeit}

Du zeugtest ein edles Geschlecht;
kein Zager kann je ihm entschlagen:
der weihlichste Held---ich weiß es---
entblüht dem Wälsungenstamm.
 

\Wotanspeaks
Schweig' von dem Wälsungenstamm!
Von dir geschieden, schied ich von ihm:
vernichten mußt' ihn der Neid!
 

\Brunnhildespeaks
Die von dir sich riß, rettete ihn.
 


\direct{heimlich}

Sieglinde hegt die heiligste Frucht;
in Schmerz und Leid, wie kein Weib sie gelitten,
wird sie gebären,
was bang sie birgt.
 

\Wotanspeaks
Nie suche bei mir Schutz für die Frau,
noch für ihres Schoßes Frucht!
 

\Brunnhildespeaks

\direct{heimlich}

Sie wahret das Schwert, das du Siegmund schufest.
 

\Wotanspeaks

\direct{heftig}

Und das ich ihm in Stücken schlug!
Nicht streb', o Maid, den Mut mir zu stören;
erwarte dein Los, wie sich's dir wirft;
nicht kiesen kann ich es dir!
Doch fort muß ich jetzt, fern mich verziehn;
zuviel schon zögert' ich hier;
von der Abwendigen wend' ich mich ab;
nicht wissen darf ich, was sie sich wünscht:
die Strafe nur muß vollstreckt ich sehn!
 

\Brunnhildespeaks
Was hast du erdacht, daß ich erdulde?
 

\Wotanspeaks
In festen Schlaf verschließ' ich dich:
wer so die Wehrlose weckt,
dem ward, erwacht, sie zum Weib!
 

\Brunnhildespeaks

\direct{stürzt auf ihre Knie}

Soll fesselnder Schlaf fest mich binden,
dem feigsten Manne zur leichten Beute:
dies eine muß du erhören,
was heil'ge Angst zu dir fleht!
Die Schlafende schütze mit scheuchenden Schrecken,
daß nur ein furchtlos freiester Held
hier auf dem Felsen einst mich fänd'!
 

\Wotanspeaks
Zu viel begehrst du, zu viel der Gunst!
 

\Brunnhildespeaks

\direct{seine Knie umfassend}

Dies eine mußt du erhören!
Zerknicke dein Kind, das dein Knie umfaßt;
zertritt die Traute, zertrümmre die Maid,
ihres Leibes Spur zerstöre dein Speer:
doch gib, Grausamer, nicht
der gräßlichsten Schmach sie preis!

\direct{mit wilder Begeisterung}

Auf dein Gebot entbrenne ein Feuer;
den Felsen umglühe lodernde Glut;
es leck' ihre Zung', es fresse ihr Zahn
den Zagen, der frech sich wagte,
dem freislichen Felsen zu nahn!
 
\Wotanspeaks

\direct{überwältigt und tief ergriffen, wendet sich lebhhaft zu Brünnhilde, erhebt sie von den Knien und blickt ihr gerührt in das Auge}

Leb' wohl, du kühnes, herrliches Kind!
Du meines Herzens heiligster Stolz!
Leb' wohl! Leb' wohl! Leb' wohl!

\direct{sehr leidenschaftlich}

Muß ich dich meiden,
und darf nicht minnig
mein Gruß dich mehr grüßen;
sollst du nun nicht mehr neben mir reiten,
noch Met beim Mahl mir reichen;
muß ich verlieren dich, die ich liebe,
du lachende Lust meines Auges:
ein bräutliches Feuer soll dir nun brennen,
wie nie einer Braut es gebrannt!
Flammende Glut umglühe den Fels;
mit zehrenden Schrecken
scheuch' es den Zagen;
der Feige fliehe Brünnhildes Fels!
Denn einer nur freie die Braut,
der freier als ich, der Gott!
 


\direct{Brünnhilde sinkt, gerührt und begeistert, an Wotans Brust; er hält sie lange umfangen. Sie schlägt das Haupt wieder zurück und blickt, immer noch ihn umfassend, feierlich ergriffen Wotan in das Auge}

Der Augen leuchtendes Paar,
das oft ich lächelnd gekost,
wenn Kampfeslust ein Kuß dir lohnte,
wenn kindisch lallend der Helden Lob
von holden Lippen dir floß:
dieser Augen strahlendes Paar,
das oft im Sturm mir geglänzt,
wenn Hoffnungssehnen das Herz mir sengte,
nach Weltenwonne mein Wunsch verlangte
aus wild webendem Bangen:
zum letztenmal
letz' es mich heut'
mit des Lebewohles letztem Kuß!
Dem glücklichen Manne
glänze sein Stern:
dem unseligen Ew'gen
muß es scheidend sich schließen.
 


\direct{Er faßt ihr Haupt in beide Hände}

Denn so kehrt der Gott sich dir ab,
so küßt er die Gottheit von dir!
 


\direct{Er küßt sie lange auf die Augen. Sie sinkt mit geschlossenen Augen, sanft ermattend, in seinen Armen zurück. Er geleitet sie zart auf einen niedrigen Mooshügel zu liegen, über den sich eine breitästige Tanne ausstreckt. Er betrachtet sie und schließt ihr den Helm: sein Auge weilt dann auf der Gestalt der Schlafenden, die er mit dem großen Stahlschilde der Walküre ganz zudeckt. Langsam kehrt er sich ab, mit einem schmerzlichen Blicke wendet er sich noch einmal um. Dann schreitet er mit feierlichem Entschlusse in die Mitte der Bühne und kehrt seines Speeres Spitze gegen einen mächtigen Felsstein.}

Loge, hör'! Lausche hieher!
Wie zuerst ich dich fand, als feurige Glut,
wie dann einst du mir schwandest,
als schweifende Lohe;
wie ich dich band, bann ich dich heut'!
Herauf, wabernde Lohe,
umlodre mir feurig den Fels!
 


\direct{Er stößt mit dem Folgenden dreimal mit dem Speer auf den Stein}

Loge! Loge! Hieher!
 


\StageDir{Dem Stein entfährt ein Feuerstrahl, der zur allmählich immer helleren Flammenglut anschwillt. Lichte Flackerlohe bricht aus. Lichte Brunst umgibt Wotan mit wildem Flackern. Er weist mit dem Speere gebieterisch dem Feuermeere den Umkreis des Felsenrandes zur Strömung an; alsbald zieht es sich nach dem Hintergrunde, wo es nun fortwährend den Bergsaum umlodert.}

Wer meines Speeres Spitze fürchtet,
durchschreite das Feuer nie!
 

\StageDir{Er streckt den Speer wie zum Banne aus, dann blickt er schmerzlich auf Brünnhilde zurück, wendet sich langsam zum Gehen und blickt noch einmal zurück, ehe er durch das Feuer verschwindet. Der Vorhang fällt.}



\end{drama}


\part{Siegfried}
\setcounter{act}{0}
\renewcommand{\playname}{Siegfried}
\begin{drama}
\act

\scene

\StageDir{Wald.

Den Vordergrund bildet ein Teil einer Felsenhöhle, die sich links tiefer nach innen zieht, nach rechts aber gegen drei Vierteile der Bühne einnimmt. Zwei natürlich gebildete Eingänge stehen dem Walde zu offen: der eine nach rechts, unmittelbar im Hintergrunde, der andere, breitere, ebenda seitwärts. An der Hinterwand, nach links zu, steht ein grosser Schmiedeherd, aus Felsstücken natürlich geformt; künstlich ist nur der grosse Blasebalg: die rohe Esse geht---Lebenfalls natürlich---durch das Felsendach hinauf. Ein sehr grosser Amboss und andre Schmiedegerätschaften.}

\Mimespeaks

\direct{sitzt, als der Vorhang nach einem
kurzen Orchestervorspiel aufgeht,
am Ambosse und hämmert mit
wachsender Unruhe an einem
Schwerte: endlich hält er unmutig ein}

Zwangvolle Plage!
Müh' ohne Zweck!
Das beste Schwert,
das je ich geschweisst,
in der Riesen Fäusten
hielte es fest;
doch dem ich's geschmiedet,
der schmähliche Knabe,
er knickt und schmeisst es entzwei,
als schüf' ich Kindergeschmeid!

\direct{Mime wirft das Schwert unmutig
auf den Amboss, stemmt die Arme
ein und blickt sinnend zu Boden}

Es gibt ein Schwert,
das er nicht zerschwänge:
Notungs Trümmer
zertrotzt' er mir nicht,
könnt' ich die starken
Stücke schweissen,
die meine Kunst
nicht zu kitten weiss!
Könnt' ich's dem Kühnen schmieden,
meiner Schmach erlangt' ich da Lohn!

\direct{Er sinkt tiefer zurück und
neigt sinnend das Haupt}

Fafner, der wilde Wurm,
lagert im finstren Wald;
mit des furchtbaren Leibes Wucht
der Niblungen Hort
hütet er dort.
Siegfrieds kindischer Kraft
erläge wohl Fafners Leib:
des Niblungen Ring
erränge er mir.
Nur ein Schwert taugt zu der Tat;
nur Notung nützt meinem Neid,
wenn Siegfried sehrend ihn schwingt:
und ich kann's nicht schweissen,
Notung, das Schwert!

\direct{Er hat das Schwert wieder
zurechtgelegt und hämmert in
höchstem Unmut daran weiter}

Zwangvolle Plage!
Müh' ohne Zweck!
Das beste Schwert,
das je ich geschweisst,
nie taugt es je
zu der einzigen Tat!
Ich tappre und hämmre nur,
weil der Knabe es heischt:
er knickt und schmeisst es entzwei,
und schmäht doch, schmied' ich ihm nicht!

\direct{Er lässt den Hammer fallen}

\Siegfriedspeaks

\direct{Siegfried, in wilder Waldkleidung, mit einem silbernen Horn
an einer Kette, kommt mit jähem Ungestüm
aus dem Walde herein; er hat einen grossen Bären
mit einen Bastseile gezäumt und treibt diesen
mit lustigem Übermute gegen Mime an}

Hoiho! Hoiho!
Hau' ein! Hau' ein!
Friss ihn! Friss ihn,
Den Fratzenschmied!

\direct{Er lacht unbändig. Mimen entsinkt vor Schreck das Schwert;
er flüchtet hinter den Herd; Siegfried treibt ihm
den Bären überall nach}

\Mimespeaks

Fort mit dem Tier!
Was taugt mir der Bär?

\Siegfriedspeaks

Zu zwei komm ich,
dich besser zu zwicken:
Brauner, frag' nach dem Schwert!

\Mimespeaks

He! Lass das Wild!
Dort liegt die Waffe:
fertig fegt' ich sie heut'.

\Siegfriedspeaks

So fährst du heute noch heil!

\direct{Er löst dem Bären den Zaum
und gibt ihm damit einen
Schlag auf den Rücken}

Lauf', Brauner!
Dich brauch' ich nicht mehr!

\direct{Der Bär läuft in den Wald zurück}

\Mimespeaks:

\direct{kommt zitternd
hinter dem Herde hervor}

Wohl leid' ich's gern,
erlegst du Bären:
was bringst du lebend
die braunen heim?

\Siegfriedspeaks

\direct{setzt sich, um sich vom
Lachen zu erholen}

Nach bessrem Gesellen sucht' ich,
als daheim mir einer sitzt;
im tiefen Walde mein Horn
liess ich hallend da ertönen:
ob sich froh mir gesellte
ein guter Freund,
das frug ich mit dem Getön'!
Aus dem Busche kam ein Bär,
der hörte mir brummend zu;
er gefiel mir besser als du,
doch bessre fänd' ich wohl noch!
Mit dem zähen Baste
zäumt' ich ihn da,
dich, Schelm, nach dem Schwerte zu fragen.

\direct{Er springt auf und geht
auf den Amboss zu}

\Mimespeaks

\direct{nimmt das Schwert auf,
um es Siegfried zu reichen}

Ich schuf die Waffe scharf,
ihrer Schneide wirst du dich freun.

\direct{Er hält das Schwert ängstlich
in der Hand fest, das Siegfried
ihm heftig entwindet}

\Siegfriedspeaks

Was frommt seine helle Schneide,
ist der Stahl nicht hart und fest?

\direct{das Schwert mit der Hand prüfend}

Hei! Was ist das
für müss'ger Tand!
Den schwachen Stift
nennst du ein Schwert?

\direct{Er zerschlägt es auf dem Amboss,
dass die Stücken ringsum fliegen;
Mime weicht erschrocken aus}

Da hast du die Stücken,
schändlicher Stümper:
hätt' ich am Schädel
dir sie zerschlagen!
Soll mich der Prahler
länger noch prellen?
Schwatzt mir von Riesen
und rüstigen Kämpfen,
von kühnen Taten
und tüchtiger Wehr;
will Waffen mir schmieden,
Schwerte schaffen;
rühmt seine Kunst,
als könnt' er was Rechts:
nehm' ich zur Hand nun,
was er gehämmert,
mit einem Griff
zergreif' ich den Quark!
Wär' mir nicht schier
zu schäbig der Wicht,
ich zerschmiedet' ihn selbst
mit seinem Geschmeid,
den alten albernen Alp!
Des Ärgers dann hätt' ich ein End'!

\direct{Siegfried wirft sich wütend auf eine Steinbank
zur Seite rechts. Mime ist ihm immer vorsichtig ausgewichen.}

\Mimespeaks

Nun tobst du wieder wie toll:
dein Undank, traun, ist arg!
Mach' ich dem bösen Buben
nicht alles gleich zu best,
was ich ihm Gutes schuf,
vergisst er gar zu schnell!
Willst du denn nie gedenken,
was ich dich lehrt' vom Danke?
Dem sollst du willig gehorchen,
der je sich wohl dir erwies.

\direct{Siegfried wendet sich unmutig um,
mit dem Gesicht nach der Wand,
so dass er Mime den Rücken kehrt}

Das willst du wieder nicht hören!

\direct{Er steht verlegen; dann geht er in
die Küche am Herd}

Doch speisen magst du wohl?
Vom Spiesse bring' ich den Braten:
versuchtest du gern den Sud?
Für dich sott ich ihn gar.

\direct{Er bietet Siegfried Speise hin;
dieser, ohne sich umzuwenden,
schmeisst ihm Topf und Braten aus der Hand}

\Siegfriedspeaks

Braten briet ich mir selbst:
deinen Sudel sauf' allein!

\Mimespeaks

\direct{stellt sich empfindlich.
Mit kläglich kreischender Stimme}

Das ist nun der Liebe
schlimmer Lohn!
Das der Sorgen
schmählicher Sold!
Als zullendes Kind
zog ich dich auf,
wärmte mit Kleiden
den kleinen Wurm:
Speise und Trank
trug ich dir zu,
hütete dich
wie die eigne Haut.
Und wie du erwuchsest,
wartet' ich dein;
dein Lager schuf ich,
dass leicht du schliefst.
Dir schmiedet' ich Tand
und ein tönend Horn;
dich zu erfreun,
müht' ich mich froh:
mit klugem Rate
riet ich dir klug,
mit lichtem Wissen
lehrt' ich dich Witz.
Sitz' ich daheim
in Fleiss und Schweiss,
nach Herzenslust
schweifst du umher:
für dich nur in Plage,
in Pein nur für dich
verzehr' ich mich alter,
armer Zwerg!

\direct{schluchzend}

Und aller Lasten
ist das nun mein Lohn,
dass der hastige Knabe
mich quält und hasst!

\direct{schluchzend}

\direct{Siegfried hat sich wieder umgewendet und
ruhig in Mimes Blick geforscht. Mime begegnet Siegfrieds Blick
und sucht den seinigen scheu zu bergen}

\Siegfriedspeaks

Vieles lehrtest du, Mime,
und manches lernt' ich von dir;
doch was du am liebsten mich lehrtest,
zu lernen gelang mir nie:
wie ich dich leiden könnt'.
Trägst du mir Trank
und Speise herbei,
der Ekel speist mich allein;
schaffst du ein leichtes
Lager zum Schlaf,
der Schlummer wird mir da schwer;
willst du mich weisen,
witzig zu sein,
gern bleib' ich taub und dumm.
Seh' ich dir erst
mit den Augen zu,
zu übel erkenn' ich,
was alles du tust:
seh' ich dich stehn,
gangeln und gehn,
knicken und nicken,
mit den Augen zwicken:
beim Genick möcht' ich
den Nicker packen,
den Garaus geben
dem garst'gen Zwicker!
So lernt' ich, Mime, dich leiden.
Bist du nun weise,
so hilf mir wissen,
worüber umsonst ich sann:
in den Wald lauf' ich,
dich zu verlassen,
wie kommt das, kehr ich zurück?
Alle Tiere sind
mir teurer als du:
Baum und Vogel,
die Fische im Bach,
lieber mag ich sie
leiden als dich:
wie kommt das nun, kehr' ich zurück?
Bist du klug, so tu mir's kund.

\Mimespeaks

\direct{setzt sich in einiger Entfernung
ihm traulich gegenüber}

Mein Kind, das lehrt dich kennen,
wie lieb ich am Herzen dir lieg'.

\Siegfriedspeaks

\direct{lachend}

Ich kann dich ja nicht leiden,
vergiss das nicht so leicht!

\Mimespeaks

\direct{fährt zurück und setzt sich wieder
abseits, Siegfried gegenüber}

Des ist deine Wildheit schuld,
die du, Böser, bänd'gen sollst.
Jammernd verlangen Junge
nach ihrer Alten Nest;
Liebe ist das Verlangen;
so lechzest du auch nach mir,
so liebst du auch deinen Mime,
so musst du ihn lieben!
Was dem Vögelein ist der Vogel,
wenn er im Nest es nährt
eh' das flügge mag fliegen:
das ist dir kind'schem Spross
der kundig sorgende Mime,
das muss er dir sein!

\Siegfriedspeaks

Ei, Mime, bist du so witzig,
so lass mich eines noch wissen!
Es sangen die Vöglein
so selig im Lenz,
das eine lockte das andre:
du sagtest selbst,
da ich's wissen wollt',
das wären Männchen und Weibchen.
Sie kosten so lieblich,
und liessen sich nicht;
sie bauten ein Nest
und brüteten drin:
da flatterte junges
Geflügel auf,
und beide pflegten der Brut.
So ruhten im Busch
auch Rehe gepaart,
selbst wilde Füchse und Wölfe:
Nahrung brachte
zum Neste das Männchen,
das Weibchen säugte die Welpen.
Da lernt' ich wohl,
was Liebe sei:
der Mutter entwandt' ich
die Welpen nie.
Wo hast du nun, Mime,
dein minniges Weibchen,
dass ich es Mutter nenne?

\Mimespeaks

\direct{ärgerlich}

Was ist dir, Tor?
Ach, bist du dumm!
Bist doch weder Vogel noch Fuchs?

\Siegfriedspeaks

Das zullende Kind
zogest du auf,
wärmtest mit Kleiden
den kleinen Wurm:
wie kam dir aber
der kindische Wurm?
Du machtest wohl gar
ohne Mutter mich?

\Mimespeaks

\direct{in grosser Verlegenheit}

Glauben sollst du,
was ich dir sage:
ich bin dir Vater
und Mutter zugleich.

\Siegfriedspeaks

Das lügst du, garstiger Gauch!
Wie die Jungen den Alten gleichen,
das hab' ich mir glücklich ersehn.
Nun kam ich zum klaren Bach:
da erspäht' ich die Bäum'
und Tier' im Spiegel;
Sonn' und Wolken,
wie sie nur sind,
im Glitzer erschienen sie gleich.
Da sah ich denn auch
mein eigen Bild;
ganz anders als du
dünkt' ich mir da:
so glich wohl der Kröte
ein glänzender Fisch;
doch kroch nie ein Fisch aus der Kröte!

\Mimespeaks

\direct{höchst ärgerlich}

Gräulichen Unsinn
kramst du da aus!

\Siegfriedspeaks

\direct{immer lebendiger}

Siehst du, nun fällt
auch selbst mir ein,
was zuvor umsonst ich besann:
wenn zum Wald ich laufe,
dich zu verlassen,
wie das kommt, kehr' ich doch heim?

\direct{er springt auf}

Von dir erst muss ich erfahren,
wer Vater und Mutter mir sei!

\Mimespeaks

\direct{weicht ihm aus}

Was Vater! Was Mutter!
Müssige Frage!

\Siegfriedspeaks

\direct{packt ihn bei der Kehle}

So muss ich dich fassen,
um was zu wissen:
gutwillig
erfahr' ich doch nichts!
So musst' ich alles
ab dir trotzen:
kaum das Reden
hätt' ich erraten,
entwandt ich's mit Gewalt
nicht dem Schuft!
Heraus damit,
räudiger Kerl!
Wer ist mir Vater und Mutter?

\Mimespeaks

\direct{nachdem er mit dem Kopfe genickt
und mit den Händen gewinkt, ist von
Siegfried losgelassen worden}

Ans Leben gehst du mir schier!
Nun lass! Was zu wissen dich geizt,
erfahr' es, ganz wie ich's weiss.
O undankbares,
arges Kind!
Jetzt hör', wofür du mich hassest!
Nicht bin ich Vater
noch Vetter dir,
und dennoch verdankst du mir dich!
Ganz fremd bist du mir,
dem einzigen Freund;
aus Erbarmen allein
barg ich dich hier:
nun hab' ich lieblichen Lohn!
Was verhofft' ich Thor mir auch Dank?
Einst lag wimmernd ein Weib
da draussen im wilden Wald:
zur Höhle half ich ihr her,
am warmen Herd sie zu hüten.
Ein Kind trug sie im Schosse;
traurig gebar sie's hier;
sie wand sich hin und her,
ich half, so gut ich konnt'.
Gross war die Not! Sie starb,
doch Siegfried, der genas.

\Siegfriedspeaks

\direct{Siegfried steht sinnend}

So starb meine Mutter an mir?

\Mimespeaks

Meinem Schutz übergab sie dich:
ich schenkt' ihn gern dem Kind.
Was hat sich Mime gemüht,
was gab sich der Gute für Not!
``Als zullendes Kind
zog ich dich auf...''

\Siegfriedspeaks

Mich dünkt, des gedachtest du schon!
Jetzt sag': woher heiss' ich Siegfried?

MIME:
So hiess mich die Mutter,
möcht' ich dich heissen:
als ``Siegfried'' würdest
du stark und schön.
``Ich wärmte mit Kleiden
den kleinen Wurm....''

\Siegfriedspeaks

Nun melde, wie hiess meine Mutter?

\Mimespeaks

Das weiss ich wahrlich kaum!
``Speise und Trank
trug ich dir zu....''

\Siegfriedspeaks

Den Namen sollst du mir nennen!

\Mimespeaks

Entfiel er mir wohl? Doch halt!
Sieglinde mochte sie heissen,
die dich in Sorge mir gab.
``Ich hütete dich
wie die eigne Haut....''

\Siegfriedspeaks

\direct{immer dringender}

Dann frag' ich, wie hiess mein Vater?

\Mimespeaks

\direct{barsch}

Den hab' ich nie gesehn.

\Siegfriedspeaks

Doch die Mutter nannte den Namen?

\Mimespeaks

Erschlagen sei er,
das sagte sie nur;
dich Vaterlosen
befahl sie mir da:
``und wie du erwuchsest,
wartet' ich dein;
dein Lager schuf ich,
dass leicht du schliefst....''

\Siegfriedspeaks

Still mit dem alten
Starenlied!
Soll ich der Kunde glauben,
hast du mir nichts gelogen,
so lass mich Zeichen sehn!

\Mimespeaks

Was soll dir's noch bezeugen?

\Siegfriedspeaks

Dir glaub' ich nicht mit dem Ohr',
dir glaub' ich nur mit dem Aug':
welch Zeichen zeugt für dich?

\Mimespeaks

\direct{holt nach einigem Besinnen
die zwei Stücke eines zerschlagenen
Schwerts herbei}

Das gab mir deine Mutter:
für Mühe, Kost und Pflege
liess sie's als schwachen Lohn.
Sieh' her, ein zerbrochnes Schwert!
Dein Vater, sagte sie, führt' es,
als im letzten Kampf er erlag.

\Siegfriedspeaks

\direct{begeistert}

Und diese Stücke
sollst du mir schmieden:
dann schwing' ich ein rechtes Schwert!
Auf! Eile dich, Mime!
Mühe dich rasch;
kannst du was Rechts,
nun zeig' deine Kunst!
Täusche mich nicht
mit schlechtem Tand:
den Trümmern allein
trau' ich was zu!
Find' ich dich faul,
fügst du sie schlecht,
flickst du mit Flausen
den festen Stahl,
dir Feigem fahr' ich zu Leib',
das Fegen lernst du von mir!
Denn heute noch, schwör' ich,
will ich das Schwert;
die Waffe gewinn' ich noch heut'!

\Mimespeaks

\direct{erschrocken}

Was willst du noch heut' mit dem Schwert?

\Siegfriedspeaks

Aus dem Wald fort
in die Welt ziehn:
nimmer kehr' ich zurück!
Wie ich froh bin,
dass ich frei ward,
nichts mich bindet und zwingt!
Mein Vater bist du nicht;
in der Ferne bin ich heim;
dein Herd ist nicht mein Haus,
meine Decke nicht dein Dach.
Wie der Fisch froh
in der Flut schwimmt,
wie der Fink frei
sich davon schwingt:
flieg' ich von hier,
flute davon,
wie der Wind übern Wald
weh' ich dahin,
dich, Mime, nie wieder zu sehn!

\direct{Er stürmt in den Wald fort}

\Mimespeaks

\direct{in höchster Angst}

Halte! Halte! Wohin?


\direct{Er ruft mit der grössten
Anstrengung in den Wald}

He! Siegfried!
Siegfried! He!

\direct{Er sieht dem Fortstürmenden
eine Weile staunend nach; dann
kehrt er in die Schmiede zurück
und setzt sich hinter den Amboss}

Da stürmt er hin!
Nun sitz' ich da:
zur alten Not
hab' ich die neue;
vernagelt bin ich nun ganz!
Wie helf' ich mir jetzt?
Wie halt' ich ihn fest?
Wie führ' ich den Huien
zu Fafners Nest?
Wie füg' ich die Stücken
des tückischen Stahls?
Keines Ofens Glut
glüht mir die echten;
keines Zwergen Hammer
zwingt mir die harten.
Des Niblungen Neid,
Not und Schweiss
nietet mir Notung nicht,
schweisst mir das Schwert nicht zu ganz!

\direct{Mime knickt verzweifelnd auf dem Schemel
hinter dem Amboss zusammen}

\scene

\StageDir{Der Wanderer [Wotan] tritt aus dem Wald an das hintere Tor der Höhle heran. Er trägt einen dunkelblauen, langen Mantel; einen Speer führt er als Stab. Auf dem Haupte hat er einen grossen Hut mit breiter runder Krämpe, die über das fehlende eine Auge tief hereinhängt.}

\Wandererspeaks

Heil dir, weiser Schmied!
Dem wegmüden Gast
gönne hold
des Hauses Herd!

\Mimespeaks

\direct{ist erschrocken aufgefahren}

Wer ist's, der im wilden
Walde mich sucht?
Wer verfolgt mich im öden Forst?

\Wandererspeaks

\direct{sehr langsam, immer nur einen
Schritt sich nähernd}

``Wand'rer'' heisst mich die Welt;
weit wandert' ich schon:
auf der Erde Rücken
rührt' ich mich viel!

\Mimespeaks

So rühre dich fort
und raste nicht hier,
heisst dich ``Wand'rer'' die Welt!

\Wandererspeaks

Gastlich ruht' ich bei Guten,
Gaben gönnten viele mir:
denn Unheil fürchtet,
wer unhold ist.

\Mimespeaks

Unheil wohnte
immer bei mir:
willst du dem Armen es mehren?

\Wandererspeaks

\direct{langsam immer näherschreitend}

Viel erforscht' ich,
erkannte viel:
Wicht'ges konnt' ich
manchem künden,
manchem wehren,
was ihn mühte:
nagende Herzensnot.

\Mimespeaks

Spürtest du klug
und erspähtest du viel,
hier brauch' ich nicht Spürer noch Späher.
Einsam will ich
und einzeln sein,
Lungerern lass' ich den Lauf.

\Wandererspeaks

\direct{tritt wieder etwas näher}

Mancher wähnte
weise zu sein,
nur was ihm not tat,
wusste er nicht;
was ihm frommte,
liess ich erfragen:
lohnend lehrt' ihn mein Wort.

\Mimespeaks

\direct{immer ängstlicher, da er
den Wanderer sich nahen sieht}

Müss'ges Wissen
wahren manche:
ich weiss mir grade genug;

\direct{Der Wanderer schreitet
vollends bis an den Herd vor}

mir genügt mein Witz,
ich will nicht mehr:
dir Weisem weis' ich den Weg!

\Wandererspeaks

\direct{am Herd sich setzend}

Hier sitz' ich am Herd
und setze mein Haupt
der Wissenswette zum Pfand:
mein Kopf ist dein,
du hast ihn erkiest,
entfrägst du dir nicht,
was dir frommt,
lös' ich's mit Lehren nicht ein.

\Mimespeaks

\direct{der zuletzt den Wanderer mit
offenem Munde angestaunt hat,
schrickt jetzt zusammen;
kleinmütig für sich}

Wie werd' ich den Lauernden los?
Verfänglich muss ich ihn fragen.

\direct{Er ermannt sich wie zu Strenge}

Dein Haupt pfänd' ich
für den Herd:
nun sorg', es sinnig zu lösen!
Drei der Fragen
stell' ich mir frei.

\Wandererspeaks

Dreimal muss ich's treffen.

\Mimespeaks

\direct{sammelt sich zum Nachdenken}

Du rührtest dich viel
auf der Erde Rücken,
die Welt durchwandert'st du weit;
nun sage mir schlau:
welches Geschlecht
tagt in der Erde Tiefe?

\Wandererspeaks

In der Erde Tiefe
tagen die Nibelungen:
Nibelheim ist ihr Land.
Schwarzalben sind sie;
Schwarz-Alberich
hütet' als Herrscher sie einst!
Eines Zauberringes
zwingende Kraft
zähmt' ihm das fleissige Volk.
Reicher Schätze
schimmernden Hort
häuften sie ihm:
der sollte die Welt ihm gewinnen.
Zum zweiten was frägst du, Zwerg?

\Mimespeaks

\direct{versinkt in immer tieferes Nachsinnen}

Viel, Wanderer,
weisst du mir
aus der Erde Nabelnest;
nun sage mir schlicht,
welches Geschlecht
ruht auf der Erde Rücken?

\Wandererspeaks

Auf der Erde Rücken
wuchtet der Riesen Geschlecht:
Riesenheim ist ihr Land.
Fasolt und Fafner,
der Rauhen Fürsten,
neideten Nibelungs Macht;
den gewaltigen Hort
gewannen sie sich,
errangen mit ihm den Ring.
Um den entbrannte
den Brüdern Streit;
der Fasolt fällte,
als wilder Wurm
hütet nun Fafner den Hort.
Die dritte Frage nun droht.

\Mimespeaks

\direct{der ganz in Träumerei
entrückt ist}

Viel, Wanderer,
weisst du mir
von der Erde rauhem Rücken.
Nun sage mir wahr,
welches Geschlecht
wohnt auf wolkigen Höh'n?

\Wandererspeaks

Auf wolkigen Höhn
wohnen die Götter:
Walhall heisst ihr Saal.
Lichtalben sind sie;
Licht-Alberich,
Wotan, waltet der Schar.
Aus der Welt-Esche
weihlichstem Aste
schuf er sich einen Schaft:
dorrt der Stamm,
nie verdirbt doch der Speer;
mit seiner Spitze
sperrt Wotan die Welt.
Heil'ger Verträge
Treuerunen
schnitt in den Schaft er ein.
Den Haft der Welt
hält in der Hand,
wer den Speer führt,
den Wotans Faust umspannt.
Ihm neigte sich
der Niblungen Heer;
der Riesen Gezücht
zähmte sein Rat:
ewig gehorchen sie alle
des Speeres starkem Herrn.

\direct{Er stösst wie unwillkürlich mit
dem Speer auf den Boden;
ein leiser Donner lässt sich
vernehmen, wovon Mime
heftig erschrickt}

Nun rede, weiser Zwerg:
wusst' ich der Fragen Rat?
Behalte mein Haupt ich frei?

\Mimespeaks

\direct{nachdem er den Wanderer mit
dem Speer aufmerksam beobachtet
hat, gerät nun in grosse Angst,
sucht verwirrt nach seinen
Gerätschaften und blickt scheu
zur Seite}

Fragen und Haupt
hast du gelöst:
nun, Wand'rer, geh' deines Wegs!

\Wandererspeaks

Was zu wissen dir frommt,
solltest du fragen:
Kunde verbürgte mein Kopf.
Dass du nun nicht weisst,
was dir nützt,
des fass' ich jetzt deines als Pfand.
Gastlich nicht
galt mir dein Gruss,
mein Haupt gab ich
in deine Hand,
um mich des Herdes zu freun.
Nach Wettens Pflicht
pfänd' ich nun dich,
lösest du drei
der Fragen nicht leicht.
Drum frische dir, Mime, den Muth!

\Mimespeaks

\direct{sehr schüchtern und zögernd, endlich
in furchtsamer Ergebung sich fassend}

Lang' schon mied ich
mein Heimatland,
lang' schon schied ich
aus der Mutter Schoss;
mir leuchtete Wotans Auge,
zur Höhle lugt' es herein:
vor ihm magert
mein Mutterwitz.
Doch frommt mir's nun weise zu sein,
Wand'rer, frage denn zu!
Vielleicht glückt mir's, gezwungen
zu lösen des Zwerges Haupt.

\Wandererspeaks

\direct{wieder gemächlich sich niederlassend}

Nun, ehrlicher Zwerg,
sag' mir zum ersten:
welches ist das Geschlecht,
dem Wotan schlimm sich zeigte
und das doch das liebste ihm lebt?

\Mimespeaks

\direct{sich ermunternd}

Wenig hört' ich
von Heldensippen;
der Frage doch mach' ich mich frei.
Die Wälsungen sind
das Wunschgeschlecht,
das Wotan zeugte
und zärtlich liebte,
zeigt' er auch Ungunst ihm.
Siegmund und Sieglind'
stammten von Wälse,
ein wild-verzweifeltes
Zwillingspaar:
Siegfried zeugten sie selbst,
den stärksten Wälsungenspross.
Behalt' ich, Wand'rer,
zum ersten mein Haupt?

\Wandererspeaks

\direct{gemütlich}

Wie doch genau
das Geschlecht du mir nennst:
schlau eracht' ich dich Argen!
Der ersten Frage
wardst du frei.
Zum zweiten nun sag' mir, Zwerg:
ein weiser Niblung
wahret Siegfried;
Fafner soll er ihm fällen,
dass den Ring er erränge,
des Hortes Herrscher zu sein.
Welches Schwert
muss Siegfried nun schwingen,
taug' es zu Fafners Tod?

\Mimespeaks

\direct{seine gegenwärtige Lage
immer mehr vergessend und von
dem Gegenstande lebhaft angezogen,
reibt sich vergnügt die Hände}

Notung heisst
ein neidliches Schwert;
in einer Esche Stamm
stiess es Wotan:
dem sollt' es geziemen,
der aus dem Stamm es zög'.
Der stärksten Helden
keiner bestand's:
Siegmund, der Kühne,
konnt's allein:
fechtend führt' er's im Streit,
bis an Wotans Speer es zersprang.
Nun verwahrt die Stücken
ein weiser Schmied;
denn er weiss, dass allein
mit dem Wotansschwert
ein kühnes dummes Kind,
Siegfried, den Wurm versehrt.

\direct{ganz vergnügt}

Behalt' ich Zwerg
auch zweitens mein Haupt?

\Wandererspeaks

\direct{lachend}

Der witzigste bist du
unter den Weisen:
wer käm' dir an Klugheit gleich?
Doch bist du so klug,
den kindischen Helden
für Zwergenzwecke zu nützen,
mit der dritten Frage
droh' ich nun!
Sag' mir, du weiser
Waffenschmied:
wer wird aus den starken Stücken
Notung, das Schwert, wohl schweissen?

\Mimespeaks

\direct{fährt im höchsten Schrecken auf}

Die Stücken! Das Schwert!
O weh! Mir schwindelt!
Was fang' ich an?
Was fällt mir ein?
Verfluchter Stahl,
dass ich dich gestohlen!
Er hat mich vernagelt
in Pein und Not!
Mir bleibt er hart,
ich kann ihn nicht hämmern:
Niet' und Löte
lässt mich im Stich!

\direct{Er wirft wie sinnlos sein Gerät
durcheinander und bricht in helle
Verzweiflung aus}

Der weiseste Schmied
weiss sich nicht Rat!
Wer schweisst nun das Schwert,
schaff' ich es nicht?
Das Wunder, wie soll ich's wissen?

\Wandererspeaks

\direct{ist ruhig vom Herd aufgestanden}

Dreimal solltest du fragen,
dreimal stand ich dir frei:
nach eitlen Fernen
forschtest du;
doch was zunächst dir sich fand,
was dir nützt, fiel dir nicht ein.
Nun ich's errate,
wirst du verrückt:
gewonnen hab' ich
das witzige Haupt!
Jetzt, Fafners kühner Bezwinger,
hör', verfall'ner Zwerg:
``Nur wer das Fürchten
nie erfuhr,
schmiedet Notung neu.''

\direct{Mime starrt ihn gross an:
er wendet sich zum Fortgange}

Dein weises Haupt
wahre von heut':
verfallen lass' ich es dem,
der das Fürchten nicht gelernt!

\direct{Er wendet sich lächelnd ab und verschwindet schnell im Walde. Mime ist wie vernichtet auf den Schemel hinter dem Amboss zurückgesunken}

\scene

\Mimespeaks

\direct{starrt grad vor sich aus in den sonnig
beleuchteten Wald hinein und gerät
zunehmend in heftiges Zittern}

Verfluchtes Licht!
Was flammt dort die Luft?
Was flackert und lackert,
was flimmert und schwirrt,
was schwebt dort und webt
und wabert umher?
Da glimmert's und glitzt's
in der Sonne Glut!
Was säuselt und summt
und saust nun gar?
Es brummt und braust
und prasselt hieher!
Dort bricht's durch den Wald,
will auf mich zu!

\direct{Er bäumt sich vor Entsetzen auf}

Ein grässlicher Rachen
reisst sich mir auf:
der Wurm will mich fangen!
Fafner! Fafner!

\direct{Er sinkt laut schreiend hinter
dem breiten Amboss zusammen}

\Siegfriedspeaks

\direct{bricht aus dem Waldgesträuch
hervor und ruft noch hinter der
Szene, während man seine Bewegung
an dem zerkrachenden Gezweige
des Gesträuches gewahrt}

Heda! Du Fauler!
Bist du nun fertig!

\direct{Er tritt in die Höhle herein
und hält verwundert an}

Schnell! Wie steht's mit dem Schwert?
Wo steckt der Schmied?
Stahl er sich fort?
Hehe! Mime, du Memme!
Wo bist du? Wo birgst du dich?

\Mimespeaks

\direct{mit schwacher Stimme hinter
dem Amboss}

Bist du es, Kind?
Kommst du allein?

\Siegfriedspeaks

\direct{lachend}

Hinter dem Amboss?
Sag', was schufest du dort?
Schärftest du mir das Schwert?

\Mimespeaks

\direct{höchst verstört und
zerstreut hervorkommend}

Das Schwert? Das Schwert?
Wie möcht' ich's schweissen?

\direct{halb für sich}

``Nur wer das Fürchten
nie erfuhr,
schmiedet Notung neu.''
Zu weise ward ich
für solches Werk!

\Siegfriedspeaks

\direct{heftig}

Wirst du mir reden?
Soll ich dir raten?

\Mimespeaks

\direct{wie zuvor}

Wo nähm' ich redlichen Rat?
Mein weises Haupt
hab' ich verwettet:

\direct{vor sich hin starrend}

verfallen, verlor ich's an den,
``der das Fürchten nicht gelernt''.

\Siegfriedspeaks

\direct{ungestüm}

Sind mir das Flausen?
Willst du mir fliehn?

\Mimespeaks

\direct{allmählich sich etwas fassend}

Wohl flöh' ich dem,
der's Fürchten kennt!
Doch das liess ich dem Kinde zu lehren!
Ich Dummer vergass,
was einzig gut:
Liebe zu mir
sollt' er lernen;
das gelang nun leider faul!
Wie bring' ich das Fürchten ihm bei?

\Siegfriedspeaks

\direct{packt ihn}

He! Muss ich helfen?
Was fegtest du heut'?

\Mimespeaks

Um dich nur besorgt,
versank ich in Sinnen,
wie ich dich Wichtiges wiese.

\Siegfriedspeaks

\direct{lachend}

Bis unter den Sitz
warst du versunken:
was Wichtiges fandest du da?

\Mimespeaks

\direct{sich immer mehr fassend}

Das Fürchten lernt' ich für dich,
dass ich's dich Dummen lehre.

\Siegfriedspeaks

\direct{mit ruhiger Verwunderung}

Was ist's mit dem Fürchten?

\Mimespeaks

Erfuhrst du's noch nie
und willst aus dem Wald
doch fort in die Welt?
Was frommte das festeste Schwert,
blieb dir das Fürchten fern?

\Siegfriedspeaks

\direct{ungeduldig}

Faulen Rat
erfindest du wohl?

\Mimespeaks

\direct{immer zutraulicher Siegfried
näher tretend}

Deiner Mutter Rat
redet aus mir;
was ich gelobte,
muss ich nun lösen:
in die listige Welt
dich nicht zu entlassen,
eh' du nicht das Fürchten gelernt.

\Siegfriedspeaks

\direct{heftig}

Ist's eine Kunst,
was kenn' ich sie nicht?
Heraus! Was ist's mit dem Fürchten?

\Mimespeaks

Fühltest du nie
im finstren Wald,
bei Dämmerschein
am dunklen Ort,
wenn fern es säuselt,
summt und saust,
wildes Brummen
näher braust,
wirres Flackern
um dich flimmert,
schwellend Schwirren
zu Leib dir schwebt:
fühltest du dann nicht grieselnd
Grausen die Glieder dir fahen?
Glühender Schauer
schüttelt die Glieder,
in der Brust bebend und bang
berstet hämmernd das Herz?
Fühltest du das noch nicht,
das Fürchten blieb dir dann fremd.

\Siegfriedspeaks

\direct{nachsinnend}

Sonderlich seltsam
muss das sein!
Hart und fest,
fühl' ich, steht mir das Herz.
Das Grieseln und Grausen,
das Glühen und Schauern,
Hitzen und Schwindeln,
Hämmern und Beben:
gern begehr' ich das Bangen,
sehnend verlangt mich's der Lust!
Doch wie bringst du,
Mime, mir's bei?
Wie wärst du, Memme, mir Meister?

\Mimespeaks

Folge mir nur,
ich führe dich wohl:
sinnend fand ich es aus.
Ich weiss einen schlimmen Wurm,
der würgt' und schlang schon viel:
Fafner lehrt dich das Fürchten,
folgst du mir zu seinem Nest.

\Siegfriedspeaks

Wo liegt er im Nest?

\Mimespeaks

Neidhöhle
wird es genannt:
im Ost, am Ende des Walds.

\Siegfriedspeaks

Dann wär's nicht weit von der Welt?

\Mimespeaks

Bei Neidhöhle liegt sie ganz nah.

\Siegfriedspeaks

Dahin denn sollst du mich führen:
lernt' ich das Fürchten,
dann fort in die Welt!
Drum schnell! Schaffe das Schwert,
in der Welt will ich es schwingen.

\Mimespeaks

Das Schwert? O Not!

\Siegfriedspeaks

Rasch in die Schmiede!
Weis', was du schufst!

\Mimespeaks

Verfluchter Stahl!
Zu flicken versteh' ich ihn nicht:
den zähen Zauber
bezwingt keines Zwergen Kraft.
Wer das Fürchten nicht kennt,
der fänd' wohl eher die Kunst.

\Siegfriedspeaks

Feine Finten
weiss mir der Faule;
dass er ein Stümper,
sollt' er gestehn:
nun lügt er sich listig heraus!
Her mit den Stücken,
fort mit dem Stümper!

\direct{auf den Herd zuschreitend}

Des Vaters Stahl
fügt sich wohl mir:
ich selbst schweisse das Schwert!

\direct{Er macht sich, Mimes Gerät
durcheinander werfend,
mit Ungestüm an die Arbeit}

\Mimespeaks

Hättest du fleissig
die Kunst gepflegt,
jetzt käm' dir's wahrlich zugut;
doch lässig warst du
stets in der Lehr':
was willst du Rechtes nun rüsten?

\Siegfriedspeaks

Was der Meister nicht kann,
vermöcht' es der Knabe,
hätt' er ihm immer gehorcht?

\direct{Er dreht ihm eine Nase}

Jetzt mach' dich fort,
misch' dich nicht drein:
sonst fällst du mir mit ins Feuer!

\direct{Er hat eine grosse Menge Kohlen
auf dem Herd aufgehäuft und unterhält
in einem fort die Glut, während er die
Schwertstücke in den Schraubstock
einspannt und sie zu Spänen zerfeilt}

\Mimespeaks

\direct{der sich etwas abseits niedergesetzt
hat, sieht Siegfried bei der Arbeit zu}

Was machst du denn da?
Nimm doch die Löte:
den Brei braut' ich schon längst.

\Siegfriedspeaks

Fort mit dem Brei!
Ich brauch' ihn nicht:
Mit Bappe back' ich kein Schwert!

\Mimespeaks

Du zerfeilst die Feile,
zerreibst die Raspel:
wie willst du den Stahl zerstampfen?

\Siegfriedspeaks

Zersponnen muss ich
in Späne ihn sehn:
was entzwei ist, zwing' ich mir so.

\direct{Er feilt mit grossem Eifer fort}

\Mimespeaks

\direct{für sich}

Hier hilft kein Kluger,
das seh' ich klar:
hier hilft dem Dummen
die Dummheit allein!
Wie er sich rührt
und mächtig regt!
lhm schwindet der Stahl,
doch wird ihm nicht schwül!

\direct{Siegfried hat das Herdfeuer
zur hellsten Glut angefacht}

Nun ward ich so alt
wie Höhl' und Wald,
und hab' nicht so was geseh'n!

\direct{Während Siegfried mit ungestümem
Eifer fortfährt, die Schwertstücken
zu zerfeilen, setzt sich Mime noch
mehr beiseite}

Mit dem Schwert gelingt's,
das lern' ich wohl:
furchtlos fegt er's zu ganz.
Der Wand'rer wusst' es gut!
Wie berg' ich nun
mein banges Haupt?
Dem kühnen Knaben verfiel's,
lehrt' ihn nicht Fafner die Furcht!

\direct{mit wachsender Unruhe
aufspringend und sich beugend}

Doch weh' mir Armen!
Wie würgt' er den Wurm,
erführ' er das Fürchten von ihm?
Wie erräng' er mir den Ring?
Verfluchte Klemme!
Da klebt' ich fest,
fänd' ich nicht klugen Rat,
wie den Furchtlosen selbst ich bezwäng'.

\Siegfriedspeaks

\direct{hat nun die Stücken zerfeilt und
in einem Schmelztiegel gefangen,
den er jetzt in die Herdglut stellt}

He, Mime! Geschwind!
Wie heisst das Schwert,
das ich in Späne zersponnen?

\Mimespeaks

\direct{fährt zusammen und wendet sich
zu Siegfried}

Notung nennt sich
das neidliche Schwert:
deine Mutter gab mir die Mär.

\Siegfriedspeaks

\direct{nährt unter dem folgenden
die Glut mit dem Blasebalg}

Notung! Notung!
Neidliches Schwert!
Was musstest du zerspringen?
Zu Spreu nun schuf ich
die scharfe Pracht,
im Tiegel brat' ich die Späne.
Hoho! Hoho!
Hohei! Hohei!
Blase, Balg!
Blase die Glut!
Wild im Walde
wuchs ein Baum,
den hab' ich im Forst gefällt:
die braune Esche
brannt' ich zur Kohl',
auf dem Herd nun liegt sie gehäuft.
Hoho! Hoho!
Hohei! Hohei!
Blase, Balg!
Blase die Glut!
Des Baumes Kohle,
wie brennt sie kühn;
wie glüht sie hell und hehr!
In springenden Funken
sprühet sie auf:
Hohei! Hohei! Hohei!
Zerschmilzt mir des Stahles Spreu.
Hoho! Hoho!
Hohei! Hoho!
Blase, Balg!
Blase die Glut!

\Mimespeaks

\direct{immer für sich, entfernt sitzend}

Er schmiedet das Schwert,
und Fafner fällt er:
das seh' ich nun sicher voraus.
Hort und Ring
erringt er im Harst:
wie erwerb' ich mir den Gewinn?
Mit Witz und List
erlang' ich beides
und berge heil mein Haupt.

\Siegfriedspeaks

\direct{nochmals am Blasebalg}

Hoho! Hoho!
Hohei! Hohei!

\Mimespeaks

\direct{im Vordergrunde für sich}

Rang er sich müd mit dem Wurm,
von der Müh' erlab' ihn ein Trunk:
aus würz'gen Säften,
die ich gesammelt,
brau' ich den Trank für ihn;
wenig Tropfen nur
braucht er zu trinken,
sinnenlos sinkt er in Schlaf.
Mit der eignen Waffe,
die er sich gewonnen,
räum' ich ihn leicht aus dem Weg,
erlange mir Ring und Hort.

\direct{Er reibt sich vergnügt die Hände}

Hei! Weiser Wand'rer!
Dünkt' ich dich dumm?
Wie gefällt dir nun
mein feiner Witz?
Fand ich mir wohl
Rat und Ruh'?

\Siegfriedspeaks

Notung! Notung!
Neidliches Schwert!
Nun schmolz deines Stahles Spreu!
Im eignen Schweisse
schwimmst du nun.

\direct{Er giesst den glühenden Inhalt
des Tiegels in eine Stangenform
und hält diese in die Höhe}

Bald schwing' ich dich als mein Schwert!

\direct{Er stösst die gefüllte Stangenform
in den Wassereimer; Dampf und
lautes Gezisch der Kühlung erfolgen}

In das Wasser floss
ein Feuerfluss:
grimmiger Zorn
zischt' ihm da auf!
Wie sehrend er floss,
in des Wassers Flut
fliesst er nicht mehr.
Starr ward er und steif,
herrisch der harte Stahl:
heisses Blut doch
fliesst ihm bald!

\direct{Er stösst den Stahl in die Herdglut
und zieht die Blasebälge mächtig an}

Nun schwitze noch einmal,
dass ich dich schweisse,
Notung, neidliches Schwert!

\direct{Mime ist vergnügt aufgesprungen;
er holt verschiedene Gefässe hervor,
schüttet aus ihnen Gewürz und Kräuter
in einen Kochtopf und sucht, diesen
auf dem Herd anzubringen}

\direct{Siegfried beobachtet während der
Arbeit Mime, welcher vom andern
Ende des Herdes her seinen Topf
sorgsam an die Glut stellt}

Was schafft der Tölpel
dort mit dem Topf?
Brenn' ich hier Stahl,
braust du dort Sudel?

\Mimespeaks

Zuschanden kam ein Schmied,
den Lehrer sein Knabe lehrt:
mit der Kunst nun ist's beim Alten aus,
als Koch dient er dem Kind.
Brennt es das Eisen zu Brei,
aus Eiern braut
der Alte ihm Sud.

\direct{er fährt fort zu kochen}

\Siegfriedspeaks

Mime, der Künstler,
lernt jetzt kochen;
das Schmieden schmeckt ihm nicht mehr.
Seine Schwerter alle
hab' ich zerschmissen;
was er kocht, ich kost' es ihm nicht!

\direct{Unter dem Folgenden zieht Siegfried
die Stangenform aus der Glut,
zerschlägt sie und legt den glühenden
Stahl auf dem Amboss zurecht}

Das Fürchten zu lernen,
will er mich führen;
ein Ferner soll es mich lehren:
was am besten er kann,
mir bringt er's nicht bei:
als Stümper besteht er in allem!

\direct{während des Schmiedens}

Hoho! Hoho! Hohei!
Schmiede, mein Hammer,
ein hartes Schwert!
Hoho! Hahei!
Hoho! Hahei!

Einst färbte Blut
dein falbes Blau;
sein rotes Rieseln
rötete dich:
kalt lachtest du da,
das warme lecktest du kühl!
Heiaho! Haha!
Haheiaha!
Nun hat die Glut
dich rot geglüht;
deine weiche Härte
dem Hammer weicht:
zornig sprühst du mir Funken,
dass ich dich Spröden gezähmt!
Heiaho! Heiaho!
Heiahohoho!
Hahei!

\Mimespeaks

\direct{beiseite}

Er schafft sich ein scharfes Schwert,
Fafner zu fällen,
der Zwerge Feind:
ich braut' ein Truggetränk,
Siegfried zu fangen,
dem Fafner fiel.
Gelingen muss mir die List;
lachen muss mir der Lohn!

\direct{Er beschäftigt sich während des folgenden damit, den Inhalt des Topfes in eine Flasche zu giessen}

\Siegfriedspeaks

Hoho! Hoho!
Hahei!
Schmiede, mein Hammer,
ein hartes Schwert!
Hoho! Hahei!
Hoho! Hahei!
Der frohen Funken
wie freu' ich mich;
es ziert den Kühnen
des Zornes Kraft:
lustig lachst du mich an,
stellst du auch grimm dich und gram!
Heiaho, haha,
haheiaha!
Durch Glut und Hammer
glückt' es mir;
mit starken Schlägen
streckt' ich dich:
nun schwinde die rote Scham;
werde kalt und hart, wie du kannst.
Heiaho! Heiaho!
Heiahohoho!
Heiah!

\direct{Er schwingt den Stahl und stösst ihn in den Wassereimer. Er lacht bei dem Gezisch laut auf}

\direct{Während Siegfried die geschmiedete Schwertklinge in dem Griffhefte befestigt, treibt sich Mime mit der Flasche im Vordergrunde umher}

\Mimespeaks

Den der Bruder schuf,
den schimmernden Reif,
in den er gezaubert
zwingende Kraft,
das helle Gold,
das zum Herrscher macht,
ihn hab' ich gewonnen!
Ich walte sein!

\direct{Er trippelt, während Siegfried mit
dem kleinen Hammer arbeitet und
schleift und feilt, mit zunehmender
Vergnügtheit lebhaft umher}

Alberich selbst,
der einst mich band,
zur Zwergenfrone
zwing' ich ihn nun;
als Niblungenfürst
fahr' ich darnieder;
gehorchen soll mir
alles Heer!
Der verachtete Zwerg,
wie wird er geehrt!
Zu dem Horte hin drängt sich
Gott und Held:

\direct{mit immer lebhafteren Gebärden}

vor meinem Nicken
neigt sich die Welt,
vor meinem Zorne
zittert sie hin!
Dann wahrlich müht sich
Mime nicht mehr:
ihm schaffen andre
den ew'gen Schatz.
Mime, der kühne,
Mime ist König,
Fürst der Alben,
Walter des Alls!
Hei, Mime! Wie glückte dir das!
Wer hätte wohl das gedacht!

\Siegfriedspeaks

\direct{hat während der letzten Absätze
von Mimes Lied mit den letzten
Schlägen die Nieten des Griffheftes
geglättet und fasst nun das Schwert}

Notung! Notung!
Neidliches Schwert!
Jetzt haftest du wieder im Heft.
Warst du entzwei,
ich zwang dich zu ganz;
kein Schlag soll nun dich mehr zerschlagen.
Dem sterbenden Vater
zersprang der Stahl,
der lebende Sohn
schuf ihn neu:
nun lacht ihm sein heller Schein,
seine Schärfe schneidet ihm hart.

\direct{das Schwert vor sich schwingend}

Notung! Notung!
Neidliches Schwert!
Zum Leben weckt' ich dich wieder,
tot lagst du
in Trümmern dort,
jetzt leuchtest du trotzig und hehr.
Zeige den Schächern
nun deinen Schein!
Schlage den Falschen,
fälle den Schelm!
Schau, Mime, du Schmied:

\direct{Er holt mit dem Schwert aus}

so schneidet Siegfrieds Schwert!

\direct{Er schlägt auf den Amboss, welcher von oben bis unten in zwei Stücke zerspaltet, so dass er unter grossem Gepolter auseinander fällt. Mime, welcher in höchster Verzückung sich auf einen Schemel geschwungen hatte, fällt vor Schreck sitzlings zu Boden. Siegfried hält jauchzend das Schwert in die Höhe. Der Vorhang fällt}

\act

\scene

\StageDir{Tiefer Wald.

Ganz im Hintergrunde die Öffnung einer Höhle. Der Boden hebt sich bis zur Mitte der Bühne, wo er eine kleine Hochebene bildet; von da senkt er sich nach hinten, der Höhle zu, wieder abwärts, so dass von dieser nur der obere Teil der Öffnung dem Zuschauer sichtbar ist. Links gewahrt man durch Waldbäume eine zerklüftete Felsenwand. Finstere Nacht, am dichtesten über dem Hintergrunde, wo anfänglich der Blick des Zuschauers gar nichts zu unterscheiden vermag.}

\Alberichspeaks

\direct{an der Felsenwand zur Seite
gelagert, düster brütend}

In Wald und Nacht
vor Neidhöhl' halt' ich Wacht:
es lauscht mein Ohr,
mühvoll lugt mein Aug'.
Banger Tag,
bebst du schon auf?
Dämmerst du dort
durch das Dunkel her?

\direct{Aus dem Walde von rechts her
erhebt sich ein Sturmwind;
ein bläulicher Glanz leuchtet
von ebendaher}

Welcher Glanz glitzert dort auf?
Näher schimmert
ein heller Schein;
es rennt wie ein leuchtendes Ross,
bricht durch den Wald
brausend daher.
Naht schon des Wurmes Würger?
Ist's schon, der Fafner fällt?

\direct{Der Sturmwind legt sich wieder;
der Glanz verlischt}

Das Licht erlischt,
der Glanz barg sich dem Blick:
Nacht ist's wieder.

\direct{Der Wanderer tritt aus dem Wald
und hält Alberich gegenüber an}

Wer naht dort schimmernd im Schatten?

\Wandererspeaks

Zur Neidhöhle
fuhr ich bei Nacht:
wen gewahr' ich im Dunkel dort?

\direct{Wie aus einem plötzlich zerreissenden
Gewölk bricht Mondschein herein und
beleuchtet des Wanderers Gestalt}

\Alberichspeaks

\direct{erkennt den Wanderer, fährt
erschrocken zurück, bricht aber
sogleich in höchste Wut aus}

Du selbst lässt dich hier sehn?
Was willst du hier?
Fort, aus dem Weg!
Von dannen, schamloser Dieb!

\Wandererspeaks

\direct{ruhig}

Schwarz-Alberich,
schweifst du hier?
Hütest du Fafners Haus?

\Alberichspeaks

Jagst du auf neue
Neidtat umher?
Weile nicht hier,
weiche von hinnen!
Genug des Truges
tränkte die Stätte mit Not.
Drum, du Frecher,
lass sie jetzt frei!

\Wandererspeaks

Zu schauen kam ich,
nicht zu schaffen:
wer wehrte mir Wand'rers Fahrt?

\Alberichspeaks

\direct{lacht tückisch auf}

Du Rat wütender Ränke!
Wär' ich dir zulieb
doch noch dumm wie damals,
als du mich Blöden bandest,
wie leicht geriet' es,
den Ring mir nochmals zu rauben!
Hab' acht! Deine Kunst
kenne ich wohl;
doch wo du schwach bist,
blieb mir auch nicht verschwiegen.
Mit meinen Schätzen
zahltest du Schulden;
mein Ring lohnte
der Riesen Müh',
die deine Burg dir gebaut.
Was mit den Trotzigen
einst du vertragen,
des Runen wahrt noch heut'
deines Speeres herrischer Schaft.
Nicht du darfst,
was als Zoll du gezahlt,
den Riesen wieder entreissen:
du selbst zerspelltest
deines Speeres Schaft;
in deiner Hand
der herrische Stab,
der starke, zerstiebte wie Spreu!

\Wandererspeaks

Durch Vertrages Treuerunen
band er dich
Bösen mir nicht:
dich beugt' er mir durch seine Kraft;
zum Krieg drum wahr' ich ihn wohl!

\Alberichspeaks

Wie stolz du dräust
in trotziger Stärke,
und wie dir's im Busen doch bangt!
Verfallen dem Tod
durch meinen Fluch
ist des Hortes Hüter:
wer wird ihn beerben?
Wird der neidliche Hort
dem Niblungen wieder gehören?
Das sehrt dich mit ew'ger Sorge!
Denn fass' ich ihn wieder
einst in der Faust,
anders als dumme Riesen
üb' ich des Ringes Kraft:
dann zittre der Helden
heiliger Hüter!
Walhalls Höhen
stürm' ich mit Hellas Heer:
der Welt walte dann ich!

\Wandererspeaks

\direct{ruhig}

Deinen Sinn kenn' ich wohl;
doch sorgt er mich nicht.
Des Ringes waltet,
wer ihn gewinnt.

\Alberichspeaks

Wie dunkel sprichst du,
was ich deutlich doch weiss!
An Heldensöhne
hält sich dein Trotz,

\direct{höhnisch}

die traut deinem Blute entblüht.
Pflegtest du wohl eines Knaben,
der klug die Frucht dir pflücke,

\direct{immer heftiger}

die du nicht brechen darfst?

\Wandererspeaks

Mit mir nicht,
hadre mit Mime:
dein Bruder bringt dir Gefahr;
einen Knaben führt er daher,
der Fafner ihm fällen soll.
Nichts weiss der von mir;
der Niblung nützt ihn für sich.
Drum sag' ich dir, Gesell:
tue frei, wie dir's frommt!

\direct{Alberich macht eine Gebärde
heftiger Neugierde}

Höre mich wohl,
sei auf der Hut!
Nicht kennt der Knabe den Ring;
doch Mime kundet' ihn aus.

\Alberichspeaks

\direct{heftig}

Deine Hand hieltest du vom Hort?

\Wandererspeaks

Wen ich liebe,
lass' ich für sich gewähren;
er steh' oder fall',
sein Herr ist er:
Helden nur können mir frommen.

\Alberichspeaks

Mit Mime räng' ich
allein um den Ring?

\Wandererspeaks

Ausser dir begehrt er
einzig das Gold.

\Alberichspeaks

Und dennoch gewänn' ich ihn nicht?

\Wandererspeaks

\direct{ruhig nähertretend}

Ein Helde naht,
den Hort zu befrei'n;
zwei Niblungen geizen das Gold;
Fafner fällt,
der den Ring bewacht:
wer ihn rafft, hat ihn gewonnen.
Willst du noch mehr?
Dort liegt der Wurm:

\direct{er wendet sich nach der Höhle}

warnst du ihn vor dem Tod,
willig wohl liess' er den Tand.
Ich selber weck' ihn dir auf.

\direct{Er stellt sich auf die Anhöhe
vor der Höhle und ruft hinein}

Fafner! Fafner!
Erwache, Wurm!

\Alberichspeaks

\direct{in gespanntem Erstaunen, für sich}

Was beginnt der Wilde?
Gönnt er mir's wirklich?

\direct{Aus der finstern Tiefe des Hintergrundes hört man Fafners Stimme durch ein starkes Sprachrohr}

\Fafnerspeaks

Wer stört mir den Schlaf?

\Wandererspeaks

\direct{der Höhle zugewandt}

Gekommen ist einer,
Not dir zu künden:
er lohnt dir's mit dem Leben,
lohnst du das Leben ihm
mit dem Horte, den du hütest?

\direct{Er beugt sein Ohr lauschend
der Höhle zu}

\speaker{\Fafner (Stimme)}
Was will er?

\Alberichspeaks

\direct{ist dem Wanderer zur Seite
getreten und ruft in die Höhle}

Wache, Fafner!
Wache, du Wurm!
Ein starker Helde naht,
dich heil'gen will er bestehn.

\speaker{\Fafner (Stimme)}
Mich hungert sein.

\Wandererspeaks

Kühn ist des Kindes Kraft,
scharf schneidet sein Schwert.

\Alberichspeaks

Den goldnen Reif
geizt er allein:
lass mir den Ring zum Lohn,
so wend' ich den Streit;
du wahrest den Hort,
und ruhig lebst du lang'!

\speaker{\Fafner (Stimme)}
Ich lieg' und besitz':

\direct{gähnend}

lasst mich schlafen!

\Wandererspeaks

\direct{lacht auf und wendet sich
dann wieder zu Alberich}

Nun, Alberich, das schlug fehl.
Doch schilt mich nicht mehr Schelm!
Dies eine, rat' ich,
achte noch wohl:

\direct{vertraulich zum ihm tretend}

Alles ist nach seiner Art:
an ihr wirst du nichts ändern.
Ich lass' dir die Stätte,
stelle dich fest!
Versuch's mit Mime, dem Bruder:
der Art ja versiehst du dich besser.

\direct{zum Abgange gewendet}

Was anders ist,
das lerne nun auch!

\direct{Er verschwindet im Walde. Sturmwind erhebt sich, heller Glanz bricht aus; dann vergeht beides schnell}

\Alberichspeaks

\direct{blickt dem davonjagenden
Wanderer nach}

Da reitet er hin,
auf lichtem Ross;
mich lässt er in Sorg' und Spott.
Doch lacht nur zu,
ihr leichtsinniges,
lustgieriges
Göttergelichter!
Euch seh' ich
noch alle vergehn!
Solang' das Gold
am Lichte glänzt,
hält ein Wissender Wacht:
Trügen wird euch sein Trotz!

\direct{Er schlüpft zur Seite in das Geklüft. Die Bühne bleibt leer. Morgendämmerung}

\scene

\StageDir{Bei anbrechendem Tage treten Mime und Siegfried auf. Siegfried trägt das Schwert in einem Gehenke von Bastseil. Mime erspäht genau die Stätte; er forscht endlich dem Hintergrunde zu, welcher während die Anhöhe im mittleren Vordergrunde später immer heller von der Sonne beleuchtet wird in finstrem Schatten bleibt; dann bedeutet er Siegfried.}

\Mimespeaks

Wir sind zur Stelle!
Bleib hier stehn!

\Siegfriedspeaks

\direct{setzt sich unter einer grossen
Linde nieder und schaut sich um}

Hier soll ich das Fürchten lernen?
Fern hast du mich geleitet:
eine volle Nacht im Walde
selbander wanderten wir.
Nun sollst du, Mime,
mich meiden!
Lern' ich hier nicht,
was ich lernen muss,
allein zieh' ich dann weiter:
dich endlich werd' ich da los!

\Mimespeaks

\direct{setzt sich ihm gegenüber,
so dass er die Höhle immer
noch im Auge behält}

Glaube, Liebster!
Lernst du heut' und hier
das Fürchten nicht,
an andrem Ort,
zu andrer Zeit
schwerlich erfährst du's je.
Siehst du dort
den dunklen Höhlenschlund?
Darin wohnt
ein greulich wilder Wurm:
unmassen grimmig
ist er und gross;
ein schrecklicher Rachen
reisst sich ihm auf;
mit Haut und Haar
auf einen Happ
verschlingt der Schlimme dich wohl.

\Siegfriedspeaks

\direct{immer unter der Linde sitzend}

Gut ist's, den Schlund ihm zu schliessen:
drum biet' ich mich nicht dem Gebiss.

\Mimespeaks

Giftig giesst sich
ein Geifer ihm aus:
wen mit des Speichels
Schweiss er bespeit,
dem schwinden wohl Fleisch und Gebein.

\Siegfriedspeaks

Dass des Geifers Gift mich nicht sehre,
weich' ich zur Seite dem Wurm.

\Mimespeaks

Ein Schlangenschweif
schlägt sich ihm auf:
wen er damit umschlingt
und fest umschliesst,
dem brechen die Glieder wie Glas!

\Siegfriedspeaks

Vor des Schweifes Schwang mich zu wahren,
halt' ich den Argen im Aug'.
Doch heisse mich das:
hat der Wurm ein Herz?

\Mimespeaks

Ein grimmiges, hartes Herz!

\Siegfriedspeaks

Das sitzt ihm doch,
wo es jedem schlägt,
trag' es Mann oder Tier?

\Mimespeaks

Gewiss, Knabe,
da führt's auch der Wurm.
Jetzt kommt dir das Fürchten wohl an?

\Siegfriedspeaks

\direct{bisher nachlässig ausgestreckt,
erhebt sich rasch zum Sitz}

Notung stoss' ich
dem Stolzen ins Herz!
Soll das etwa Fürchten heissen?
He, du Alter!
Ist das alles,
was deine List
mich lehren kann?
Fahr' deines Wegs dann weiter;
das Fürchten lern' ich hier nicht.

\Mimespeaks

Wart' es nur ab!
Was ich dir sage,
dünke dich tauber Schall:
ihn selber musst du
hören und sehn,
die Sinne vergehn dir dann schon!
Wenn dein Blick verschwimmt,
der Boden dir schwankt,
im Busen bang
dein Herz erbebt:

\direct{sehr freundlich}

dann dankst du mir, der dich führte,
gedenkst, wie Mime dich liebt.

\Siegfriedspeaks

Du sollst mich nicht lieben!
Sagt' ich dir's nicht?
Fort aus den Augen mir!
Lass mich allein:
sonst halt' ich's hier länger nicht aus,
fängst du von Liebe gar an!
Das eklige Nicken
und Augenzwicken,
wann endlich soll ich's
nicht mehr sehn,
wann werd' ich den Albernen los?

\Mimespeaks

Ich lass' dich schon.
Am Quell dort lagr' ich mich;
steh' du nur hier;
steigt dann die Sonne zur Höh',
merk' auf den Wurm:
aus der Höhle wälzt er sich her,
hier vorbei
biegt er dann,
am Brunnen sich zu tränken.

\Siegfriedspeaks

\direct{lachend}

Mime, weilst du am Quell,
dahin lass' ich den Wurm wohl gehn:
Notung stoss' ich
ihm erst in die Nieren,
wenn er dich selbst dort
mit weggesoffen.
Darum, hör' meinen Rat,
raste nicht dort am Quell;
kehre dich weg,
so weit du kannst,
und komm' nie mehr zu mir!

\Mimespeaks

Nach freislichem Streit
dich zu erfrischen,
wirst du mir wohl nicht wehren?

\direct{Siegfried wehrt ihn hastig ab}

Rufe mich auch,
darbst du des Rates,

\direct{Siegfried wiederholt die
Gebärde mit Ungestüm}

oder wenn dir das Fürchten gefällt.

\direct{Siegfried erhebt sich und treibt Mime
mit wütender Gebärde zum Fortgehen}

\direct{im Abgehen für sich}

Fafner und Siegfried
Siegfried und Fafner
O brächten beide sich um!

\direct{Er verschwindet rechts im Wald}

\Siegfriedspeaks

\direct{streckt sich behaglich unter
der Linde aus und blickt dem
davongehenden Mime nach}

Dass der mein Vater nicht ist,
wie fühl' ich mich drob so froh!
Nun erst gefällt mir
der frische Wald;
nun erst lacht mir
der lustige Tag,
da der Garstige von mir schied
und ich gar nicht ihn wiederseh'!

\direct{Er verfällt in schweigendes Sinnen}

Wie sah mein Vater wohl aus?
Ha, gewiss wie ich selbst!
Denn wär' wo von Mime ein Sohn,
müsst' er nicht ganz
Mime gleichen?
Grade so garstig,
griesig und grau,
klein und krumm,
höckrig und kinkend,
mit hängenden Ohren,
triefigen Augen
fort mit dem Alp!
Ich mag ihn nicht mehr seh'n.

\direct{Er lehnt sich tiefer zurück und
blickt durch die Baumwipfel auf.
Tiefe Stille. Waldweben}

Aber wie sah
meine Mutter wohl aus?
Das kann ich
nun gar nicht mir denken!
Der Rehhindin gleich
glänzten gewiss
ihr hell schimmernde Augen,
nur noch viel schöner!
Da bang sie mich geboren,
warum aber starb sie da?
Sterben die Menschenmütter
an ihren Söhnen
alle dahin?
Traurig wäre das, traun!
Ach, möcht' ich Sohn
meine Mutter sehen!
Meine Mutter
ein Menschenweib!

\direct{Er seufzt leise und streckt
sich tiefer zurück. Grosse Stille.
Wachsendes Waldweben. Siegfrieds
Aufmerksamkeit wird endlich durch
den Gesang der Waldvögel gefesselt.
Er lauscht mit wachsender Teilnahme
einem Waldvogel in den Zweigen
über ihm}

Du holdes Vöglein!
Dich hört' ich noch nie:
bist du im Wald hier daheim?
Verstünd' ich sein süsses Stammeln!
Gewiss sagt' es mir was,
vielleicht von der lieben Mutter?
Ein zankender Zwerg
hat mir erzählt,
der Vöglein Stammeln
gut zu verstehn,
dazu könnte man kommen.
Wie das wohl möglich wär'?

\direct{Er sinnt nach. Sein Blick fällt
auf ein Rohrgebüsch unweit
der Linde}

Hei! Ich versuch's;
sing' ihm nach:
auf dem Rohr tön' ich ihm ähnlich!
Entrat' ich der Worte,
achte der Weise,
sing' ich so seine Sprache,
versteh' ich wohl auch, was es spricht.

\direct{Er eilt an den nahen Quell, schneidet
mit dem Schwerte ein Rohr ab und
schnitzt sich hastig eine Pfeife daraus.
Währenddem lauscht er wieder}

Es schweigt und lauscht:
so schwatz' ich denn los!

\direct{Er bläst auf dem Rohr. Er setzt ab,
schnitzt wieder und bessert. Er bläst
wieder. Er schüttelt mit dem Kopfe und
bessert wieder. Er wird ärgerlich, drückt
das Rohr mit der Hand und versucht
wieder. Er setzt lächelnd ganz ab}

Das tönt nicht recht;
auf dem Rohre taugt
die wonnige Weise mir nicht.
Vöglein, mich dünkt,
ich bleibe dumm:
von dir lernt sich's nicht leicht!

\direct{Er hört den Vogel wieder
und blickt zu ihm auf}

Nun schäm' ich mich gar
vor dem schelmischen Lauscher:
er lugt und kann nichts erlauschen.
Heida! So höre
nun auf mein Horn.

\direct{Er schwingt das Rohr
und wirft es weit fort}

Auf dem dummen Rohre
gerät mir nichts.
Einer Waldweise,
wie ich sie kann,
der lustigen sollst du nun lauschen.
Nach liebem Gesellen
lockt' ich mit ihr:
nichts Bessres kam noch
als Wolf und Bär.
Nun lass mich sehn,
wen jetzt sie mir lockt:
ob das mir ein lieber Gesell?

\direct{Er nimmt das silberne Hifthorn und bläst darauf. Im Hintergrunde regt es sich. Fafner, in der Gestalt eines ungeheuren eidechsenartigen Schlangenwurmes, hat sich in der Höhle von seinem Lager erhoben; er bricht durch das Gesträuch und wälzt sich aus der Tiefe nach der höheren Stelle vor, so dass er mit dem Vorderleibe bereits auf ihr angelangt ist, als er jetzt einen starken, gähnenden Laut ausstösst.}

\direct{sieht sich um und heftet den
Blick verwundert auf Fafner}

Haha! Da hätte mein Lied
mir was Liebes erblasen!
Du wärst mir ein saub'rer Gesell!

\Fafnerspeaks

\direct{hat beim Anblick Siegfrieds
auf der Höhe angehalten
und verweilt nun daselbst}

Was ist da?

\Siegfriedspeaks

Ei, bist du ein Tier,
das zum Sprechen taugt,
wohl liess' sich von dir was lernen?
Hier kennt einer
das Fürchten nicht:
kann er's von dir erfahren?

\Fafnerspeaks

Hast du Übermut?

\Siegfriedspeaks

Mut oder Übermut
was weiss ich!
Doch dir fahr' ich zu Leibe,
lehrst du das Fürchten mich nicht!

\Fafnerspeaks

\direct{stösst einen lachenden Laut aus}

Trinken wollt' ich:
nun treff' ich auch Frass!

\direct{Er öffnet seinen Rachen
und zeigt die Zähne}

\Siegfriedspeaks

Eine zierliche Fresse
zeigst du mir da,
lachende Zähne
im Leckermaul!
Gut wär' es, den Schlund dir zu schliessen;
dein Rachen reckt sich zu weit!

\Fafnerspeaks

Zu tauben Reden
taugt er schlecht:
dich zu verschlingen,
frommt der Schlund.

\direct{Er droht mit dem Schweife}

\Siegfriedspeaks

Hoho! Du grausam
grimmiger Kerl!
Von dir verdaut sein,
dünkt mich übel:
rätlich und fromm doch scheint's,
du verrecktest hier ohne Frist.

\Fafnerspeaks

\direct{brüllend}

Pruh! Komm,
prahlendes Kind!

\Siegfriedspeaks

Hab' acht, Brüller!
Der Prahler naht!

\direct{Er zieht sein Schwert, springt Fafner an und
bleibt herausfordernd stehen. Fafner wälzt
sich weiter auf die Höhe herauf und sprüht aus den Nüstern
auf Siegfried. Dieser weicht dem Geifer aus,
springt näher zu und stellt sich zur Seite. Fafner sucht ihn
mit dem Schweife zu erreichen. Siegfried, welchen Fafner fast erreicht hat,
springt mit einem Satze über diesen hinweg
und verwundet ihn an dem Schweife.
Fafner brüllt, zieht den Schweif heftig zurück und bäumt
den Vorderleib, um mit dessen voller Wucht
sich auf Siegfried zu werfen; so bietet er diesem die Brust dar;
Siegfried erspäht schnell die Stelle des Herzens und stösst sein
Schwert bis an das Heft hinein. Fafner bäumt sich vor Schmerz
noch höher und sinkt, als Siegfried das Schwert losgelassen
und zur Seite gesprungen ist,
auf die Wunde zusammen}

Da lieg', neidischer Kerl!
Notung trägst du im Herzen.

\Fafnerspeaks

\direct{mit schwächerer Stimme}

Wer bist du, kühner Knabe,
der das Herz mir traf?
Wer reizte des Kindes Mut
zu der mordlichen Tat?
Dein Hirn brütete nicht,
was du vollbracht.

\Siegfriedspeaks

Viel weiss ich noch nicht,
noch nicht auch, wer ich bin.
Mit dir mordlich zu ringen,
reiztest du selbst meinen Mut.

\Fafnerspeaks

Du helläugiger Knabe,
unkund deiner selbst,
wen du gemordet
meld' ich dir.
Der Riesen ragend Geschlecht,
Fasolt und Fafner,
die Brüder fielen nun beide.
Um verfluchtes Gold,
von Göttern vergabt,
traf ich Fasolt zu Tod.
Der nun als Wurm
den Hort bewachte,
Fafner, den letzten Riesen,
fällte ein rosiger Held.
Blicke nun hell,
blühender Knabe:
der dich Blinden reizte zur Tat,
berät jetzt des Blühenden Tod!

\direct{ersterbend}

Merk', wie's endet!
Acht' auf mich!

\Siegfriedspeaks

Woher ich stamme,
rate mir noch;
weise ja scheinst du,
Wilder, im Sterben:
rat' es nach meinem Namen:
Siegfried bin ich genannt.

\Fafnerspeaks

Siegfried...!

\direct{Er seufzt, hebt sich und stirbt}

\Siegfriedspeaks

Zur Kunde taugt kein Toter.
So leite mich denn
mein lebendes Schwert!

\direct{Fafner hat sich im Sterben
zur Seite gewälzt. Siegfried zieht
ihm jetzt das Schwert aus der Brust:
dabei wird seine Hand vom Blute
benetzt: er fährt heftig mit der Hand auf}

Wie Feuer brennt das Blut!

\direct{Er führt unwillkürlich die Finger
zum Munde, um das Blut von ihnen
abzusaugen. Wie er sinnend vor
sich hinblickt, wird seine
Aufmerksamkeit immer mehr von
dem Gesange der Waldvögel angezogen}

Ist mir doch fast,
als sprächen die Vöglein zu mir!
Nützte mir das
des Blutes Genuss?
Das seltne Vöglein hier,
horch, was singt es nur?

\speaker{Eine Waldvogel (Stimme)}
\direct{aus den Zweigen der Linde
über Siegfried}

Hei! Siegfried gehört
nun der Niblungen Hort!
O, fänd' in der Höhle
den Hort er jetzt!
Wollt' er den Tarnhelm gewinnen,
der taugt' ihm zu wonniger Tat:
doch möcht' er den Ring sich erraten,
der macht' ihn zum Walter der Welt!

\Siegfriedspeaks

\direct{hat mit verhaltenem Atem
und verzückter Miene gelauscht}

Dank, liebes Vöglein,
für deinen Rat!
Gern folg' ich dem Ruf!

\direct{Er wendet sich nach hinten und
steigt in die Höhle hinab, wo er
alsbald gänzlich verschwindet}

\scene

\StageDir{Mime schleicht heran, scheu umherblickend, um sich von Fafners Tod zu überzeugen. Gleichzeitig kommt von der anderen Seite Alberich aus dem Geklüft; er beobachtet Mime genau. Als dieser Siegfried nicht mehr gewahrt und vorsichtig sich nach hinten der Höhle zuwendet, stürzt Alberich auf ihn zu und vertritt ihm den Weg}

\Alberichspeaks

Wohin schleichst du
eilig und schlau,
schlimmer Gesell?

\Mimespeaks

Verfluchter Bruder,
dich braucht' ich hier!
Was bringt dich her?

\Alberichspeaks

Geizt es dich, Schelm,
nach meinem Gold?
Verlangst du mein Gut?

\Mimespeaks

Fort von der Stelle!
Die Stätte ist mein:
was stöberst du hier?

\Alberichspeaks

Stör' ich dich wohl
im stillen Geschäft,
wenn du hier stiehlst?

\Mimespeaks

Was ich erschwang
mit schwerer Müh',
soll mir nicht schwinden.

\Alberichspeaks

Hast du dem Rhein
das Gold zum Ringe geraubt?
Erzeugtest du gar
den zähen Zauber im Reif?

\Mimespeaks

Wer schuf den Tarnhelm,
der die Gestalten tauscht?
Der seiner bedurfte,
erdachtest du ihn wohl?

\Alberichspeaks

Was hättest du Stümper
je wohl zu stampfen verstanden?
Der Zauberring
zwang mir den Zwerg erst zur Kunst.

\Mimespeaks

Wo hast du den Ring?
Dir Zagem entrissen ihn Riesen!
Was du verlorst,
meine List erlangt es für mich.

\Alberichspeaks

Mit des Knaben Tat
will der Knicker nun knausern?
Dir gehört sie gar nicht,
der Helle ist selbst ihr Herr!

\Mimespeaks

Ich zog ihn auf;
für die Zucht zahlt er mir nun:
für Müh' und Last
erlauert' ich lang meinen Lohn!

\Alberichspeaks

Für des Knaben Zucht
will der knickrige
schäbige Knecht
keck und kühn
wohl gar König nun sein?
Dem räudigsten Hund
wäre der Ring
geratner als dir:
nimmer erringst
du Rüpel den Herrscherreif!

\Mimespeaks

\direct{kratzt sich den Kopf}

Behalt' ihn denn:
und hüt' ihn wohl,
den hellen Reif!
Sei du Herr:
doch mich heisse auch Bruder!
Um meines Tarnhelms
lustigen Tand
tausch' ich ihn dir:
uns beiden taugt's,
teilen die Beute wir so.

\direct{Er reibt sich zutraulich die Hände}

\Alberichspeaks

\direct{mit Hohnlachen}

Teilen mit dir?
Und den Tarnhelm gar?
Wie schlau du bist!
Sicher schlief' ich
niemals vor deinen Schlingen!

\Mimespeaks

\direct{ausser sich}

Selbst nicht tauschen?
Auch nicht teilen?
Leer soll ich gehn?
Ganz ohne Lohn?

\direct{kreischend}

Gar nichts willst du mir lassen?

\Alberichspeaks

Nichts von allem!
Nicht einen Nagel
sollst du dir nehmen!

\Mimespeaks

\direct{in höchster Wut}

Weder Ring noch Tarnhelm
soll dir denn taugen!
Nicht teil' ich nun mehr!
Gegen dich doch ruf' ich
Siegfried zu Rat
und des Recken Schwert;
der rasche Held,
der richte, Brüderchen, dich!

\direct{Siegfried erscheint im Hintergrund}

\Alberichspeaks

Kehre dich um!
Aus der Höhle kommt er daher!

\Mimespeaks

\direct{sich umblickend}

Kindischen Tand
erkor er gewiss.

\Alberichspeaks

Den Tarnhelm hält er!

\Mimespeaks

Doch auch den Ring!

\Alberichspeaks

Verflucht! Den Ring!

\Mimespeaks

\direct{hämisch lachend}

Lass ihn den Ring dir doch geben!
Ich will ihn mir schon gewinnen.

\direct{Er schlüpft mit den letzten
Worten in den Wald zurück}

\Alberichspeaks

Und doch seinem Herrn
soll er allein noch gehören!

\direct{Er verschwindet im Geklüfte}

\direct{Siegfried ist mit Tarnhelm und Ring während des letzteren langsam und sinnend aus der Höhle vorgeschritten: er betrachtet gedankenvoll seine Beute und hält, nahe dem Baume, auf der Höhe des Mittelgrundes wieder an}

\Siegfriedspeaks

Was ihr mir nützt,
weiss ich nicht;
doch nahm ich euch
aus des Horts gehäuftem Gold,
weil guter Rat mir es riet.
So taug' eure Zier
als des Tages Zeuge,
es mahne der Tand,
dass ich kämpfend Fafner erlegt,
doch das Fürchten noch nicht gelernt!

\direct{Er steckt den Tarnhelm sich in den Gürtel und den Reif an den Finger. Stillschweigen. Wachsendes Waldweben. Siegfried achtet unwillkürlich wieder des Vogels und lauscht ihm mit verhaltenem Atem}

\speaker{Eine Waldvogel (Stimme)}
Hei! Siegfried gehört
nun der Helm und der Ring!
O, traute er Mime,
dem treulosen, nicht!
Hörte Siegfried nur scharf
auf des Schelmen Heuchlergered'!
Wie sein Herz es meint,
kann er Mime verstehn:
so nützt' ihm des Blutes Genuss.

\direct{Siegfrieds Miene und Gebärde drücken aus,
dass er den Sinn des Vogelgesanges wohl vernommen. Er sieht
Mime sich nähern und bleibt, ohne sich zu rühren, auf sein Schwert
gestützt, beobachtend und in sich geschlossen,in seiner Stellung
auf der Anhöhe bis zum Schlusse des folgenden Auftrittes}

\Mimespeaks

\direct{schleicht heran und beobachtet
vom Vordergrunde aus Siegfried}

Er sinnt und erwägt
der Beute Wert:
weilte wohl hier
ein weiser Wand'rer,
schweifte umher,
beschwatzte das Kind
mit list'ger Runen Rat?
Zwiefach schlau
sei nun der Zwerg;
die listigste Schlinge
leg' ich jetzt aus,
dass ich mit traulichem
Truggerede
betöre das trotzige Kind.

\direct{er tritt näher an Siegfried heran
und bewillkommt diesen mit
schmeichelnden Gebärden}

Willkommen, Siegfried!
Sag', du Kühner,
hast du das Fürchten gelernt?

\Siegfriedspeaks

Den Lehrer fand ich noch nicht!

\Mimespeaks

Doch den Schlangenwurm,
du hast ihn erschlagen?
Das war doch ein schlimmer Gesell?

\Siegfriedspeaks

So grimm und tückisch er war,
sein Tod grämt mich doch schier,
da viel üblere Schächer
unerschlagen noch leben!
Der mich ihn morden hiess,
den hass' ich mehr als den Wurm!

\Mimespeaks

\direct{sehr freundlich}

Nur sachte! Nicht lange
siehst du mich mehr:
zum ew'gen Schlaf
schliess' ich dir die Augen bald!
Wozu ich dich brauchte,

\direct{zärtlich}

hast du vollbracht;
jetzt will ich nur noch
die Beute dir abgewinnen.
Mich dünkt, das soll mir gelingen;
zu betören bist du ja leicht!

SIEGFRIED:
So sinnst du auf meinen Schaden?

\Mimespeaks

\direct{verwundert}

Wie sagt' ich denn das?
Siegfried! Hör doch, mein Söhnchen!
Dich und deine Art
hasst' ich immer von Herzen;

\direct{zärtlich}

aus Liebe erzog ich
dich Lästigen nicht:
dem Horte in Fafners Hut,
dem Golde galt meine Müh'.

\direct{als verspräche er ihm hübsche Sachen}

Gibst du mir das
gutwillig nun nicht,

\direct{als wäre er bereit, sein Leben
für ihn zu lassen}

Siegfried, mein Sohn,
das siehst du wohl selbst,

\direct{mit freundlichem Scherze}

dein Leben musst du mir lassen!

\Siegfriedspeaks

Dass du mich hassest,
hör' ich gern:
doch auch mein Leben muss ich dir lassen?

\Mimespeaks

\direct{ärgerlich}

Das sagt' ich doch nicht?
Du verstehst mich ja falsch!

\direct{Er sucht sein Fläschchen hervor.
Er gibt sich die ersichtlichste
Mühe zur Verstellung}

Sieh', du bist müde
von harter Müh';
brünstig wohl brennt dir der Leib:
dich zu erquicken
mit queckem Trank
säumt' ich Sorgender nicht.
Als dein Schwert du dir branntest,
braut' ich den Sud;
trinkst du nun den,
gewinn' ich dein trautes Schwert,
und mit ihm Helm und Hort.

\direct{er kichert dazu}

\Siegfriedspeaks

So willst du mein Schwert
und was ich erschwungen,
Ring und Beute, mir rauben?

\Mimespeaks

\direct{heftig}

Was du doch falsch mich verstehst!
Stamml' ich, fasl' ich wohl gar?
Die grösste Mühe
geb' ich mir doch,
mein heimliches Sinnen
heuchelnd zu bergen,
und du dummer Bube
deutest alles doch falsch!
Öffne die Ohren,
und vernimm genau:
Höre, was Mime meint!

\direct{wieder sehr freundlich,
mit ersichtlicher Mühe}

Hier nimm und trinke die Labung!
Mein Trank labte dich oft:
tat'st du wohl unwirsch,
stelltest dich arg:
was ich dir bot,
erbost auch nahmst du's doch immer.

\Siegfriedspeaks

\direct{ohne eine Miene zu verziehen}

Einen guten Trank
hätt' ich gern:
wie hast du diesen gebraut?

\Mimespeaks

\direct{lustig scherzend, als schildere
er ihm einen angenehm berauschten
Zustand, den ihm der Saft bereiten soll}

Hei! So trink nur,
trau' meiner Kunst!
In Nacht und Nebel
sinken die Sinne dir bald:
ohne Wach' und Wissen
stracks streckst du die Glieder.
Liegst du nun da,
leicht könnt' ich
die Beute nehmen und bergen:
doch erwachtest du je,
nirgends wär' ich
sicher vor dir,
hätt' ich selbst auch den Ring.
Drum mit dem Schwert,
das so scharf du schufst,

\direct{mit einer Gebärde
ausgelassener Lustigkeit}

hau' ich dem Kind
den Kopf erst ab:
dann hab' ich mir Ruh' und auch den Ring!

\direct{Er kichert wieder}

\Siegfriedspeaks

Im Schlafe willst du mich morden?

\Mimespeaks

\direct{wütend ärgerlich}

Was möcht' ich? Sagt' ich denn das?

\direct{Er bemüht sich, den zärtlichsten
Ton anzunehmen}

Ich will dem Kind

\direct{mit sorglichster Deutlichkeit}

nur den Kopf abhau'n!

\direct{mit dem Ausdruck herzlicher
Besorgtheit für Siegfrieds Gesundheit}

Denn hasste ich dich
auch nicht so sehr,
und hätt' ich des Schimpfs
und der schändlichen Mühe
auch nicht so viel zu rächen:

\direct{sanft}

aus dem Wege dich zu räumen,
darf ich doch nicht rasten:
wie käm' ich sonst anders zur Beute,
da Alberich auch nach ihr lugt?

\direct{Er giesst den Saft in das Trinkhorn
und führt dieses Siegfried mit
aufdringlicher Gebärde zu}

Nun, mein Wälsung!
Wolfssohn du!
Sauf', und würg' dich zu Tod:
Nie tust du mehr 'nen Schluck!

\direct{Siegfried holt mit dem Schwert aus. Er führt, wie in einer Anwandlung heftigen Ekels einen jähen Streich nach Mime; dieser stürzt sogleich tot zu Boden. Man hört Alberichs höhnisches Gelächter aus dem Geklüfte}

\Siegfriedspeaks

Schmeck' du mein Schwert,
ekliger Schwätzer!

\direct{Er henkt, auf den am
Boden Liegenden blickend,
ruhig sein Schwert wieder ein}

Neides Zoll
zahlt Notung:
dazu durft' ich ihn schmieden.

\direct{Er rafft Mimes Leichnam auf,
trägt ihn auf die Anhöhe vor
den Eingang der Höhle und
wirft ihn dort hinein}

In der Höhle hier
lieg' auf dem Hort!
Mit zäher List
erzieltest du ihn:
jetzt magst du des wonnigen walten!
Einen guten Wächter
geb' ich dir auch,
dass er vor Dieben dich deckt.

\direct{Er wälzt mit grosser Anstrengung
den Leichnam des Wurmes vor den
Eingang der Höhle, so dass er diesen
ganz damit verstopft}

Da lieg' auch du,
dunkler Wurm!
Den gleissenden Hort
hüte zugleich
mit dem beuterührigen Feind:
so fandet beide ihr nun Ruh'!

\direct{Er blickt eine Weile sinnend
in die Höhle hinab und wendet
sich dann langsam, wie ermüdet,
in den Vordergrund. Es ist Mittag.
Er führt sich die Hand über die Stirn}

Heiss ward mir
von der harten Last!
Brausend jagt
mein brünst'ges Blut;
die Hand brennt mir am Haupt.
Hoch steht schon die Sonne:
aus lichtem Blau
blickt ihr Aug'
auf den Scheitel steil mir herab.
Linde Kühlung
erkies' ich unter der Linde!

\direct{Er streckt sich unter der Linde
aus und blickt wieder die Zweige
hinauf}

Noch einmal, liebes Vöglein,
da wir so lang
lästig gestört,
lauscht' ich gerne deinem Sange:
auf dem Zweige seh' ich
wohlig dich wiegen;
zwitschernd umschwirren
dich Brüder und Schwestern,
umschweben dich lustig und lieb!
Doch ich bin so allein,
hab' nicht Brüder noch Schwestern:
meine Mutter schwand,
mein Vater fiel:
nie sah sie der Sohn!
Mein einz'ger Gesell
war ein garstiger Zwerg;
Güte zwang
uns nie zu Liebe;
listige Schlingen
warf mir der Schlaue;
nun musst' ich ihn gar erschlagen!

\direct{Er blickt schmerzlich bewegt
wieder nach den Zweigen auf}

Freundliches Vöglein,
dich frage ich nun:
gönntest du mir
wohl ein gut Gesell?
Willst du mir das Rechte raten?
Ich lockte so oft,
und erlost' es mir nie:
Du, mein Trauter,
träfst es wohl besser,
so recht ja rietest du schon.
Nun sing'! Ich lausche dem Gesang.

\speaker{Eine Waldvogel (Stimme)}
Hei! Siegfried erschlug
nun den schlimmen Zwerg!
Jetzt wüsst' ich ihm noch
das herrlichste Weib:
auf hohem Felsen sie schläft,
Feuer umbrennt ihren Saal:
durchschritt' er die Brunst,
weckt' er die Braut,
Brünnhilde wäre dann sein!

\Siegfriedspeaks

\direct{fährt mit jäher Heftigkeit
vom Sitze auf}

O holder Sang!
Süssester Hauch!
Wie brennt sein Sinn
mir sehrend die Brust!
Wie zückt er heftig
zündend mein Herz!
Was jagt mir so jach
durch Herz und Sinne?
Sag' es mir, süsser Freund!

\direct{er lauscht}

\speaker{Eine Waldvogel (Stimme)}
Lustig im Leid
sing' ich von Liebe;
wonnig aus Weh
web' ich mein Lied:
nur Sehnende kennen den Sinn!

\Siegfriedspeaks

Fort jagt mich's
jauchzend von hinnen,
fort aus dem Wald auf den Fels!
Noch einmal sage mir,
holder Sänger:
werd' ich das Feuer durchbrechen?
Kann ich erwecken die Braut?

\direct{Siegfried lauscht noch mal}

\speaker{Eine Waldvogel (Stimme)}
Die Braut gewinnt,
Brünnhilde erweckt
ein Feiger nie:
nur wer das Fürchten nicht kennt!

\Siegfriedspeaks

\direct{lacht auf vor Entzücken}

Der dumme Knab',
der das Fürchten nicht kennt,
mein Vöglein, der bin ja ich!
Noch heute gab ich
vergebens mir Müh,
das Fürchten von Fafner zu lernen:
nun brenn' ich vor Lust,
es von Brünnhilde zu wissen!
Wie find' ich zum Felsen den Weg?

\direct{Der Vogel flattert auf, kreist über Siegfried und fliegt ihm zögernd voran jauchzend}

So wird mir der Weg gewiesen:
wohin du flatterst
folg' ich dem Flug!

\direct{Er läuft dem Vogel, welcher ihn neckend einige Zeitlang
unstet nach verschiedenen Richtungen hinleitet, nach und folgt
ihm endlich, als dieser mit einer bestimmten Wendung
nach dem Hintergrunde davonfliegt.}

\act

\scene

\StageDir{Wilde Gegend

am Fusse eines Felsenberges, welcher links nach hinten steil aufsteigt. Nacht, Sturm und Wetter, Blitz und heftiger Donner, welch letzterer dann schweigt, während Blitze noch längere Zeit die Wolken durchkreuzen.}

\Wandererspeaks

\direct{schreitet entschlossen auf ein
gruftähnliches Höhlentor in einem Felsen
des Vordergrundes zu und nimmt dort,
auf seinen Speer gestützt, eine Stellung ein,
während er das Folgende dem Eingange
der Höhle zu ruft}

Wache, Wala!
Wala! Erwach'!
Aus langem Schlaf
weck' ich dich Schlummernde wach.
Ich rufe dich auf:
Herauf! Herauf!
Aus nebliger Gruft,
aus nächtigem Grunde herauf!
Erda! Erda!
Ewiges Weib!
Aus heimischer Tiefe
tauche zur Höh!
Dein Wecklied sing' ich,
dass du erwachest;
aus sinnendem Schlafe
weck' ich dich auf.
Allwissende!
Urweltweise!
Erda! Erda!
Ewiges Weib!
Wache, erwache,
du Wala! Erwache!

\direct{Die Höhlengruft erdämmert. Bläulicher Lichtschein: von ihm beleuchtet steigt mit dem Folgenden Erda sehr allmählich aus der Tiefe auf. Sie erscheint wie von Reif bedeckt: Haar und Gewand werfen einen glitzernden Schimmer von sich}

\Erdaspeaks

Stark ruft das Lied;
kräftig reizt der Zauber.
Ich bin erwacht
aus wissendem Schlaf:
wer scheucht den Schlummer mir?

\Wandererspeaks

Der Weckrufer bin ich,
und Weisen üb' ich,
dass weithin wache,
was fester Schlaf umschliesst.
Die Welt durchzog ich,
wanderte viel,
Kunde zu werben,
urweisen Rat zu gewinnen.
Kundiger gibt es
keine als dich;
bekannt ist dir,
was die Tiefe birgt,
was Berg und Tal,
Luft und Wasser durchwebt.
Wo Wesen sind,
wehet dein Atem;
wo Hirne sinnen,
haftet dein Sinn:
alles, sagt man,
sei dir bekannt.
Dass ich nun Kunde gewänne,
weck' ich dich aus dem Schlaf!

\Erdaspeaks

Mein Schlaf ist Träumen,
mein Träumen Sinnen,
mein Sinnen Walten des Wissens.
Doch wenn ich schlafe,
wachen Nornen:
sie weben das Seil
und spinnen fromm, was ich weiss.
Was frägst du nicht die Nornen?

\Wandererspeaks

Im Zwange der Welt
weben die Nornen:
sie können nichts wenden noch wandeln.
Doch deiner Weisheit
dankt' ich den Rat wohl,
wie zu hemmen ein rollendes Rad?

\Erdaspeaks

Männertaten
umdämmern mir den Mut:
mich Wissende selbst
bezwang ein Waltender einst.
Ein Wunschmädchen
gebar ich Wotan:
der Helden Wal
hiess für sich er sie küren.
Kühn ist sie
und weise auch:
was weckst du mich
und frägst um Kunde
nicht Erdas und Wotans Kind?

\Wandererspeaks

Die Walküre meinst du,
Brünnhild', die Maid?
Sie trotzte dem Stürmebezwinger,
wo er am stärksten selbst sich bezwang:
was den Lenker der Schlacht
zu tun verlangte,
doch dem er wehrte
---zuwider sich selbst---,
allzu vertraut
wagte die Trotzige,
das für sich zu vollbringen,
Brünnhild' in brennender Schlacht.
Streitvater
strafte die Maid:
in ihr Auge drückte er Schlaf;
auf dem Felsen schläft sie fest:
erwachen wird
die Weihliche nur,
um einen Mann zu minnen als Weib.
Frommten mir Fragen an sie?

\Erdaspeaks

\direct{ist in Sinnen versunken und
beginnt erst nach längerem
Schweigen}

Wirr wird mir,
seit ich erwacht:
wild und kraus
kreist die Welt!
Die Walküre,
der Wala Kind,
büsst' in Banden des Schlafs,
als die wissende Mutter schlief?
Der den Trotz lehrte,
straft den Trotz?
Der die Tat entzündet,
zürnt um die Tat?
Der die Rechte wahrt,
der die Eide hütet,
wehret dem Recht,
herrscht durch Meineid?
Lass mich wieder hinab!
Schlaf verschliesse mein Wissen!

\Wandererspeaks

Dich, Mutter, lass' ich nicht ziehn,
da des Zaubers mächtig ich bin.
Urwissend
stachest du einst
der Sorge Stachel
in Wotans wagendes Herz:
mit Furcht vor schmachvoll
feindlichem Ende
füllt' ihn dein Wissen,
dass Bangen band seinen Mut.
Bist du der Welt
weisestes Weib,
sage mir nun:
wie besiegt die Sorge der Gott?

\Erdaspeaks

Du bist nicht
was du dich nennst!
Was kamst du, störrischer Wilder,
zu stören der Wala Schlaf?

\Wandererspeaks

Du bist nicht,
was du dich wähnst!
Urmütter-Weisheit
geht zu Ende:
dein Wissen verweht
vor meinem Willen.
Weisst du, was Wotan will?

\direct{Langes Schweigen}

Dir Unweisen
ruf' ich ins Ohr,
dass sorglos ewig du nun schläfst!
Um der Götter Ende
grämt mich die Angst nicht,
seit mein Wunsch es will!
Was in des Zwiespalts wildem Schmerze
verzweifelnd einst ich beschloss,
froh und freudig
führe frei ich nun aus.
Weiht' ich in wütendem Ekel
des Niblungen Neid schon die Welt,
dem herrlichsten Wälsung
weis' ich mein Erbe nun an.
Der von mir erkoren,
doch nie mich gekannt,
ein kühnester Knabe,
bar meines Rates,
errang des Niblungen Ring:
liebesfroh,
ledig des Neides,
erlahmt an dem Edlen
Alberichs Fluch;
denn fremd bleibt ihm die Furcht.
Die du mir gebarst,
Brünnhild',
weckt sich hold der Held:
wachend wirkt
dein wissendes Kind
erlösende Weltentat.
Drum schlafe nun du,
schliesse dein Auge;
träumend erschau' mein Ende!
Was jene auch wirken,
dem ewig Jungen
weicht in Wonne der Gott.
Hinab denn, Erda!
Urmütterfurcht!
Ursorge!
Hinab! Hinab,
zu ewigem Schlaf!

\direct{Nachdem Erda bereits die Augen geschlossen hat und allmählich tiefer versunken ist, verschwindet sie jetzt gänzlich; auch die Höhle ist jetzt wiederum durchaus verfinstert. Monddämmerung erhellt die Bühne, der Sturm hat aufgehört}

\scene

\StageDir{Der Wanderer ist dicht an die Höhle getreten und lehnt sich dann mit dem Rücken an das Gestein derselben, das Gesicht der Szene zugewandt}

\Wandererspeaks

Dort seh' ich Siegfried nahn.

\direct{Er verbleibt in seiner Stellung an der Höhle.
Siegfrieds Waldvogel flattert dem Vordergrunde zu.
Plötzlich hält der Vogel in seiner Richtung ein, flattert
ängstlich hin und her und verschwindet hastig
dem Hintergrunde zu}

\Siegfriedspeaks

\direct{tritt rechts im Vordergrunde
auf und hält an}

Mein Vöglein schwebte mir fort!
Mit flatterndem Flug
und süssem Sang
wies es mich wonnig des Wegs:
nun schwand es fern mir davon!
Am besten find' ich mir
selbst nun den Berg:
wohin mein Führer mich wiess,
dahin wandr' ich jetzt fort.

\direct{Er schreitet weiter nach hinten}

\Wandererspeaks

\direct{in seiner Stellung an der
Höhle verbleibend}

Wohin, Knabe,
heisst dich dein Weg?

\Siegfriedspeaks

\direct{hält an und wendet sich um}

Da redet's ja:
wohl rät das mir den Weg.

\direct{Er tritt dem Wanderer näher}

Einen Felsen such' ich,
von Feuer ist der umwabert:
dort schläft ein Weib,
das ich wecken will.

\Wandererspeaks

Wer sagt' es dir,
den Fels zu suchen?
Wer, nach der Frau dich zu sehnen?

\Siegfriedspeaks

Mich wies ein singend
Waldvöglein:
das gab mir gute Kunde.

\Wandererspeaks

Ein Vöglein schwatzt wohl manches;
kein Mensch doch kann's verstehn.
Wie mochtest du Sinn
dem Sang entnehmen?

\Siegfriedspeaks

Das wirkte das Blut
eines wilden Wurms,
der mir vor Neidhöhl' erblasste:
kaum netzt' es zündend
die Zunge mir,
da verstand ich der Vöglein Gestimm'.

\Wandererspeaks

Erschlugst den Riesen du,
wer reizte dich,
den starken Wurm zu bestehn?

\Siegfriedspeaks

Mich führte Mime,
ein falscher Zwerg;
das Fürchten wollt' er mich lehren:
zum Schwertstreich aber,
der ihn erschlug,
reizte der Wurm mich selbst;
seinen Rachen riss er mir auf.

\Wandererspeaks

Wer schuf das Schwert
so scharf und hart,
dass der stärkste Feind ihm fiel?

\Siegfriedspeaks

Das schweisst' ich mir selbst,
da's der Schmied nicht konnte:
schwertlos noch wär' ich wohl sonst.

\Wandererspeaks

Doch, wer schuf
die starken Stücken,
daraus das Schwert du dir geschweisst?

\Siegfriedspeaks

Was weiss ich davon?
Ich weiss allein,
dass die Stücke mir nichts nützten,
schuf ich das Schwert mir nicht neu.

\Wandererspeaks

\direct{bricht in ein freudig
gemütliches Lachen aus}

Das mein' ich wohl auch!

\direct{Er betrachtet Siegfried
wohlgefällig}

\Siegfriedspeaks

\direct{verwundert}

Was lachst du mich aus?
Alter Frager!
Hör' einmal auf;
lass mich nicht länger hier schwatzen!
Kannst du den Weg
mir weisen, so rede:
vermagst du's nicht,
so halte dein Maul!

\Wandererspeaks

Geduld, du Knabe!
Dünk' ich dich alt,
so sollst du Achtung mir bieten.

\Siegfriedspeaks

Das wär' nicht übel!
Solang' ich lebe,
stand mir ein Alter
stets im Wege;
den hab' ich nun fortgefegt.
Stemmst du dort länger
steif dich mir entgegen,
sieh dich vor, sag' ich,

\direct{mit entsprechender Gebärde}

dass du wie Mime nicht fährst!

\direct{Er tritt noch näher an
den Wanderer heran}

Wie siehst du denn aus?
Was hast du gar
für 'nen grossen Hut?
Warum hängt er dir so ins Gesicht?

\Wandererspeaks

\direct{immer ohne seine
Stellung zu verlassen}

Das ist so Wand'rers Weise,
wenn dem Wind entgegen er geht.

\Siegfriedspeaks

\direct{immer näher ihn betrachtend}

Doch darunter fehlt dir ein Auge!
Das schlug dir einer
gewiss schon aus,
dem du zu trotzig
den Weg vertratst?
Mach dich jetzt fort,
sonst könntest du leicht
das andere auch noch verlieren.

\Wandererspeaks

Ich seh', mein Sohn,
wo du nichts weisst,
da weisst du dir leicht zu helfen.
Mit dem Auge,
das als andres mir fehlt,
erblickst du selber das eine,
das mir zum Sehen verblieb.

\Siegfriedspeaks

\direct{der sinnend zugehört hat,
bricht jetzt unwillkürlich
in helles Lachen aus}

Zum Lachen bist du mir lustig!
Doch hör', nun schwatz' ich nicht länger:
geschwind, zeig' mir den Weg,
deines Weges ziehe dann du;
zu nichts andrem
acht' ich dich nütz':
drum sprich, sonst spreng' ich dich fort!

\Wandererspeaks

\direct{weich}

Kenntest du mich,
kühner Spross,
den Schimpf spartest du mir!
Dir so vertraut,
trifft mich schmerzlich dein Dräuen.
Liebt' ich von je
deine lichte Art,
Grauen auch zeugt' ihr
mein zürnender Grimm.
Dem ich so hold bin,
Allzuhehrer,
heut' nicht wecke mir Neid:
er vernichtete dich und mich!

\Siegfriedspeaks

Bleibst du mir stumm,
störrischer Wicht?
Weich' von der Stelle,
denn dorthin, ich weiss,
führt es zur schlafenden Frau.
So wies es mein Vöglein,
das hier erst flüchtig entfloh.

\direct{Es wird schnell wieder ganz finster}

\Wandererspeaks

\direct{in Zorn ausbrechend und
in gebieterischer Stellung}

Es floh dir zu seinem Heil!
Den Herrn der Raben
erriet es hier:
weh' ihm, holen sie's ein!
Den Weg, den es zeigte,
sollst du nicht ziehn!

\Siegfriedspeaks

\direct{tritt mit Verwunderung in
trotziger Stellung zurück}

Hoho! Du Verbieter!
Wer bist du denn,
dass du mir wehren willst?

\Wandererspeaks

Fürchte des Felsens Hüter!
Verschlossen hält
meine Macht die schlafende Maid:
wer sie erweckte,
wer sie gewänne,
machtlos macht' er mich ewig!
Ein Feuermeer
umflutet die Frau,
glühende Lohe
umleckt den Fels:
wer die Braut begehrt,
dem brennt entgegen die Brunst.

\direct{Er winkt mit dem Speere
nach der Felsenhöhe}

Blick' nach der Höh'!
Erlugst du das Licht?
Es wächst der Schein,
es schwillt die Glut;
sengende Wolken,
wabernde Lohe
wälzen sich brennend
und prasselnd herab:
ein Lichtmeer
umleuchtet dein Haupt:

\direct{Mit wachsender Helle zeigt
sich von der Höhe des Felsens
her ein wabernder Feuerschein}

bald frisst und zehrt dich
zündendes Feuer.
Zurück denn, rasendes Kind!

\Siegfriedspeaks

Zurück, du Prahler, mit dir!

\direct{Er schreitet weiter, der Wanderer
stellt sich ihm entgegen}

Dort, wo die Brünste brennen,
zu Brünnhilde muss ich dahin!

\Wandererspeaks

Fürchtest das Feuer du nicht,

\direct{den Speer vorhaltend}

so sperre mein Speer dir den Weg!
Noch hält meine Hand
der Herrschaft Haft:
das Schwert, das du schwingst,
zerschlug einst dieser Schaft:
noch einmal denn
zerspring' es am ew'gen Speer!

\direct{Er streckt den Speer vor}

\Siegfriedspeaks

\direct{das Schwert ziehend}

Meines Vaters Feind!
Find' ich dich hier?
Herrlich zur Rache
geriet mir das!
Schwing' deinen Speer:
in Stücken spalt' ihn mein Schwert!

\direct{Er haut dem Wanderer mit einem Schlage den Speer
in zwei Stücken; ein Blitzstrahl fährt daraus nach der
Felsenhöhe zu, wo von nun an der bisher mattere Schein
in immer helleren Feuerflammen zu lodern beginnt. Starker
Donner, der schnell sich abschwächt, begleitet den Schlag.
Die Speerstücken rollen zu des Wanderers Füssen.
Er rafft sie ruhig auf}

\Wandererspeaks

\direct{zurückweichend}

Zieh hin! Ich kann dich nicht halten!

\direct{Er verschwindet plötzlich
in völliger Finsternis}

\Siegfriedspeaks

Mit zerfocht'ner Waffe
wich mir der Feige?

\direct{Die wachsende Helle der immer
tiefer sich senkenden Feuerwolken
trifft Siegfrieds Blick}

Ha! Wonnige Glut!
Leuchtender Glanz!
Strahlend nun offen
steht mir die Strasse.
Im Feuer mich baden!
Im Feuer zu finden die Braut
Hoho! Hahei!
Jetzt lock' ich ein liebes Gesell!

\direct{Siegfried setzt sein Horn an und stürzt, seine Lockweise
blasend, sich in das wogende Feuer, welches sich,
von der Höhe herabdringend, nun auch über den
Vordergrund ausbreitet. Siegfried, den man bald nicht mehr erblickt,
scheint sich nach der Höhe zu entfernen. Hellstes Leuchten
der Flammen. Danach beginnt die Glut zu erbleichen und
löst sich allmählich in ein immer feineres, wie durch
die Morgenröte beleuchtetes Gewölk auf}

\scene

\StageDir{Das immer zarter gewordene Gewölk hat sich in einen feinen Nebelschleier von rosiger Färbung aufgelöst und zerteilt sich nun in der Weise, dass der Duft sich gänzlich nach oben verzieht und endlich nur noch den heiteren, blauen Tageshimmel erblicken lässt, während am Saume der nun sichtbar werdenden Felsenhöhe ganz die gleiche Szene wie im dritten Aufzug der ``Walküre'' ein morgenrötlicher Nebelschleier haften bleibt, welcher zugleich an die in der Tiefe noch lodernde Zauberlohe erinnert. Die Anordnung der Szene ist durchaus dieselbe wie am Schlusse der ``Walküre'': im Vordergrunde, unter der breitästigen Tanne, liegt Brünnhilde in vollständiger, glänzender Panzerrüstung, mit dem Helm auf dem Haupte, den langen Schild über sich gedeckt, in tiefem Schlafe}

\Siegfriedspeaks

\direct{gelangt von aussen her auf den
felsigen Saum der Höhe und zeigt sich
dort zuerst nur mit dem Oberleibe: so
blickt er lange staunend um sich}

Selige Öde
auf sonniger Höh'!

\direct{Er steigt vollends herauf und
betrachtet, auf einem Felsensteine
des hinteren Abhanges stehend,
mit Verwunderung die Szene.
Er blickt zur Seite in den Tann
und schreitet etwas vor}

Was ruht dort schlummernd
im schattigen Tann?
Ein Ross ist's,
rastend in tiefem Schlaf!

\direct{Langsam näher kommend,
hält er verwundert an, als er noch
aus einiger Entfernung Brünnhildes
Gestalt wahrnimmt}

Was strahlt mir dort entgegen?
Welch glänzendes Stahlgeschmeid?
Blendet mir noch
die Lohe den Blick?

\direct{Er tritt näher hinzu}

Helle Waffen!
Heb' ich sie auf?

\direct{Er hebt den Schild ab und erblickt
Brünnhildes Gestalt, während ihr
Gesicht jedoch noch zum grossen
Teil vom Helm verdeckt ist}

Ha! In Waffen ein Mann:
wie mahnt mich wonnig sein Bild!
Das hehre Haupt
drückt wohl der Helm?
Leichter würd' ihm,
löst' ich den Schmuck.

\direct{Vorsichtig löst er den Helm
und hebt ihn der Schlafenden
vom Haupte ab: langes lockiges
Haar bricht hervor. Siegfried erschrickt}

Ach! Wie schön!

\direct{Er bleibt in den Anblick versunken}

Schimmernde Wolken
säumen in Wellen
den hellen Himmelssee;
leuchtender Sonne
lachendes Bild
strahlt durch das Wogengewölk!

\direct{Er neigt sich tiefer zu der
Schlafenden hinab}

Von schwellendem Atem
schwingt sich die Brust:
brech' ich die engende Brünne?

\direct{Er versucht mit grosser
Behutsamkeit die Brünne zu lösen}

Komm, mein Schwert,
schneide das Eisen!

\direct{Er zieht sein Schwert, durchschneidet
mit zarter Vorsicht die Panzerringe zu
beiden Seiten der ganzen Rüstung
und hebt dann die Brünne und die
Schienen ab, so dass nun Brünnhilde
in einem weichen weiblichen Gewande
vor ihm liegt. Er fährt erschreckt
und staunend auf}

Das ist kein Mann!

\direct{Er starrt mit höchster Aufgeregtheit
auf die Schlafende hin}

Brennender Zauber
zückt mir ins Herz;
feurige Angst
fasst meine Augen:
mir schwankt und schwindelt der Sinn!

\direct{Er gerät in höchste Beklemmung}

Wen ruf' ich zum Heil,
dass er mir helfe?
Mutter! Mutter!
Gedenke mein!

\direct{Er sinkt, wie ohnmächtig, an Brünnhildes
Busen. Langes Schweigen. Dann fährt er
seufzend auf}

Wie weck' ich die Maid,
dass sie ihr Auge mir öffne?
Das Auge mir öffne?
Blende mich auch noch der Blick?
Wagt' es mein Trotz?
Ertrüg' ich das Licht?
Mir schwebt und schwankt
und schwirrt es umher!
Sehrendes Sehnen
zehrt meine Sinne;
am zagenden Herzen
zittert die Hand!
Wie ist mir Feigem?
Ist dies das Fürchten?
O Mutter! Mutter!
Dein mutiges Kind!
Im Schlafe liegt eine Frau:
die hat ihn das Fürchten gelehrt!
Wie end' ich die Furcht?
Wie fass' ich Mut?
Dass ich selbst erwache,
muss die Maid mich erwecken!

\direct{Indem er sich der Schlafenden von
neuem nähert, wird er wieder von
zarteren Empfindungen an ihren
Anblick gefesselt. Er neigt sich
tiefer hinab}

Süss erbebt mir
ihr blühender Mund.
Wie mild erzitternd
mich Zagen er reizt!
Ach! Dieses Atems
wonnig warmes Gedüft!

\direct{wie in Verzweiflung}

Erwache! Erwache!
Heiliges Weib!

\direct{Er starrt auf sie hin}

Sie hört mich nicht.

\direct{gedehnt mit gepresstem,
drängendem Ausdruck}

So saug' ich mir Leben
aus süssesten Lippen,
sollt' ich auch sterbend vergeh'n!

\direct{Er sinkt, wie ersterbend, auf die Schlafende und heftet mit geschlossenen Augen seine Lippen auf ihren Mund. Brünnhilde schlägt die Augen auf. Siegfried fährt auf und bleibt vor ihr stehen. Brünnhilde richtet sich langsam zum Sitze auf. Sie begrüsst mit feierlichen Gebärden der erhobenen Arme ihre Rückkehr zur Wahrnehmung der Erde und des Himmels}

\Brunnhildespeaks

Heil dir, Sonne!
Heil dir, Licht!
Heil dir, leuchtender Tag!
Lang war mein Schlaf;
ich bin erwacht.
Wer ist der Held,
der mich erweckt'?

\Siegfriedspeaks

\direct{von ihrem Blicke und ihrer
Stimme feierlich ergriffen,
steht wie festgebannt}

Durch das Feuer drang ich,
das den Fels umbrann;
ich erbrach dir den festen Helm:
Siegfried bin ich,
der dich erweckt'.

\Brunnhildespeaks

\direct{hoch aufgerichtet sitzend}

Heil euch, Götter!
Heil dir, Welt!
Heil dir, prangende Erde!
Zu End' ist nun mein Schlaf;
erwacht, seh' ich:
Siegfried ist es,
der mich erweckt!

\Siegfriedspeaks

\direct{in erhabenste Verzückung
ausbrechend}

O Heil der Mutter,
die mich gebar;
Heil der Erde,
die mich genährt!
Dass ich das Aug' erschaut,
das jetzt mir Seligem lacht!

\Brunnhildespeaks

\direct{mit grösster Bewegtheit}

O Heil der Mutter,
die dich gebar!
Heil der Erde,
die dich genährt!
Nur dein Blick durfte mich schau'n,
erwachen durft' ich nur dir!

\direct{Beide bleiben voll strahlenden Entzückens
in ihren gegenseitigen Anblick verloren}

O Siegfried! Siegfried!
Seliger Held!
Du Wecker des Lebens,
siegendes Licht!
O wüsstest du, Lust der Welt,
wie ich dich je geliebt!
Du warst mein Sinnen,
mein Sorgen du!
Dich Zarten nährt' ich,
noch eh' du gezeugt;
noch eh' du geboren,
barg dich mein Schild:
so lang' lieb' ich dich, Siegfried!

\Siegfriedspeaks

\direct{leise und schüchtern}

So starb nicht meine Mutter?
Schlief die minnige nur?

\Brunnhildespeaks

\direct{lächelnd, freundlich die
Hand nach ihm ausstreckend}

Du wonniges Kind!
Deine Mutter kehrt dir nicht wieder.
Du selbst bin ich,
wenn du mich Selige liebst.
Was du nicht weisst,
weiss ich für dich;
doch wissend bin ich
nur---weil ich dich liebe!
O Siegfried! Siegfried!
Siegendes Licht!
Dich liebt' ich immer;
denn mir allein
erdünkte Wotans Gedanke
der Gedanke, den ich nie
nennen durfte;
den ich nicht dachte,
sondern nur fühlte;
für den ich focht,
kämpfte und stritt;
für den ich trotzte
dem, der ihn dachte;
für den ich büsste,
Strafe mich band,
weil ich nicht ihn dachte
und nur empfand!
Denn der Gedanke
dürftest du's lösen!
mir war er nur Liebe zu dir!

\Siegfriedspeaks

Wie Wunder tönt,
was wonnig du singst;
doch dunkel dünkt mich der Sinn.
Deines Auges Leuchten
seh' ich licht;
deines Atems Wehen
fühl' ich warm;
deiner Stimme Singen
hör' ich süss:
doch was du singend mir sagst,
staunend versteh' ich's nicht.
Nicht kann ich das Ferne
sinnig erfassen,
wenn alle Sinne
dich nur sehen und fühlen!
Mit banger Furcht
fesselst du mich:
du Einz'ge hast
ihre Angst mich gelehrt.
Den du gebunden
in mächtigen Banden,
birg meinen Mut mir nicht mehr!

\direct{Er verweilt in grosser Aufregung,
sehnsuchtsvollen Blick auf sie heftend}

\Brunnhildespeaks

\direct{wendet sanft das Haupt zur Seite
und richtet ihren Blick nach dem Tann}

Dort seh' ich Grane,
mein selig Ross:
wie weidet er munter,
der mit mir schlief!
Mit mir hat ihn Siegfried erweckt.

\Siegfriedspeaks

\direct{in der vorigen Stellung verbleibend}

Auf wonnigem Munde
weidet mein Auge:
in brünstigem Durst
doch brennen die Lippen,
dass der Augen Weide sie labe!

\Brunnhildespeaks

\direct{deutet ihm mit der Hand nach
ihren Waffen, die sie gewahrt}

Dort seh' ich den Schild,
der Helden schirmte;
dort seh' ich den Helm,
der das Haupt mir barg:
er schirmt, er birgt mich nicht mehr!

\Siegfriedspeaks

Eine selige Maid
versehrte mein Herz;
Wunden dem Haupte
schlug mir ein Weib:
ich kam ohne Schild und Helm!

\Brunnhildespeaks

\direct{mit gesteigertem Wehmut}

Ich sehe der Brünne
prangenden Stahl:
ein scharfes Schwert
schnitt sie entzwei;
von dem maidlichen Leibe
löst' es die Wehr:
ich bin ohne Schutz und Schirm,
ohne Trutz ein trauriges Weib!

\Siegfriedspeaks

Durch brennendes Feuer
fuhr ich zu dir!
Nicht Brünne noch Panzer
barg meinen Leib:
nun brach die Lohe
mir in die Brust.
Es braust mein Blut
in blühender Brunst;
ein zehrendes Feuer
ist mir entzündet:
die Glut, die Brünnhilds
Felsen umbrann,
die brennt mir nun in der Brust!
O Weib, jetzt lösche den Brand!
Schweige die schäumende Glut!

\direct{Er hat sie heftig umfasst: sie springt auf,
wehrt ihm mit der höchsten Kraft der Angst,
und entflieht nach der anderen Seite}

\Brunnhildespeaks

Kein Gott nahte mir je!
Der Jungfrau neigten
scheu sich die Helden:
heilig schied sie aus Walhall!
Wehe! Wehe!
Wehe der Schmach,
der schmählichen Not!
Verwundet hat mich,
der mich erweckt!
Er erbrach mir Brünne und Helm:
Brünnhilde bin ich nicht mehr!

\Siegfriedspeaks

Noch bist du mir
die träumende Maid:
Brünnhildes Schlaf
brach ich noch nicht.
Erwache, sei mir ein Weib!

\Brunnhildespeaks

\direct{in Betäubung}

Mir schwirren die Sinne,
mein Wissen schweigt:
soll mir die Weisheit schwinden?

\Siegfriedspeaks

Sangst du mir nicht,
dein Wissen sei
das Leuchten der Liebe zu mir?

\Brunnhildespeaks

\direct{vor sich hinstarrend}

Trauriges Dunkel
trübt meinen Blick;
mein Auge dämmert,
das Licht verlischt:
Nacht wird's um mich.
Aus Nebel und Grau'n
windet sich wütend
ein Angstgewirr:
Schrecken schreitet
und bäumt sich empor!

\direct{Sie birgt heftig die Augen
mit beiden Händen}

\Siegfriedspeaks

\direct{indem er ihr sanft die Hände
von den Augen löst}

Nacht umfängt
gebund'ne Augen.
Mit den Fesseln schwindet
das finstre Grau'n.
Tauch' aus dem Dunkel und sieh:
sonnenhell leuchtet der Tag!

\Brunnhildespeaks

\direct{in höchster Ergriffenheit}

Sonnenhell
leuchtet der Tag meiner Schmach!
O Siegfried! Siegfried!
Sieh' meine Angst!

\direct{Ihre Miene verrät, dass ihr ein
anmutiges Bild vor die Seele tritt,
von welchem ab sie den Blick mit
Sanftmut wieder auf Siegfried richtet}

Ewig war ich,
ewig bin ich,
ewig in süss
sehnender Wonne,
doch ewig zu deinem Heil!
O Siegfried! Herrlicher!
Hort der Welt!
Leben der Erde!
Lachender Held!
Lass, ach lass,
lasse von mir!
Nahe mir nicht
mit der wütenden Nähe!
Zwinge mich nicht
mit dem brechenden Zwang,
zertrümmre die Traute dir nicht!
Sahst du dein Bild
im klaren Bach?
Hat es dich Frohen erfreut?
Rührtest zur Woge
das Wasser du auf,
zerflösse die klare
Fläche des Bachs:
dein Bild sähst du nicht mehr,
nur der Welle schwankend Gewog'!
So berühre mich nicht,
trübe mich nicht!
Ewig licht
lachst du selig dann
aus mir dir entgegen,
froh und heiter ein Held!
O Siegfried!
Leuchtender Spross!
Liebe dich,
und lasse von mir:
vernichte dein Eigen nicht!

\Siegfriedspeaks

Dich lieb' ich:
o liebtest mich du!
Nicht hab' ich mehr mich:
o, hätte ich dich!
Ein herrlich Gewässer
wogt vor mir;
mit allen Sinnen
seh' ich nur sie,
die wonnig wogende Welle:
Brach sie mein Bild,
so brenn' ich nun selbst,
sengende Glut
in der Flut zu kühlen;
ich selbst, wie ich bin,
spring' in den Bach:
o, dass seine Wogen
mich selig verschlängen,
mein Sehnen schwänd' in der Flut!
Erwache, Brünnhilde!
Wache, du Maid!
Lache und lebe,
süsseste Lust!
Sei mein! Sei mein! Sei mein!

\Brunnhildespeaks

\direct{sehr innig}

O Siegfried! Dein
war ich von je!

\Siegfriedspeaks

\direct{feurig}

Warst du's von je,
so sei es jetzt!

\Brunnhildespeaks

Dein werd' ich
ewig sein!

\Siegfriedspeaks

Was du sein wirst,
sei es mir heut'!
Fasst dich mein Arm,
umschling' ich dich fest;
schlägt meine Brust
brünstig die deine;
zünden die Blicke,
zehren die Atem sich;
Aug' in Auge,
Mund an Mund:
dann bist du mir,
was bang du mir warst und wirst!
Dann brach sich die brennende Sorge,
ob jetzt Brünnhilde mein?

\direct{Er hat sie umfasst}

\Brunnhildespeaks

Ob jetzt ich dein?
Göttliche Ruhe
rast mir in Wogen;
keuschestes Licht
lodert in Gluten:
himmlisches Wissen
stürmt mir dahin,
Jauchzen der Liebe
jagt es davon!
Ob jetzt ich dein?
Siegfried! Siegfried!
Siehst du mich nicht?
Wie mein Blick dich verzehrt,
erblindest du nicht?
Wie mein Arm dich presst,
entbrennst du mir nicht?
Wie in Strömen mein Blut
entgegen dir stürmt,
das wilde Feuer,
fühlst du es nicht?
Fürchtest du, Siegfried,
fürchtest du nicht
das wild wütende Weib?

\direct{Sie umfasst ihn heftig}

\Siegfriedspeaks

\direct{in freudigem Schreck}

Ha!
Wie des Blutes Ströme sich zünden,
wie der Blicke Strahlen sich zehren,
Wie die Arme brünstig sich pressen,
kehrt mir zurück
mein kühner Mut,
und das Fürchten, ach!
Das ich nie gelernt,
das Fürchten, das du
mich kaum gelehrt:
das Fürchten, mich dünkt
ich Dummer vergass es nun ganz!

\direct{Er hat bei den letzten Worten
Brünnhilde unwillkürlich losgelassen}

\Brunnhildespeaks

\direct{im höchsten Liebesjubel wild auflachend}

O kindischer Held!
O herrlicher Knabe!
Du hehrster Taten
töriger Hort!
Lachend muss ich dich lieben,
lachend will ich erblinden,
lachend lass uns verderben
lachend zu Grunde gehn!
Fahr' hin, Walhalls
leuchtende Welt!
Zerfall in Staub
deine stolze Burg!
Leb' wohl, prangende
Götterpracht!
End' in Wonne,
du ewig Geschlecht!
Zerreisst, ihr Nornen,
das Runenseil!
Götterdämm'rung,
dunkle herauf!
Nacht der Vernichtung,
neble herein!
Mir strahlt zur Stunde
Siegfrieds Stern;
er ist mir ewig,
ist mir immer,
Erb' und Eigen,
ein' und all':
leuchtende Liebe,
lachender Tod!

\Siegfriedspeaks

Lachend erwachst
du Wonnige mir:
Brünnhilde lebt,
Brünnhilde lacht!
Heil dem Tage,
der uns umleuchtet!

Heil der Sonne,
die uns bescheint!

Heil dem Licht,
das ser Nacht enttaucht!

Heil der Welt,
der Brünnhilde lebt!
Sie wacht, sie lebt,
sie lacht mir entgegen.
Prangend strahlt
mir Brünnhildes Stern!
Sie ist mir ewig,
ist mir immer,
Erb' und Eigen,
ein' und all':
leuchtende Liebe,
lachender Tod!
\end{drama}


%% Götterdämmerung has some scenes outside of
%% any act. We thus also have to reset the Scene
%% counter ourselves, else we will just continue
%% numbering as if we were still in Siegfried, Act 3.
\renewcommand{\playname}{Götterdämmerung}
\setcounter{act}{0}
\setcounter{scene}{0}
\part{G\"otterd\"ammerung}
 \begin{drama}
   \let\oldthescene\thescene
   \let\oldscenename\scenename
   \let\oldscenecontentsline\scenecontentsline
   \renewcommand{\scenecontentsline}{}
   \renewcommand{\thescene}{}
   \renewcommand{\scenename}{}
   \let\oldsaythescene\saythescene
   \renewcommand{\saythescene}{Vorspiel}
\Scene{Vorspiel}

\StageDir{Auf dem Walkürenfelsen
  
Die Szene ist dieselbe wie am Schlusse des zweiten Tages. Nacht. Aus der Tiefe des Hintergrundes leuchtet Feuerschein. Die drei Nornen, hohe Frauengestalten in langen, dunklen und schleierartigen Faltengewändern. Die erste (älteste) lagert im Vordergrunde rechts unter der breitästigen Tanne; die zweite (jüngere) ist an einer Steinbank vor dem Felsengemache hingestreckt; die dritte (jüngste) sitzt in der Mitte des Hintergrundes auf einem Felssteine des Höhensaumes. Eine Zeitlang herrscht düsteres Schweigen.}

\DieErsteNornspeaks


\direct{ohne sich zu bewegen}

Welch Licht leuchtet dort?

\DieZweiteNornspeaks

Dämmert der Tag schon auf?
 

\DieDritteNornspeaks

Loges Heer lodert feurig um den Fels.
Noch ist's Nacht.
Was spinnen und singen wir nicht?
 

\DieZweiteNornspeaks


\direct{zu der ersten}

Wollen wir spinnen und singen,
woran spannst du das Seil?
 

\DieErsteNornspeaks


\direct{erhebt sich, während sie ein goldenes Seil von sich löst und mit dem einen Ende es an einen Ast der Tanne knüpft}

So gut und schlimm es geh',
schling' ich das Seil und singe.
An der Weltesche wob ich einst,
da groß und stark dem Stamm entgrünte
weihlicher Äste Wald.
Im kühlen Schatten rauscht' ein Quell,
Weisheit raunend rann sein Gewell';
da sang ich heil'gen Sinn.
Ein kühner Gott
trat zum Trunk an den Quell;
seiner Augen eines
zahlt' er als ewigen Zoll.
Von der Weltesche
brach da Wotan einen Ast;
eines Speeres Schaft
entschnitt der Starke dem Stamm.
In langer Zeiten Lauf
zehrte die Wunde den Wald;
falb fielen die Blätter,
dürr darbte der Baum,
traurig versiegte des Quelles Trank:
trüben Sinnes ward mein Gesang.
Doch, web' ich heut'
an der Weltesche nicht mehr,
muß mir die Tanne
taugen zu fesseln das Seil:
singe, Schwester, dir werf' ich's zu.
Weißt du, wie das wird?
 

\DieZweiteNornspeaks


\direct{windet das zugeworfene Seil um einen hervorspringenden Felsstein am Eingange des Gemaches}

Treu beratner Verträge Runen
schnitt Wotan in des Speeres Schaft:
den hielt er als Haft der Welt.
Ein kühner Held
zerhieb im Kampfe den Speer;
in Trümmer sprang
der Verträge heiliger Haft.
Da hieß Wotan Walhalls Helden
der Weltesche welkes Geäst
mit dem Stamm in Stücke zu fällen.
Die Esche sank;
ewig versiegte der Quell!
Fessle ich heut'
an den scharfen Fels das Seil:
singe, Schwester, dir werf' ich's zu.
Weißt du, wie das wird?
 

\DieDritteNornspeaks


\direct{das Seil auffangend und dessen Ende hinter sich werfend}

Es ragt die Burg, von Riesen gebaut:
mit der Götter und Helden heiliger Sippe
sitzt dort Wotan im Saal.
Gehau'ner Scheite hohe Schicht
ragt zuhauf rings um die Halle:
die Weltesche war dies einst!
Brennt das Holz
heilig brünstig und hell,
sengt die Glut
sehrend den glänzenden Saal:
der ewigen Götter Ende
dämmert ewig da auf.
Wisset ihr noch,
so windet von neuem das Seil;
von Norden wieder werf' ich's dir nach.
 


\direct{Sie wirft das Seil der zweiten Norn zu}


\DieZweiteNornspeaks


\direct{schwingt das Seil der ersten hin, die es vom Zweige löst und es an einen andern Ast wieder anknüpft}

Spinne, Schwester, und singe!
 

\DieErsteNornspeaks


\direct{nach hinten blickend}

Dämmert der Tag?
Oder leuchtet die Lohe?
Getrübt trügt sich mein Blick;
nicht hell eracht' ich das heilig Alte,
da Loge einst entbrannte in lichter Brunst.
Weißt du, was aus ihm ward?
 

\DieZweiteNornspeaks


\direct{das zugeworfene Seil wieder um den Stein windend}

Durch des Speeres Zauber
zähmte ihn Wotan;
Räte raunt' er dem Gott.
An des Schaftes Runen,
frei sich zu raten,
nagte zehrend sein Zahn:
da, mit des Speeres
zwingender Spitze
bannte ihn Wotan,
Brünnhildes Fels zu umbrennen.
Weißt du, was aus ihm wird?
 

\DieDritteNornspeaks


\direct{das zugeschwungene Seil wieder hinter sich werfend}

Des zerschlagnen Speeres
stechende Splitter
taucht einst Wotan
dem Brünstigen tief in die Brust:
zehrender Brand zündet da auf;
den wirft der Gott in der Weltesche
zuhauf geschichtete Scheite.
 


\direct{Sie wirft das Seil zurück, die zweite Norn windet es auf und wirft es der ersten wieder zu}


\DieZweiteNornspeaks

Wollt ihr wissen,
wann das wird?
Schwinget, Schwestern, das Seil!
 

\DieErsteNornspeaks


\direct{das Seil von neuem anknüpfend}

Die Nacht weicht;
nichts mehr gewahr' ich:
des Seiles Fäden find' ich nicht mehr;
verflochten ist das Geflecht.
Ein wüstes Gesicht wirrt mir wütend den Sinn:
das Rheingold raubte Alberich einst:
weißt du, was aus ihm ward?
 

\DieZweiteNornspeaks


\direct{mit mühevoller Hand das Seil um den zackigen Stein des Gemaches windend}

Des Steines Schärfe schnitt in das Seil;
nicht fest spannt mehr der Fäden Gespinst;
verwirrt ist das Geweb'.
Aus Not und Neid
ragt mir des Niblungen Ring:
ein rächender Fluch
nagt meiner Fäden Geflecht.
Weißt du, was daraus wird?
 

\DieDritteNornspeaks


\direct{das zugeworfene Seil hastig fassend}

Zu locker das Seil, mir langt es nicht.
Soll ich nach Norden neigen das Ende,
straffer sei es gestreckt!

\direct{Sie zieht gewaltsam das Seil an: dieses reißt in der Mitte}

Es riß!
 

\DieZweiteNornspeaks

Es riß!
 

\DieErsteNornspeaks

Es riß!
 


\direct{Erschreckt sind die drei Nornen aufgefahren und nach der Mitte der Bühne zusammengetreten: sie fassen die Stücke des zerrissenen Seiles und binden damit ihre Leiber aneinander}


\speaker{Die Drei Nornen}
Zu End' ewiges Wissen!
Der Welt melden Weise nichts mehr.
Hinab! Zur Mutter! Hinab!
 


\direct{Sie verschwinden}


\direct{Tagesgrauen. Wachsende Morgenröte, immer schwächeres Leuchten des Feuerscheines aus der Tiefe}


\direct{Tagesgrauen---Sonnenaufgang---Heller Tag.}


\direct{treten aus dem Steingemache auf. Siegfried ist in vollen Waffen, Brünnhilde führt ihr Roß am Zaume}


\Brunnhildespeaks

Zu neuen Taten, teurer Helde,
wie liebt' ich dich, ließ ich dich nicht?
Ein einzig' Sorgen läßt mich säumen:
daß dir zu wenig mein Wert gewann!
Was Götter mich wiesen, gab ich dir:
heiliger Runen reichen Hort;
doch meiner Stärke magdlichen Stamm
nahm mir der Held, dem ich nun mich neige.
Des Wissens bar, doch des Wunsches voll:
an Liebe reich, doch ledig der Kraft:
mögst du die Arme nicht verachten,
die dir nur gönnen, nicht geben mehr kann!
 

\Siegfriedspeaks

Mehr gabst du, Wunderfrau,
als ich zu wahren weiß.
Nicht zürne, wenn dein Lehren
mich unbelehret ließ!
Ein Wissen doch wahr' ich wohl:
daß mir Brünnhilde lebt;
eine Lehre lernt' ich leicht:
Brünnhildes zu gedenken!
 

\Brunnhildespeaks

Willst du mir Minne schenken,
gedenke deiner nur,
gedenke deiner Taten:
gedenk' des wilden Feuers,
das furchtlos du durchschrittest,
da den Fels es rings umbrann.
 

\Siegfriedspeaks

Brünnhilde zu gewinnen!
 

\Brunnhildespeaks

Gedenk' der beschildeten Frau,
die in tiefem Schlaf du fandest,
der den festen Helm du erbrachst.
 

\Siegfriedspeaks

Brünnhilde zu erwecken!
 

\Brunnhildespeaks

Gedenk' der Eide, die uns einen;
gedenk' der Treue, die wir tragen;
gedenk' der Liebe, der wir leben:
Brünnhilde brennt dann ewig
heilig dir in der Brust!
 


\direct{Sie umarmt Siegfried}


\Siegfriedspeaks

Laß ich, Liebste, dich hier
in der Lohe heiliger Hut;

\direct{Er hat den Ring Alberichs von seinem Finger gezogen und reicht ihn jetzt Brünnhilde dar}

zum Tausche deiner Runen
reich' ich dir diesen Ring.
Was der Taten je ich schuf,
des Tugend schließt er ein.
Ich erschlug einen wilden Wurm,
der grimmig lang' ihn bewacht.
Nun wahre du seine Kraft
als Weihegruß meiner Treu'!
 

\Brunnhildespeaks


\direct{voll Entzücken den Ring sich ansteckend}

Ihn geiz' ich als einziges Gut!
Für den Ring nimm nun auch mein Roß!
Ging sein Lauf mit mir
einst kühn durch die Lüfte,
mit mir verlor es die mächt'ge Art;
über Wolken hin auf blitzenden Wettern
nicht mehr schwingt es sich mutig des Wegs;
doch wohin du ihn führst---
sei es durchs Feuer---
grauenlos folgt dir Grane;
denn dir, o Helde,
soll er gehorchen!
Du hüt' ihn wohl;
er hört dein Wort:
o bringe Grane oft Brünnhildes Gruß!
 

\Siegfriedspeaks

Durch deine Tugend allein
soll so ich Taten noch wirken?
Meine Kämpfe kiesest du,
meine Siege kehren zu dir:
auf deines Rosses Rücken,
in deines Schildes Schirm,
nicht Siegfried acht' ich mich mehr,
ich bin nur Brünnhildes Arm.
 

\Brunnhildespeaks

O wäre Brünnhild' deine Seele!
 

\Siegfriedspeaks

Durch sie entbrennt mir der Mut.
 

\Brunnhildespeaks

So wärst du Siegfried und Brünnhild'?
 

\Siegfriedspeaks

Wo ich bin, bergen sich beide.
 

\Brunnhildespeaks


\direct{lebhaft}

So verödet mein Felsensaal?
 

\Siegfriedspeaks

Vereint, faßt er uns zwei!
 

\Brunnhildespeaks


\direct{in großer Ergriffenheit}

O heilige Götter!
Hehre Geschlechter!
Weidet eu'r Aug' an dem weihvollen Paar!
Getrennt---wer will es scheiden?
Geschieden---trennt es sich nie!
 

\Siegfriedspeaks

Heil dir, Brünnhilde, prangender Stern!
Heil, strahlende Liebe!
 

\Brunnhildespeaks

Heil dir, Siegfried, siegendes Licht!
Heil, strahlendes Leben!
 

\speaker{Beide}
Heil! Heil! Heil! Heil!
 

\StageDir{(Siegfried geleitet schnell das Roß dem Felsenabhange zu, wohin ihm Brünnhilde folgt. Siegfried ist mit dem Rosse hinter dem Felsenvorsprunge abwärts verschwunden, so daß der Zuschauer ihn nicht mehr sieht: Brünnhilde steht so plötzlich allein am Abhange und blickt Siegfried in die Tiefe nach. Man hört Siegfrieds Horn aus der Tiefe. Brünnhilde lauscht. Sie tritt weiter auf den Abhang hinaus und erblickt Siegfried nochmals in der Tiefe: sie winkt ihm mit entzückter Gebärde zu. Aus ihrem freudigen Lächeln deutet sich der Anblick des lustig davonziehenden Helden. Der Vorhang fällt schnell).
  
(Das Orchester nimmt die Weise des Hornes auf und führt sie in einem kräftigen Satze durch. Darauf beginnt sogleich der erste Aufzug)}


\let\saythescene\oldsaythescene
\let\thescene\oldthescene
\let\scenename\oldscenename
\let\scenecontentsline\oldscenecontentsline
\act

\scene

\StageDir{Die Halle der Gibichungen am Rhein.
  
Diese ist dem Hintergrunde zu ganz offen; den Hintergrund selbst nimmt ein freier Uferraum bis zum Flusse hin ein; felsige Anhöhen umgrenzen das Ufer.}
 

\direct{Gunther und Gutrune auf dem Hochsitze zur Seite, vor welchem ein Tisch mit Trinkgerät steht; davor sitzt Hagen}


\Guntherspeaks

Nun hör', Hagen, sage mir, Held:
sitz' ich herrlich am Rhein,
Gunther zu Gibichs Ruhm?
 

\Hagenspeaks

Dich echt genannten acht' ich zu neiden:
die beid' uns Brüder gebar,
Frau Grimhild' hieß mich's begreifen.
 

\Guntherspeaks

Dich neide ich: nicht neide mich du!
Erbt' ich Erstlingsart,
Weisheit ward dir allein:
Halbbrüderzwist bezwang sich nie besser.
Deinem Rat nur red' ich Lob,
frag' ich dich nach meinem Ruhm.
 

\Hagenspeaks

So schelt' ich den Rat,
da schlecht noch dein Ruhm;
denn hohe Güter weiß ich,
die der Gibichung noch nicht gewann.
 

\Guntherspeaks

Verschwiegest du sie,
so schelt' auch ich.
 

\Hagenspeaks

In sommerlich reifer Stärke
seh' ich Gibichs Stamm,
dich, Gunther, unbeweibt,
dich, Gutrun', ohne Mann.
 


\direct{Gunther und Gutrune sind in schweigendes Sinnen verloren}


\Guntherspeaks

Wen rätst du nun zu frein,
daß unsrem Ruhm' es fromm'?
 

\Hagenspeaks

Ein Weib weiß ich,
das herrlichste der Welt:
auf Felsen hoch ihr Sitz;
ein Feuer umbrennt ihren Saal;
nur wer durch das Feuer bricht,
darf Brünnhildes Freier sein.
 

\Guntherspeaks

Vermag das mein Mut zu bestehn?
 

\Hagenspeaks

Einem Stärkren noch ist's nur bestimmt.
 

\Guntherspeaks

Wer ist der streitlichste Mann?
 

\Hagenspeaks

Siegfried, der Wälsungen Sproß:
der ist der stärkste Held.
Ein Zwillingspaar,
von Liebe bezwungen,
Siegmund und Sieglinde,
zeugten den echtesten Sohn.
Der im Walde mächtig erwuchs,
den wünsch' ich Gutrun' zum Mann.
 

\Gutrunespeaks


\direct{schüchtern beginnend}

Welche Tat schuf er so tapfer,
daß als herrlichster Held er genannt?
 

\Hagenspeaks

Vor Neidhöhle den Niblungenhort
bewachte ein riesiger Wurm:
Siegfried schloß ihm den freislichen Schlund,
erschlug ihn mit siegendem Schwert.
Solch ungeheurer Tat
enttagte des Helden Ruhm.
 

\Guntherspeaks


\direct{in Nachsinnen}

Vom Niblungenhort vernahm ich:
er birgt den neidlichsten Schatz?
 

\Hagenspeaks

Wer wohl ihn zu nützen wüßt',
dem neigte sich wahrlich die Welt.
 

\Guntherspeaks

Und Siegfried hat ihn erkämpft?
 

\Hagenspeaks

Knecht sind die Niblungen ihm.
 

\Guntherspeaks

Und Brünnhild' gewänne nur er?
 

\Hagenspeaks

Keinem andren wiche die Brunst.
 

\Guntherspeaks


\direct{unwillig sich vom Sitze erhebend}

Wie weckst du Zweifel und Zwist!
Was ich nicht zwingen soll,
darnach zu verlangen machst du mir Lust?
 


\direct{Er schreitet bewegt in der Halle auf und ab. Hagen, ohne seinen Sitz zu verlassen, hält Gunther, als dieser wieder in seine Nähe kommt, durch einen geheimnisvollen Wink fest.}


\Hagenspeaks

Brächte Siegfried die Braut dir heim,
wär' dann nicht Brünnhilde dein?
 

\Guntherspeaks


\direct{wendet sich wieder zweifelnd und unmutig ab}

Was zwänge den frohen Mann,
für mich die Braut zu frein?
 

\Hagenspeaks


\direct{wie vorher}

Ihn zwänge bald deine Bitte,
bänd' ihn Gutrun' zuvor.
 

\Gutrunespeaks

Du Spötter, böser Hagen!
Wie sollt' ich Siegfried binden?
Ist er der herrlichste Held der Welt,
der Erde holdeste Frauen
friedeten längst ihn schon.
 

\Hagenspeaks


\direct{sehr vertraulich zu Gutrune hinneigend}

Gedenk' des Trankes im Schrein;

\direct{heimlicher}

vertraue mir, der ihn gewann:
den Helden, des du verlangst,
bindet er liebend an dich.

\direct{Gunther ist wieder an den Tisch getreten und hört, auf ihn gelehnt, jetzt aufmerksam zu}

Träte nun Siegfried ein,
genöss' er des würzigen Tranks,
daß vor dir ein Weib er ersah,
daß je ein Weib ihm genaht,
vergessen müßt' er des ganz.
Nun redet: wie dünkt euch Hagens Rat?
 

\Guntherspeaks


\direct{lebhaft auffahrend}

Gepriesen sei Grimhild',
die uns den Bruder gab!
 

\Gutrunespeaks

Möcht' ich Siegfried je ersehn!
 

\Guntherspeaks

Wie suchten wir ihn auf?
 


\direct{Ein Horn auf dem Theater klingt aus dem Hintergrunde von links her. Hagen lauscht}


\Hagenspeaks

Jagt er auf Taten wonnig umher,
zum engen Tann wird ihm die Welt:
wohl stürmt er in rastloser Jagd
auch zu Gibichs Strand an den Rhein.
 

\Guntherspeaks

Willkommen hieß' ich ihn gern!

\direct{Horn auf dem Theater, näher, aber immer noch fern. Beide lauschen}

Vom Rhein ertönt das Horn.
 

\Hagenspeaks


\direct{ist an das Ufer gegangen, späht den Fluß hinab und ruft zurück}

In einem Nachen Held und Roß!
Der bläst so munter das Horn!

\direct{Gunther bleibt auf halbem Wege lauschend zurück}

Ein gemächlicher Schlag,
wie von müßiger Hand,
treibt jach den Kahn wider den Strom;
so rüstiger Kraft in des Ruders Schwung
rühmt sich nur der, der den Wurm erschlug.
Siegfried ist es, sicher kein andrer!
 

\Guntherspeaks

Jagt er vorbei?
 

\Hagenspeaks


\direct{durch die hohlen Hände nach dem Flusse rufend}

Hoiho! Wohin,
du heitrer Held?
 

\speaker{\Siegfried (Stimme)}

\direct{aus der Ferne, vom Flusse her}

Zu Gibichs starkem Sohne.
 

\Hagenspeaks

Zu seiner Halle entbiet' ich dich.
 


\direct{Siegfried erscheint im Kahn am Ufer}

Hieher! Hier lege an!

\scene

\StageDir{Siegfried legt mit dem Kahne an und springt, nachdem Hagen den Kahn mit der Kette am Ufer festgeschlossen hat, mit dem Rosse auf den Strand}


\Hagenspeaks

Heil! Siegfried, teurer Held!
 


\direct{Gunther ist zu Hagen an das Ufer getreten. Gutrune blickt vom Hochsitze aus in staunender Bewunderung auf Siegfried. Gunther will freundlichen Gruß bieten. Alle sind in gegenseitiger stummer Betrachtung gefesselt}


\Siegfriedspeaks


\direct{auf sein Roß gelehnt, bleibt ruhig am Kahne stehen}

Wer ist Gibichs Sohn?
 

\Guntherspeaks

Gunther, ich, den du suchst.
 

\Siegfriedspeaks

Dich hört' ich rühmen weit am Rhein:
nun ficht mit mir, oder sei mein Freund!
 

\Guntherspeaks

Laß den Kampf!
Sei willkommen!
 

\Siegfriedspeaks


\direct{sieht sich ruhig um}

Wo berg' ich mein Roß?
 

\Hagenspeaks

Ich biet' ihm Rast.
 

\Siegfriedspeaks


\direct{zu Hagen gewendet}

Du riefst mich Siegfried:
sahst du mich schon?
 

\Hagenspeaks

Ich kannte dich nur an deiner Kraft.
 

\Siegfriedspeaks


\direct{indem er an Hagen das Roß übergibt}

Wohl hüte mir Grane! Du hieltest nie
von edlerer Zucht am Zaume ein Roß.
 


\direct{Hagen führt das Roß rechts hinter die Halle ab. Während Siegfried ihm gedankenvoll nachblickt, entfernt sich auch Gutrune, durch einen Wink Hagens bedeutet, von Siegfried unbemerkt, nach links durch eine Tür in ihr Gemach}


\direct{Gunther schreitet mit Siegfried, den er dazu einlädt, in die Halle vor}


\Guntherspeaks

Begrüße froh, o Held,
die Halle meines Vaters;
wohin du schreitest,
was du ersiehst,
das achte nun dein Eigen:
dein ist mein Erbe, Land und Leut',
hilf, mein Leib, meinem Eide!
Mich selbst geb' ich zum Mann.
 

\Siegfriedspeaks

Nicht Land noch Leute biete ich,
noch Vaters Haus und Hof:
einzig erbt' ich den eignen Leib;
lebend zehr' ich den auf.
Nur ein Schwert hab' ich,
selbst geschmiedet:
hilf, mein Schwert, meinem Eide!
Das biet' ich mit mir zum Bund.
 

\Hagenspeaks


\direct{der zurückgekommen ist und jetzt hinter Siegfried steht}

Doch des Niblungenhortes
nennt die Märe dich Herrn?
 

\Siegfriedspeaks


\direct{sich zu Hagen umwendend}

Des Schatzes vergaß ich fast:
so schätz' ich sein müß'ges Gut!
In einer Höhle ließ ich's liegen,
wo ein Wurm es einst bewacht'.
 

\Hagenspeaks

Und nichts entnahmst du ihm?
 

\Siegfriedspeaks


\direct{auf das stählerne Netzgewirk deutend, das er im Gürtel hängen hat}

Dies Gewirk, unkund seiner Kraft.
 

\Hagenspeaks

Den Tarnhelm kenn' ich,
der Niblungen künstliches Werk:
er taugt, bedeckt er dein Haupt,
dir zu tauschen jede Gestalt;
verlangt dich's an fernsten Ort,
er entführt flugs dich dahin.
Sonst nichts entnahmst du dem Hort?
 

\Siegfriedspeaks

Einen Ring.
 

\Hagenspeaks

Den hütest du wohl?
 

\Siegfriedspeaks

Den hütet ein hehres Weib.
 

\Hagenspeaks


\direct{für sich}

Brünnhild'!...
 

\Guntherspeaks

Nicht, Siegfried, sollst du mir tauschen:
Tand gäb' ich für dein Geschmeid,
nähmst all' mein Gut du dafür.
Ohn' Entgelt dien' ich dir gern.
 


\direct{Hagen ist zu Gutrunes Türe gegangen und öffnet sie jetzt. Gutrune tritt heraus, sie trägt ein gefülltes Trinkhorn und naht damit Siegfried}


\Gutrunespeaks

Willkommen, Gast, in Gibichs Haus!
Seine Tochter reicht dir den Trank.
 

\Siegfriedspeaks


\direct{neigt sich ihr freundlich und ergreift das Horn; er hält es gedankenvoll vor sich hin und sagt leise}

Vergäß' ich alles, was du mir gabst,
von einer Lehre lass' ich doch nie:
den ersten Trunk zu treuer Minne,
Brünnhilde, bring' ich dir!
 


\direct{Er setzt das Trinkhorn an und trinkt in einem langen Zuge. Er reicht das Horn an Gutrune zurück, die verschämt und verwirrt ihre Augen vor ihm niederschlägt}


\Siegfriedspeaks


\direct{heftet den Blick mit schnell entbrannter Leidenschaft auf sie}

Die so mit dem Blitz den Blick du mir sengst,
was senkst du dein Auge vor mir?
 


\direct{Gutrune schlägt errötend das Auge zu ihm auf}


\Siegfriedspeaks

Ha, schönstes Weib!
Schließe den Blick;
das Herz in der Brust
brennt mir sein Strahl:
zu feurigen Strömen fühl' ich
ihn zehrend zünden mein Blut!

\direct{mit bebender Stimme}

Gunther, wie heißt deine Schwester?
 

\Guntherspeaks

Gutrune.
 

\Siegfriedspeaks


\direct{leise}

Sind's gute Runen,
die ihrem Aug' ich entrate?

\direct{Er faßt Gutrune mit feurigem Ungestüm bei der Hand}

Deinem Bruder bot ich mich zum Mann:
der Stolze schlug mich aus;
trügst du, wie er, mir Übermut,
böt' ich mich dir zum Bund?
 


\direct{Gutrune trifft unwillkürlich auf Hagens Blick. Sie neigt demütig das Haupt, und mit einer Gebärde, als fühle sie sich seiner nicht wert, verläßt sie schwankenden Schrittes wieder die Halle}


\Siegfriedspeaks


\direct{von Hagen und Gunther aufmerksam beobachtet, blickt ihr, wie festgezaubert, nach; dann, ohne sich umzuwenden, fragt er}

Hast du, Gunther, ein Weib?
 

\Guntherspeaks

Nicht freit' ich noch,
und einer Frau soll ich mich schwerlich freun!
Auf eine setzt' ich den Sinn,
die kein Rat mir je gewinnt.
 

\Siegfriedspeaks


\direct{wendet sich lebhaft zu Gunther}

Was wär' dir versagt, steh' ich zu dir?
 

\Guntherspeaks

Auf Felsen hoch ihr Sitz---
 

\Siegfriedspeaks


\direct{mit verwunderungsvoller Hast einfallend}

``Auf Felsen hoch ihr Sitz;''
 

\Guntherspeaks

ein Feuer umbrennt den Saal---
 

\Siegfriedspeaks

``ein Feuer umbrennt den Saal''... ?
 

\Guntherspeaks

Nur wer durch das Feuer bricht---
 

\Siegfriedspeaks


\direct{mit der heftigsten Anstrengung, um eine Erinnerung festzuhalten}

``Nur wer durch das Feuer bricht''... ?
 

\Guntherspeaks

---darf Brünnhildes Freier sein.
 


\direct{Siegfried drückt durch eine Gebärde aus, daß bei Nennung von Brünnhildes Namen die Erinnerung ihm vollends ganz schwindet}


\Guntherspeaks

Nun darf ich den Fels nicht erklimmen;
das Feuer verglimmt mir nie!
 

\Siegfriedspeaks


\direct{kommt aus einem traumartigen Zustand zu sich und wendet sich mit übermütiger Lustigkeit zu Gunther}

Ich fürchte kein Feuer,
für dich frei ich die Frau;
denn dein Mann bin ich,
und mein Mut ist dein,
gewinn' ich mir Gutrun' zum Weib.
 

\Guntherspeaks

Gutrune gönn' ich dir gerne.
 

\Siegfriedspeaks

Brünnhilde bring' ich dir.
 

\Guntherspeaks

Wie willst du sie täuschen?
 

\Siegfriedspeaks

Durch des Tarnhelms Trug
tausch' ich mir deine Gestalt.
 

\Guntherspeaks

So stelle Eide zum Schwur!
 

\Siegfriedspeaks

Blut-Brüderschaft schwöre ein Eid!
 


\direct{Hagen füllt ein Trinkhorn mit frischem Wein; dieses hält er dann Siegfried und Gunther hin, welche sich mit ihren Schwertern die Arme ritzen und diese eine kurze Zeit über die Öffnung des Trinkhornes halten}


\direct{Siegfried und Gunther legen zwei ihrer Finger auf das Horn, welches Hagen fortwährend in ihrer Mitte hält}


\Siegfriedspeaks

Blühenden Lebens labendes Blut
träufelt' ich in den Trank.
 

\Guntherspeaks

Bruder-brünstig mutig gemischt,
blüh' im Trank unser Blut.
 

\speaker{Beide}
Treue trink' ich dem Freund.
Froh und frei entblühe dem Bund,
Blut-Brüderschaft heut'!
 

\Guntherspeaks

Bricht ein Bruder den Bund,
 

\Siegfriedspeaks

Trügt den Treuen der Freund,
 

\speaker{Beide}
Was in Tropfen heut' hold wir tranken,
in Strahlen ström' es dahin,
fromme Sühne dem Freund!
 

\Guntherspeaks


\direct{trinkt und reicht das Horn Siegfried}

So biet' ich den Bund.
 

\Siegfriedspeaks

So trink' ich dir Treu'!
 


\direct{Er trinkt und hält das geleerte Trinkhorn Hagen hin. Hagen zerschlägt mit seinem Schwerte das Horn in zwei Stücke. Siegfried und Gunther reichen sich die Hände}


\Siegfriedspeaks


\direct{betrachtet Hagen, welcher während des Schwures hinter ihm gestanden}

Was nahmst du am Eide nicht teil?
 

\Hagenspeaks

Mein Blut verdürb' euch den Trank;
nicht fließt mir's echt und edel wie euch;
störrisch und kalt stockt's in mir;
nicht will's die Wange mir röten.
Drum bleibt ich fern vom feurigen Bund.
 

\Guntherspeaks


\direct{zu Siegfried}

Laß den unfrohen Mann!
 

\Siegfriedspeaks


\direct{hängt sich den Schild wieder über}

Frisch auf die Fahrt!
Dort liegt mein Schiff;
schnell führt es zum Felsen.

\direct{Er tritt näher zu Gunther und bedeutet diesen}

Eine Nacht am Ufer harrst du im Nachen;
die Frau fährst du dann heim.

\direct{Er wendet sich zum Fortgehen und winkt Gunther, ihm zu folgen}


\Guntherspeaks

Rastest du nicht zuvor?
 

\Siegfriedspeaks

Um die Rückkehr ist mir's jach!
 


\direct{Er geht zum Ufer, um das Schiff loszubinden}


\Guntherspeaks

Du, Hagen, bewache die Halle!
 


\direct{Er folgt Siegfried zum Ufer.  Während Siegfried und Gunther, nachdem sie ihre Waffen darin niedergelegt, im Schiff das Segel aufstecken und alles zur Abfahrt bereit machen, nimmt Hagen seinen Speer und Schild}


\direct{Gutrune erscheint an der Tür ihres Gemachs, als soeben Siegfried das Schiff abstößt, welches sogleich der Mitte des Stromes zutreibt}


\Gutrunespeaks

Wohin eilen die Schnellen?
 

\Hagenspeaks


\direct{während er sich gemächlich mit Schild und Speer vor der Halle niedersetzt}

Zu Schiff---Brünnhild' zu frein.
 

\Gutrunespeaks

Siegfried?
 

\Hagenspeaks

Sieh', wie's ihn treibt,
zum Weib dich zu gewinnen!
 

\Gutrunespeaks

Siegfried mein!
 


\direct{Sie geht, lebhaft erregt, in ihr Gemach zurück}


\direct{Siegfried hat das Ruder erfaßt und treibt jetzt mit dessen Schlägen den Nachen stromabwärts, so daß dieser bald gänzlich außer Gesicht kommt}


\Hagenspeaks


\direct{sitzt mit dem Rücken an den Pfosten der Halle gelehnt, bewegungslos}

Hier sitz' ich zur Wacht, wahre den Hof,
wehre die Halle dem Feind.
Gibichs Sohne wehet der Wind,
auf Werben fährt er dahin.
lhm führt das Steuer ein starker Held,
Gefahr ihm will er bestehn:
Die eigne Braut ihm bringt er zum Rhein;
mir aber bringt er den Ring!
Ihr freien Söhne, frohe Gesellen,
segelt nur lustig dahin!
Dünkt er euch niedrig, ihr dient ihm doch,
des Niblungen Sohn.
 


\direct{Ein Teppich, welcher dem Vordergrunde zu die Halle einfaßte, schlägt zusammen und schließt die Bühne vor dem Zuschauer ab. Nachdem während eines kurzen Orchester-Zwischenspieles der Schauplatz verwandelt ist, wird der Teppich gänzlich aufgezogen}

 

\scene

\StageDir{Die Felsenhöhle (wie im Vorspiel)

  Brünnhilde sitzt am Eingange des Steingemaches, in stummen Sinnen Siegfrieds Ring betrachtend; von wonniger Erinnerung überwältigt, bedeckt sie ihn mit Küssen. Ferner Donner läßt sich vernehmen, sie blickt auf und lauscht. Dann wendet sie sich wieder zu dem Ring. Ein feuriger Blitz. Sie lauscht von neuem und späht nach der Ferne, von woher eine finstre Gewitterwolke dem Felsensaume zuzieht}


\Brunnhildespeaks

Altgewohntes Geräusch
raunt meinem Ohr die Ferne.
Ein Luftroß jagt im Laufe daher;
auf der Wolke fährt es wetternd zum Fels.
Wer fand mich Einsame auf?
 

\speaker{\Waltraute (Stimme)}

\direct{aus der Ferne}

Brünnhilde! Schwester!
Schläfst oder wachst du?
 

\Brunnhildespeaks


\direct{fährt vom Sitze auf}

Waltrautes Ruf, so wonnig mir kund!

\direct{in die Szene rufend}

Kommst du, Schwester?
Schwingst dich kühn zu mir her?

\direct{Sie eilt nach dem Felsrande}

Dort im Tann
---dir noch vertraut---
steige vom Roß
und stell' den Renner zur Rast!


\direct{Sie stürmt in den Tann, von wo ein starkes Geräusch, gleich einem Gewitterschlage, sich vernehmen läßt. Dann kommt sie in heftiger Bewegung mit Waltraute zurück; sie bleibt freudig erregt, ohne Waltrautes ängstliche Scheu zu beachten}


Kommst du zu mir?
Bist du so kühn,
magst ohne Grauen
Brünnhild' bieten den Gruß?
 

\Waltrautespeaks

Einzig dir nur galt meine Eil'!
 

\Brunnhildespeaks


\direct{in höchster freudiger Aufgeregtheit}

So wagtest du, Brünnhild' zulieb,
Walvaters Bann zu brechen?
Oder wie---o sag'---
wär' wider mich Wotans Sinn erweicht?
Als dem Gott entgegen Siegmund ich schützte,
fehlend---ich weiß es---
erfüllt' ich doch seinen Wunsch.
Daß sein Zorn sich verzogen,
weiß ich auch;
denn verschloß er mich gleich in Schlaf,
fesselt' er mich auf den Fels,
wies er dem Mann mich zur Magd,
der am Weg mich fänd' und erweckt',
meiner bangen Bitte doch gab er Gunst:
mit zehrendem Feuer umzog er den Fels,
dem Zagen zu wehren den Weg.
So zur Seligsten schuf mich die Strafe:
der herrlichste Held
gewann mich zum Weib!
In seiner Liebe leucht' und lach' ich heut' auf.
 


\direct{Sie umarmt Waltraute unter stürmischen Freudenbezeigungen, welche diese mit scheuer Ungeduld abzuwehren sucht}

Lockte dich, Schwester, mein Los?
An meiner Wonne willst du dich weiden,
teilen, was mich betraf?
 

\Waltrautespeaks


\direct{heftig}

Teilen den Taumel, der dich Törin erfaßt?
Ein andres bewog mich in Angst,
zu brechen Wotans Gebot.
 


\direct{Brünnhilde gewahrt hier erst mit Befremdung die wildaufgeregte Stimmung Waltrautes}


\Brunnhildespeaks

Angst und Furcht fesseln dich Arme?
So verzieh der Strenge noch nicht?
Du zagst vor des Strafenden Zorn?
 

\Waltrautespeaks


\direct{düster}

Dürft' ich ihn fürchten,
meiner Angst fänd' ich ein End'!
 

\Brunnhildespeaks

Staunend versteh' ich dich nicht!
 

\Waltrautespeaks

Wehre der Wallung:
achtsam höre mich an!
Nach Walhall wieder
drängt mich die Angst,
die von Walhall hierher mich trieb.
 

\Brunnhildespeaks


\direct{erschrocken}

Was ist's mit den ewigen Göttern?
 

\Waltrautespeaks

Höre mit Sinn, was ich dir sage!
Seit er von dir geschieden,
zur Schlacht nicht mehr schickte uns Wotan;
irr und ratlos ritten wir ängstlich zu Heer;
Walhalls mutige Helden mied Walvater.
Einsam zu Roß, ohne Ruh' noch Rast,
durchschweift er als Wandrer die Welt.
Jüngst kehrte er heim;
in der Hand hielt er seines Speeres Splitter:
die hatte ein Held ihm geschlagen.
Mit stummem Wink Walhalls Edle
wies er zum Forst, die Weltesche zu fällen.
Des Stammes Scheite hieß er sie schichten
zu ragendem Hauf rings um der Seligen Saal.
Der Götter Rat ließ er berufen;
den Hochsitz nahm heilig er ein:
ihm zu Seiten hieß er die Bangen sich setzen,
in Ring und Reih' die Hall' erfüllen die Helden.
So sitzt er, sagt kein Wort,
auf hehrem Sitze stumm und ernst,
des Speeres Splitter fest in der Faust;
Holdas Äpfel rührt er nicht an.
Staunen und Bangen binden starr die Götter.
Seine Raben beide sandt' er auf Reise:
kehrten die einst mit guter Kunde zurück,
dann noch einmal zum letztenmal
lächelte ewig der Gott.
Seine Knie umwindend, liegen wir Walküren;
blind bleibt er den flehenden Blicken;
uns alle verzehrt Zagen und endlose Angst.
An seine Brust preßt' ich mich weinend:
da brach sich sein Blick---
er gedachte, Brünnhilde, dein'!
Tief seufzt' er auf, schloß das Auge,
und wie im Traume
raunt' er das Wort:
``Des tiefen Rheines Töchtern
gäbe den Ring sie wieder zurück,
von des Fluches Last
erlöst wär' Gott und Welt!''
Da sann ich nach: von seiner Seite
durch stumme Reihen stahl ich mich fort;
in heimlicher Hast bestieg ich mein Roß
und ritt im Sturme zu dir.
Dich, o Schwester, beschwör' ich nun:
was du vermagst, vollend' es dein Mut!
Ende der Ewigen Qual!
 


\direct{Sie hat sich vor Brünnhilde niedergeworfen}


\Brunnhildespeaks


\direct{ruhig}

Welch' banger Träume Mären
meldest du Traurige mir!
Der Götter heiligem Himmelsnebel
bin ich Törin enttaucht:
nicht faß ich, was ich erfahre.
Wirr und wüst scheint mir dein Sinn;
in deinem Aug'---so übermüde---
glänzt flackernde Glut.
Mit blasser Wange, du bleiche Schwester,
was willst du Wilde von mir?
 

\Waltrautespeaks


\direct{heftig}

An deiner Hand, der Ring,
er ist's; hör' meinen Rat:
für Wotan wirf ihn von dir!
 

\Brunnhildespeaks

Den Ring? Von mir?
 

\Waltrautespeaks

Den Rheintöchtern gib ihn zurück!
 

\Brunnhildespeaks

Den Rheintöchtern---ich---den Ring?
Siegfrieds Liebespfand?
Bist du von Sinnen?
 

\Waltrautespeaks

Hör' mich! Hör' meine Angst!
Der Welt Unheil haftet sicher an ihm.
Wirf ihn von dir, fort in die Welle!
Walhalls Elend zu enden,
den verfluchten wirf in die Flut!
 

\Brunnhildespeaks

Ha! Weißt du, was er mir ist?
Wie kannst du's fassen, fühllose Maid!
Mehr als Walhalls Wonne,
mehr als der Ewigen Ruhm
ist mir der Ring:
ein Blick auf sein helles Gold,
ein Blitz aus dem hehren Glanz
gilt mir werter
als aller Götter ewig währendes Glück!
Denn selig aus ihm leuchtet mir Siegfrieds Liebe:
Siegfrieds Liebe!
O ließ' sich die Wonne dir sagen!
Sie wahrt mir der Reif.
Geh' hin zu der Götter heiligem Rat!
Von meinem Ringe raune ihnen zu:
die Liebe ließe ich nie,
mir nähmen nie sie die Liebe,
stürzt' auch in Trümmern
Walhalls strahlende Pracht!
 

\Waltrautespeaks

Dies deine Treue?
So in Trauer
entlässest du lieblos die Schwester?
 

\Brunnhildespeaks

Schwinge dich fort!
Fliege zu Roß!
Den Ring entführst du mir nicht!
 

\Waltrautespeaks

Wehe! Wehe!
Weh' dir, Schwester!
Walhalls Göttern weh'!
 


\direct{Sie stürzt fort. Bald erhebt sich unter Sturm eine Gewitterwolke aus dem Tann}


\Brunnhildespeaks


\direct{während sie der davonjagenden, hell erleuchteten Gewitterwolke, die sich bald gänzlich in der Ferne verliert, nachblickt}

Blitzend Gewölk,
vom Wind getragen,
stürme dahin:
zu mir nie steure mehr her!

\direct{Es ist Abend geworden. Aus der Tiefe leuchtet der Feuerschein allmählich heller auf. Brünnhilde blickt ruhig in die Landschaft hinaus}

Abendlich Dämmern deckt den Himmel;
heller leuchtet die hütende Lohe herauf.

\direct{Der Feuerschein nähert sich aus der Tiefe. Immer glühendere Flammenzungen lecken über den Felsensaum auf}

Was leckt so wütend
die lodernde Welle zum Wall?
Zur Felsenspitze wälzt sich der feurige Schwall.

\direct{Man hört aus der Tiefe Siegfrieds Hornruf nahen. Brünnhilde lauscht und fährt entzückt auf}

Siegfried! Siegfried zurück?
Seinen Ruf sendet er her!
Auf! Auf! Ihm entgegen!
In meines Gottes Arm!
 


\direct{Sie eilt in höchstem Entzücken dem Felsrande zu. Feuerflammen schlagen herauf: aus ihnen springt Siegfried auf einen hochragenden Felsstein empor, worauf die Flammen sogleich wieder zurückweichen und abermals nur aus der Tiefe heraufleuchten. Siegfried, auf dem Haupte den Tarnhelm, der ihm bis zur Hälfte das Gesicht verdeckt und nur die Augen freiläßt, erscheint in Gunthers Gestalt}


\Brunnhildespeaks


\direct{voll Entsetzen zurückweichend}

Verrat! Wer drang zu mir?
 


\direct{Sie flieht bis in den Vordergrund und heftet von da aus in sprachlosem Erstaunen ihren Blick auf Siegfried}


\Siegfriedspeaks


\direct{im Hintergrunde auf dem Steine verweilend, betrachtet sie lange, regungslos auf seinen Schild gelehnt; dann redet er sie mit verstellter---tieferer---Stimme an}

Brünnhild'! Ein Freier kam,
den dein Feuer nicht geschreckt.
Dich werb' ich nun zum Weib:
du folge willig mir!
 

\Brunnhildespeaks


\direct{heftig zitternd}

Wer ist der Mann,
der das vermochte,
was dem Stärksten nur bestimmt?
 

\Siegfriedspeaks


\direct{unverändert wie zuvor}

Ein Helde, der dich zähmt,
bezwingt Gewalt dich nur.
 

\Brunnhildespeaks


\direct{von Grausen erfaßt}

Ein Unhold schwang sich auf jenen Stein!
Ein Aar kam geflogen,
mich zu zerfleischen!
Wer bist du, Schrecklicher?
 


\direct{Langes Schweigen}

Stammst du von Menschen?
Kommst du von Hellas nächtlichem Heer?
 

\Siegfriedspeaks


\direct{wie zuvor, mit etwas bebender Stimme beginnend, alsbald aber wieder sicherer fortfahrend}

Ein Gibichung bin ich,
und Gunther heißt der Held,
dem, Frau, du folgen sollst.
 

\Brunnhildespeaks


\direct{in Verzweiflung ausbrechend}

Wotan! Ergrimmter, grausamer Gott!
Weh'! Nun erseh' ich der Strafe Sinn:
zu Hohn und Jammer jagst du mich hin!
 

\Siegfriedspeaks


\direct{springt vom Stein herab und tritt näher heran}

Die Nacht bricht an:
in diesem Gemach
mußt du dich mir vermählen!
 

\Brunnhildespeaks


\direct{indem sie den Finger, an dem sie Siegfrieds Ring trägt, drohend ausstreckt}

Bleib' fern! Fürchte dies Zeichen!
Zur Schande zwingst du mich nicht,
solang' der Ring mich beschützt.
 

\Siegfriedspeaks

Mannesrecht gebe er Gunther,
durch den Ring sei ihm vermählt!
 

\Brunnhildespeaks

Zurück, du Räuber!
Frevelnder Dieb!
Erfreche dich nicht, mir zu nahn!
Stärker als Stahl macht mich der Ring:
nie raubst du ihn mir!
 

\Siegfriedspeaks

Von dir ihn zu lösen,
lehrst du mich nun!
 


\direct{Er dringt auf sie ein; sie ringen miteinander. Brünnhilde windet sich los, flieht und wendet sich um, wie zur Wehr. Siegfried greift sie von neuem an. Sie flieht, er erreicht sie. Beide ringen heftig miteinander. Er faßt sie bei der Hand und entzieht ihrem Finger den Ring. Sie schreit heftig auf. Als sie wie zerbrochen in seinen Armen niedersinkt, streift ihr Blick bewußtlos die Augen Siegfrieds}


\Siegfriedspeaks


\direct{läßt die Machtlose auf die Steinbank vor dem Felsengemach niedergleiten}

Jetzt bist du mein,
Brünnhilde, Gunthers Braut.
Gönne mir nun dein Gemach!
 

\Brunnhildespeaks


\direct{starrt ohnmächtig vor sich hin, matt}

Was könntest du wehren, elendes Weib!
 


\direct{Siegfried treibt sie mit einer gebietenden Bewegung an. Zitternd und wankenden Schrittes geht sie in das Gemach}


\Siegfriedspeaks


\direct{das Schwert ziehend, mit seiner natürlichen Stimme}

Nun, Notung, zeuge du,
daß ich in Züchten warb.
Die Treue wahrend dem Bruder,
trenne mich von seiner Braut!
 


\direct{Er folgt Brünnhilde}


\direct{Der Vorhang fällt}


   

\act

\scene

\StageDir{Uferraum vor der Halle der Gibichungen: rechts der offene Eingang zur Halle; links das Rheinufer; von diesem aus erhebt sich eine durch verschiedene Bergpfade gespaltene, felsige Anhöhe quer über die Bühne, nach rechts dem Hintergrunde zu aufsteigend. Dort sieht man einen der Fricka errichteten Weihstein, welchem höher hinauf ein größerer für Wotan, sowie seitwärts ein gleicher dem Donner geweihter entspricht. Es ist Nacht.}
 


\direct{Hagen, den Speer im Arm, den Schild zur Seite, sitzt schlafend an einen Pfosten der Halle gelehnt. Der Mond wirft plötzlich ein grelles Licht auf ihn und seine nächste Umgebung; man gewahrt Alberich vor Hagen kauernd, die Arme auf dessen Knie gelehnt}


\Alberichspeaks


\direct{leise}

Schläfst du, Hagen, mein Sohn?
Du schläfst und hörst mich nicht,
den Ruh' und Schlaf verriet?
 

\Hagenspeaks


\direct{leise, ohne sich zu rühren, so daß er immerfort zu schlafen scheint, obwohl er die Augen offen hat}

Ich höre dich, schlimmer Albe:
was hast du meinem Schlaf zu sagen?
 

\Alberichspeaks

Gemahnt sei der Macht,
der du gebietest,
bist du so mutig,
wie die Mutter dich mir gebar!
 

\Hagenspeaks


\direct{immer wie zuvor}

Gab mir die Mutter Mut,
nicht mag ich ihr doch danken,
daß deiner List sie erlag:
frühalt, fahl und bleich,
hass' ich die Frohen, freue mich nie!
 

\Alberichspeaks


\direct{wie zuvor}

Hagen, mein Sohn! Hasse die Frohen!
Mich Lustfreien, Leidbelasteten
liebst du so, wie du sollst!
Bist du kräftig, kühn und klug:
die wir bekämpfen mit nächtigem Krieg,
schon gibt ihnen Not unser Neid.
Der einst den Ring mir entriß,
Wotan, der wütende Räuber,
vom eignen Geschlechte ward er geschlagen:
an den Wälsung verlor er Macht und Gewalt;
mit der Götter ganzer Sippe
in Angst ersieht er sein Ende.
Nicht ihn fürcht' ich mehr:
fallen muß er mit allen!
Schläfst du, Hagen, mein Sohn?
 

\Hagenspeaks


\direct{bleibt unverändert wie zuvor}

Der Ewigen Macht, wer erbte sie?
 

\Alberichspeaks

Ich und du! Wir erben die Welt.
Trüg' ich mich nicht in deiner Treu',
teilst du meinen Gram und Grimm.
Wotans Speer zerspellte der Wälsung,
der Fafner, den Wurm, im Kampfe gefällt
und kindisch den Reif sich errang.
Jede Gewalt hat er gewonnen;
Walhall und Nibelheim neigen sich ihm.

\direct{immer heimlich}

An dem furchtlosen Helden
erlahmt selbst mein Fluch:
denn nicht kennt er des Ringes Wert,
zu nichts nützt er die neidlichste Macht.
Lachend in liebender Brunst,
brennt er lebend dahin.
Ihn zu verderben, taugt uns nun einzig!
Schläfst du, Hagen, mein Sohn?
 

\Hagenspeaks


\direct{wie zuvor}

Zu seinem Verderben dient er mir schon.
 

\Alberichspeaks

Den goldnen Ring,
den Reif gilt's zu erringen!
Ein weises Weib lebt dem Wälsung zulieb':
riet es ihm je des Rheines Töchtern,
die in Wassers Tiefen einst mich betört,
zurückzugeben den Ring,
verloren ging' mir das Gold,
keine List erlangte es je.
Drum, ohne Zögern ziel' auf den Reif!
Dich Zaglosen zeugt' ich mir ja,
daß wider Helden hart du mir hieltest.
Zwar stark nicht genug, den Wurm zu bestehn,
---was allein dem Wälsung bestimmt---
zu zähem Haß doch erzog ich Hagen,
der soll mich nun rächen,
den Ring gewinnen
dem Wälsung und Wotan zum Hohn!
Schwörst du mir's, Hagen, mein Sohn?
 


\direct{Von hier an bedeckt ein immer finsterer werdender Schatten wieder Alberich. Zugleich beginnt das erste Tagesgrauen}


\Hagenspeaks


\direct{immer wie zuvor}

Den Ring soll ich haben:
harre in Ruh'!
 

\Alberichspeaks

Schwörst du mir's, Hagen, mein Held?
 

\Hagenspeaks

Mir selbst schwör' ich's;
schweige die Sorge!
 

\Alberichspeaks


\direct{wie er allmählich immer mehr dem Blicke entschwindet, wird auch seine Stimme immer unvernehmbarer}

Sei treu, Hagen, mein Sohn!
Trauter Helde! Sei treu!
Sei treu! Treu!
 


\direct{Alberich ist gänzlich verschwunden. Hagen, der unverändert in seiner Stellung verblieben, blickt regungslos und starren Auges nach dem Rheine hin, auf welchem sich die Morgendämmerung ausbreitet}


\scene

\StageDir{Der Rhein färbt sich immer stärker vom erglühenden Morgenrot. Hagen macht eine zuckende Bewegung. Siegfried tritt plötzlich, dicht am Ufer, hinter einem Busche hervor. Er ist in seiner eignen Gestalt; nur den Tarnhelm hat er noch auf dem Haupte: er zieht ihn jetzt ab und hängt ihn, während er hervorschreitet, in den Gürtel}


\Siegfriedspeaks

Hoiho, Hagen! Müder Mann!
Siehst du mich kommen?
 

\Hagenspeaks


\direct{gemächlich sich erhebend}

Hei, Siegfried?
Geschwinder Helde?
Wo brausest du her?
 

\Siegfriedspeaks

Vom Brünnhildenstein!
Dort sog ich den Atem ein,
mit dem ich dich rief:
so schnell war meine Fahrt!
Langsamer folgt mir ein Paar:
zu Schiff gelangt das her!
 

\Hagenspeaks

So zwangst du Brünnhild'?
 

\Siegfriedspeaks

Wacht Gutrune?
 

\Hagenspeaks


\direct{in die Halle rufend}

Hoiho, Gutrune! Komm' heraus!
Siegfried ist da:
was säumst du drin?
 

\Siegfriedspeaks


\direct{zur Halle sich wendend}

Euch beiden meld' ich,
wie ich Brünnhild' band.
 


\direct{Gutrune tritt ihm aus der Halle entgegen}


\Siegfriedspeaks

Heiß' mich willkommen, Gibichskind!
Ein guter Bote bin ich dir.
 

\Gutrunespeaks

Freia grüße dich zu aller Frauen Ehre!
 

\Siegfriedspeaks

Frei und hold sei nun mir Frohem:
zum Weib gewann ich dich heut'.
 

\Gutrunespeaks

So folgt Brünnhild' meinem Bruder?
 

\Siegfriedspeaks

Leicht ward die Frau ihm gefreit.
 

\Gutrunespeaks

Sengte das Feuer ihn nicht?
 

\Siegfriedspeaks

Ihn hätt' es auch nicht versehrt,
doch ich durchschritt es für ihn,
da dich ich wollt' erwerben.
 

\Gutrunespeaks

Und dich hat es verschont?
 

\Siegfriedspeaks

Mich freute die schwelende Brunst.
 

\Gutrunespeaks

Hielt Brünnhild' dich für Gunther?
 

\Siegfriedspeaks

Ihm glich ich auf ein Haar:
der Tarnhelm wirkte das,
wie Hagen tüchtig es wies.
 

\Hagenspeaks

Dir gab ich guten Rat.
 

\Gutrunespeaks

So zwangst du das kühne Weib?
 

\Siegfriedspeaks

Sie wich Gunthers Kraft.
 

\Gutrunespeaks

Und vermählte sie sich dir?
 

\Siegfriedspeaks

Ihrem Mann gehorchte Brünnhild'
eine volle bräutliche Nacht.
 

\Gutrunespeaks

Als ihr Mann doch galtest du?
 

\Siegfriedspeaks

Bei Gutrune weilte Siegfried.
 

\Gutrunespeaks

Doch zur Seite war ihm Brünnhild'?
 

\Siegfriedspeaks


\direct{auf sein Schwert deutend}

Zwischen Ost und West der Nord:
so nah war Brünnhild' ihm fern.
 

\Gutrunespeaks

Wie empfing Gunther sie nun von dir?
 

\Siegfriedspeaks

Durch des Feuers verlöschende Lohe,
im Frühnebel vom Felsen folgte sie mir zu Tal;
dem Strande nah,
flugs die Stelle tauschte Gunther mit mir:
durch des Geschmeides Tugend
wünscht' ich mich schnell hieher.
Ein starker Wind nun treibt
die Trauten den Rhein herauf:
drum rüstet jetzt den Empfang!
 

\Gutrunespeaks

Siegfried, mächtigster Mann!
Wie faßt mich Furcht vor dir!
 

\Hagenspeaks


\direct{von der Höhe im Hintergrunde den Fluß hinabspähend}

In der Ferne seh' ich ein Segel.
 

\Siegfriedspeaks

So sagt dem Boten Dank!
 

\Gutrunespeaks

Lasset uns sie hold empfangen,
daß heiter sie und gern hier weile!
Du, Hagen, minnig rufe die Mannen
nach Gibichs Hof zur Hochzeit!
Frohe Frauen ruf' ich zum Fest:
der Freudigen folgen sie gern.
 


\direct{Nach der Halle schreitend, wendet sie sich wieder um}

Rastest du, schlimmer Held?
 

\Siegfriedspeaks

Dir zu helfen, ruh' ich aus.
 


\direct{Er reicht ihr die Hand und geht mit ihr in die Halle}



\scene

\Hagenspeaks

\direct{er hat einen Felsstein in der Höhe des Hintergrundes erstiegen; dort setzt er, der Landseite zugewendet, sein Stierhorn zum Blasen an}

Hoiho! Hoihohoho!
Ihr Gibichsmannen, machet euch auf!
Wehe! Wehe! Waffen! Waffen!
Waffen durchs Land! Gute Waffen!
Starke Waffen! Scharf zum Streit.
Not ist da! Not! Wehe! Wehe!
Hoiho! Hoihohoho!
 


\direct{Hagen bleibt immer in seiner Stellung auf der Anhöhe. Er bläst abermals. Aus verschiedenen Gegenden vom Lande her antworten Heerhörner. Auf den verschiedenen Höhenpfaden stürmen in Hast und Eile gewaffnete Mannen herbei, erst einzelne, dann immer mehrere zusammen, welche sich dann auf dem Uferraum vor der Halle anhäufen}


\speaker{Die Mannen}

\direct{erst einzelne, dann immer neu hinzukommende}

Was tost das Horn?
Was ruft es zu Heer?
Wir kommen mit Wehr,
Wir kommen mit Waffen!
Hagen! Hagen!
Hoiho! Hoiho!
Welche Not ist da?
Welcher Feind ist nah?
Wer gibt uns Streit?
Ist Gunther in Not?
Wir kommen mit Waffen,
mit scharfer Wehr.
Hoiho! Ho! Hagen!
 

\Hagenspeaks


\direct{immer von der Anhöhe herab}

Rüstet euch wohl und rastet nicht;
Gunther sollt ihr empfahn:
ein Weib hat der gefreit.
 

\speaker{Die Mannen}
Drohet ihm Not?
Drängt ihn der Feind?
 

\Hagenspeaks

Ein freisliches Weib führet er heim.
 

\speaker{Die Mannen}
Ihm folgen der Magen feindliche Mannen?
 

\Hagenspeaks

Einsam fährt er: keiner folgt.
 

\speaker{Die Mannen}
So bestand er die Not?
So bestand er den Kampf?
Sag' es an!
 

\Hagenspeaks

Der Wurmtöter wehrte der Not:
Siegfried, der Held, der schuf ihm Heil!
 

\speaker{Ein Mann}
Was soll ihm das Heer nun noch helfen?
 

\speaker{Zehn Weitere}
Was hilft ihm nun das Heer?
 

\Hagenspeaks

Starke Stiere sollt ihr schlachten;
am Weihstein fließe Wotan ihr Blut!
 

\speaker{Ein Mann}
Was, Hagen, was heißest du uns dann?
 

\speaker{Acht Mannen}
Was heißest du uns dann?
 

\speaker{Vier Weitere}
Was soll es dann?
 

\speaker{Alle}
Was heißest du uns dann?
 

\Hagenspeaks

Einen Eber fällen sollt ihr für Froh!
Einen stämmigen Bock stechen für Donner!
Schafe aber schlachtet für Fricka,
daß gute Ehe sie gebe!
 

\speaker{Die Mannen}

\direct{mit immer mehr ausbrechender Heiterkeit}

Schlugen wir Tiere,
was schaffen wir dann?
 

\Hagenspeaks

Das Trinkhorn nehmt,
von trauten Frau'n
mit Met und Wein wonnig gefüllt!
 

\speaker{Die Mannen}
Das Trinkhorn zur Hand,
wie halten wir es dann?
 

\Hagenspeaks

Rüstig gezecht, bis der Rausch euch zähmt!
Alles den Göttern zu Ehren,
daß gute Ehe sie geben!
 

\speaker{Die Mannen}

\direct{brechen in ein schallendes Gelächter aus}

Groß Glück und Heil lacht nun dem Rhein,
da Hagen, der Grimme, so lustig mag sein!
Der Hagedorn sticht nun nicht mehr;
zum Hochzeitsrufer ward er bestellt.
 

\Hagenspeaks


\direct{der immer sehr ernst geblieben, ist zu den Mannen herabgestiegen und steht jetzt unter ihnen}

Nun laßt das Lachen, mut'ge Mannen!
Empfangt Gunthers Braut!
Brünnhilde naht dort mit ihm.
 


\direct{Er deutet die Mannen nach dem Rhein hin: diese eilen zum Teil nach der Anhöhe, während andere sich am Ufer aufstellen, um die Ankommenden zu erblicken}


\direct{Näher zu einigen Mannen tretend}

Hold seid der Herrin,
helfet ihr treu:
traf sie ein Leid,
rasch seid zur Rache!
 


\direct{Er wendet sich langsam zur Seite, in den Hintergrund}


\direct{Während des Folgenden kommt der Nachen mit Gunther und Brünnhilde auf dem Rheine an}


\speaker{Die Mannen}

\direct{diejenigen, welche von der Höhe ausgeblickt hatten, kommen zum Ufer herab}

Heil! Heil!
Willkommen! Willkommen!
 


\direct{Einige der Mannen springen in den Fluß und ziehen den Kahn an das Land. Alles drängt sich immer dichter an das Ufer}

Willkommen, Gunther!
Heil! Heil!



\direct{Gunther steigt mit Brünnhilde aus dem Kahne; die Mannen reihen sich ehrerbietig zu ihren Empfange. Während des Folgenden geleitet Gunther Brünnhilde feierlich an der Hand}


\speaker{Die Mannen}
Heil dir, Gunther!
Heil dir und deiner Braut!
Willkommen!
 


\direct{Sie schlagen die Waffen tosend zusammen}


\Guntherspeaks


\direct{Brünnhilde, welche bleich und gesenkten Blickes ihm folgt, den Mannen vorstellend}

Brünnhild', die hehrste Frau,
bring' ich euch her zum Rhein.
Ein edleres Weib ward nie gewonnen.
Der Gibichungen Geschlecht,
gaben die Götter ihm Gunst,
zum höchsten Ruhm rag' es nun auf!
 

\speaker{Die Mannen}

\direct{feierlich an ihre Waffen schlagend}

Heil! Heil dir,
glücklicher Gibichung!
 


\direct{Gunther geleitet Brünnhilde, die nie aufblickt, zur Halle, aus welcher jetzt Siegfried und Gutrune, von Frauen begleitet, heraustreten}


\Guntherspeaks


\direct{hält vor der Halle an}

Gegrüßt sei, teurer Held;
gegrüßt, holde Schwester!
Dich seh' ich froh ihm zur Seite,
der dich zum Weib gewann.
Zwei sel'ge Paare
seh ich hier prangen:
 


\direct{Er führt Brünnhilde näher heran}

Brünnhild' und Gunther,
Gutrun' und Siegfried!
 


\direct{Brünnhilde schlägt erschreckt die Augen auf und erblickt Siegfried; wie in Erstaunen bleibt ihr Blick auf ihn gerichtet. Gunther, welcher Brünnhildes heftig zuckende Hand losgelassen hat, sowie alle übrigen zeigen starre Betroffenheit über Brünnhildes Benehmen}


\speaker{Mannen und Frauen}
Was ist ihr? Ist sie entrückt?
 


\direct{Brünnhilde beginnt zu zittern}


\Siegfriedspeaks


\direct{geht ruhig einige Schritte auf Brünnhilde zu}

Was müht Brünnhildes Blick?
 

\Brunnhildespeaks


\direct{kaum ihrer mächtig}

Siegfried... hier...! Gutrune...?
 

\Siegfriedspeaks

Gunthers milde Schwester:
mir vermählt wie Gunther du.
 

\Brunnhildespeaks


\direct{furchtbar heftig}

Ich.... Gunther... ? Du lügst!

\direct{Sie schwankt und droht umzusinken: Siegfried, ihr zunächst, stützt sie}

Mir schwindet das Licht ....

\direct{Sie blickt in seinen Armen matt zu Siegfried auf}

Siegfried kennt mich nicht!
 

\Siegfriedspeaks

Gunther, deinem Weib ist übel!

\direct{Gunther tritt hinzu}

Erwache, Frau!
Hier steht dein Gatte.
 

\Brunnhildespeaks


\direct{erblickt am ausgestreckten Finger Siegfrieds den Ring und schrickt mit furchtbarer Heftigkeit auf}

Ha! Der Ring 
an seiner Hand!
Er? Siegfried?
 

\speaker{Mannen und Frauen}
Was ist?
 

\Hagenspeaks


\direct{aus dem Hintergrunde unter die Mannen tretend}

Jetzt merket klug,
was die Frau euch klagt!
 

\Brunnhildespeaks


\direct{sucht sich zu ermannen, indem sie die schrecklichste Aufregung gewaltsam zurückhält}

Einen Ring sah ich an deiner Hand,
nicht dir gehört er,
ihn entriß mir

\direct{auf Gunther deutend}

dieser Mann!
Wie mochtest von ihm
den Ring du empfahn?
 

\Siegfriedspeaks


\direct{aufmerksam den Ring an seiner Hand betrachtend}

Den Ring empfing ich nicht von ihm.
 

\Brunnhildespeaks


\direct{zu Gunther}

Nahmst du von mir den Ring,
durch den ich dir vermählt;
so melde ihm dein Recht,
fordre zurück das Pfand!
 

\Guntherspeaks


\direct{in großer Verwirrung}

Den Ring? Ich gab ihm keinen:
doch kennst du ihn auch gut?
 

\Brunnhildespeaks

Wo bärgest du den Ring,
den du von mir erbeutet?
 


\direct{Gunther schweigt in höchster Betroffenheit}


\Brunnhildespeaks


\direct{wütend auffahrend}

Ha! Dieser war es,
der mir den Ring entriß:
Siegfried, der trugvolle Dieb!
 


\direct{Alles blickt erwartungsvoll auf Siegfried, welcher über der Betrachtung des Ringes in fernes Sinnen entrückt ist}


\Siegfriedspeaks

Von keinem Weib kam mir der Reif;
noch war's ein Weib, dem ich ihn abgewann:
genau erkenn' ich des Kampfes Lohn,
den vor Neidhöhl' einst ich bestand,
als den starken Wurm ich erschlug.
 

\Hagenspeaks


\direct{zwischen sie tretend}

Brünnhild', kühne Frau,
kennst du genau den Ring?
Ist's der, den du Gunthern gabst,
so ist er sein,
und Siegfried gewann ihn durch Trug,
den der Treulose büßen sollt'!
 

\Brunnhildespeaks


\direct{in furchtbarstem Schmerze aufschreiend}

Betrug! Betrug! Schändlichster Betrug!
Verrat! Verrat! Wie noch nie er gerächt!
 

\Gutrunespeaks

Verrat? An wem?
 

\speaker{Mannen und Frauen}
Verrat? Verrat?
 

\Brunnhildespeaks

Heil'ge Götter, himmlische Lenker!
Rauntet ihr dies in eurem Rat?
Lehrt ihr mich Leiden, wie keiner sie litt?
Schuft ihr mir Schmach, wie nie sie geschmerzt?
Ratet nun Rache, wie nie sie gerast!
Zündet mir Zorn, wie noch nie er gezähmt!
Heißet Brünnhild' ihr Herz zu zerbrechen,
den zu zertrümmern, der sie betrog!
 

\Guntherspeaks

Brünnhild', Gemahlin!
Mäß'ge dich!
 

\Brunnhildespeaks

Weich' fern, Verräter!
Selbst Verrat'ner
Wisset denn alle: nicht ihm,
dem Manne dort bin ich vermählt.
 

\speaker{Frauen}
Siegfried? Gutruns Gemahl?
 

\speaker{Mannen}
Gutruns Gemahl?
 

\Brunnhildespeaks

Er zwang mir Lust und Liebe ab.
 

\Siegfriedspeaks

Achtest du so der eignen Ehre?
Die Zunge, die sie lästert,
muß ich der Lüge sie zeihen?
Hört, ob ich Treue brach!
Blutbrüderschaft
hab' ich Gunther geschworen:
Notung, das werte Schwert,
wahrte der Treue Eid;
mich trennte seine Schärfe
von diesem traur'gen Weib.
 

\Brunnhildespeaks

Du listiger Held, sieh', wie du lügst!
Wie auf dein Schwert du schlecht dich berufst!
Wohl kenn' ich seine Schärfe,
doch kenn' auch die Scheide,
darin so wonnig ruht' an der Wand
Notung, der treue Freund,
als die Traute sein Herr sich gefreit.
 

\speaker{Die Mannen}

\direct{in lebhafter Entrüstung zusammentretend}

Wie? Brach er die Treue?
Trübte er Gunthers Ehre?
 

\speaker{Die Frauen}
Brach er die Treue?
 

\Guntherspeaks


\direct{zu Siegfried}

Geschändet wär' ich, schmählich bewahrt,
gäbst du die Rede nicht ihr zurück!
 

\Gutrunespeaks

Treulos, Siegfried, sannest du Trug?
Bezeuge, daß jene falsch dich zeiht!
 

\speaker{Die Mannen}
Reinige dich, bist du im Recht!
Schweige die Klage!
Schwöre den Eid!
 

\Siegfriedspeaks

Schweig' ich die Klage,
schwör' ich den Eid:
wer von euch wagt seine Waffe daran?
 

\Hagenspeaks

Meines Speeres Spitze wag' ich daran:
sie wahr' in Ehren den Eid.
 


\direct{Die Mannen schließen einen Ring um Siegfried und Hagen. Hagen hält den Speer hin; Siegfried legt zwei Finger seiner rechten Hand auf die Speerspitze}


\Siegfriedspeaks

Helle Wehr! Heilige Waffe!
Hilf meinem ewigen Eide!
Bei des Speeres Spitze sprech' ich den Eid:
Spitze, achte des Spruchs!
Wo Scharfes mich schneidet,
schneide du mich;
wo der Tod mich soll treffen,
treffe du mich:
klagte das Weib dort wahr,
brach ich dem Bruder den Eid!
 

\Brunnhildespeaks


\direct{tritt wütend in den Ring, reißt Siegfrieds Hand vom Speere hinweg und faßt dafür mit der ihrigen die Spitze}

Helle Wehr! Heilige Waffe!
Hilf meinem ewigen Eide!
Bei des Speeres Spitze sprech' ich den Eid:
Spitze, achte des Spruchs!
Ich weihe deine Wucht,
daß sie ihn werfe!
Deine Schärfe segne ich,
daß sie ihn schneide:
denn, brach seine Eide er all',
schwur Meineid jetzt dieser Mann!
 

\speaker{Die Mannen}

\direct{im höchsten Aufruhr}

Hilf, Donner, tose dein Wetter,
zu schweigen die wütende Schmach!
 

\Siegfriedspeaks

Gunther! Wehr' deinem Weibe,
das schamlos Schande dir lügt!
Gönnt ihr Weil' und Ruh',
der wilden Felsenfrau,
daß ihre freche Wut sich lege,
die eines Unholds arge List
wider uns alle erregt!
Ihr Mannen, kehret euch ab!
Laßt das Weibergekeif'!
Als Zage weichen wir gern,
gilt es mit Zungen den Streit.

\direct{Er tritt dicht zu Gunther}

Glaub', mehr zürnt es mich als dich,
daß schlecht ich sie getäuscht:
der Tarnhelm, dünkt mich fast,
hat halb mich nur gehehlt.
Doch Frauengroll friedet sich bald:
daß ich dir es gewann,
dankt dir gewiß noch das Weib.

\direct{Er wendet sich wieder zu den Mannen}

Munter, ihr Mannen!
Folgt mir zum Mahl!

\direct{zu den Frauen}

Froh zur Hochzeit, helfet, ihr Frauen!
Wonnige Lust lache nun auf!
In Hof und Hain,
heiter vor allen sollt ihr heute mich sehn.
Wen die Minne freut,
meinem frohen Mute
tu' es der Glückliche gleich!
 


\direct{Er schlingt in ausgelassenem Übermute seinen Arm um Gutrune und zieht sie mit sich in die Halle fort. Die Mannen und Frauen, von seinem Beispiele hingerissen, folgen ihm nach}


\direct{Die Bühne ist leer geworden. Nur Brünnhilde, Gunther und Hagen bleiben zurück. Gunther hat sich in tiefer Scham und furchtbarer Verstimmung mit verhülltem Gesichte abseits niedergesetzt. Brünnhilde, im Vordergrunde stehend, blickt Siegfried und Gutrune noch eine Zeitlang schmerzlich nach und senkt dann das Haupt.}


\Brunnhildespeaks


\direct{in starrem Nachsinnen befangen}

Welches Unholds List liegt hier verhohlen?
Welches Zaubers Rat regte dies auf?
Wo ist nun mein Wissen gegen dies Wirrsal?
Wo sind meine Runen gegen dies Rätsel?
Ach Jammer! Jammer! Weh', ach Wehe!
All mein Wissen wies ich ihm zu!
In seiner Macht hält er die Magd;
in seinen Banden faßt er die Beute,
die, jammernd ob ihrer Schmach,
jauchzend der Reiche verschenkt!
Wer bietet mir nun das Schwert,
mit dem ich die Bande zerschnitt'?
 

\Hagenspeaks


\direct{dicht an sie herantretend}

Vertraue mir, betrog'ne Frau!
Wer dich verriet, das räche ich.
 

\Brunnhildespeaks


\direct{matt sich umblickend}

An wem?
 

\Hagenspeaks

An Siegfried, der dich betrog.
 

\Brunnhildespeaks

An Siegfried?... Du?

\direct{bitter lächelnd}

Ein einz'ger Blick seines blitzenden Auges,
das selbst durch die Lügengestalt
leuchtend strahlte zu mir,
deinen besten Mut
machte er bangen!
 

\Hagenspeaks

Doch meinem Speere
spart ihn sein Meineid?
 

\Brunnhildespeaks

Eid und Meineid, müßige Acht!
Nach Stärkrem späh',
deinen Speer zu waffnen,
willst du den Stärksten bestehn!
 

\Hagenspeaks

Wohl kenn' ich Siegfrieds siegende Kraft,
wie schwer im Kampf er zu fällen;
drum raune nun du mir klugen Rat,
wie doch der Recke mir wich'?
 

\Brunnhildespeaks

O Undank, schändlichster Lohn!
Nicht eine Kunst war mir bekannt,
die zum Heil nicht half seinem Leib'!
Unwissend zähmt' ihn mein Zauberspiel,
das ihn vor Wunden nun gewahrt.
 

\Hagenspeaks

So kann keine Wehr ihm schaden?
 

\Brunnhildespeaks

Im Kampfe nicht; doch
träfst du im Rücken ihn....
Niemals---das wußt ich---
wich' er dem Feind,
nie reicht' er fliehend ihm den Rücken:
an ihm drum spart' ich den Segen.
 

\Hagenspeaks

Und dort trifft ihn mein Speer!

\direct{Er wendet sich rasch von Brünnhilde ab zu Gunther}

Auf, Gunther, edler Gibichung!
Hier steht dein starkes Weib:
was hängst du dort in Harm?
 

\Guntherspeaks


\direct{leidenschaftlich auffahrend}

O Schmach! O Schande!
Wehe mir, dem jammervollsten Manne!
 

\Hagenspeaks

In Schande liegst du;
leugn' ich das?
 

\Brunnhildespeaks


\direct{zu Gunther}

O feiger Mann! Falscher Genoss'!
Hinter dem Helden hehltest du dich,
daß Preise des Ruhmes er dir erränge!
Tief wohl sank das teure Geschlecht,
das solche Zagen gezeugt!
 

\Guntherspeaks


\direct{außer sich}

Betrüger ich und betrogen!
Verräter ich und verraten!
Zermalmt mir das Mark!
Zerbrecht mir die Brust!
Hilf, Hagen! Hilf meiner Ehre!
Hilf deiner Mutter,
die mich auch ja gebar!
 

\Hagenspeaks

Dir hilft kein Hirn,
dir hilft keine Hand:
dir hilft nur Siegfrieds Tod!
 

\Guntherspeaks


\direct{von Grausen erfaßt}

Siegfrieds Tod!
 

\Hagenspeaks

Nur der sühnt deine Schmach!
 

\Guntherspeaks


\direct{vor sich hinstarrend}

Blutbrüderschaft schwuren wir uns!
 

\Hagenspeaks

Des Bundes Bruch sühne nun Blut!
 

\Guntherspeaks

Brach er den Bund?
 

\Hagenspeaks

Da er dich verriet!
 

\Guntherspeaks

Verriet er mich?
 

\Brunnhildespeaks

Dich verriet er,
und mich verrietet ihr alle!
Wär' ich gerecht, alles Blut der Welt
büßte mir nicht eure Schuld!
Doch des einen Tod taugt mir für alle:
Siegfried falle zur Sühne für sich und euch!
 

\Hagenspeaks


\direct{heimlich zu Gunther}

Er falle---dir zum Heil!
Ungeheure Macht wird dir,
gewinnst von ihm du den Ring,
den der Tod ihm wohl nur entreißt.
 

\Guntherspeaks


\direct{leise}

Brünnhildes Ring?
 

\Hagenspeaks

Des Nibelungen Reif.
 

\Guntherspeaks


\direct{schwer seufzend}

So wär' es Siegfrieds Ende!
 

\Hagenspeaks

Uns allen frommt sein Tod.
 

\Guntherspeaks

Doch Gutrune, ach, der ich ihn gönnte!
Straften den Gatten wir so,
wie bestünden wir vor ihr?
 

\Brunnhildespeaks


\direct{wild auffahrend}

Was riet mir mein Wissen?
Was wiesen mich Runen?
Im hilflosen Elend achtet mir's hell:
Gutrune heißt der Zauber,
der den Gatten mir entrückt!
Angst treffe sie!
 

\Hagenspeaks


\direct{zu Gunther}

Muß sein Tod sie betrüben,
verhehlt sei ihr die Tat.
Auf muntres Jagen ziehen wir morgen:
der Edle braust uns voran,
ein Eber bracht' ihn da um.
 

\speaker{\Gunther und \Brunnhilde}
So soll es sein! Siegfried falle!
Sühn' er die Schmach, die er mir schuf!
Des Eides Treue hat er getrogen:
mit seinem Blut büß' er die Schuld!
Allrauner, rächender Gott!
Schwurwissender Eideshort!
Wotan! Wende dich her!
Weise die schrecklich heilige Schar,
hieher zu horchen dem Racheschwur!
 

\Hagenspeaks

Sterb' er dahin, der strahlende Held!
Mein ist der Hort, mir muß er gehören.
Drum sei der Reif ihm entrissen.
Alben-Vater, gefallner Fürst!
Nachthüter! Niblungenherr!
Alberich! Achte auf mich!
Weise von neuem der Niblungen Schar,
dir zu gehorchen, des Ringes Herrn!
 

\StageDir{Als Gunther mit Brünnhilde heftig der Halle sich zuwendet, tritt ihnen der von dort heraustretende Brautzug entgegen. Knaben und Mädchen, Blumenstäbe schwingend, springen lustig voraus. Siegfried wird auf einem Schilde, Gutrune auf einem Sessel von den Männern getragen. Auf der Anhöhe des Hintergrundes führen Knechte und Mägde auf verschiedenen Bergpfaden Opfergeräte und Opfertiere zu den Weihsteinen herbei und schmücken diese mit Blumen. Siegfried und die Mannen blasen auf ihren Hörnern den Hochzeitsruf. Die Frauen fordern Brünnhilde auf, an Gutrunes Seite sie zu geleiten. Brünnhilde blickt starr zu Gutrune auf, welche ihr mit freundlichem Lächeln zuwinkt. Als Brünnhilde heftig zurücktreten will, tritt Hagen rasch dazwischen und drängt sie an Gunther, der jetzt von neuem ihre Hand erfaßt, worauf er selbst von den Männern sich auf den Schild heben läßt. Während der Zug, kaum unterbrochen, schnell der Höhe zu sich wieder in Bewegung setzt, fällt der Vorhang}
   

\act
 
\scene

\StageDir{Wildes Wald- und Felsental am Rheine, welcher im Hintergrunde an einem steilen Abhange vorbeifließt.}


\direct{Die drei Rheintöchter, Woglinde, Wellgunde und Flosshilde, tauchen aus der Flut auf und schwimmen, wie im Reigentanze, im Kreise umher}


\speaker{Die Drei Rheintöchter}

\direct{im Schwimmen mäßig einhaltend}

Frau Sonne sendet lichte Strahlen;
Nacht liegt in der Tiefe:
einst war sie hell,
da heil und hehr
des Vaters Gold noch in ihr glänzte.
Rheingold! Klares Gold!
Wie hell du einstens strahltest,
hehrer Stern der Tiefe!

\direct{Sie schließen wieder den Schwimmreigen}

Weialala leia, wallala leialala.

\direct{Ferner Hornruf. Sie lauschen. Sie schlagen jauchzend das Wasser}

Frau Sonne, sende uns den Helden,
der das Gold uns wiedergäbe!
Ließ' er es uns, dein lichtes Auge
neideten dann wir nicht länger.
Rheingold! Klares Gold!
Wie froh du dann strahltest,
freier Stern der Tiefe!

\direct{man hört Siegfrieds Horn von der Höhe her}


\Woglindespeaks

Ich höre sein Horn.
 

\Wellgundespeaks

Der Helde naht.
 

\Flosshildespeaks

Laßt uns beraten!
 


\direct{Sie tauchen alle drei schnell unter}


\direct{Siegfried erscheint auf dem Abhange in vollen Waffen}


\Siegfriedspeaks

Ein Albe führte mich irr,
daß ich die Fährte verlor:
He, Schelm, in welchem Berge
bargst du so schnell mir das Wild?
 

\speaker{Die Drei Rheintöchter}

\direct{tauchen wieder auf und schwimmen im Reigen}

Siegfried!
 

\Flosshildespeaks

Was schiltst du so in den Grund?
 

\Wellgundespeaks

Welchem Alben bist du gram?
 

\Woglindespeaks

Hat dich ein Nicker geneckt?
 

\speaker{Alle Drei}
Sag' es, Siegfried, sag' es uns!
 

\Siegfriedspeaks


\direct{sie lächelnd betrachtend}

Entzücktet ihr zu euch den zottigen Gesellen,
der mir verschwand?
Ist's euer Friedel,
euch lustigen Frauen lass' ich ihn gern.

\direct{Die Mädchen lachen laut auf}


\Woglindespeaks

Siegfried, was gibst du uns,
wenn wir das Wild dir gönnen?
 

\Siegfriedspeaks

Noch bin ich beutelos;
so bittet, was ihr begehrt.
 

\Wellgundespeaks

Ein goldner Ring ragt dir am Finger!
 

\speaker{Die Drei Rheintöchter}
Den gib uns!
 

\Siegfriedspeaks

Einen Riesenwurm erschlug ich um den Reif:
für eines schlechten Bären Tatzen
böt' ich ihn nun zum Tausch?
 

\Woglindespeaks

Bist du so karg?
 

\Wellgundespeaks

So geizig beim Kauf?
 

\Flosshildespeaks

Freigebig solltest Frauen du sein.
 

\Siegfriedspeaks

Verzehrt' ich an euch mein Gut,
des zürnte mir wohl mein Weib.
 

\Flosshildespeaks

Sie ist wohl schlimm?
 

\Wellgundespeaks

Sie schlägt dich wohl?
 

\Woglindespeaks

Ihre Hand fühlt schon der Held!
 


\direct{Sie lachen unmäßig}


\Siegfriedspeaks

Nun lacht nur lustig zu!
In Harm lass' ich euch doch:
denn giert ihr nach dem Ring,
euch Nickern geb' ich ihn nie!
 


\direct{Die Rheintöchter haben sich wieder zum Reigen gefaßt}


\Flosshildespeaks

So schön!
 

\Wellgundespeaks

So stark!
 

\Woglindespeaks

So gehrenswert!
 

\speaker{Alle Drei}
Wie schade, daß er geizig ist!
 


\direct{Sie lachen und tauchen unter}


\Siegfriedspeaks


\direct{tiefer in den Grund hinabsteigend}

Was leid' ich doch das karge Lob?
Lass' ich so mich schmähn?
Kämen sie wieder zum Wasserrand,
den Ring könnten sie haben.

\direct{laut rufend}

He! He, he! Ihr muntren Wasserminnen!
Kommt rasch! Ich schenk' euch den Ring!


\direct{Er hat den Ring vom Finger gezogen und hält ihn in die Höhe}


\direct{Die drei Rheintöchter tauchen wieder auf. Sie zeigen sich ernst und feierlich}


\Flosshildespeaks

Behalt' ihn, Held, und wahr' ihn wohl,
bis du das Unheil errätst -
 

\speaker{\Woglinde und \Wellgunde}
das in dem Ring du hegst.
 

\speaker{Alle Drei}
Froh fühlst du dich dann,
befrein wir dich von dem Fluch.
 

\Siegfriedspeaks


\direct{steckt gelassen den Ring wieder an seinen Finger}

So singet, was ihr wißt!
 

\speaker{Die Rheintöchter}
Siegfried! Siegfried! Siegfried!
Schlimmes wissen wir dir.
 

\Wellgundespeaks

Zu deinem Unheil wahrst du den Reif!
 

\speaker{Alle Drei}
Aus des Rheines Gold ist der Reif geglüht.
 

\Wellgundespeaks

Der ihn listig geschmiedet und schmählich verlor -
 

\speaker{Alle Drei}
der verfluchte ihn, in fernster Zeit
zu zeugen den Tod dem, der ihn trüg'.
 

\Flosshildespeaks

Wie den Wurm du fälltest -
 

\speaker{\Wellgunde und \Flosshilde}
so fällst auch du -
 

\speaker{Alle Drei}
und heute noch:
So heißen wir's dir,
tauschest den Ring du uns nicht -
 

\speaker{\Wellgunde und \Flosshilde}
im tiefen Rhein ihn zu bergen:
 

\speaker{Alle Drei}
Nur seine Flut sühnet den Fluch!
 

\Siegfriedspeaks

Ihr listigen Frauen, laßt das sein!
Traut' ich kaum eurem Schmeicheln,
euer Drohen schreckt mich noch minder!
 

\speaker{Die Drei Rheintöchter}
Siegfried! Siegfried!
Wir weisen dich wahr.
Weiche, weiche dem Fluch!
Ihn flochten nächtlich webende Nornen
in des Urgesetzes Seil!
 

\Siegfriedspeaks

Mein Schwert zerschwang einen Speer:
des Urgesetzes ewiges Seil,
flochten sie wilde Flüche hinein,
Notung zerhaut es den Nornen!
Wohl warnte mich einst
vor dem Fluch ein Wurm,
doch das Fürchten lehrt' er mich nicht!

\direct{Er betrachtet den Ring}

Der Welt Erbe gewänne mir ein Ring:
für der Minne Gunst miss' ich ihn gern;
ich geb' ihn euch, gönnt ihr mir Lust.
Doch bedroht ihr mir Leben und Leib:
faßte er nicht eines Fingers Wert,
den Reif entringt ihr mir nicht!
Denn Leben und Leib,
seht:so werf' ich sie weit von mir!
 


\direct{Er hebt eine Erdscholle vom Boden auf, hält sie über seinem Haupte und wirft sie mit den letzten Worten hinter sich}


\speaker{Die Drei Rheintöchter}
Kommt, Schwestern!
Schwindet dem Toren!
So weise und stark verwähnt sich der Held,
als gebunden und blind er doch ist.

\direct{Sie schwimmen, wild aufgeregt, in weiten Schwenkungen dicht an das Ufer heran}

Eide schwur er---und achtet sie nicht.

\direct{Wieder heftige Bewegung}

Runen weiß er---und rät sie nicht!
 

\speaker{\Flosshilde, dann \Woglinde}
Ein hehrstes Gut ward ihm vergönnt.
 

\speaker{Alle Drei}
Daß er's verworfen, weiß er nicht;
 

\Flosshildespeaks

nur den Ring, -
 

\Wellgundespeaks

der zum Tod ihm taugt, -
 

\speaker{Alle Drei}
den Reif nur will er sich wahren!
Leb' wohl, Siegfried!
Ein stolzes Weib
wird noch heute dich Argen beerben:
sie beut uns besseres Gehör:
Zu ihr! Zu ihr! Zu ihr!
 


\direct{Sie wenden sich schnell zum Reigen, mit welchem sie gemächlich dem Hintergrunde zu fortschwimmen}


\direct{Siegfried sieht ihnen lächelnd nach, stemmt ein Bein auf ein Felsstück am Ufer und verweilt mit auf der Hand gestütztem Kinne}


\speaker{Alle Drei}
Weialala leia, wallala leialala.
 

\Siegfriedspeaks

Im Wasser, wie am Lande
lernte nun ich Weiberart:
wer nicht ihrem Schmeicheln traut,
den schrecken sie mit Drohen;
wer dem kühnlich trotzt,
dem kommt dann ihr Keifen dran.

\direct{Die Rheintöchter sind hier gänzlich verschwunden}

Und doch, trüg' ich nicht Gutrun' Treu, -
der zieren Frauen eine
hätt' ich mir frisch gezähmt!

\direct{Er blickt ihnen unverwandt nach}


\speaker{Die Rheintöchter}

\direct{in größerer Entfernung}

La, la!

\direct{Jagdhornrufe kommen von der Höhe näher}




\speaker{\Hagen (Stimme)}

\direct{von fern}

Hoiho!
 


\direct{Siegfried fährt aus seiner träumerischen Entrücktheit auf und antwortet dem vernommenen Rufe auf seinem Horne}

\scene

\speaker{Die Mannen}

\direct{außerhalb der Szene}

Hoiho! Hoiho!
 

\Siegfriedspeaks


\direct{antwortend}

Hoiho! Hoiho! Hoihe!
 

\Hagenspeaks


\direct{kommt auf der Höhe hervor. Gunther folgt ihm. Siegfried erblickend}

Finden wir endlich,
wohin du flogest?
 

\Siegfriedspeaks

Kommt herab! Hier ist's frisch und kühl!
 


\direct{Die Mannen kommen alle auf der Höhe an und steigen nun mit Hagen und Gunther herab}


\Hagenspeaks

Hier rasten wir und rüsten das Mahl.

\direct{Jagdbeute wird zuhauf gelegt}

Laßt ruhn die Beute und bietet die Schläuche!

\direct{Trinkhörner und Schläuche werden hervorgeholt, dann lagert sich alles}

Der uns das Wild verscheuchte,
nun sollt ihr Wunder hören,
was Siegfried sich erjagt.
 

\Siegfriedspeaks


\direct{lachend}

Schlimm steht es um mein Mahl:
von eurer Beute bitte ich für mich.
 

\Hagenspeaks

Du beutelos?
 

\Siegfriedspeaks

Auf Waldjagd zog ich aus,
doch Wasserwild zeigte sich nur.
War ich dazu recht beraten,
drei wilde Wasservögel
hätt' ich euch wohl gefangen,
die dort auf dem Rheine mir sangen,
erschlagen würd' ich noch heut'.
 


\direct{Er lagert sich zwischen Gunther und Hagen}


\direct{Gunther erschrickt und blickt düster auf Hagen}


\Hagenspeaks

Das wäre üble Jagd,
wenn den Beutelosen selbst
ein lauernd Wild erlegte!
 

\Siegfriedspeaks

Mich dürstet!
 

\Hagenspeaks


\direct{indem er für Siegfried ein Trinkhorn füllen läßt und es diesem dann darreicht}

Ich hörte sagen, Siegfried,
der Vögel Sangessprache
verstündest du wohl:
so wäre das wahr?
 

\Siegfriedspeaks

Seit lange acht' ich des Lallens nicht mehr.
 


\direct{Er faßt das Trinkhorn und wendet sich damit zu Gunther. Er trinkt und reicht das Horn Gunther hin}

Trink', Gunther, trink'!
 

Dein Bruder bringt es dir!
 

\Guntherspeaks


\direct{gedankenvoll und schwermütig in das Horn blickend, dumpf}

Du mischtest matt und bleich:

\direct{noch gedämpfter}

dein Blut allein darin!
 

\Siegfriedspeaks


\direct{lachend}

So misch' ich's mit dem deinen!

\direct{Er gießt aus Gunthers Horn in das seine, so daß dieses überläuft}

Nun floß gemischt es über:
der Mutter Erde laß das ein Labsal sein!
 

\Guntherspeaks


\direct{mit einem heftigen Seufzer}

Du überfroher Held!
 

\Siegfriedspeaks


\direct{leise zu Hagen}

Ihm macht Brünnhilde Müh?
 

\Hagenspeaks


\direct{leise zu Siegfried}

Verstünd' er sie so gut,
wie du der Vögel Sang!
 

\Siegfriedspeaks

Seit Frauen ich singen hörte,
vergaß ich der Vöglein ganz.
 

\Hagenspeaks

Doch einst vernahmst du sie?
 

\Siegfriedspeaks


\direct{sich lebhaft zu Gunther wendend}

Hei! Gunther, grämlicher Mann!
Dankst du es mir,
so sing' ich dir Mären
aus meinen jungen Tagen.
 

\Guntherspeaks

Die hör' ich so gern.
 


\direct{Alle lagern sich nah an Siegfried, welcher allein aufrecht sitzt, während die andern tiefer gestreckt liegen}


\Hagenspeaks

So singe, Held!
 

\Siegfriedspeaks

Mime hieß ein mürrischer Zwerg:
in des Neides Zwang zog er mich auf,
daß einst das Kind, wann kühn es erwuchs,
einen Wurm ihm fällt' im Wald,
der faul dort hütet' einen Hort.
Er lehrte mich schmieden und Erze schmelzen;
doch was der Künstler selber nicht konnt',
des Lehrlings Mute mußt' es gelingen:
eines zerschlagnen Stahles Stücke
neu zu schmieden zum Schwert.
Des Vaters Wehr fügt' ich mir neu:
nagelfest schuf ich mir Notung.
Tüchtig zum Kampf dünkt' er dem Zwerg;
der führte mich nun zum Wald:
dort fällt' ich Fafner, den Wurm.
Jetzt aber merkt wohl auf die Mär':
Wunder muß ich euch melden.
Von des Wurmes Blut
mir brannten die Finger;
sie führt' ich kühlend zum Mund:
kaum netzt' ein wenig
die Zunge das Naß, -
was da die Vöglein sangen,
das konnt' ich flugs verstehn.
Auf den Ästen saß es und sang:
``Hei! Siegfried gehört nun
der Niblungen Hort!
Oh! Fänd' in der Höhle
den Hort er jetzt!
Wollt' er den Tarnhelm gewinnen,
der taugt' ihm zu wonniger Tat!
Doch möcht' er den Ring sich erraten,
der macht ihn zum Walter der Welt!''
 

\Hagenspeaks

Ring und Tarnhelm trugst du nun fort?
 

\speaker{Die Mannen}
Das Vöglein hörtest du wieder?
 

\Siegfriedspeaks

Ring und Tarnhelm hatt' ich gerafft:
da lauscht' ich wieder dem wonnigen Laller;
der saß im Wipfel und sang:
``Hei, Siegfried gehört nun der Helm und der Ring.
O traute er Mime, dem Treulosen, nicht!
Ihm sollt' er den Hort nur erheben;
nun lauert er listig am Weg:
nach dem Leben trachtet er Siegfried.
Oh, traute Siegfried nicht Mime!''
 

\Hagenspeaks

Es mahnte dich gut?
 

\speaker{Vier Mannen}
Vergaltest du Mime?
 

\Siegfriedspeaks

Mit tödlichem Tranke trat er zu mir;
bang und stotternd gestand er mir Böses:
Notung streckte den Strolch!
 

\Hagenspeaks


\direct{grell lachend}

Was er nicht geschmiedet,
schmeckte doch Mime!
 

\speaker{Zwei Mannen}

\direct{nacheinander}

Was wies das Vöglein dich wieder?
 

\Hagenspeaks


\direct{läßt ein Trinkhorn neu füllen und träufelt den Saft eines Krautes hinein}

Trink' erst, Held, aus meinem Horn:
ich würzte dir holden Trank,
die Erinnerung hell dir zu wecken,

\direct{er reicht Siegfried das Horn}

daß Fernes nicht dir entfalle!
 

\Siegfriedspeaks


\direct{blickt gedankenvoll in das Horn und trinkt dann langsam}

In Leid zu dem Wipfel lauscht' ich hinauf;
da saß es noch und sang:
``Hei, Siegfried erschlug nun den schlimmen Zwerg!
Jetzt wüßt' ich ihm noch das herrlichste Weib.
Auf hohem Felsen sie schläft,
Feuer umbrennt ihren Saal;
durchschritt' er die Brunst,
weckt' er die Braut -
Brünnhilde wäre dann sein!''
 

\Hagenspeaks

Und folgtest du des Vögleins Rate?
 

\Siegfriedspeaks

Rasch ohne Zögern zog ich nun aus,

\direct{Gunther hört mit wachsendem Erstaunen zu}

bis den feurigen Fels ich traf:
die Lohe durchschritt ich
und fand zum Lohn -

\direct{in immer größere Verzückung geratend}

schlafend ein wonniges Weib
in lichter Waffen Gewand.
Den Helm löst' ich der herrlichen Maid;
mein Kuß erweckte sie kühn:
oh, wie mich brünstig da umschlang
der schönen Brünnhilde Arm!
 

\Guntherspeaks


\direct{in höchstem Schrecken aufspringend}

Was hör' ich!
 


\direct{Zwei Raben fliegen aus einem Busche auf, kreisen über Siegfried und fliegen dann, dem Rheine zu, davon}


\Hagenspeaks

Errätst du auch dieser Raben Geraun'?
 


\direct{Siegfried fährt heftig auf und blickt, Hagen den Rücken zukehrend, den Raben nach}


\Hagenspeaks

Rache rieten sie mir!
 


\direct{Er stößt seinen Speer in Siegfrieds Rücken: Gunther fällt ihm---zu spät---in den Arm. Siegfried schwingt mit beiden Händen seinen Schild hoch empor, um Hagen damit zu zerschmettern: die Kraft verläßt ihn, der Schild entsinkt ihm rückwärts; er selbst stürzt krachend über dem Schilde zusammen}


\speaker{Vier Mannen}

\direct{welche vergebens Hagen zurückzuhalten versucht}

Hagen! Was tust du?
 

\speaker{Zwei Andere}
Was tatest du?
 

\Guntherspeaks

Hagen, was tatest du?
 

\Hagenspeaks


\direct{auf den zu Boden Gestreckten deutend}

Meineid rächt' ich!
 


\direct{Er wendet sich ruhig zur Seite ab und verliert sich dann einsam über die Höhe, wo man ihn langsam durch die bereits mit der Erscheinung der Raben eingebrochenen Dämmerung von dannen schreiten sieht. Gunther beugt sich schmerzergriffen zu Siegfrieds Seite nieder. Die Mannen umstehen teilnahmsvoll den Sterbenden}


\Siegfriedspeaks


\direct{von zwei Mannen sitzend erhalten, schlägt die Augen glanzvoll auf}

Brünnhilde! Heilige Braut!
Wach' auf! Öffne dein Auge!
Wer verschloß dich wieder in Schlaf?
Wer band dich in Schlummer so bang?
Der Wecker kam; er küßt dich wach,
und aber der Braut bricht er die Bande:
da lacht ihm Brünnhildes Lust!
Ach! Dieses Auge, ewig nun offen!
Ach, dieses Atems wonniges Wehen!
Süßes Vergehen, seliges Grauen:
Brünnhild' bietet mir---Gruß!
 


\StageDir{Er sinkt zurück und stirbt. Regungslose Trauer der Umstehenden. Die Nacht ist hereingebrochen. Auf die stumme Ermahnung Gunthers erheben die Mannen Siegfrieds Leiche und geleiten mit dem Folgenden sie in feierlichem Zuge über die Felsenhöhe langsam von dannen. Gunther folgt der Leiche zunächst.

Der Mond bricht durch die Wolken hervor und beleuchtet immer heller den die Berghöhe erreichenden Trauerzug. Dann steigen Nebel aus dem Rheine auf und erfüllen allmählich die ganze Bühne, auf welcher der Trauerzug bereits unsichtbar geworden ist, bis nach vorne, so daß diese während des Zwischenspiels gänzlich verhüllt bleibt. Als sich die Nebel wieder verteilen, tritt die Halle der Gibichungen, wie im ersten Aufzuge, immer erkennbarer hervor}


\scene

\StageDir{Es ist Nacht. Mondschein spiegelt sich auf dem Rheine. Gutrune tritt aus ihrem Gemache in die Halle hinaus}


\Gutrunespeaks

War das sein Horn?

\direct{Sie lauscht}

Nein!---Noch kehrt er nicht heim.---
Schlimme Träume störten mir den Schlaf!
Wild wieherte sein Roß;
Lachen Brünnhildes weckte mich auf.
Wer war das Weib,
das ich zum Ufer schreiten sah?
Ich fürchte Brünnhild'!
Ist sie daheim?

\direct{Sie lauscht an der Tür rechts und ruft dann leise}

Brünnhild'! Brünnhild'!
Bist du wach?

\direct{Sie öffnet schüchtern und blickt in das innere Gemach}

Leer das Gemach.
So war es sie,
die ich zum Rheine schreiten sah!

\direct{Sie erschrickt und lauscht nach der Ferne}

War das sein Horn?
Nein! Öd' alles!
Säh' ich Siegfried nur bald!
 


\direct{Sie will sich wieder ihrem Gemache zuwenden: als sie jedoch Hagens Stimme vernimmt, hält sie an und bleibt, von Furcht gefesselt, eine Zeitlang unbeweglich stehen}


\speaker{\Hagen (Stimme)}

\direct{von außen sich nähernd}

Hoiho! Hoiho!
Wacht auf! Wacht auf!
Lichte! Lichte! Helle Brände!
Jagdbeute bringen wir heim.
Hoiho! Hoiho!

\direct{Licht und wachsender Feuerschein von außen}


\Hagenspeaks


\direct{tritt in die Halle}

Auf, Gutrun'! Begrüße Siegfried!
Der starke Held, er kehret heim!
 

\Gutrunespeaks


\direct{im großer Angst}

Was geschah? Hagen!
Nicht hört' ich sein Horn!
 


\direct{Männer und Frauen, mit Lichtern und Feuerbränden, geleiten den Zug der mit Siegfrieds Leiche Heimkehrenden, unter denen Gunther}


\Hagenspeaks

Der bleiche Held,
nicht bläst er es mehr;
nicht stürmt er zur Jagd,
zum Streite nicht mehr,
noch wirbt er um wonnige Frauen.
 

\Gutrunespeaks


\direct{mit wachsendem Entsetzen}

Was bringen die?
 


\direct{Der Zug gelangt in die Mitte der Halle, und die Mannen setzen dort die Leiche auf einer schnell errichteten Erhöhung nieder}


\Hagenspeaks

Eines wilden Ebers Beute:
Siegfried, deinen toten Mann.
 


\direct{Gutrune schreit auf und stürzt über die Leiche hin. Allgemeine Erschütterung und Trauer}


\Guntherspeaks


\direct{bemüht sich um die Ohnmächtige}

Gutrun'! Holde Schwester,
hebe dein Auge, schweige mir nicht!
 

\Gutrunespeaks


\direct{wieder zu sich kommend}

Siegfried! Siegfried erschlagen!

\direct{Sie stößt Gunther heftig zurück}

Fort, treuloser Bruder,
du Mörder meines Mannes!
O Hilfe! Hilfe! Wehe! Wehe!
Sie haben Siegfried erschlagen!
 

\Guntherspeaks

Nicht klage wider mich!
Dort klage wider Hagen.
Er ist der verfluchte Eber,
der diesen Edlen zerfleischt'.
 

\Hagenspeaks

Bist du mir gram darum?
 

\Guntherspeaks

Angst und Unheil greife dich immer!
 

\Hagenspeaks


\direct{mit furchtbarem Trotze herantretend}

Ja denn! Ich hab' ihn erschlagen!
Ich, Hagen, schlug ihn zu Tod.
Meinem Speer war er gespart,
bei dem er Meineid sprach.
Heiliges Beuterecht
hab' ich mir nun errungen:
drum fordr' ich hier diesen Ring.
 

\Guntherspeaks

Zurück! Was mir verfiel,
sollst nimmer du empfahn.
 

\Hagenspeaks

Ihr Mannen, richtet mein Recht!
 

\Guntherspeaks

Rührst du an Gutrunes Erbe,
schamloser Albensohn?
 

\Hagenspeaks


\direct{sein Schwert ziehend}

Des Alben Erbe fordert so sein Sohn!
 


\direct{Er dringt auf Gunther ein, dieser wehrt sich; sie fechten. Die Mannen werfen sich dazwischen. Gunther fällt von einem Streiche Hagens darnieder}

Her den Ring!
 


\direct{Er greift nach Siegfrieds Hand; diese hebt sich drohend empor. Gutrune und die Frauen schreien entsetzt laut auf. Alles bleibt in Schauder regungslos gefesselt}


\direct{Vom Hintergrunde her schreitet Brünnhilde fest und feierlich dem Vordergrunde zu}


\Brunnhildespeaks


\direct{noch im Hintergrunde}

Schweigt eures Jammers
jauchzenden Schwall!
Das ihr alle verrietet,
zur Rache schreitet sein Weib.

\direct{Sie schreitet ruhig weiter vor}

Kinder hört' ich greinen nach der Mutter,
da süße Milch sie verschüttet:
doch nicht erklang mir würdige Klage,
des hehrsten Helden wert.
 

\Gutrunespeaks


\direct{vom Boden heftig sich aufrichtend}

Brünnhilde! Neiderboste!
Du brachtest uns diese Not:
die du die Männer ihm verhetztest,
weh, daß du dem Haus genaht!
 

\Brunnhildespeaks

Armselige, schweig'!
Sein Eheweib warst du nie,
als Buhlerin bandest du ihn.
Sein Mannesgemahl bin ich,
der ewige Eide er schwur,
eh' Siegfried je dich ersah.
 

\Gutrunespeaks


\direct{in jähe Verzweiflung ausbrechend}

Verfluchter Hagen!
Daß du das Gift mir rietest,
das ihr den Gatten entrückt!
Ach, Jammer!
Wie jäh nun weiß ich's,
Brünnhilde war die Traute,
die durch den Trank er vergaß! -
 


\direct{Sie wendet sich voll Scheu von Siegfried ab und beugt sich, im Schmerz aufgelöst, über Gunthers Leiche; so verbleibt sie regungslos bis zum Ende. Hagen steht, trotzig auf Speer und Schild gelehnt, in finsteres Sinnen versunken, auf der entgegengesetzen Seite}


\Brunnhildespeaks


\direct{allein in der Mitte; nachdem sie lange, zuerst mit tiefer Erschütterung, dann mit fast überwältigender Wehmut das Angesicht Siegfrieds betrachtet, wendet sie sich mit feierlicher Erhebung an die Männer und Frauen. Zu den Mannen}

Starke Scheite schichtet mir dort
am Rande des Rheins zuhauf!
Hoch und hell lodre die Glut,
die den edlen Leib
des hehrsten Helden verzehrt.
Sein Roß führet daher,
daß mit mir dem Recken es folge:
denn des Helden heiligste Ehre zu teilen,
verlangt mein eigener Leib.
Vollbringt Brünnhildes Wunsch!
 


\direct{Die jüngeren Männer errichten während des Folgenden vor der Halle nahe am Rheinufer einen mächtigen Scheiterhaufen, Frauen schmücken ihn mit Decken, auf die sie Kräuter und Blumen streuen}


\Brunnhildespeaks


\direct{versinkt von neuem in die Betrachtung des Antlitzes der Leiche Siegfrieds. Ihre Mienen nehmen eine immer sanftere Verklärung an}

Wie Sonne lauter strahlt mir sein Licht:
der Reinste war er, der mich verriet!
Die Gattin trügend,---treu dem Freunde,---
von der eignen Trauten---einzig ihm teuer---
schied er sich durch sein Schwert.
Echter als er schwur keiner Eide;
treuer als er hielt keiner Verträge;
lautrer als er liebte kein andrer:
und doch, alle Eide, alle Verträge,
die treueste Liebe trog keiner wie er! -
Wißt ihr, wie das ward?

\direct{nach oben blickend}

O ihr, der Eide ewige Hüter!
Lenkt euren Blick auf mein blühendes Leid:
erschaut eure ewige Schuld!
Meine Klage hör', du hehrster Gott!
Durch seine tapferste Tat,
dir so tauglich erwünscht,
weihtest du den, der sie gewirkt,
dem Fluche, dem du verfielest:
mich mußte der Reinste verraten,
daß wissend würde ein Weib!
Weiß ich nun, was dir frommt? -
Alles, alles, alles weiß ich,
alles ward mir nun frei!
Auch deine Raben hör' ich rauschen;
mit bang ersehnter Botschaft
send' ich die beiden nun heim.
Ruhe, ruhe, du Gott! -


\direct{Sie winkt den Mannen, Siegfrieds Leiche auf den Scheiterhaufen zu tragen; zugleich zieht sie von Siegfrieds Finger den Ring ab und betrachtet ihn sinnend}


Mein Erbe nun nehm' ich zu eigen.
Verfluchter Reif! Furchtbarer Ring!
Dein Gold fass' ich und geb' es nun fort.
Der Wassertiefe weise Schwestern,
des Rheines schwimmende Töchter,
euch dank' ich redlichen Rat.
Was ihr begehrt, ich geb' es euch:
aus meiner Asche nehmt es zu eigen!
Das Feuer, das mich verbrennt,
rein'ge vom Fluche den Ring!
Ihr in der Flut löset ihn auf,
und lauter bewahrt das lichte Gold,
das euch zum Unheil geraubt.
 


\direct{Sie hat sich den Ring angesteckt und wendet sich jetzt zu dem Scheiterhaufen, auf welchem Siegfrieds Leiche ausgestreckt liegt. Sie entreißt einem Manne den mächtigen Feuerbrand, schwingt diesen und deutet nach dem Hintergrunde}


Fliegt heim, ihr Raben!
Raunt es eurem Herren,
was hier am Rhein ihr gehört!
An Brünnhildes Felsen fahrt vorbei! -
Der dort noch lodert,
weiset Loge nach Walhall!
Denn der Götter Ende dämmert nun auf.
So---werf' ich den Brand
in Walhalls prangende Burg.
 


\direct{Sie schleudert den Brand in den Holzstoß, der sich schnell hell entzündet. Zwei Raben sind vom Felsen am Ufer aufgeflogen und verschwinden nach den Hintergrunde zu}


\direct{Brünnhilde gewahrt ihr Roß, welches zwei junge Männer hereinführen. Sie ist ihm entgegengesprungen, faßt es und entzäumt es schnell; dann neigt sie sich traulich zu ihm}


Grane, mein Roß!
Sei mir gegrüßt!
Weißt du auch, mein Freund,
wohin ich dich führe?
Im Feuer leuchtend, liegt dort dein Herr,
Siegfried, mein seliger Held.
Dem Freunde zu folgen, wieherst du freudig?
Lockt dich zu ihm die lachende Lohe?
Fühl' meine Brust auch, wie sie entbrennt;
helles Feuer das Herz mir erfaßt,
ihn zu umschlingen, umschlossen von ihm,
in mächtigster Minne vermählt ihm zu sein!
Heiajoho! Grane!
Grüß' deinen Herren!
Siegfried! Siegfried! Sieh!
Selig grüßt dich dein Weib!
 

\StageDir{Sie hat sich auf das Roß geschwungen und hebt es jetzt zum Sprunge. Sie sprengt es mit einem Satze in den brennenden Scheiterhaufen. Sogleich steigt prasselnd der Brand hoch auf, so daß das Feuer den ganzen Raum vor der Halle erfüllt und diese selbst schon zu ergreifen scheint. Entsetzt drängen sich Männer und Frauen nach dem äußersten Vordergrunde.}



\StageDir{Als der ganze Bühnenraum nur noch von Feuer erfüllt erscheint, verlischt plötzlich der Glutschein, so daß bald bloß ein Dampfgewölk zurückbleibt, welches sich dem Hintergrunde zu verzieht und dort am Horizont sich als finstere Wolkenschicht lagert. Zugleich ist vom Ufer her der Rhein mächtig angeschwollen und hat seine Flut über die Brandstätte gewälzt. Auf den Wogen sind die drei Rheintöchter herbeigeschwommen und erscheinen jetzt über der Brandstätte. Hagen, der seit dem Vorgange mit dem Ringe Brünnhildes Benehmen mit wachsender Angst beobachtet hat, gerät beim Anblick der Rheintöchter in höchsten Schreck. Er wirft hastig Speer, Schild und Helm von sich und stürzt wie wahnsinnig sich in die Flut.}



\Hagenspeaks


Zurück vom Ring!
 


\StageDir{Woglinde und Wellgunde umschlingen mit ihren Armen seinen Nacken und ziehen ihn so, zurückschwimmend, mit sich in die Tiefe. Flosshilde, den anderen voran dem Hintergrunde zuschwimmend, hält jubelnd den gewonnenen Ring in die Höhe. Durch die Wolkenschicht, welche sich am Horizont gelagert, bricht ein rötlicher Glutschein mit wachsender Helligkeit aus. Von dieser Helligkeit beleuchtet, sieht man die drei Rheintöchter auf den ruhigeren Wellen des allmählich wieder in sein Bett zurückgetretenen Rheines, lustig mit dem Ringe spielend, im Reigen schwimmen. Aus den Trümmern der zusammengestürzten Halle sehen die Männer und Frauen in höchster Ergriffenheit dem wachsenden Feuerschein am Himmel zu. Als dieser endlich in lichtester Helligkeit leuchtet, erblickt man darin den Saal Walhalls, in welchem die Götter und Helden, ganz nach der Schilderung Waltrautes im ersten Aufzuge, versammelt sitzen. Helle Flammen scheinen in dem Saal der Götter aufzuschlagen. Als die Götter von den Flammen gänzlich verhüllt sind, fällt der Vorhang}




\end{drama}


\end{document}


%%% Local Variables:
%%% mode: latex
%%% TeX-master: t
%%% End:
