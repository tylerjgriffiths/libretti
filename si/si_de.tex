\begin{drama}
\act

\scene

\StageDir{Wald.

Den Vordergrund bildet ein Teil einer Felsenhöhle, die sich links tiefer nach innen zieht, nach rechts aber gegen drei Vierteile der Bühne einnimmt. Zwei natürlich gebildete Eingänge stehen dem Walde zu offen: der eine nach rechts, unmittelbar im Hintergrunde, der andere, breitere, ebenda seitwärts. An der Hinterwand, nach links zu, steht ein grosser Schmiedeherd, aus Felsstücken natürlich geformt; künstlich ist nur der grosse Blasebalg: die rohe Esse geht---Lebenfalls natürlich---durch das Felsendach hinauf. Ein sehr grosser Amboss und andre Schmiedegerätschaften.}

\Mimespeaks

\direct{sitzt, als der Vorhang nach einem
kurzen Orchestervorspiel aufgeht,
am Ambosse und hämmert mit
wachsender Unruhe an einem
Schwerte: endlich hält er unmutig ein}

Zwangvolle Plage!
Müh' ohne Zweck!
Das beste Schwert,
das je ich geschweisst,
in der Riesen Fäusten
hielte es fest;
doch dem ich's geschmiedet,
der schmähliche Knabe,
er knickt und schmeisst es entzwei,
als schüf' ich Kindergeschmeid!

\direct{Mime wirft das Schwert unmutig
auf den Amboss, stemmt die Arme
ein und blickt sinnend zu Boden}

Es gibt ein Schwert,
das er nicht zerschwänge:
Notungs Trümmer
zertrotzt' er mir nicht,
könnt' ich die starken
Stücke schweissen,
die meine Kunst
nicht zu kitten weiss!
Könnt' ich's dem Kühnen schmieden,
meiner Schmach erlangt' ich da Lohn!

\direct{Er sinkt tiefer zurück und
neigt sinnend das Haupt}

Fafner, der wilde Wurm,
lagert im finstren Wald;
mit des furchtbaren Leibes Wucht
der Niblungen Hort
hütet er dort.
Siegfrieds kindischer Kraft
erläge wohl Fafners Leib:
des Niblungen Ring
erränge er mir.
Nur ein Schwert taugt zu der Tat;
nur Notung nützt meinem Neid,
wenn Siegfried sehrend ihn schwingt:
und ich kann's nicht schweissen,
Notung, das Schwert!

\direct{Er hat das Schwert wieder
zurechtgelegt und hämmert in
höchstem Unmut daran weiter}

Zwangvolle Plage!
Müh' ohne Zweck!
Das beste Schwert,
das je ich geschweisst,
nie taugt es je
zu der einzigen Tat!
Ich tappre und hämmre nur,
weil der Knabe es heischt:
er knickt und schmeisst es entzwei,
und schmäht doch, schmied' ich ihm nicht!

\direct{Er lässt den Hammer fallen}

\Siegfriedspeaks

\direct{Siegfried, in wilder Waldkleidung, mit einem silbernen Horn
an einer Kette, kommt mit jähem Ungestüm
aus dem Walde herein; er hat einen grossen Bären
mit einen Bastseile gezäumt und treibt diesen
mit lustigem Übermute gegen Mime an}

Hoiho! Hoiho!
Hau' ein! Hau' ein!
Friss ihn! Friss ihn,
Den Fratzenschmied!

\direct{Er lacht unbändig. Mimen entsinkt vor Schreck das Schwert;
er flüchtet hinter den Herd; Siegfried treibt ihm
den Bären überall nach}

\Mimespeaks

Fort mit dem Tier!
Was taugt mir der Bär?

\Siegfriedspeaks

Zu zwei komm ich,
dich besser zu zwicken:
Brauner, frag' nach dem Schwert!

\Mimespeaks

He! Lass das Wild!
Dort liegt die Waffe:
fertig fegt' ich sie heut'.

\Siegfriedspeaks

So fährst du heute noch heil!

\direct{Er löst dem Bären den Zaum
und gibt ihm damit einen
Schlag auf den Rücken}

Lauf', Brauner!
Dich brauch' ich nicht mehr!

\direct{Der Bär läuft in den Wald zurück}

\Mimespeaks:

\direct{kommt zitternd
hinter dem Herde hervor}

Wohl leid' ich's gern,
erlegst du Bären:
was bringst du lebend
die braunen heim?

\Siegfriedspeaks

\direct{setzt sich, um sich vom
Lachen zu erholen}

Nach bessrem Gesellen sucht' ich,
als daheim mir einer sitzt;
im tiefen Walde mein Horn
liess ich hallend da ertönen:
ob sich froh mir gesellte
ein guter Freund,
das frug ich mit dem Getön'!
Aus dem Busche kam ein Bär,
der hörte mir brummend zu;
er gefiel mir besser als du,
doch bessre fänd' ich wohl noch!
Mit dem zähen Baste
zäumt' ich ihn da,
dich, Schelm, nach dem Schwerte zu fragen.

\direct{Er springt auf und geht
auf den Amboss zu}

\Mimespeaks

\direct{nimmt das Schwert auf,
um es Siegfried zu reichen}

Ich schuf die Waffe scharf,
ihrer Schneide wirst du dich freun.

\direct{Er hält das Schwert ängstlich
in der Hand fest, das Siegfried
ihm heftig entwindet}

\Siegfriedspeaks

Was frommt seine helle Schneide,
ist der Stahl nicht hart und fest?

\direct{das Schwert mit der Hand prüfend}

Hei! Was ist das
für müss'ger Tand!
Den schwachen Stift
nennst du ein Schwert?

\direct{Er zerschlägt es auf dem Amboss,
dass die Stücken ringsum fliegen;
Mime weicht erschrocken aus}

Da hast du die Stücken,
schändlicher Stümper:
hätt' ich am Schädel
dir sie zerschlagen!
Soll mich der Prahler
länger noch prellen?
Schwatzt mir von Riesen
und rüstigen Kämpfen,
von kühnen Taten
und tüchtiger Wehr;
will Waffen mir schmieden,
Schwerte schaffen;
rühmt seine Kunst,
als könnt' er was Rechts:
nehm' ich zur Hand nun,
was er gehämmert,
mit einem Griff
zergreif' ich den Quark!
Wär' mir nicht schier
zu schäbig der Wicht,
ich zerschmiedet' ihn selbst
mit seinem Geschmeid,
den alten albernen Alp!
Des Ärgers dann hätt' ich ein End'!

\direct{Siegfried wirft sich wütend auf eine Steinbank
zur Seite rechts. Mime ist ihm immer vorsichtig ausgewichen.}

\Mimespeaks

Nun tobst du wieder wie toll:
dein Undank, traun, ist arg!
Mach' ich dem bösen Buben
nicht alles gleich zu best,
was ich ihm Gutes schuf,
vergisst er gar zu schnell!
Willst du denn nie gedenken,
was ich dich lehrt' vom Danke?
Dem sollst du willig gehorchen,
der je sich wohl dir erwies.

\direct{Siegfried wendet sich unmutig um,
mit dem Gesicht nach der Wand,
so dass er Mime den Rücken kehrt}

Das willst du wieder nicht hören!

\direct{Er steht verlegen; dann geht er in
die Küche am Herd}

Doch speisen magst du wohl?
Vom Spiesse bring' ich den Braten:
versuchtest du gern den Sud?
Für dich sott ich ihn gar.

\direct{Er bietet Siegfried Speise hin;
dieser, ohne sich umzuwenden,
schmeisst ihm Topf und Braten aus der Hand}

\Siegfriedspeaks

Braten briet ich mir selbst:
deinen Sudel sauf' allein!

\Mimespeaks

\direct{stellt sich empfindlich.
Mit kläglich kreischender Stimme}

Das ist nun der Liebe
schlimmer Lohn!
Das der Sorgen
schmählicher Sold!
Als zullendes Kind
zog ich dich auf,
wärmte mit Kleiden
den kleinen Wurm:
Speise und Trank
trug ich dir zu,
hütete dich
wie die eigne Haut.
Und wie du erwuchsest,
wartet' ich dein;
dein Lager schuf ich,
dass leicht du schliefst.
Dir schmiedet' ich Tand
und ein tönend Horn;
dich zu erfreun,
müht' ich mich froh:
mit klugem Rate
riet ich dir klug,
mit lichtem Wissen
lehrt' ich dich Witz.
Sitz' ich daheim
in Fleiss und Schweiss,
nach Herzenslust
schweifst du umher:
für dich nur in Plage,
in Pein nur für dich
verzehr' ich mich alter,
armer Zwerg!

\direct{schluchzend}

Und aller Lasten
ist das nun mein Lohn,
dass der hastige Knabe
mich quält und hasst!

\direct{schluchzend}

\direct{Siegfried hat sich wieder umgewendet und
ruhig in Mimes Blick geforscht. Mime begegnet Siegfrieds Blick
und sucht den seinigen scheu zu bergen}

\Siegfriedspeaks

Vieles lehrtest du, Mime,
und manches lernt' ich von dir;
doch was du am liebsten mich lehrtest,
zu lernen gelang mir nie:
wie ich dich leiden könnt'.
Trägst du mir Trank
und Speise herbei,
der Ekel speist mich allein;
schaffst du ein leichtes
Lager zum Schlaf,
der Schlummer wird mir da schwer;
willst du mich weisen,
witzig zu sein,
gern bleib' ich taub und dumm.
Seh' ich dir erst
mit den Augen zu,
zu übel erkenn' ich,
was alles du tust:
seh' ich dich stehn,
gangeln und gehn,
knicken und nicken,
mit den Augen zwicken:
beim Genick möcht' ich
den Nicker packen,
den Garaus geben
dem garst'gen Zwicker!
So lernt' ich, Mime, dich leiden.
Bist du nun weise,
so hilf mir wissen,
worüber umsonst ich sann:
in den Wald lauf' ich,
dich zu verlassen,
wie kommt das, kehr ich zurück?
Alle Tiere sind
mir teurer als du:
Baum und Vogel,
die Fische im Bach,
lieber mag ich sie
leiden als dich:
wie kommt das nun, kehr' ich zurück?
Bist du klug, so tu mir's kund.

\Mimespeaks

\direct{setzt sich in einiger Entfernung
ihm traulich gegenüber}

Mein Kind, das lehrt dich kennen,
wie lieb ich am Herzen dir lieg'.

\Siegfriedspeaks

\direct{lachend}

Ich kann dich ja nicht leiden,
vergiss das nicht so leicht!

\Mimespeaks

\direct{fährt zurück und setzt sich wieder
abseits, Siegfried gegenüber}

Des ist deine Wildheit schuld,
die du, Böser, bänd'gen sollst.
Jammernd verlangen Junge
nach ihrer Alten Nest;
Liebe ist das Verlangen;
so lechzest du auch nach mir,
so liebst du auch deinen Mime,
so musst du ihn lieben!
Was dem Vögelein ist der Vogel,
wenn er im Nest es nährt
eh' das flügge mag fliegen:
das ist dir kind'schem Spross
der kundig sorgende Mime,
das muss er dir sein!

\Siegfriedspeaks

Ei, Mime, bist du so witzig,
so lass mich eines noch wissen!
Es sangen die Vöglein
so selig im Lenz,
das eine lockte das andre:
du sagtest selbst,
da ich's wissen wollt',
das wären Männchen und Weibchen.
Sie kosten so lieblich,
und liessen sich nicht;
sie bauten ein Nest
und brüteten drin:
da flatterte junges
Geflügel auf,
und beide pflegten der Brut.
So ruhten im Busch
auch Rehe gepaart,
selbst wilde Füchse und Wölfe:
Nahrung brachte
zum Neste das Männchen,
das Weibchen säugte die Welpen.
Da lernt' ich wohl,
was Liebe sei:
der Mutter entwandt' ich
die Welpen nie.
Wo hast du nun, Mime,
dein minniges Weibchen,
dass ich es Mutter nenne?

\Mimespeaks

\direct{ärgerlich}

Was ist dir, Tor?
Ach, bist du dumm!
Bist doch weder Vogel noch Fuchs?

\Siegfriedspeaks

Das zullende Kind
zogest du auf,
wärmtest mit Kleiden
den kleinen Wurm:
wie kam dir aber
der kindische Wurm?
Du machtest wohl gar
ohne Mutter mich?

\Mimespeaks

\direct{in grosser Verlegenheit}

Glauben sollst du,
was ich dir sage:
ich bin dir Vater
und Mutter zugleich.

\Siegfriedspeaks

Das lügst du, garstiger Gauch!
Wie die Jungen den Alten gleichen,
das hab' ich mir glücklich ersehn.
Nun kam ich zum klaren Bach:
da erspäht' ich die Bäum'
und Tier' im Spiegel;
Sonn' und Wolken,
wie sie nur sind,
im Glitzer erschienen sie gleich.
Da sah ich denn auch
mein eigen Bild;
ganz anders als du
dünkt' ich mir da:
so glich wohl der Kröte
ein glänzender Fisch;
doch kroch nie ein Fisch aus der Kröte!

\Mimespeaks

\direct{höchst ärgerlich}

Gräulichen Unsinn
kramst du da aus!

\Siegfriedspeaks

\direct{immer lebendiger}

Siehst du, nun fällt
auch selbst mir ein,
was zuvor umsonst ich besann:
wenn zum Wald ich laufe,
dich zu verlassen,
wie das kommt, kehr' ich doch heim?

\direct{er springt auf}

Von dir erst muss ich erfahren,
wer Vater und Mutter mir sei!

\Mimespeaks

\direct{weicht ihm aus}

Was Vater! Was Mutter!
Müssige Frage!

\Siegfriedspeaks

\direct{packt ihn bei der Kehle}

So muss ich dich fassen,
um was zu wissen:
gutwillig
erfahr' ich doch nichts!
So musst' ich alles
ab dir trotzen:
kaum das Reden
hätt' ich erraten,
entwandt ich's mit Gewalt
nicht dem Schuft!
Heraus damit,
räudiger Kerl!
Wer ist mir Vater und Mutter?

\Mimespeaks

\direct{nachdem er mit dem Kopfe genickt
und mit den Händen gewinkt, ist von
Siegfried losgelassen worden}

Ans Leben gehst du mir schier!
Nun lass! Was zu wissen dich geizt,
erfahr' es, ganz wie ich's weiss.
O undankbares,
arges Kind!
Jetzt hör', wofür du mich hassest!
Nicht bin ich Vater
noch Vetter dir,
und dennoch verdankst du mir dich!
Ganz fremd bist du mir,
dem einzigen Freund;
aus Erbarmen allein
barg ich dich hier:
nun hab' ich lieblichen Lohn!
Was verhofft' ich Thor mir auch Dank?
Einst lag wimmernd ein Weib
da draussen im wilden Wald:
zur Höhle half ich ihr her,
am warmen Herd sie zu hüten.
Ein Kind trug sie im Schosse;
traurig gebar sie's hier;
sie wand sich hin und her,
ich half, so gut ich konnt'.
Gross war die Not! Sie starb,
doch Siegfried, der genas.

\Siegfriedspeaks

\direct{Siegfried steht sinnend}

So starb meine Mutter an mir?

\Mimespeaks

Meinem Schutz übergab sie dich:
ich schenkt' ihn gern dem Kind.
Was hat sich Mime gemüht,
was gab sich der Gute für Not!
``Als zullendes Kind
zog ich dich auf...''

\Siegfriedspeaks

Mich dünkt, des gedachtest du schon!
Jetzt sag': woher heiss' ich Siegfried?

MIME:
So hiess mich die Mutter,
möcht' ich dich heissen:
als ``Siegfried'' würdest
du stark und schön.
``Ich wärmte mit Kleiden
den kleinen Wurm....''

\Siegfriedspeaks

Nun melde, wie hiess meine Mutter?

\Mimespeaks

Das weiss ich wahrlich kaum!
``Speise und Trank
trug ich dir zu....''

\Siegfriedspeaks

Den Namen sollst du mir nennen!

\Mimespeaks

Entfiel er mir wohl? Doch halt!
Sieglinde mochte sie heissen,
die dich in Sorge mir gab.
``Ich hütete dich
wie die eigne Haut....''

\Siegfriedspeaks

\direct{immer dringender}

Dann frag' ich, wie hiess mein Vater?

\Mimespeaks

\direct{barsch}

Den hab' ich nie gesehn.

\Siegfriedspeaks

Doch die Mutter nannte den Namen?

\Mimespeaks

Erschlagen sei er,
das sagte sie nur;
dich Vaterlosen
befahl sie mir da:
``und wie du erwuchsest,
wartet' ich dein;
dein Lager schuf ich,
dass leicht du schliefst....''

\Siegfriedspeaks

Still mit dem alten
Starenlied!
Soll ich der Kunde glauben,
hast du mir nichts gelogen,
so lass mich Zeichen sehn!

\Mimespeaks

Was soll dir's noch bezeugen?

\Siegfriedspeaks

Dir glaub' ich nicht mit dem Ohr',
dir glaub' ich nur mit dem Aug':
welch Zeichen zeugt für dich?

\Mimespeaks

\direct{holt nach einigem Besinnen
die zwei Stücke eines zerschlagenen
Schwerts herbei}

Das gab mir deine Mutter:
für Mühe, Kost und Pflege
liess sie's als schwachen Lohn.
Sieh' her, ein zerbrochnes Schwert!
Dein Vater, sagte sie, führt' es,
als im letzten Kampf er erlag.

\Siegfriedspeaks

\direct{begeistert}

Und diese Stücke
sollst du mir schmieden:
dann schwing' ich ein rechtes Schwert!
Auf! Eile dich, Mime!
Mühe dich rasch;
kannst du was Rechts,
nun zeig' deine Kunst!
Täusche mich nicht
mit schlechtem Tand:
den Trümmern allein
trau' ich was zu!
Find' ich dich faul,
fügst du sie schlecht,
flickst du mit Flausen
den festen Stahl,
dir Feigem fahr' ich zu Leib',
das Fegen lernst du von mir!
Denn heute noch, schwör' ich,
will ich das Schwert;
die Waffe gewinn' ich noch heut'!

\Mimespeaks

\direct{erschrocken}

Was willst du noch heut' mit dem Schwert?

\Siegfriedspeaks

Aus dem Wald fort
in die Welt ziehn:
nimmer kehr' ich zurück!
Wie ich froh bin,
dass ich frei ward,
nichts mich bindet und zwingt!
Mein Vater bist du nicht;
in der Ferne bin ich heim;
dein Herd ist nicht mein Haus,
meine Decke nicht dein Dach.
Wie der Fisch froh
in der Flut schwimmt,
wie der Fink frei
sich davon schwingt:
flieg' ich von hier,
flute davon,
wie der Wind übern Wald
weh' ich dahin,
dich, Mime, nie wieder zu sehn!

\direct{Er stürmt in den Wald fort}

\Mimespeaks

\direct{in höchster Angst}

Halte! Halte! Wohin?


\direct{Er ruft mit der grössten
Anstrengung in den Wald}

He! Siegfried!
Siegfried! He!

\direct{Er sieht dem Fortstürmenden
eine Weile staunend nach; dann
kehrt er in die Schmiede zurück
und setzt sich hinter den Amboss}

Da stürmt er hin!
Nun sitz' ich da:
zur alten Not
hab' ich die neue;
vernagelt bin ich nun ganz!
Wie helf' ich mir jetzt?
Wie halt' ich ihn fest?
Wie führ' ich den Huien
zu Fafners Nest?
Wie füg' ich die Stücken
des tückischen Stahls?
Keines Ofens Glut
glüht mir die echten;
keines Zwergen Hammer
zwingt mir die harten.
Des Niblungen Neid,
Not und Schweiss
nietet mir Notung nicht,
schweisst mir das Schwert nicht zu ganz!

\direct{Mime knickt verzweifelnd auf dem Schemel
hinter dem Amboss zusammen}

\scene

\StageDir{Der Wanderer [Wotan] tritt aus dem Wald an das hintere Tor der Höhle heran. Er trägt einen dunkelblauen, langen Mantel; einen Speer führt er als Stab. Auf dem Haupte hat er einen grossen Hut mit breiter runder Krämpe, die über das fehlende eine Auge tief hereinhängt.}

\Wandererspeaks

Heil dir, weiser Schmied!
Dem wegmüden Gast
gönne hold
des Hauses Herd!

\Mimespeaks

\direct{ist erschrocken aufgefahren}

Wer ist's, der im wilden
Walde mich sucht?
Wer verfolgt mich im öden Forst?

\Wandererspeaks

\direct{sehr langsam, immer nur einen
Schritt sich nähernd}

``Wand'rer'' heisst mich die Welt;
weit wandert' ich schon:
auf der Erde Rücken
rührt' ich mich viel!

\Mimespeaks

So rühre dich fort
und raste nicht hier,
heisst dich ``Wand'rer'' die Welt!

\Wandererspeaks

Gastlich ruht' ich bei Guten,
Gaben gönnten viele mir:
denn Unheil fürchtet,
wer unhold ist.

\Mimespeaks

Unheil wohnte
immer bei mir:
willst du dem Armen es mehren?

\Wandererspeaks

\direct{langsam immer näherschreitend}

Viel erforscht' ich,
erkannte viel:
Wicht'ges konnt' ich
manchem künden,
manchem wehren,
was ihn mühte:
nagende Herzensnot.

\Mimespeaks

Spürtest du klug
und erspähtest du viel,
hier brauch' ich nicht Spürer noch Späher.
Einsam will ich
und einzeln sein,
Lungerern lass' ich den Lauf.

\Wandererspeaks

\direct{tritt wieder etwas näher}

Mancher wähnte
weise zu sein,
nur was ihm not tat,
wusste er nicht;
was ihm frommte,
liess ich erfragen:
lohnend lehrt' ihn mein Wort.

\Mimespeaks

\direct{immer ängstlicher, da er
den Wanderer sich nahen sieht}

Müss'ges Wissen
wahren manche:
ich weiss mir grade genug;

\direct{Der Wanderer schreitet
vollends bis an den Herd vor}

mir genügt mein Witz,
ich will nicht mehr:
dir Weisem weis' ich den Weg!

\Wandererspeaks

\direct{am Herd sich setzend}

Hier sitz' ich am Herd
und setze mein Haupt
der Wissenswette zum Pfand:
mein Kopf ist dein,
du hast ihn erkiest,
entfrägst du dir nicht,
was dir frommt,
lös' ich's mit Lehren nicht ein.

\Mimespeaks

\direct{der zuletzt den Wanderer mit
offenem Munde angestaunt hat,
schrickt jetzt zusammen;
kleinmütig für sich}

Wie werd' ich den Lauernden los?
Verfänglich muss ich ihn fragen.

\direct{Er ermannt sich wie zu Strenge}

Dein Haupt pfänd' ich
für den Herd:
nun sorg', es sinnig zu lösen!
Drei der Fragen
stell' ich mir frei.

\Wandererspeaks

Dreimal muss ich's treffen.

\Mimespeaks

\direct{sammelt sich zum Nachdenken}

Du rührtest dich viel
auf der Erde Rücken,
die Welt durchwandert'st du weit;
nun sage mir schlau:
welches Geschlecht
tagt in der Erde Tiefe?

\Wandererspeaks

In der Erde Tiefe
tagen die Nibelungen:
Nibelheim ist ihr Land.
Schwarzalben sind sie;
Schwarz-Alberich
hütet' als Herrscher sie einst!
Eines Zauberringes
zwingende Kraft
zähmt' ihm das fleissige Volk.
Reicher Schätze
schimmernden Hort
häuften sie ihm:
der sollte die Welt ihm gewinnen.
Zum zweiten was frägst du, Zwerg?

\Mimespeaks

\direct{versinkt in immer tieferes Nachsinnen}

Viel, Wanderer,
weisst du mir
aus der Erde Nabelnest;
nun sage mir schlicht,
welches Geschlecht
ruht auf der Erde Rücken?

\Wandererspeaks

Auf der Erde Rücken
wuchtet der Riesen Geschlecht:
Riesenheim ist ihr Land.
Fasolt und Fafner,
der Rauhen Fürsten,
neideten Nibelungs Macht;
den gewaltigen Hort
gewannen sie sich,
errangen mit ihm den Ring.
Um den entbrannte
den Brüdern Streit;
der Fasolt fällte,
als wilder Wurm
hütet nun Fafner den Hort.
Die dritte Frage nun droht.

\Mimespeaks

\direct{der ganz in Träumerei
entrückt ist}

Viel, Wanderer,
weisst du mir
von der Erde rauhem Rücken.
Nun sage mir wahr,
welches Geschlecht
wohnt auf wolkigen Höh'n?

\Wandererspeaks

Auf wolkigen Höhn
wohnen die Götter:
Walhall heisst ihr Saal.
Lichtalben sind sie;
Licht-Alberich,
Wotan, waltet der Schar.
Aus der Welt-Esche
weihlichstem Aste
schuf er sich einen Schaft:
dorrt der Stamm,
nie verdirbt doch der Speer;
mit seiner Spitze
sperrt Wotan die Welt.
Heil'ger Verträge
Treuerunen
schnitt in den Schaft er ein.
Den Haft der Welt
hält in der Hand,
wer den Speer führt,
den Wotans Faust umspannt.
Ihm neigte sich
der Niblungen Heer;
der Riesen Gezücht
zähmte sein Rat:
ewig gehorchen sie alle
des Speeres starkem Herrn.

\direct{Er stösst wie unwillkürlich mit
dem Speer auf den Boden;
ein leiser Donner lässt sich
vernehmen, wovon Mime
heftig erschrickt}

Nun rede, weiser Zwerg:
wusst' ich der Fragen Rat?
Behalte mein Haupt ich frei?

\Mimespeaks

\direct{nachdem er den Wanderer mit
dem Speer aufmerksam beobachtet
hat, gerät nun in grosse Angst,
sucht verwirrt nach seinen
Gerätschaften und blickt scheu
zur Seite}

Fragen und Haupt
hast du gelöst:
nun, Wand'rer, geh' deines Wegs!

\Wandererspeaks

Was zu wissen dir frommt,
solltest du fragen:
Kunde verbürgte mein Kopf.
Dass du nun nicht weisst,
was dir nützt,
des fass' ich jetzt deines als Pfand.
Gastlich nicht
galt mir dein Gruss,
mein Haupt gab ich
in deine Hand,
um mich des Herdes zu freun.
Nach Wettens Pflicht
pfänd' ich nun dich,
lösest du drei
der Fragen nicht leicht.
Drum frische dir, Mime, den Muth!

\Mimespeaks

\direct{sehr schüchtern und zögernd, endlich
in furchtsamer Ergebung sich fassend}

Lang' schon mied ich
mein Heimatland,
lang' schon schied ich
aus der Mutter Schoss;
mir leuchtete Wotans Auge,
zur Höhle lugt' es herein:
vor ihm magert
mein Mutterwitz.
Doch frommt mir's nun weise zu sein,
Wand'rer, frage denn zu!
Vielleicht glückt mir's, gezwungen
zu lösen des Zwerges Haupt.

\Wandererspeaks

\direct{wieder gemächlich sich niederlassend}

Nun, ehrlicher Zwerg,
sag' mir zum ersten:
welches ist das Geschlecht,
dem Wotan schlimm sich zeigte
und das doch das liebste ihm lebt?

\Mimespeaks

\direct{sich ermunternd}

Wenig hört' ich
von Heldensippen;
der Frage doch mach' ich mich frei.
Die Wälsungen sind
das Wunschgeschlecht,
das Wotan zeugte
und zärtlich liebte,
zeigt' er auch Ungunst ihm.
Siegmund und Sieglind'
stammten von Wälse,
ein wild-verzweifeltes
Zwillingspaar:
Siegfried zeugten sie selbst,
den stärksten Wälsungenspross.
Behalt' ich, Wand'rer,
zum ersten mein Haupt?

\Wandererspeaks

\direct{gemütlich}

Wie doch genau
das Geschlecht du mir nennst:
schlau eracht' ich dich Argen!
Der ersten Frage
wardst du frei.
Zum zweiten nun sag' mir, Zwerg:
ein weiser Niblung
wahret Siegfried;
Fafner soll er ihm fällen,
dass den Ring er erränge,
des Hortes Herrscher zu sein.
Welches Schwert
muss Siegfried nun schwingen,
taug' es zu Fafners Tod?

\Mimespeaks

\direct{seine gegenwärtige Lage
immer mehr vergessend und von
dem Gegenstande lebhaft angezogen,
reibt sich vergnügt die Hände}

Notung heisst
ein neidliches Schwert;
in einer Esche Stamm
stiess es Wotan:
dem sollt' es geziemen,
der aus dem Stamm es zög'.
Der stärksten Helden
keiner bestand's:
Siegmund, der Kühne,
konnt's allein:
fechtend führt' er's im Streit,
bis an Wotans Speer es zersprang.
Nun verwahrt die Stücken
ein weiser Schmied;
denn er weiss, dass allein
mit dem Wotansschwert
ein kühnes dummes Kind,
Siegfried, den Wurm versehrt.

\direct{ganz vergnügt}

Behalt' ich Zwerg
auch zweitens mein Haupt?

\Wandererspeaks

\direct{lachend}

Der witzigste bist du
unter den Weisen:
wer käm' dir an Klugheit gleich?
Doch bist du so klug,
den kindischen Helden
für Zwergenzwecke zu nützen,
mit der dritten Frage
droh' ich nun!
Sag' mir, du weiser
Waffenschmied:
wer wird aus den starken Stücken
Notung, das Schwert, wohl schweissen?

\Mimespeaks

\direct{fährt im höchsten Schrecken auf}

Die Stücken! Das Schwert!
O weh! Mir schwindelt!
Was fang' ich an?
Was fällt mir ein?
Verfluchter Stahl,
dass ich dich gestohlen!
Er hat mich vernagelt
in Pein und Not!
Mir bleibt er hart,
ich kann ihn nicht hämmern:
Niet' und Löte
lässt mich im Stich!

\direct{Er wirft wie sinnlos sein Gerät
durcheinander und bricht in helle
Verzweiflung aus}

Der weiseste Schmied
weiss sich nicht Rat!
Wer schweisst nun das Schwert,
schaff' ich es nicht?
Das Wunder, wie soll ich's wissen?

\Wandererspeaks

\direct{ist ruhig vom Herd aufgestanden}

Dreimal solltest du fragen,
dreimal stand ich dir frei:
nach eitlen Fernen
forschtest du;
doch was zunächst dir sich fand,
was dir nützt, fiel dir nicht ein.
Nun ich's errate,
wirst du verrückt:
gewonnen hab' ich
das witzige Haupt!
Jetzt, Fafners kühner Bezwinger,
hör', verfall'ner Zwerg:
``Nur wer das Fürchten
nie erfuhr,
schmiedet Notung neu.''

\direct{Mime starrt ihn gross an:
er wendet sich zum Fortgange}

Dein weises Haupt
wahre von heut':
verfallen lass' ich es dem,
der das Fürchten nicht gelernt!

\direct{Er wendet sich lächelnd ab und verschwindet schnell im Walde. Mime ist wie vernichtet auf den Schemel hinter dem Amboss zurückgesunken}

\scene

\Mimespeaks

\direct{starrt grad vor sich aus in den sonnig
beleuchteten Wald hinein und gerät
zunehmend in heftiges Zittern}

Verfluchtes Licht!
Was flammt dort die Luft?
Was flackert und lackert,
was flimmert und schwirrt,
was schwebt dort und webt
und wabert umher?
Da glimmert's und glitzt's
in der Sonne Glut!
Was säuselt und summt
und saust nun gar?
Es brummt und braust
und prasselt hieher!
Dort bricht's durch den Wald,
will auf mich zu!

\direct{Er bäumt sich vor Entsetzen auf}

Ein grässlicher Rachen
reisst sich mir auf:
der Wurm will mich fangen!
Fafner! Fafner!

\direct{Er sinkt laut schreiend hinter
dem breiten Amboss zusammen}

\Siegfriedspeaks

\direct{bricht aus dem Waldgesträuch
hervor und ruft noch hinter der
Szene, während man seine Bewegung
an dem zerkrachenden Gezweige
des Gesträuches gewahrt}

Heda! Du Fauler!
Bist du nun fertig!

\direct{Er tritt in die Höhle herein
und hält verwundert an}

Schnell! Wie steht's mit dem Schwert?
Wo steckt der Schmied?
Stahl er sich fort?
Hehe! Mime, du Memme!
Wo bist du? Wo birgst du dich?

\Mimespeaks

\direct{mit schwacher Stimme hinter
dem Amboss}

Bist du es, Kind?
Kommst du allein?

\Siegfriedspeaks

\direct{lachend}

Hinter dem Amboss?
Sag', was schufest du dort?
Schärftest du mir das Schwert?

\Mimespeaks

\direct{höchst verstört und
zerstreut hervorkommend}

Das Schwert? Das Schwert?
Wie möcht' ich's schweissen?

\direct{halb für sich}

``Nur wer das Fürchten
nie erfuhr,
schmiedet Notung neu.''
Zu weise ward ich
für solches Werk!

\Siegfriedspeaks

\direct{heftig}

Wirst du mir reden?
Soll ich dir raten?

\Mimespeaks

\direct{wie zuvor}

Wo nähm' ich redlichen Rat?
Mein weises Haupt
hab' ich verwettet:

\direct{vor sich hin starrend}

verfallen, verlor ich's an den,
``der das Fürchten nicht gelernt''.

\Siegfriedspeaks

\direct{ungestüm}

Sind mir das Flausen?
Willst du mir fliehn?

\Mimespeaks

\direct{allmählich sich etwas fassend}

Wohl flöh' ich dem,
der's Fürchten kennt!
Doch das liess ich dem Kinde zu lehren!
Ich Dummer vergass,
was einzig gut:
Liebe zu mir
sollt' er lernen;
das gelang nun leider faul!
Wie bring' ich das Fürchten ihm bei?

\Siegfriedspeaks

\direct{packt ihn}

He! Muss ich helfen?
Was fegtest du heut'?

\Mimespeaks

Um dich nur besorgt,
versank ich in Sinnen,
wie ich dich Wichtiges wiese.

\Siegfriedspeaks

\direct{lachend}

Bis unter den Sitz
warst du versunken:
was Wichtiges fandest du da?

\Mimespeaks

\direct{sich immer mehr fassend}

Das Fürchten lernt' ich für dich,
dass ich's dich Dummen lehre.

\Siegfriedspeaks

\direct{mit ruhiger Verwunderung}

Was ist's mit dem Fürchten?

\Mimespeaks

Erfuhrst du's noch nie
und willst aus dem Wald
doch fort in die Welt?
Was frommte das festeste Schwert,
blieb dir das Fürchten fern?

\Siegfriedspeaks

\direct{ungeduldig}

Faulen Rat
erfindest du wohl?

\Mimespeaks

\direct{immer zutraulicher Siegfried
näher tretend}

Deiner Mutter Rat
redet aus mir;
was ich gelobte,
muss ich nun lösen:
in die listige Welt
dich nicht zu entlassen,
eh' du nicht das Fürchten gelernt.

\Siegfriedspeaks

\direct{heftig}

Ist's eine Kunst,
was kenn' ich sie nicht?
Heraus! Was ist's mit dem Fürchten?

\Mimespeaks

Fühltest du nie
im finstren Wald,
bei Dämmerschein
am dunklen Ort,
wenn fern es säuselt,
summt und saust,
wildes Brummen
näher braust,
wirres Flackern
um dich flimmert,
schwellend Schwirren
zu Leib dir schwebt:
fühltest du dann nicht grieselnd
Grausen die Glieder dir fahen?
Glühender Schauer
schüttelt die Glieder,
in der Brust bebend und bang
berstet hämmernd das Herz?
Fühltest du das noch nicht,
das Fürchten blieb dir dann fremd.

\Siegfriedspeaks

\direct{nachsinnend}

Sonderlich seltsam
muss das sein!
Hart und fest,
fühl' ich, steht mir das Herz.
Das Grieseln und Grausen,
das Glühen und Schauern,
Hitzen und Schwindeln,
Hämmern und Beben:
gern begehr' ich das Bangen,
sehnend verlangt mich's der Lust!
Doch wie bringst du,
Mime, mir's bei?
Wie wärst du, Memme, mir Meister?

\Mimespeaks

Folge mir nur,
ich führe dich wohl:
sinnend fand ich es aus.
Ich weiss einen schlimmen Wurm,
der würgt' und schlang schon viel:
Fafner lehrt dich das Fürchten,
folgst du mir zu seinem Nest.

\Siegfriedspeaks

Wo liegt er im Nest?

\Mimespeaks

Neidhöhle
wird es genannt:
im Ost, am Ende des Walds.

\Siegfriedspeaks

Dann wär's nicht weit von der Welt?

\Mimespeaks

Bei Neidhöhle liegt sie ganz nah.

\Siegfriedspeaks

Dahin denn sollst du mich führen:
lernt' ich das Fürchten,
dann fort in die Welt!
Drum schnell! Schaffe das Schwert,
in der Welt will ich es schwingen.

\Mimespeaks

Das Schwert? O Not!

\Siegfriedspeaks

Rasch in die Schmiede!
Weis', was du schufst!

\Mimespeaks

Verfluchter Stahl!
Zu flicken versteh' ich ihn nicht:
den zähen Zauber
bezwingt keines Zwergen Kraft.
Wer das Fürchten nicht kennt,
der fänd' wohl eher die Kunst.

\Siegfriedspeaks

Feine Finten
weiss mir der Faule;
dass er ein Stümper,
sollt' er gestehn:
nun lügt er sich listig heraus!
Her mit den Stücken,
fort mit dem Stümper!

\direct{auf den Herd zuschreitend}

Des Vaters Stahl
fügt sich wohl mir:
ich selbst schweisse das Schwert!

\direct{Er macht sich, Mimes Gerät
durcheinander werfend,
mit Ungestüm an die Arbeit}

\Mimespeaks

Hättest du fleissig
die Kunst gepflegt,
jetzt käm' dir's wahrlich zugut;
doch lässig warst du
stets in der Lehr':
was willst du Rechtes nun rüsten?

\Siegfriedspeaks

Was der Meister nicht kann,
vermöcht' es der Knabe,
hätt' er ihm immer gehorcht?

\direct{Er dreht ihm eine Nase}

Jetzt mach' dich fort,
misch' dich nicht drein:
sonst fällst du mir mit ins Feuer!

\direct{Er hat eine grosse Menge Kohlen
auf dem Herd aufgehäuft und unterhält
in einem fort die Glut, während er die
Schwertstücke in den Schraubstock
einspannt und sie zu Spänen zerfeilt}

\Mimespeaks

\direct{der sich etwas abseits niedergesetzt
hat, sieht Siegfried bei der Arbeit zu}

Was machst du denn da?
Nimm doch die Löte:
den Brei braut' ich schon längst.

\Siegfriedspeaks

Fort mit dem Brei!
Ich brauch' ihn nicht:
Mit Bappe back' ich kein Schwert!

\Mimespeaks

Du zerfeilst die Feile,
zerreibst die Raspel:
wie willst du den Stahl zerstampfen?

\Siegfriedspeaks

Zersponnen muss ich
in Späne ihn sehn:
was entzwei ist, zwing' ich mir so.

\direct{Er feilt mit grossem Eifer fort}

\Mimespeaks

\direct{für sich}

Hier hilft kein Kluger,
das seh' ich klar:
hier hilft dem Dummen
die Dummheit allein!
Wie er sich rührt
und mächtig regt!
lhm schwindet der Stahl,
doch wird ihm nicht schwül!

\direct{Siegfried hat das Herdfeuer
zur hellsten Glut angefacht}

Nun ward ich so alt
wie Höhl' und Wald,
und hab' nicht so was geseh'n!

\direct{Während Siegfried mit ungestümem
Eifer fortfährt, die Schwertstücken
zu zerfeilen, setzt sich Mime noch
mehr beiseite}

Mit dem Schwert gelingt's,
das lern' ich wohl:
furchtlos fegt er's zu ganz.
Der Wand'rer wusst' es gut!
Wie berg' ich nun
mein banges Haupt?
Dem kühnen Knaben verfiel's,
lehrt' ihn nicht Fafner die Furcht!

\direct{mit wachsender Unruhe
aufspringend und sich beugend}

Doch weh' mir Armen!
Wie würgt' er den Wurm,
erführ' er das Fürchten von ihm?
Wie erräng' er mir den Ring?
Verfluchte Klemme!
Da klebt' ich fest,
fänd' ich nicht klugen Rat,
wie den Furchtlosen selbst ich bezwäng'.

\Siegfriedspeaks

\direct{hat nun die Stücken zerfeilt und
in einem Schmelztiegel gefangen,
den er jetzt in die Herdglut stellt}

He, Mime! Geschwind!
Wie heisst das Schwert,
das ich in Späne zersponnen?

\Mimespeaks

\direct{fährt zusammen und wendet sich
zu Siegfried}

Notung nennt sich
das neidliche Schwert:
deine Mutter gab mir die Mär.

\Siegfriedspeaks

\direct{nährt unter dem folgenden
die Glut mit dem Blasebalg}

Notung! Notung!
Neidliches Schwert!
Was musstest du zerspringen?
Zu Spreu nun schuf ich
die scharfe Pracht,
im Tiegel brat' ich die Späne.
Hoho! Hoho!
Hohei! Hohei!
Blase, Balg!
Blase die Glut!
Wild im Walde
wuchs ein Baum,
den hab' ich im Forst gefällt:
die braune Esche
brannt' ich zur Kohl',
auf dem Herd nun liegt sie gehäuft.
Hoho! Hoho!
Hohei! Hohei!
Blase, Balg!
Blase die Glut!
Des Baumes Kohle,
wie brennt sie kühn;
wie glüht sie hell und hehr!
In springenden Funken
sprühet sie auf:
Hohei! Hohei! Hohei!
Zerschmilzt mir des Stahles Spreu.
Hoho! Hoho!
Hohei! Hoho!
Blase, Balg!
Blase die Glut!

\Mimespeaks

\direct{immer für sich, entfernt sitzend}

Er schmiedet das Schwert,
und Fafner fällt er:
das seh' ich nun sicher voraus.
Hort und Ring
erringt er im Harst:
wie erwerb' ich mir den Gewinn?
Mit Witz und List
erlang' ich beides
und berge heil mein Haupt.

\Siegfriedspeaks

\direct{nochmals am Blasebalg}

Hoho! Hoho!
Hohei! Hohei!

\Mimespeaks

\direct{im Vordergrunde für sich}

Rang er sich müd mit dem Wurm,
von der Müh' erlab' ihn ein Trunk:
aus würz'gen Säften,
die ich gesammelt,
brau' ich den Trank für ihn;
wenig Tropfen nur
braucht er zu trinken,
sinnenlos sinkt er in Schlaf.
Mit der eignen Waffe,
die er sich gewonnen,
räum' ich ihn leicht aus dem Weg,
erlange mir Ring und Hort.

\direct{Er reibt sich vergnügt die Hände}

Hei! Weiser Wand'rer!
Dünkt' ich dich dumm?
Wie gefällt dir nun
mein feiner Witz?
Fand ich mir wohl
Rat und Ruh'?

\Siegfriedspeaks

Notung! Notung!
Neidliches Schwert!
Nun schmolz deines Stahles Spreu!
Im eignen Schweisse
schwimmst du nun.

\direct{Er giesst den glühenden Inhalt
des Tiegels in eine Stangenform
und hält diese in die Höhe}

Bald schwing' ich dich als mein Schwert!

\direct{Er stösst die gefüllte Stangenform
in den Wassereimer; Dampf und
lautes Gezisch der Kühlung erfolgen}

In das Wasser floss
ein Feuerfluss:
grimmiger Zorn
zischt' ihm da auf!
Wie sehrend er floss,
in des Wassers Flut
fliesst er nicht mehr.
Starr ward er und steif,
herrisch der harte Stahl:
heisses Blut doch
fliesst ihm bald!

\direct{Er stösst den Stahl in die Herdglut
und zieht die Blasebälge mächtig an}

Nun schwitze noch einmal,
dass ich dich schweisse,
Notung, neidliches Schwert!

\direct{Mime ist vergnügt aufgesprungen;
er holt verschiedene Gefässe hervor,
schüttet aus ihnen Gewürz und Kräuter
in einen Kochtopf und sucht, diesen
auf dem Herd anzubringen}

\direct{Siegfried beobachtet während der
Arbeit Mime, welcher vom andern
Ende des Herdes her seinen Topf
sorgsam an die Glut stellt}

Was schafft der Tölpel
dort mit dem Topf?
Brenn' ich hier Stahl,
braust du dort Sudel?

\Mimespeaks

Zuschanden kam ein Schmied,
den Lehrer sein Knabe lehrt:
mit der Kunst nun ist's beim Alten aus,
als Koch dient er dem Kind.
Brennt es das Eisen zu Brei,
aus Eiern braut
der Alte ihm Sud.

\direct{er fährt fort zu kochen}

\Siegfriedspeaks

Mime, der Künstler,
lernt jetzt kochen;
das Schmieden schmeckt ihm nicht mehr.
Seine Schwerter alle
hab' ich zerschmissen;
was er kocht, ich kost' es ihm nicht!

\direct{Unter dem Folgenden zieht Siegfried
die Stangenform aus der Glut,
zerschlägt sie und legt den glühenden
Stahl auf dem Amboss zurecht}

Das Fürchten zu lernen,
will er mich führen;
ein Ferner soll es mich lehren:
was am besten er kann,
mir bringt er's nicht bei:
als Stümper besteht er in allem!

\direct{während des Schmiedens}

Hoho! Hoho! Hohei!
Schmiede, mein Hammer,
ein hartes Schwert!
Hoho! Hahei!
Hoho! Hahei!

Einst färbte Blut
dein falbes Blau;
sein rotes Rieseln
rötete dich:
kalt lachtest du da,
das warme lecktest du kühl!
Heiaho! Haha!
Haheiaha!
Nun hat die Glut
dich rot geglüht;
deine weiche Härte
dem Hammer weicht:
zornig sprühst du mir Funken,
dass ich dich Spröden gezähmt!
Heiaho! Heiaho!
Heiahohoho!
Hahei!

\Mimespeaks

\direct{beiseite}

Er schafft sich ein scharfes Schwert,
Fafner zu fällen,
der Zwerge Feind:
ich braut' ein Truggetränk,
Siegfried zu fangen,
dem Fafner fiel.
Gelingen muss mir die List;
lachen muss mir der Lohn!

\direct{Er beschäftigt sich während des folgenden damit, den Inhalt des Topfes in eine Flasche zu giessen}

\Siegfriedspeaks

Hoho! Hoho!
Hahei!
Schmiede, mein Hammer,
ein hartes Schwert!
Hoho! Hahei!
Hoho! Hahei!
Der frohen Funken
wie freu' ich mich;
es ziert den Kühnen
des Zornes Kraft:
lustig lachst du mich an,
stellst du auch grimm dich und gram!
Heiaho, haha,
haheiaha!
Durch Glut und Hammer
glückt' es mir;
mit starken Schlägen
streckt' ich dich:
nun schwinde die rote Scham;
werde kalt und hart, wie du kannst.
Heiaho! Heiaho!
Heiahohoho!
Heiah!

\direct{Er schwingt den Stahl und stösst ihn in den Wassereimer. Er lacht bei dem Gezisch laut auf}

\direct{Während Siegfried die geschmiedete Schwertklinge in dem Griffhefte befestigt, treibt sich Mime mit der Flasche im Vordergrunde umher}

\Mimespeaks

Den der Bruder schuf,
den schimmernden Reif,
in den er gezaubert
zwingende Kraft,
das helle Gold,
das zum Herrscher macht,
ihn hab' ich gewonnen!
Ich walte sein!

\direct{Er trippelt, während Siegfried mit
dem kleinen Hammer arbeitet und
schleift und feilt, mit zunehmender
Vergnügtheit lebhaft umher}

Alberich selbst,
der einst mich band,
zur Zwergenfrone
zwing' ich ihn nun;
als Niblungenfürst
fahr' ich darnieder;
gehorchen soll mir
alles Heer!
Der verachtete Zwerg,
wie wird er geehrt!
Zu dem Horte hin drängt sich
Gott und Held:

\direct{mit immer lebhafteren Gebärden}

vor meinem Nicken
neigt sich die Welt,
vor meinem Zorne
zittert sie hin!
Dann wahrlich müht sich
Mime nicht mehr:
ihm schaffen andre
den ew'gen Schatz.
Mime, der kühne,
Mime ist König,
Fürst der Alben,
Walter des Alls!
Hei, Mime! Wie glückte dir das!
Wer hätte wohl das gedacht!

\Siegfriedspeaks

\direct{hat während der letzten Absätze
von Mimes Lied mit den letzten
Schlägen die Nieten des Griffheftes
geglättet und fasst nun das Schwert}

Notung! Notung!
Neidliches Schwert!
Jetzt haftest du wieder im Heft.
Warst du entzwei,
ich zwang dich zu ganz;
kein Schlag soll nun dich mehr zerschlagen.
Dem sterbenden Vater
zersprang der Stahl,
der lebende Sohn
schuf ihn neu:
nun lacht ihm sein heller Schein,
seine Schärfe schneidet ihm hart.

\direct{das Schwert vor sich schwingend}

Notung! Notung!
Neidliches Schwert!
Zum Leben weckt' ich dich wieder,
tot lagst du
in Trümmern dort,
jetzt leuchtest du trotzig und hehr.
Zeige den Schächern
nun deinen Schein!
Schlage den Falschen,
fälle den Schelm!
Schau, Mime, du Schmied:

\direct{Er holt mit dem Schwert aus}

so schneidet Siegfrieds Schwert!

\direct{Er schlägt auf den Amboss, welcher von oben bis unten in zwei Stücke zerspaltet, so dass er unter grossem Gepolter auseinander fällt. Mime, welcher in höchster Verzückung sich auf einen Schemel geschwungen hatte, fällt vor Schreck sitzlings zu Boden. Siegfried hält jauchzend das Schwert in die Höhe. Der Vorhang fällt}

\act

\scene

\StageDir{Tiefer Wald.

Ganz im Hintergrunde die Öffnung einer Höhle. Der Boden hebt sich bis zur Mitte der Bühne, wo er eine kleine Hochebene bildet; von da senkt er sich nach hinten, der Höhle zu, wieder abwärts, so dass von dieser nur der obere Teil der Öffnung dem Zuschauer sichtbar ist. Links gewahrt man durch Waldbäume eine zerklüftete Felsenwand. Finstere Nacht, am dichtesten über dem Hintergrunde, wo anfänglich der Blick des Zuschauers gar nichts zu unterscheiden vermag.}

\Alberichspeaks

\direct{an der Felsenwand zur Seite
gelagert, düster brütend}

In Wald und Nacht
vor Neidhöhl' halt' ich Wacht:
es lauscht mein Ohr,
mühvoll lugt mein Aug'.
Banger Tag,
bebst du schon auf?
Dämmerst du dort
durch das Dunkel her?

\direct{Aus dem Walde von rechts her
erhebt sich ein Sturmwind;
ein bläulicher Glanz leuchtet
von ebendaher}

Welcher Glanz glitzert dort auf?
Näher schimmert
ein heller Schein;
es rennt wie ein leuchtendes Ross,
bricht durch den Wald
brausend daher.
Naht schon des Wurmes Würger?
Ist's schon, der Fafner fällt?

\direct{Der Sturmwind legt sich wieder;
der Glanz verlischt}

Das Licht erlischt,
der Glanz barg sich dem Blick:
Nacht ist's wieder.

\direct{Der Wanderer tritt aus dem Wald
und hält Alberich gegenüber an}

Wer naht dort schimmernd im Schatten?

\Wandererspeaks

Zur Neidhöhle
fuhr ich bei Nacht:
wen gewahr' ich im Dunkel dort?

\direct{Wie aus einem plötzlich zerreissenden
Gewölk bricht Mondschein herein und
beleuchtet des Wanderers Gestalt}

\Alberichspeaks

\direct{erkennt den Wanderer, fährt
erschrocken zurück, bricht aber
sogleich in höchste Wut aus}

Du selbst lässt dich hier sehn?
Was willst du hier?
Fort, aus dem Weg!
Von dannen, schamloser Dieb!

\Wandererspeaks

\direct{ruhig}

Schwarz-Alberich,
schweifst du hier?
Hütest du Fafners Haus?

\Alberichspeaks

Jagst du auf neue
Neidtat umher?
Weile nicht hier,
weiche von hinnen!
Genug des Truges
tränkte die Stätte mit Not.
Drum, du Frecher,
lass sie jetzt frei!

\Wandererspeaks

Zu schauen kam ich,
nicht zu schaffen:
wer wehrte mir Wand'rers Fahrt?

\Alberichspeaks

\direct{lacht tückisch auf}

Du Rat wütender Ränke!
Wär' ich dir zulieb
doch noch dumm wie damals,
als du mich Blöden bandest,
wie leicht geriet' es,
den Ring mir nochmals zu rauben!
Hab' acht! Deine Kunst
kenne ich wohl;
doch wo du schwach bist,
blieb mir auch nicht verschwiegen.
Mit meinen Schätzen
zahltest du Schulden;
mein Ring lohnte
der Riesen Müh',
die deine Burg dir gebaut.
Was mit den Trotzigen
einst du vertragen,
des Runen wahrt noch heut'
deines Speeres herrischer Schaft.
Nicht du darfst,
was als Zoll du gezahlt,
den Riesen wieder entreissen:
du selbst zerspelltest
deines Speeres Schaft;
in deiner Hand
der herrische Stab,
der starke, zerstiebte wie Spreu!

\Wandererspeaks

Durch Vertrages Treuerunen
band er dich
Bösen mir nicht:
dich beugt' er mir durch seine Kraft;
zum Krieg drum wahr' ich ihn wohl!

\Alberichspeaks

Wie stolz du dräust
in trotziger Stärke,
und wie dir's im Busen doch bangt!
Verfallen dem Tod
durch meinen Fluch
ist des Hortes Hüter:
wer wird ihn beerben?
Wird der neidliche Hort
dem Niblungen wieder gehören?
Das sehrt dich mit ew'ger Sorge!
Denn fass' ich ihn wieder
einst in der Faust,
anders als dumme Riesen
üb' ich des Ringes Kraft:
dann zittre der Helden
heiliger Hüter!
Walhalls Höhen
stürm' ich mit Hellas Heer:
der Welt walte dann ich!

\Wandererspeaks

\direct{ruhig}

Deinen Sinn kenn' ich wohl;
doch sorgt er mich nicht.
Des Ringes waltet,
wer ihn gewinnt.

\Alberichspeaks

Wie dunkel sprichst du,
was ich deutlich doch weiss!
An Heldensöhne
hält sich dein Trotz,

\direct{höhnisch}

die traut deinem Blute entblüht.
Pflegtest du wohl eines Knaben,
der klug die Frucht dir pflücke,

\direct{immer heftiger}

die du nicht brechen darfst?

\Wandererspeaks

Mit mir nicht,
hadre mit Mime:
dein Bruder bringt dir Gefahr;
einen Knaben führt er daher,
der Fafner ihm fällen soll.
Nichts weiss der von mir;
der Niblung nützt ihn für sich.
Drum sag' ich dir, Gesell:
tue frei, wie dir's frommt!

\direct{Alberich macht eine Gebärde
heftiger Neugierde}

Höre mich wohl,
sei auf der Hut!
Nicht kennt der Knabe den Ring;
doch Mime kundet' ihn aus.

\Alberichspeaks

\direct{heftig}

Deine Hand hieltest du vom Hort?

\Wandererspeaks

Wen ich liebe,
lass' ich für sich gewähren;
er steh' oder fall',
sein Herr ist er:
Helden nur können mir frommen.

\Alberichspeaks

Mit Mime räng' ich
allein um den Ring?

\Wandererspeaks

Ausser dir begehrt er
einzig das Gold.

\Alberichspeaks

Und dennoch gewänn' ich ihn nicht?

\Wandererspeaks

\direct{ruhig nähertretend}

Ein Helde naht,
den Hort zu befrei'n;
zwei Niblungen geizen das Gold;
Fafner fällt,
der den Ring bewacht:
wer ihn rafft, hat ihn gewonnen.
Willst du noch mehr?
Dort liegt der Wurm:

\direct{er wendet sich nach der Höhle}

warnst du ihn vor dem Tod,
willig wohl liess' er den Tand.
Ich selber weck' ihn dir auf.

\direct{Er stellt sich auf die Anhöhe
vor der Höhle und ruft hinein}

Fafner! Fafner!
Erwache, Wurm!

\Alberichspeaks

\direct{in gespanntem Erstaunen, für sich}

Was beginnt der Wilde?
Gönnt er mir's wirklich?

\direct{Aus der finstern Tiefe des Hintergrundes hört man Fafners Stimme durch ein starkes Sprachrohr}

\Fafnerspeaks

Wer stört mir den Schlaf?

\Wandererspeaks

\direct{der Höhle zugewandt}

Gekommen ist einer,
Not dir zu künden:
er lohnt dir's mit dem Leben,
lohnst du das Leben ihm
mit dem Horte, den du hütest?

\direct{Er beugt sein Ohr lauschend
der Höhle zu}

\speaker{\Fafner (Stimme)}
Was will er?

\Alberichspeaks

\direct{ist dem Wanderer zur Seite
getreten und ruft in die Höhle}

Wache, Fafner!
Wache, du Wurm!
Ein starker Helde naht,
dich heil'gen will er bestehn.

\speaker{\Fafner (Stimme)}
Mich hungert sein.

\Wandererspeaks

Kühn ist des Kindes Kraft,
scharf schneidet sein Schwert.

\Alberichspeaks

Den goldnen Reif
geizt er allein:
lass mir den Ring zum Lohn,
so wend' ich den Streit;
du wahrest den Hort,
und ruhig lebst du lang'!

\speaker{\Fafner (Stimme)}
Ich lieg' und besitz':

\direct{gähnend}

lasst mich schlafen!

\Wandererspeaks

\direct{lacht auf und wendet sich
dann wieder zu Alberich}

Nun, Alberich, das schlug fehl.
Doch schilt mich nicht mehr Schelm!
Dies eine, rat' ich,
achte noch wohl:

\direct{vertraulich zum ihm tretend}

Alles ist nach seiner Art:
an ihr wirst du nichts ändern.
Ich lass' dir die Stätte,
stelle dich fest!
Versuch's mit Mime, dem Bruder:
der Art ja versiehst du dich besser.

\direct{zum Abgange gewendet}

Was anders ist,
das lerne nun auch!

\direct{Er verschwindet im Walde. Sturmwind erhebt sich, heller Glanz bricht aus; dann vergeht beides schnell}

\Alberichspeaks

\direct{blickt dem davonjagenden
Wanderer nach}

Da reitet er hin,
auf lichtem Ross;
mich lässt er in Sorg' und Spott.
Doch lacht nur zu,
ihr leichtsinniges,
lustgieriges
Göttergelichter!
Euch seh' ich
noch alle vergehn!
Solang' das Gold
am Lichte glänzt,
hält ein Wissender Wacht:
Trügen wird euch sein Trotz!

\direct{Er schlüpft zur Seite in das Geklüft. Die Bühne bleibt leer. Morgendämmerung}

\scene

\StageDir{Bei anbrechendem Tage treten Mime und Siegfried auf. Siegfried trägt das Schwert in einem Gehenke von Bastseil. Mime erspäht genau die Stätte; er forscht endlich dem Hintergrunde zu, welcher während die Anhöhe im mittleren Vordergrunde später immer heller von der Sonne beleuchtet wird in finstrem Schatten bleibt; dann bedeutet er Siegfried.}

\Mimespeaks

Wir sind zur Stelle!
Bleib hier stehn!

\Siegfriedspeaks

\direct{setzt sich unter einer grossen
Linde nieder und schaut sich um}

Hier soll ich das Fürchten lernen?
Fern hast du mich geleitet:
eine volle Nacht im Walde
selbander wanderten wir.
Nun sollst du, Mime,
mich meiden!
Lern' ich hier nicht,
was ich lernen muss,
allein zieh' ich dann weiter:
dich endlich werd' ich da los!

\Mimespeaks

\direct{setzt sich ihm gegenüber,
so dass er die Höhle immer
noch im Auge behält}

Glaube, Liebster!
Lernst du heut' und hier
das Fürchten nicht,
an andrem Ort,
zu andrer Zeit
schwerlich erfährst du's je.
Siehst du dort
den dunklen Höhlenschlund?
Darin wohnt
ein greulich wilder Wurm:
unmassen grimmig
ist er und gross;
ein schrecklicher Rachen
reisst sich ihm auf;
mit Haut und Haar
auf einen Happ
verschlingt der Schlimme dich wohl.

\Siegfriedspeaks

\direct{immer unter der Linde sitzend}

Gut ist's, den Schlund ihm zu schliessen:
drum biet' ich mich nicht dem Gebiss.

\Mimespeaks

Giftig giesst sich
ein Geifer ihm aus:
wen mit des Speichels
Schweiss er bespeit,
dem schwinden wohl Fleisch und Gebein.

\Siegfriedspeaks

Dass des Geifers Gift mich nicht sehre,
weich' ich zur Seite dem Wurm.

\Mimespeaks

Ein Schlangenschweif
schlägt sich ihm auf:
wen er damit umschlingt
und fest umschliesst,
dem brechen die Glieder wie Glas!

\Siegfriedspeaks

Vor des Schweifes Schwang mich zu wahren,
halt' ich den Argen im Aug'.
Doch heisse mich das:
hat der Wurm ein Herz?

\Mimespeaks

Ein grimmiges, hartes Herz!

\Siegfriedspeaks

Das sitzt ihm doch,
wo es jedem schlägt,
trag' es Mann oder Tier?

\Mimespeaks

Gewiss, Knabe,
da führt's auch der Wurm.
Jetzt kommt dir das Fürchten wohl an?

\Siegfriedspeaks

\direct{bisher nachlässig ausgestreckt,
erhebt sich rasch zum Sitz}

Notung stoss' ich
dem Stolzen ins Herz!
Soll das etwa Fürchten heissen?
He, du Alter!
Ist das alles,
was deine List
mich lehren kann?
Fahr' deines Wegs dann weiter;
das Fürchten lern' ich hier nicht.

\Mimespeaks

Wart' es nur ab!
Was ich dir sage,
dünke dich tauber Schall:
ihn selber musst du
hören und sehn,
die Sinne vergehn dir dann schon!
Wenn dein Blick verschwimmt,
der Boden dir schwankt,
im Busen bang
dein Herz erbebt:

\direct{sehr freundlich}

dann dankst du mir, der dich führte,
gedenkst, wie Mime dich liebt.

\Siegfriedspeaks

Du sollst mich nicht lieben!
Sagt' ich dir's nicht?
Fort aus den Augen mir!
Lass mich allein:
sonst halt' ich's hier länger nicht aus,
fängst du von Liebe gar an!
Das eklige Nicken
und Augenzwicken,
wann endlich soll ich's
nicht mehr sehn,
wann werd' ich den Albernen los?

\Mimespeaks

Ich lass' dich schon.
Am Quell dort lagr' ich mich;
steh' du nur hier;
steigt dann die Sonne zur Höh',
merk' auf den Wurm:
aus der Höhle wälzt er sich her,
hier vorbei
biegt er dann,
am Brunnen sich zu tränken.

\Siegfriedspeaks

\direct{lachend}

Mime, weilst du am Quell,
dahin lass' ich den Wurm wohl gehn:
Notung stoss' ich
ihm erst in die Nieren,
wenn er dich selbst dort
mit weggesoffen.
Darum, hör' meinen Rat,
raste nicht dort am Quell;
kehre dich weg,
so weit du kannst,
und komm' nie mehr zu mir!

\Mimespeaks

Nach freislichem Streit
dich zu erfrischen,
wirst du mir wohl nicht wehren?

\direct{Siegfried wehrt ihn hastig ab}

Rufe mich auch,
darbst du des Rates,

\direct{Siegfried wiederholt die
Gebärde mit Ungestüm}

oder wenn dir das Fürchten gefällt.

\direct{Siegfried erhebt sich und treibt Mime
mit wütender Gebärde zum Fortgehen}

\direct{im Abgehen für sich}

Fafner und Siegfried
Siegfried und Fafner
O brächten beide sich um!

\direct{Er verschwindet rechts im Wald}

\Siegfriedspeaks

\direct{streckt sich behaglich unter
der Linde aus und blickt dem
davongehenden Mime nach}

Dass der mein Vater nicht ist,
wie fühl' ich mich drob so froh!
Nun erst gefällt mir
der frische Wald;
nun erst lacht mir
der lustige Tag,
da der Garstige von mir schied
und ich gar nicht ihn wiederseh'!

\direct{Er verfällt in schweigendes Sinnen}

Wie sah mein Vater wohl aus?
Ha, gewiss wie ich selbst!
Denn wär' wo von Mime ein Sohn,
müsst' er nicht ganz
Mime gleichen?
Grade so garstig,
griesig und grau,
klein und krumm,
höckrig und kinkend,
mit hängenden Ohren,
triefigen Augen
fort mit dem Alp!
Ich mag ihn nicht mehr seh'n.

\direct{Er lehnt sich tiefer zurück und
blickt durch die Baumwipfel auf.
Tiefe Stille. Waldweben}

Aber wie sah
meine Mutter wohl aus?
Das kann ich
nun gar nicht mir denken!
Der Rehhindin gleich
glänzten gewiss
ihr hell schimmernde Augen,
nur noch viel schöner!
Da bang sie mich geboren,
warum aber starb sie da?
Sterben die Menschenmütter
an ihren Söhnen
alle dahin?
Traurig wäre das, traun!
Ach, möcht' ich Sohn
meine Mutter sehen!
Meine Mutter
ein Menschenweib!

\direct{Er seufzt leise und streckt
sich tiefer zurück. Grosse Stille.
Wachsendes Waldweben. Siegfrieds
Aufmerksamkeit wird endlich durch
den Gesang der Waldvögel gefesselt.
Er lauscht mit wachsender Teilnahme
einem Waldvogel in den Zweigen
über ihm}

Du holdes Vöglein!
Dich hört' ich noch nie:
bist du im Wald hier daheim?
Verstünd' ich sein süsses Stammeln!
Gewiss sagt' es mir was,
vielleicht von der lieben Mutter?
Ein zankender Zwerg
hat mir erzählt,
der Vöglein Stammeln
gut zu verstehn,
dazu könnte man kommen.
Wie das wohl möglich wär'?

\direct{Er sinnt nach. Sein Blick fällt
auf ein Rohrgebüsch unweit
der Linde}

Hei! Ich versuch's;
sing' ihm nach:
auf dem Rohr tön' ich ihm ähnlich!
Entrat' ich der Worte,
achte der Weise,
sing' ich so seine Sprache,
versteh' ich wohl auch, was es spricht.

\direct{Er eilt an den nahen Quell, schneidet
mit dem Schwerte ein Rohr ab und
schnitzt sich hastig eine Pfeife daraus.
Währenddem lauscht er wieder}

Es schweigt und lauscht:
so schwatz' ich denn los!

\direct{Er bläst auf dem Rohr. Er setzt ab,
schnitzt wieder und bessert. Er bläst
wieder. Er schüttelt mit dem Kopfe und
bessert wieder. Er wird ärgerlich, drückt
das Rohr mit der Hand und versucht
wieder. Er setzt lächelnd ganz ab}

Das tönt nicht recht;
auf dem Rohre taugt
die wonnige Weise mir nicht.
Vöglein, mich dünkt,
ich bleibe dumm:
von dir lernt sich's nicht leicht!

\direct{Er hört den Vogel wieder
und blickt zu ihm auf}

Nun schäm' ich mich gar
vor dem schelmischen Lauscher:
er lugt und kann nichts erlauschen.
Heida! So höre
nun auf mein Horn.

\direct{Er schwingt das Rohr
und wirft es weit fort}

Auf dem dummen Rohre
gerät mir nichts.
Einer Waldweise,
wie ich sie kann,
der lustigen sollst du nun lauschen.
Nach liebem Gesellen
lockt' ich mit ihr:
nichts Bessres kam noch
als Wolf und Bär.
Nun lass mich sehn,
wen jetzt sie mir lockt:
ob das mir ein lieber Gesell?

\direct{Er nimmt das silberne Hifthorn und bläst darauf. Im Hintergrunde regt es sich. Fafner, in der Gestalt eines ungeheuren eidechsenartigen Schlangenwurmes, hat sich in der Höhle von seinem Lager erhoben; er bricht durch das Gesträuch und wälzt sich aus der Tiefe nach der höheren Stelle vor, so dass er mit dem Vorderleibe bereits auf ihr angelangt ist, als er jetzt einen starken, gähnenden Laut ausstösst.}

\direct{sieht sich um und heftet den
Blick verwundert auf Fafner}

Haha! Da hätte mein Lied
mir was Liebes erblasen!
Du wärst mir ein saub'rer Gesell!

\Fafnerspeaks

\direct{hat beim Anblick Siegfrieds
auf der Höhe angehalten
und verweilt nun daselbst}

Was ist da?

\Siegfriedspeaks

Ei, bist du ein Tier,
das zum Sprechen taugt,
wohl liess' sich von dir was lernen?
Hier kennt einer
das Fürchten nicht:
kann er's von dir erfahren?

\Fafnerspeaks

Hast du Übermut?

\Siegfriedspeaks

Mut oder Übermut
was weiss ich!
Doch dir fahr' ich zu Leibe,
lehrst du das Fürchten mich nicht!

\Fafnerspeaks

\direct{stösst einen lachenden Laut aus}

Trinken wollt' ich:
nun treff' ich auch Frass!

\direct{Er öffnet seinen Rachen
und zeigt die Zähne}

\Siegfriedspeaks

Eine zierliche Fresse
zeigst du mir da,
lachende Zähne
im Leckermaul!
Gut wär' es, den Schlund dir zu schliessen;
dein Rachen reckt sich zu weit!

\Fafnerspeaks

Zu tauben Reden
taugt er schlecht:
dich zu verschlingen,
frommt der Schlund.

\direct{Er droht mit dem Schweife}

\Siegfriedspeaks

Hoho! Du grausam
grimmiger Kerl!
Von dir verdaut sein,
dünkt mich übel:
rätlich und fromm doch scheint's,
du verrecktest hier ohne Frist.

\Fafnerspeaks

\direct{brüllend}

Pruh! Komm,
prahlendes Kind!

\Siegfriedspeaks

Hab' acht, Brüller!
Der Prahler naht!

\direct{Er zieht sein Schwert, springt Fafner an und
bleibt herausfordernd stehen. Fafner wälzt
sich weiter auf die Höhe herauf und sprüht aus den Nüstern
auf Siegfried. Dieser weicht dem Geifer aus,
springt näher zu und stellt sich zur Seite. Fafner sucht ihn
mit dem Schweife zu erreichen. Siegfried, welchen Fafner fast erreicht hat,
springt mit einem Satze über diesen hinweg
und verwundet ihn an dem Schweife.
Fafner brüllt, zieht den Schweif heftig zurück und bäumt
den Vorderleib, um mit dessen voller Wucht
sich auf Siegfried zu werfen; so bietet er diesem die Brust dar;
Siegfried erspäht schnell die Stelle des Herzens und stösst sein
Schwert bis an das Heft hinein. Fafner bäumt sich vor Schmerz
noch höher und sinkt, als Siegfried das Schwert losgelassen
und zur Seite gesprungen ist,
auf die Wunde zusammen}

Da lieg', neidischer Kerl!
Notung trägst du im Herzen.

\Fafnerspeaks

\direct{mit schwächerer Stimme}

Wer bist du, kühner Knabe,
der das Herz mir traf?
Wer reizte des Kindes Mut
zu der mordlichen Tat?
Dein Hirn brütete nicht,
was du vollbracht.

\Siegfriedspeaks

Viel weiss ich noch nicht,
noch nicht auch, wer ich bin.
Mit dir mordlich zu ringen,
reiztest du selbst meinen Mut.

\Fafnerspeaks

Du helläugiger Knabe,
unkund deiner selbst,
wen du gemordet
meld' ich dir.
Der Riesen ragend Geschlecht,
Fasolt und Fafner,
die Brüder fielen nun beide.
Um verfluchtes Gold,
von Göttern vergabt,
traf ich Fasolt zu Tod.
Der nun als Wurm
den Hort bewachte,
Fafner, den letzten Riesen,
fällte ein rosiger Held.
Blicke nun hell,
blühender Knabe:
der dich Blinden reizte zur Tat,
berät jetzt des Blühenden Tod!

\direct{ersterbend}

Merk', wie's endet!
Acht' auf mich!

\Siegfriedspeaks

Woher ich stamme,
rate mir noch;
weise ja scheinst du,
Wilder, im Sterben:
rat' es nach meinem Namen:
Siegfried bin ich genannt.

\Fafnerspeaks

Siegfried...!

\direct{Er seufzt, hebt sich und stirbt}

\Siegfriedspeaks

Zur Kunde taugt kein Toter.
So leite mich denn
mein lebendes Schwert!

\direct{Fafner hat sich im Sterben
zur Seite gewälzt. Siegfried zieht
ihm jetzt das Schwert aus der Brust:
dabei wird seine Hand vom Blute
benetzt: er fährt heftig mit der Hand auf}

Wie Feuer brennt das Blut!

\direct{Er führt unwillkürlich die Finger
zum Munde, um das Blut von ihnen
abzusaugen. Wie er sinnend vor
sich hinblickt, wird seine
Aufmerksamkeit immer mehr von
dem Gesange der Waldvögel angezogen}

Ist mir doch fast,
als sprächen die Vöglein zu mir!
Nützte mir das
des Blutes Genuss?
Das seltne Vöglein hier,
horch, was singt es nur?

\speaker{Eine Waldvogel (Stimme)}
\direct{aus den Zweigen der Linde
über Siegfried}

Hei! Siegfried gehört
nun der Niblungen Hort!
O, fänd' in der Höhle
den Hort er jetzt!
Wollt' er den Tarnhelm gewinnen,
der taugt' ihm zu wonniger Tat:
doch möcht' er den Ring sich erraten,
der macht' ihn zum Walter der Welt!

\Siegfriedspeaks

\direct{hat mit verhaltenem Atem
und verzückter Miene gelauscht}

Dank, liebes Vöglein,
für deinen Rat!
Gern folg' ich dem Ruf!

\direct{Er wendet sich nach hinten und
steigt in die Höhle hinab, wo er
alsbald gänzlich verschwindet}

\scene

\StageDir{Mime schleicht heran, scheu umherblickend, um sich von Fafners Tod zu überzeugen. Gleichzeitig kommt von der anderen Seite Alberich aus dem Geklüft; er beobachtet Mime genau. Als dieser Siegfried nicht mehr gewahrt und vorsichtig sich nach hinten der Höhle zuwendet, stürzt Alberich auf ihn zu und vertritt ihm den Weg}

\Alberichspeaks

Wohin schleichst du
eilig und schlau,
schlimmer Gesell?

\Mimespeaks

Verfluchter Bruder,
dich braucht' ich hier!
Was bringt dich her?

\Alberichspeaks

Geizt es dich, Schelm,
nach meinem Gold?
Verlangst du mein Gut?

\Mimespeaks

Fort von der Stelle!
Die Stätte ist mein:
was stöberst du hier?

\Alberichspeaks

Stör' ich dich wohl
im stillen Geschäft,
wenn du hier stiehlst?

\Mimespeaks

Was ich erschwang
mit schwerer Müh',
soll mir nicht schwinden.

\Alberichspeaks

Hast du dem Rhein
das Gold zum Ringe geraubt?
Erzeugtest du gar
den zähen Zauber im Reif?

\Mimespeaks

Wer schuf den Tarnhelm,
der die Gestalten tauscht?
Der seiner bedurfte,
erdachtest du ihn wohl?

\Alberichspeaks

Was hättest du Stümper
je wohl zu stampfen verstanden?
Der Zauberring
zwang mir den Zwerg erst zur Kunst.

\Mimespeaks

Wo hast du den Ring?
Dir Zagem entrissen ihn Riesen!
Was du verlorst,
meine List erlangt es für mich.

\Alberichspeaks

Mit des Knaben Tat
will der Knicker nun knausern?
Dir gehört sie gar nicht,
der Helle ist selbst ihr Herr!

\Mimespeaks

Ich zog ihn auf;
für die Zucht zahlt er mir nun:
für Müh' und Last
erlauert' ich lang meinen Lohn!

\Alberichspeaks

Für des Knaben Zucht
will der knickrige
schäbige Knecht
keck und kühn
wohl gar König nun sein?
Dem räudigsten Hund
wäre der Ring
geratner als dir:
nimmer erringst
du Rüpel den Herrscherreif!

\Mimespeaks

\direct{kratzt sich den Kopf}

Behalt' ihn denn:
und hüt' ihn wohl,
den hellen Reif!
Sei du Herr:
doch mich heisse auch Bruder!
Um meines Tarnhelms
lustigen Tand
tausch' ich ihn dir:
uns beiden taugt's,
teilen die Beute wir so.

\direct{Er reibt sich zutraulich die Hände}

\Alberichspeaks

\direct{mit Hohnlachen}

Teilen mit dir?
Und den Tarnhelm gar?
Wie schlau du bist!
Sicher schlief' ich
niemals vor deinen Schlingen!

\Mimespeaks

\direct{ausser sich}

Selbst nicht tauschen?
Auch nicht teilen?
Leer soll ich gehn?
Ganz ohne Lohn?

\direct{kreischend}

Gar nichts willst du mir lassen?

\Alberichspeaks

Nichts von allem!
Nicht einen Nagel
sollst du dir nehmen!

\Mimespeaks

\direct{in höchster Wut}

Weder Ring noch Tarnhelm
soll dir denn taugen!
Nicht teil' ich nun mehr!
Gegen dich doch ruf' ich
Siegfried zu Rat
und des Recken Schwert;
der rasche Held,
der richte, Brüderchen, dich!

\direct{Siegfried erscheint im Hintergrund}

\Alberichspeaks

Kehre dich um!
Aus der Höhle kommt er daher!

\Mimespeaks

\direct{sich umblickend}

Kindischen Tand
erkor er gewiss.

\Alberichspeaks

Den Tarnhelm hält er!

\Mimespeaks

Doch auch den Ring!

\Alberichspeaks

Verflucht! Den Ring!

\Mimespeaks

\direct{hämisch lachend}

Lass ihn den Ring dir doch geben!
Ich will ihn mir schon gewinnen.

\direct{Er schlüpft mit den letzten
Worten in den Wald zurück}

\Alberichspeaks

Und doch seinem Herrn
soll er allein noch gehören!

\direct{Er verschwindet im Geklüfte}

\direct{Siegfried ist mit Tarnhelm und Ring während des letzteren langsam und sinnend aus der Höhle vorgeschritten: er betrachtet gedankenvoll seine Beute und hält, nahe dem Baume, auf der Höhe des Mittelgrundes wieder an}

\Siegfriedspeaks

Was ihr mir nützt,
weiss ich nicht;
doch nahm ich euch
aus des Horts gehäuftem Gold,
weil guter Rat mir es riet.
So taug' eure Zier
als des Tages Zeuge,
es mahne der Tand,
dass ich kämpfend Fafner erlegt,
doch das Fürchten noch nicht gelernt!

\direct{Er steckt den Tarnhelm sich in den Gürtel und den Reif an den Finger. Stillschweigen. Wachsendes Waldweben. Siegfried achtet unwillkürlich wieder des Vogels und lauscht ihm mit verhaltenem Atem}

\speaker{Eine Waldvogel (Stimme)}
Hei! Siegfried gehört
nun der Helm und der Ring!
O, traute er Mime,
dem treulosen, nicht!
Hörte Siegfried nur scharf
auf des Schelmen Heuchlergered'!
Wie sein Herz es meint,
kann er Mime verstehn:
so nützt' ihm des Blutes Genuss.

\direct{Siegfrieds Miene und Gebärde drücken aus,
dass er den Sinn des Vogelgesanges wohl vernommen. Er sieht
Mime sich nähern und bleibt, ohne sich zu rühren, auf sein Schwert
gestützt, beobachtend und in sich geschlossen,in seiner Stellung
auf der Anhöhe bis zum Schlusse des folgenden Auftrittes}

\Mimespeaks

\direct{schleicht heran und beobachtet
vom Vordergrunde aus Siegfried}

Er sinnt und erwägt
der Beute Wert:
weilte wohl hier
ein weiser Wand'rer,
schweifte umher,
beschwatzte das Kind
mit list'ger Runen Rat?
Zwiefach schlau
sei nun der Zwerg;
die listigste Schlinge
leg' ich jetzt aus,
dass ich mit traulichem
Truggerede
betöre das trotzige Kind.

\direct{er tritt näher an Siegfried heran
und bewillkommt diesen mit
schmeichelnden Gebärden}

Willkommen, Siegfried!
Sag', du Kühner,
hast du das Fürchten gelernt?

\Siegfriedspeaks

Den Lehrer fand ich noch nicht!

\Mimespeaks

Doch den Schlangenwurm,
du hast ihn erschlagen?
Das war doch ein schlimmer Gesell?

\Siegfriedspeaks

So grimm und tückisch er war,
sein Tod grämt mich doch schier,
da viel üblere Schächer
unerschlagen noch leben!
Der mich ihn morden hiess,
den hass' ich mehr als den Wurm!

\Mimespeaks

\direct{sehr freundlich}

Nur sachte! Nicht lange
siehst du mich mehr:
zum ew'gen Schlaf
schliess' ich dir die Augen bald!
Wozu ich dich brauchte,

\direct{zärtlich}

hast du vollbracht;
jetzt will ich nur noch
die Beute dir abgewinnen.
Mich dünkt, das soll mir gelingen;
zu betören bist du ja leicht!

SIEGFRIED:
So sinnst du auf meinen Schaden?

\Mimespeaks

\direct{verwundert}

Wie sagt' ich denn das?
Siegfried! Hör doch, mein Söhnchen!
Dich und deine Art
hasst' ich immer von Herzen;

\direct{zärtlich}

aus Liebe erzog ich
dich Lästigen nicht:
dem Horte in Fafners Hut,
dem Golde galt meine Müh'.

\direct{als verspräche er ihm hübsche Sachen}

Gibst du mir das
gutwillig nun nicht,

\direct{als wäre er bereit, sein Leben
für ihn zu lassen}

Siegfried, mein Sohn,
das siehst du wohl selbst,

\direct{mit freundlichem Scherze}

dein Leben musst du mir lassen!

\Siegfriedspeaks

Dass du mich hassest,
hör' ich gern:
doch auch mein Leben muss ich dir lassen?

\Mimespeaks

\direct{ärgerlich}

Das sagt' ich doch nicht?
Du verstehst mich ja falsch!

\direct{Er sucht sein Fläschchen hervor.
Er gibt sich die ersichtlichste
Mühe zur Verstellung}

Sieh', du bist müde
von harter Müh';
brünstig wohl brennt dir der Leib:
dich zu erquicken
mit queckem Trank
säumt' ich Sorgender nicht.
Als dein Schwert du dir branntest,
braut' ich den Sud;
trinkst du nun den,
gewinn' ich dein trautes Schwert,
und mit ihm Helm und Hort.

\direct{er kichert dazu}

\Siegfriedspeaks

So willst du mein Schwert
und was ich erschwungen,
Ring und Beute, mir rauben?

\Mimespeaks

\direct{heftig}

Was du doch falsch mich verstehst!
Stamml' ich, fasl' ich wohl gar?
Die grösste Mühe
geb' ich mir doch,
mein heimliches Sinnen
heuchelnd zu bergen,
und du dummer Bube
deutest alles doch falsch!
Öffne die Ohren,
und vernimm genau:
Höre, was Mime meint!

\direct{wieder sehr freundlich,
mit ersichtlicher Mühe}

Hier nimm und trinke die Labung!
Mein Trank labte dich oft:
tat'st du wohl unwirsch,
stelltest dich arg:
was ich dir bot,
erbost auch nahmst du's doch immer.

\Siegfriedspeaks

\direct{ohne eine Miene zu verziehen}

Einen guten Trank
hätt' ich gern:
wie hast du diesen gebraut?

\Mimespeaks

\direct{lustig scherzend, als schildere
er ihm einen angenehm berauschten
Zustand, den ihm der Saft bereiten soll}

Hei! So trink nur,
trau' meiner Kunst!
In Nacht und Nebel
sinken die Sinne dir bald:
ohne Wach' und Wissen
stracks streckst du die Glieder.
Liegst du nun da,
leicht könnt' ich
die Beute nehmen und bergen:
doch erwachtest du je,
nirgends wär' ich
sicher vor dir,
hätt' ich selbst auch den Ring.
Drum mit dem Schwert,
das so scharf du schufst,

\direct{mit einer Gebärde
ausgelassener Lustigkeit}

hau' ich dem Kind
den Kopf erst ab:
dann hab' ich mir Ruh' und auch den Ring!

\direct{Er kichert wieder}

\Siegfriedspeaks

Im Schlafe willst du mich morden?

\Mimespeaks

\direct{wütend ärgerlich}

Was möcht' ich? Sagt' ich denn das?

\direct{Er bemüht sich, den zärtlichsten
Ton anzunehmen}

Ich will dem Kind

\direct{mit sorglichster Deutlichkeit}

nur den Kopf abhau'n!

\direct{mit dem Ausdruck herzlicher
Besorgtheit für Siegfrieds Gesundheit}

Denn hasste ich dich
auch nicht so sehr,
und hätt' ich des Schimpfs
und der schändlichen Mühe
auch nicht so viel zu rächen:

\direct{sanft}

aus dem Wege dich zu räumen,
darf ich doch nicht rasten:
wie käm' ich sonst anders zur Beute,
da Alberich auch nach ihr lugt?

\direct{Er giesst den Saft in das Trinkhorn
und führt dieses Siegfried mit
aufdringlicher Gebärde zu}

Nun, mein Wälsung!
Wolfssohn du!
Sauf', und würg' dich zu Tod:
Nie tust du mehr 'nen Schluck!

\direct{Siegfried holt mit dem Schwert aus. Er führt, wie in einer Anwandlung heftigen Ekels einen jähen Streich nach Mime; dieser stürzt sogleich tot zu Boden. Man hört Alberichs höhnisches Gelächter aus dem Geklüfte}

\Siegfriedspeaks

Schmeck' du mein Schwert,
ekliger Schwätzer!

\direct{Er henkt, auf den am
Boden Liegenden blickend,
ruhig sein Schwert wieder ein}

Neides Zoll
zahlt Notung:
dazu durft' ich ihn schmieden.

\direct{Er rafft Mimes Leichnam auf,
trägt ihn auf die Anhöhe vor
den Eingang der Höhle und
wirft ihn dort hinein}

In der Höhle hier
lieg' auf dem Hort!
Mit zäher List
erzieltest du ihn:
jetzt magst du des wonnigen walten!
Einen guten Wächter
geb' ich dir auch,
dass er vor Dieben dich deckt.

\direct{Er wälzt mit grosser Anstrengung
den Leichnam des Wurmes vor den
Eingang der Höhle, so dass er diesen
ganz damit verstopft}

Da lieg' auch du,
dunkler Wurm!
Den gleissenden Hort
hüte zugleich
mit dem beuterührigen Feind:
so fandet beide ihr nun Ruh'!

\direct{Er blickt eine Weile sinnend
in die Höhle hinab und wendet
sich dann langsam, wie ermüdet,
in den Vordergrund. Es ist Mittag.
Er führt sich die Hand über die Stirn}

Heiss ward mir
von der harten Last!
Brausend jagt
mein brünst'ges Blut;
die Hand brennt mir am Haupt.
Hoch steht schon die Sonne:
aus lichtem Blau
blickt ihr Aug'
auf den Scheitel steil mir herab.
Linde Kühlung
erkies' ich unter der Linde!

\direct{Er streckt sich unter der Linde
aus und blickt wieder die Zweige
hinauf}

Noch einmal, liebes Vöglein,
da wir so lang
lästig gestört,
lauscht' ich gerne deinem Sange:
auf dem Zweige seh' ich
wohlig dich wiegen;
zwitschernd umschwirren
dich Brüder und Schwestern,
umschweben dich lustig und lieb!
Doch ich bin so allein,
hab' nicht Brüder noch Schwestern:
meine Mutter schwand,
mein Vater fiel:
nie sah sie der Sohn!
Mein einz'ger Gesell
war ein garstiger Zwerg;
Güte zwang
uns nie zu Liebe;
listige Schlingen
warf mir der Schlaue;
nun musst' ich ihn gar erschlagen!

\direct{Er blickt schmerzlich bewegt
wieder nach den Zweigen auf}

Freundliches Vöglein,
dich frage ich nun:
gönntest du mir
wohl ein gut Gesell?
Willst du mir das Rechte raten?
Ich lockte so oft,
und erlost' es mir nie:
Du, mein Trauter,
träfst es wohl besser,
so recht ja rietest du schon.
Nun sing'! Ich lausche dem Gesang.

\speaker{Eine Waldvogel (Stimme)}
Hei! Siegfried erschlug
nun den schlimmen Zwerg!
Jetzt wüsst' ich ihm noch
das herrlichste Weib:
auf hohem Felsen sie schläft,
Feuer umbrennt ihren Saal:
durchschritt' er die Brunst,
weckt' er die Braut,
Brünnhilde wäre dann sein!

\Siegfriedspeaks

\direct{fährt mit jäher Heftigkeit
vom Sitze auf}

O holder Sang!
Süssester Hauch!
Wie brennt sein Sinn
mir sehrend die Brust!
Wie zückt er heftig
zündend mein Herz!
Was jagt mir so jach
durch Herz und Sinne?
Sag' es mir, süsser Freund!

\direct{er lauscht}

\speaker{Eine Waldvogel (Stimme)}
Lustig im Leid
sing' ich von Liebe;
wonnig aus Weh
web' ich mein Lied:
nur Sehnende kennen den Sinn!

\Siegfriedspeaks

Fort jagt mich's
jauchzend von hinnen,
fort aus dem Wald auf den Fels!
Noch einmal sage mir,
holder Sänger:
werd' ich das Feuer durchbrechen?
Kann ich erwecken die Braut?

\direct{Siegfried lauscht noch mal}

\speaker{Eine Waldvogel (Stimme)}
Die Braut gewinnt,
Brünnhilde erweckt
ein Feiger nie:
nur wer das Fürchten nicht kennt!

\Siegfriedspeaks

\direct{lacht auf vor Entzücken}

Der dumme Knab',
der das Fürchten nicht kennt,
mein Vöglein, der bin ja ich!
Noch heute gab ich
vergebens mir Müh,
das Fürchten von Fafner zu lernen:
nun brenn' ich vor Lust,
es von Brünnhilde zu wissen!
Wie find' ich zum Felsen den Weg?

\direct{Der Vogel flattert auf, kreist über Siegfried und fliegt ihm zögernd voran jauchzend}

So wird mir der Weg gewiesen:
wohin du flatterst
folg' ich dem Flug!

\direct{Er läuft dem Vogel, welcher ihn neckend einige Zeitlang
unstet nach verschiedenen Richtungen hinleitet, nach und folgt
ihm endlich, als dieser mit einer bestimmten Wendung
nach dem Hintergrunde davonfliegt.}

\act

\scene

\StageDir{Wilde Gegend

am Fusse eines Felsenberges, welcher links nach hinten steil aufsteigt. Nacht, Sturm und Wetter, Blitz und heftiger Donner, welch letzterer dann schweigt, während Blitze noch längere Zeit die Wolken durchkreuzen.}

\Wandererspeaks

\direct{schreitet entschlossen auf ein
gruftähnliches Höhlentor in einem Felsen
des Vordergrundes zu und nimmt dort,
auf seinen Speer gestützt, eine Stellung ein,
während er das Folgende dem Eingange
der Höhle zu ruft}

Wache, Wala!
Wala! Erwach'!
Aus langem Schlaf
weck' ich dich Schlummernde wach.
Ich rufe dich auf:
Herauf! Herauf!
Aus nebliger Gruft,
aus nächtigem Grunde herauf!
Erda! Erda!
Ewiges Weib!
Aus heimischer Tiefe
tauche zur Höh!
Dein Wecklied sing' ich,
dass du erwachest;
aus sinnendem Schlafe
weck' ich dich auf.
Allwissende!
Urweltweise!
Erda! Erda!
Ewiges Weib!
Wache, erwache,
du Wala! Erwache!

\direct{Die Höhlengruft erdämmert. Bläulicher Lichtschein: von ihm beleuchtet steigt mit dem Folgenden Erda sehr allmählich aus der Tiefe auf. Sie erscheint wie von Reif bedeckt: Haar und Gewand werfen einen glitzernden Schimmer von sich}

\Erdaspeaks

Stark ruft das Lied;
kräftig reizt der Zauber.
Ich bin erwacht
aus wissendem Schlaf:
wer scheucht den Schlummer mir?

\Wandererspeaks

Der Weckrufer bin ich,
und Weisen üb' ich,
dass weithin wache,
was fester Schlaf umschliesst.
Die Welt durchzog ich,
wanderte viel,
Kunde zu werben,
urweisen Rat zu gewinnen.
Kundiger gibt es
keine als dich;
bekannt ist dir,
was die Tiefe birgt,
was Berg und Tal,
Luft und Wasser durchwebt.
Wo Wesen sind,
wehet dein Atem;
wo Hirne sinnen,
haftet dein Sinn:
alles, sagt man,
sei dir bekannt.
Dass ich nun Kunde gewänne,
weck' ich dich aus dem Schlaf!

\Erdaspeaks

Mein Schlaf ist Träumen,
mein Träumen Sinnen,
mein Sinnen Walten des Wissens.
Doch wenn ich schlafe,
wachen Nornen:
sie weben das Seil
und spinnen fromm, was ich weiss.
Was frägst du nicht die Nornen?

\Wandererspeaks

Im Zwange der Welt
weben die Nornen:
sie können nichts wenden noch wandeln.
Doch deiner Weisheit
dankt' ich den Rat wohl,
wie zu hemmen ein rollendes Rad?

\Erdaspeaks

Männertaten
umdämmern mir den Mut:
mich Wissende selbst
bezwang ein Waltender einst.
Ein Wunschmädchen
gebar ich Wotan:
der Helden Wal
hiess für sich er sie küren.
Kühn ist sie
und weise auch:
was weckst du mich
und frägst um Kunde
nicht Erdas und Wotans Kind?

\Wandererspeaks

Die Walküre meinst du,
Brünnhild', die Maid?
Sie trotzte dem Stürmebezwinger,
wo er am stärksten selbst sich bezwang:
was den Lenker der Schlacht
zu tun verlangte,
doch dem er wehrte
---zuwider sich selbst---,
allzu vertraut
wagte die Trotzige,
das für sich zu vollbringen,
Brünnhild' in brennender Schlacht.
Streitvater
strafte die Maid:
in ihr Auge drückte er Schlaf;
auf dem Felsen schläft sie fest:
erwachen wird
die Weihliche nur,
um einen Mann zu minnen als Weib.
Frommten mir Fragen an sie?

\Erdaspeaks

\direct{ist in Sinnen versunken und
beginnt erst nach längerem
Schweigen}

Wirr wird mir,
seit ich erwacht:
wild und kraus
kreist die Welt!
Die Walküre,
der Wala Kind,
büsst' in Banden des Schlafs,
als die wissende Mutter schlief?
Der den Trotz lehrte,
straft den Trotz?
Der die Tat entzündet,
zürnt um die Tat?
Der die Rechte wahrt,
der die Eide hütet,
wehret dem Recht,
herrscht durch Meineid?
Lass mich wieder hinab!
Schlaf verschliesse mein Wissen!

\Wandererspeaks

Dich, Mutter, lass' ich nicht ziehn,
da des Zaubers mächtig ich bin.
Urwissend
stachest du einst
der Sorge Stachel
in Wotans wagendes Herz:
mit Furcht vor schmachvoll
feindlichem Ende
füllt' ihn dein Wissen,
dass Bangen band seinen Mut.
Bist du der Welt
weisestes Weib,
sage mir nun:
wie besiegt die Sorge der Gott?

\Erdaspeaks

Du bist nicht
was du dich nennst!
Was kamst du, störrischer Wilder,
zu stören der Wala Schlaf?

\Wandererspeaks

Du bist nicht,
was du dich wähnst!
Urmütter-Weisheit
geht zu Ende:
dein Wissen verweht
vor meinem Willen.
Weisst du, was Wotan will?

\direct{Langes Schweigen}

Dir Unweisen
ruf' ich ins Ohr,
dass sorglos ewig du nun schläfst!
Um der Götter Ende
grämt mich die Angst nicht,
seit mein Wunsch es will!
Was in des Zwiespalts wildem Schmerze
verzweifelnd einst ich beschloss,
froh und freudig
führe frei ich nun aus.
Weiht' ich in wütendem Ekel
des Niblungen Neid schon die Welt,
dem herrlichsten Wälsung
weis' ich mein Erbe nun an.
Der von mir erkoren,
doch nie mich gekannt,
ein kühnester Knabe,
bar meines Rates,
errang des Niblungen Ring:
liebesfroh,
ledig des Neides,
erlahmt an dem Edlen
Alberichs Fluch;
denn fremd bleibt ihm die Furcht.
Die du mir gebarst,
Brünnhild',
weckt sich hold der Held:
wachend wirkt
dein wissendes Kind
erlösende Weltentat.
Drum schlafe nun du,
schliesse dein Auge;
träumend erschau' mein Ende!
Was jene auch wirken,
dem ewig Jungen
weicht in Wonne der Gott.
Hinab denn, Erda!
Urmütterfurcht!
Ursorge!
Hinab! Hinab,
zu ewigem Schlaf!

\direct{Nachdem Erda bereits die Augen geschlossen hat und allmählich tiefer versunken ist, verschwindet sie jetzt gänzlich; auch die Höhle ist jetzt wiederum durchaus verfinstert. Monddämmerung erhellt die Bühne, der Sturm hat aufgehört}

\scene

\StageDir{Der Wanderer ist dicht an die Höhle getreten und lehnt sich dann mit dem Rücken an das Gestein derselben, das Gesicht der Szene zugewandt}

\Wandererspeaks

Dort seh' ich Siegfried nahn.

\direct{Er verbleibt in seiner Stellung an der Höhle.
Siegfrieds Waldvogel flattert dem Vordergrunde zu.
Plötzlich hält der Vogel in seiner Richtung ein, flattert
ängstlich hin und her und verschwindet hastig
dem Hintergrunde zu}

\Siegfriedspeaks

\direct{tritt rechts im Vordergrunde
auf und hält an}

Mein Vöglein schwebte mir fort!
Mit flatterndem Flug
und süssem Sang
wies es mich wonnig des Wegs:
nun schwand es fern mir davon!
Am besten find' ich mir
selbst nun den Berg:
wohin mein Führer mich wiess,
dahin wandr' ich jetzt fort.

\direct{Er schreitet weiter nach hinten}

\Wandererspeaks

\direct{in seiner Stellung an der
Höhle verbleibend}

Wohin, Knabe,
heisst dich dein Weg?

\Siegfriedspeaks

\direct{hält an und wendet sich um}

Da redet's ja:
wohl rät das mir den Weg.

\direct{Er tritt dem Wanderer näher}

Einen Felsen such' ich,
von Feuer ist der umwabert:
dort schläft ein Weib,
das ich wecken will.

\Wandererspeaks

Wer sagt' es dir,
den Fels zu suchen?
Wer, nach der Frau dich zu sehnen?

\Siegfriedspeaks

Mich wies ein singend
Waldvöglein:
das gab mir gute Kunde.

\Wandererspeaks

Ein Vöglein schwatzt wohl manches;
kein Mensch doch kann's verstehn.
Wie mochtest du Sinn
dem Sang entnehmen?

\Siegfriedspeaks

Das wirkte das Blut
eines wilden Wurms,
der mir vor Neidhöhl' erblasste:
kaum netzt' es zündend
die Zunge mir,
da verstand ich der Vöglein Gestimm'.

\Wandererspeaks

Erschlugst den Riesen du,
wer reizte dich,
den starken Wurm zu bestehn?

\Siegfriedspeaks

Mich führte Mime,
ein falscher Zwerg;
das Fürchten wollt' er mich lehren:
zum Schwertstreich aber,
der ihn erschlug,
reizte der Wurm mich selbst;
seinen Rachen riss er mir auf.

\Wandererspeaks

Wer schuf das Schwert
so scharf und hart,
dass der stärkste Feind ihm fiel?

\Siegfriedspeaks

Das schweisst' ich mir selbst,
da's der Schmied nicht konnte:
schwertlos noch wär' ich wohl sonst.

\Wandererspeaks

Doch, wer schuf
die starken Stücken,
daraus das Schwert du dir geschweisst?

\Siegfriedspeaks

Was weiss ich davon?
Ich weiss allein,
dass die Stücke mir nichts nützten,
schuf ich das Schwert mir nicht neu.

\Wandererspeaks

\direct{bricht in ein freudig
gemütliches Lachen aus}

Das mein' ich wohl auch!

\direct{Er betrachtet Siegfried
wohlgefällig}

\Siegfriedspeaks

\direct{verwundert}

Was lachst du mich aus?
Alter Frager!
Hör' einmal auf;
lass mich nicht länger hier schwatzen!
Kannst du den Weg
mir weisen, so rede:
vermagst du's nicht,
so halte dein Maul!

\Wandererspeaks

Geduld, du Knabe!
Dünk' ich dich alt,
so sollst du Achtung mir bieten.

\Siegfriedspeaks

Das wär' nicht übel!
Solang' ich lebe,
stand mir ein Alter
stets im Wege;
den hab' ich nun fortgefegt.
Stemmst du dort länger
steif dich mir entgegen,
sieh dich vor, sag' ich,

\direct{mit entsprechender Gebärde}

dass du wie Mime nicht fährst!

\direct{Er tritt noch näher an
den Wanderer heran}

Wie siehst du denn aus?
Was hast du gar
für 'nen grossen Hut?
Warum hängt er dir so ins Gesicht?

\Wandererspeaks

\direct{immer ohne seine
Stellung zu verlassen}

Das ist so Wand'rers Weise,
wenn dem Wind entgegen er geht.

\Siegfriedspeaks

\direct{immer näher ihn betrachtend}

Doch darunter fehlt dir ein Auge!
Das schlug dir einer
gewiss schon aus,
dem du zu trotzig
den Weg vertratst?
Mach dich jetzt fort,
sonst könntest du leicht
das andere auch noch verlieren.

\Wandererspeaks

Ich seh', mein Sohn,
wo du nichts weisst,
da weisst du dir leicht zu helfen.
Mit dem Auge,
das als andres mir fehlt,
erblickst du selber das eine,
das mir zum Sehen verblieb.

\Siegfriedspeaks

\direct{der sinnend zugehört hat,
bricht jetzt unwillkürlich
in helles Lachen aus}

Zum Lachen bist du mir lustig!
Doch hör', nun schwatz' ich nicht länger:
geschwind, zeig' mir den Weg,
deines Weges ziehe dann du;
zu nichts andrem
acht' ich dich nütz':
drum sprich, sonst spreng' ich dich fort!

\Wandererspeaks

\direct{weich}

Kenntest du mich,
kühner Spross,
den Schimpf spartest du mir!
Dir so vertraut,
trifft mich schmerzlich dein Dräuen.
Liebt' ich von je
deine lichte Art,
Grauen auch zeugt' ihr
mein zürnender Grimm.
Dem ich so hold bin,
Allzuhehrer,
heut' nicht wecke mir Neid:
er vernichtete dich und mich!

\Siegfriedspeaks

Bleibst du mir stumm,
störrischer Wicht?
Weich' von der Stelle,
denn dorthin, ich weiss,
führt es zur schlafenden Frau.
So wies es mein Vöglein,
das hier erst flüchtig entfloh.

\direct{Es wird schnell wieder ganz finster}

\Wandererspeaks

\direct{in Zorn ausbrechend und
in gebieterischer Stellung}

Es floh dir zu seinem Heil!
Den Herrn der Raben
erriet es hier:
weh' ihm, holen sie's ein!
Den Weg, den es zeigte,
sollst du nicht ziehn!

\Siegfriedspeaks

\direct{tritt mit Verwunderung in
trotziger Stellung zurück}

Hoho! Du Verbieter!
Wer bist du denn,
dass du mir wehren willst?

\Wandererspeaks

Fürchte des Felsens Hüter!
Verschlossen hält
meine Macht die schlafende Maid:
wer sie erweckte,
wer sie gewänne,
machtlos macht' er mich ewig!
Ein Feuermeer
umflutet die Frau,
glühende Lohe
umleckt den Fels:
wer die Braut begehrt,
dem brennt entgegen die Brunst.

\direct{Er winkt mit dem Speere
nach der Felsenhöhe}

Blick' nach der Höh'!
Erlugst du das Licht?
Es wächst der Schein,
es schwillt die Glut;
sengende Wolken,
wabernde Lohe
wälzen sich brennend
und prasselnd herab:
ein Lichtmeer
umleuchtet dein Haupt:

\direct{Mit wachsender Helle zeigt
sich von der Höhe des Felsens
her ein wabernder Feuerschein}

bald frisst und zehrt dich
zündendes Feuer.
Zurück denn, rasendes Kind!

\Siegfriedspeaks

Zurück, du Prahler, mit dir!

\direct{Er schreitet weiter, der Wanderer
stellt sich ihm entgegen}

Dort, wo die Brünste brennen,
zu Brünnhilde muss ich dahin!

\Wandererspeaks

Fürchtest das Feuer du nicht,

\direct{den Speer vorhaltend}

so sperre mein Speer dir den Weg!
Noch hält meine Hand
der Herrschaft Haft:
das Schwert, das du schwingst,
zerschlug einst dieser Schaft:
noch einmal denn
zerspring' es am ew'gen Speer!

\direct{Er streckt den Speer vor}

\Siegfriedspeaks

\direct{das Schwert ziehend}

Meines Vaters Feind!
Find' ich dich hier?
Herrlich zur Rache
geriet mir das!
Schwing' deinen Speer:
in Stücken spalt' ihn mein Schwert!

\direct{Er haut dem Wanderer mit einem Schlage den Speer
in zwei Stücken; ein Blitzstrahl fährt daraus nach der
Felsenhöhe zu, wo von nun an der bisher mattere Schein
in immer helleren Feuerflammen zu lodern beginnt. Starker
Donner, der schnell sich abschwächt, begleitet den Schlag.
Die Speerstücken rollen zu des Wanderers Füssen.
Er rafft sie ruhig auf}

\Wandererspeaks

\direct{zurückweichend}

Zieh hin! Ich kann dich nicht halten!

\direct{Er verschwindet plötzlich
in völliger Finsternis}

\Siegfriedspeaks

Mit zerfocht'ner Waffe
wich mir der Feige?

\direct{Die wachsende Helle der immer
tiefer sich senkenden Feuerwolken
trifft Siegfrieds Blick}

Ha! Wonnige Glut!
Leuchtender Glanz!
Strahlend nun offen
steht mir die Strasse.
Im Feuer mich baden!
Im Feuer zu finden die Braut
Hoho! Hahei!
Jetzt lock' ich ein liebes Gesell!

\direct{Siegfried setzt sein Horn an und stürzt, seine Lockweise
blasend, sich in das wogende Feuer, welches sich,
von der Höhe herabdringend, nun auch über den
Vordergrund ausbreitet. Siegfried, den man bald nicht mehr erblickt,
scheint sich nach der Höhe zu entfernen. Hellstes Leuchten
der Flammen. Danach beginnt die Glut zu erbleichen und
löst sich allmählich in ein immer feineres, wie durch
die Morgenröte beleuchtetes Gewölk auf}

\scene

\StageDir{Das immer zarter gewordene Gewölk hat sich in einen feinen Nebelschleier von rosiger Färbung aufgelöst und zerteilt sich nun in der Weise, dass der Duft sich gänzlich nach oben verzieht und endlich nur noch den heiteren, blauen Tageshimmel erblicken lässt, während am Saume der nun sichtbar werdenden Felsenhöhe ganz die gleiche Szene wie im dritten Aufzug der ``Walküre'' ein morgenrötlicher Nebelschleier haften bleibt, welcher zugleich an die in der Tiefe noch lodernde Zauberlohe erinnert. Die Anordnung der Szene ist durchaus dieselbe wie am Schlusse der ``Walküre'': im Vordergrunde, unter der breitästigen Tanne, liegt Brünnhilde in vollständiger, glänzender Panzerrüstung, mit dem Helm auf dem Haupte, den langen Schild über sich gedeckt, in tiefem Schlafe}

\Siegfriedspeaks

\direct{gelangt von aussen her auf den
felsigen Saum der Höhe und zeigt sich
dort zuerst nur mit dem Oberleibe: so
blickt er lange staunend um sich}

Selige Öde
auf sonniger Höh'!

\direct{Er steigt vollends herauf und
betrachtet, auf einem Felsensteine
des hinteren Abhanges stehend,
mit Verwunderung die Szene.
Er blickt zur Seite in den Tann
und schreitet etwas vor}

Was ruht dort schlummernd
im schattigen Tann?
Ein Ross ist's,
rastend in tiefem Schlaf!

\direct{Langsam näher kommend,
hält er verwundert an, als er noch
aus einiger Entfernung Brünnhildes
Gestalt wahrnimmt}

Was strahlt mir dort entgegen?
Welch glänzendes Stahlgeschmeid?
Blendet mir noch
die Lohe den Blick?

\direct{Er tritt näher hinzu}

Helle Waffen!
Heb' ich sie auf?

\direct{Er hebt den Schild ab und erblickt
Brünnhildes Gestalt, während ihr
Gesicht jedoch noch zum grossen
Teil vom Helm verdeckt ist}

Ha! In Waffen ein Mann:
wie mahnt mich wonnig sein Bild!
Das hehre Haupt
drückt wohl der Helm?
Leichter würd' ihm,
löst' ich den Schmuck.

\direct{Vorsichtig löst er den Helm
und hebt ihn der Schlafenden
vom Haupte ab: langes lockiges
Haar bricht hervor. Siegfried erschrickt}

Ach! Wie schön!

\direct{Er bleibt in den Anblick versunken}

Schimmernde Wolken
säumen in Wellen
den hellen Himmelssee;
leuchtender Sonne
lachendes Bild
strahlt durch das Wogengewölk!

\direct{Er neigt sich tiefer zu der
Schlafenden hinab}

Von schwellendem Atem
schwingt sich die Brust:
brech' ich die engende Brünne?

\direct{Er versucht mit grosser
Behutsamkeit die Brünne zu lösen}

Komm, mein Schwert,
schneide das Eisen!

\direct{Er zieht sein Schwert, durchschneidet
mit zarter Vorsicht die Panzerringe zu
beiden Seiten der ganzen Rüstung
und hebt dann die Brünne und die
Schienen ab, so dass nun Brünnhilde
in einem weichen weiblichen Gewande
vor ihm liegt. Er fährt erschreckt
und staunend auf}

Das ist kein Mann!

\direct{Er starrt mit höchster Aufgeregtheit
auf die Schlafende hin}

Brennender Zauber
zückt mir ins Herz;
feurige Angst
fasst meine Augen:
mir schwankt und schwindelt der Sinn!

\direct{Er gerät in höchste Beklemmung}

Wen ruf' ich zum Heil,
dass er mir helfe?
Mutter! Mutter!
Gedenke mein!

\direct{Er sinkt, wie ohnmächtig, an Brünnhildes
Busen. Langes Schweigen. Dann fährt er
seufzend auf}

Wie weck' ich die Maid,
dass sie ihr Auge mir öffne?
Das Auge mir öffne?
Blende mich auch noch der Blick?
Wagt' es mein Trotz?
Ertrüg' ich das Licht?
Mir schwebt und schwankt
und schwirrt es umher!
Sehrendes Sehnen
zehrt meine Sinne;
am zagenden Herzen
zittert die Hand!
Wie ist mir Feigem?
Ist dies das Fürchten?
O Mutter! Mutter!
Dein mutiges Kind!
Im Schlafe liegt eine Frau:
die hat ihn das Fürchten gelehrt!
Wie end' ich die Furcht?
Wie fass' ich Mut?
Dass ich selbst erwache,
muss die Maid mich erwecken!

\direct{Indem er sich der Schlafenden von
neuem nähert, wird er wieder von
zarteren Empfindungen an ihren
Anblick gefesselt. Er neigt sich
tiefer hinab}

Süss erbebt mir
ihr blühender Mund.
Wie mild erzitternd
mich Zagen er reizt!
Ach! Dieses Atems
wonnig warmes Gedüft!

\direct{wie in Verzweiflung}

Erwache! Erwache!
Heiliges Weib!

\direct{Er starrt auf sie hin}

Sie hört mich nicht.

\direct{gedehnt mit gepresstem,
drängendem Ausdruck}

So saug' ich mir Leben
aus süssesten Lippen,
sollt' ich auch sterbend vergeh'n!

\direct{Er sinkt, wie ersterbend, auf die Schlafende und heftet mit geschlossenen Augen seine Lippen auf ihren Mund. Brünnhilde schlägt die Augen auf. Siegfried fährt auf und bleibt vor ihr stehen. Brünnhilde richtet sich langsam zum Sitze auf. Sie begrüsst mit feierlichen Gebärden der erhobenen Arme ihre Rückkehr zur Wahrnehmung der Erde und des Himmels}

\Brunnhildespeaks

Heil dir, Sonne!
Heil dir, Licht!
Heil dir, leuchtender Tag!
Lang war mein Schlaf;
ich bin erwacht.
Wer ist der Held,
der mich erweckt'?

\Siegfriedspeaks

\direct{von ihrem Blicke und ihrer
Stimme feierlich ergriffen,
steht wie festgebannt}

Durch das Feuer drang ich,
das den Fels umbrann;
ich erbrach dir den festen Helm:
Siegfried bin ich,
der dich erweckt'.

\Brunnhildespeaks

\direct{hoch aufgerichtet sitzend}

Heil euch, Götter!
Heil dir, Welt!
Heil dir, prangende Erde!
Zu End' ist nun mein Schlaf;
erwacht, seh' ich:
Siegfried ist es,
der mich erweckt!

\Siegfriedspeaks

\direct{in erhabenste Verzückung
ausbrechend}

O Heil der Mutter,
die mich gebar;
Heil der Erde,
die mich genährt!
Dass ich das Aug' erschaut,
das jetzt mir Seligem lacht!

\Brunnhildespeaks

\direct{mit grösster Bewegtheit}

O Heil der Mutter,
die dich gebar!
Heil der Erde,
die dich genährt!
Nur dein Blick durfte mich schau'n,
erwachen durft' ich nur dir!

\direct{Beide bleiben voll strahlenden Entzückens
in ihren gegenseitigen Anblick verloren}

O Siegfried! Siegfried!
Seliger Held!
Du Wecker des Lebens,
siegendes Licht!
O wüsstest du, Lust der Welt,
wie ich dich je geliebt!
Du warst mein Sinnen,
mein Sorgen du!
Dich Zarten nährt' ich,
noch eh' du gezeugt;
noch eh' du geboren,
barg dich mein Schild:
so lang' lieb' ich dich, Siegfried!

\Siegfriedspeaks

\direct{leise und schüchtern}

So starb nicht meine Mutter?
Schlief die minnige nur?

\Brunnhildespeaks

\direct{lächelnd, freundlich die
Hand nach ihm ausstreckend}

Du wonniges Kind!
Deine Mutter kehrt dir nicht wieder.
Du selbst bin ich,
wenn du mich Selige liebst.
Was du nicht weisst,
weiss ich für dich;
doch wissend bin ich
nur---weil ich dich liebe!
O Siegfried! Siegfried!
Siegendes Licht!
Dich liebt' ich immer;
denn mir allein
erdünkte Wotans Gedanke
der Gedanke, den ich nie
nennen durfte;
den ich nicht dachte,
sondern nur fühlte;
für den ich focht,
kämpfte und stritt;
für den ich trotzte
dem, der ihn dachte;
für den ich büsste,
Strafe mich band,
weil ich nicht ihn dachte
und nur empfand!
Denn der Gedanke
dürftest du's lösen!
mir war er nur Liebe zu dir!

\Siegfriedspeaks

Wie Wunder tönt,
was wonnig du singst;
doch dunkel dünkt mich der Sinn.
Deines Auges Leuchten
seh' ich licht;
deines Atems Wehen
fühl' ich warm;
deiner Stimme Singen
hör' ich süss:
doch was du singend mir sagst,
staunend versteh' ich's nicht.
Nicht kann ich das Ferne
sinnig erfassen,
wenn alle Sinne
dich nur sehen und fühlen!
Mit banger Furcht
fesselst du mich:
du Einz'ge hast
ihre Angst mich gelehrt.
Den du gebunden
in mächtigen Banden,
birg meinen Mut mir nicht mehr!

\direct{Er verweilt in grosser Aufregung,
sehnsuchtsvollen Blick auf sie heftend}

\Brunnhildespeaks

\direct{wendet sanft das Haupt zur Seite
und richtet ihren Blick nach dem Tann}

Dort seh' ich Grane,
mein selig Ross:
wie weidet er munter,
der mit mir schlief!
Mit mir hat ihn Siegfried erweckt.

\Siegfriedspeaks

\direct{in der vorigen Stellung verbleibend}

Auf wonnigem Munde
weidet mein Auge:
in brünstigem Durst
doch brennen die Lippen,
dass der Augen Weide sie labe!

\Brunnhildespeaks

\direct{deutet ihm mit der Hand nach
ihren Waffen, die sie gewahrt}

Dort seh' ich den Schild,
der Helden schirmte;
dort seh' ich den Helm,
der das Haupt mir barg:
er schirmt, er birgt mich nicht mehr!

\Siegfriedspeaks

Eine selige Maid
versehrte mein Herz;
Wunden dem Haupte
schlug mir ein Weib:
ich kam ohne Schild und Helm!

\Brunnhildespeaks

\direct{mit gesteigertem Wehmut}

Ich sehe der Brünne
prangenden Stahl:
ein scharfes Schwert
schnitt sie entzwei;
von dem maidlichen Leibe
löst' es die Wehr:
ich bin ohne Schutz und Schirm,
ohne Trutz ein trauriges Weib!

\Siegfriedspeaks

Durch brennendes Feuer
fuhr ich zu dir!
Nicht Brünne noch Panzer
barg meinen Leib:
nun brach die Lohe
mir in die Brust.
Es braust mein Blut
in blühender Brunst;
ein zehrendes Feuer
ist mir entzündet:
die Glut, die Brünnhilds
Felsen umbrann,
die brennt mir nun in der Brust!
O Weib, jetzt lösche den Brand!
Schweige die schäumende Glut!

\direct{Er hat sie heftig umfasst: sie springt auf,
wehrt ihm mit der höchsten Kraft der Angst,
und entflieht nach der anderen Seite}

\Brunnhildespeaks

Kein Gott nahte mir je!
Der Jungfrau neigten
scheu sich die Helden:
heilig schied sie aus Walhall!
Wehe! Wehe!
Wehe der Schmach,
der schmählichen Not!
Verwundet hat mich,
der mich erweckt!
Er erbrach mir Brünne und Helm:
Brünnhilde bin ich nicht mehr!

\Siegfriedspeaks

Noch bist du mir
die träumende Maid:
Brünnhildes Schlaf
brach ich noch nicht.
Erwache, sei mir ein Weib!

\Brunnhildespeaks

\direct{in Betäubung}

Mir schwirren die Sinne,
mein Wissen schweigt:
soll mir die Weisheit schwinden?

\Siegfriedspeaks

Sangst du mir nicht,
dein Wissen sei
das Leuchten der Liebe zu mir?

\Brunnhildespeaks

\direct{vor sich hinstarrend}

Trauriges Dunkel
trübt meinen Blick;
mein Auge dämmert,
das Licht verlischt:
Nacht wird's um mich.
Aus Nebel und Grau'n
windet sich wütend
ein Angstgewirr:
Schrecken schreitet
und bäumt sich empor!

\direct{Sie birgt heftig die Augen
mit beiden Händen}

\Siegfriedspeaks

\direct{indem er ihr sanft die Hände
von den Augen löst}

Nacht umfängt
gebund'ne Augen.
Mit den Fesseln schwindet
das finstre Grau'n.
Tauch' aus dem Dunkel und sieh:
sonnenhell leuchtet der Tag!

\Brunnhildespeaks

\direct{in höchster Ergriffenheit}

Sonnenhell
leuchtet der Tag meiner Schmach!
O Siegfried! Siegfried!
Sieh' meine Angst!

\direct{Ihre Miene verrät, dass ihr ein
anmutiges Bild vor die Seele tritt,
von welchem ab sie den Blick mit
Sanftmut wieder auf Siegfried richtet}

Ewig war ich,
ewig bin ich,
ewig in süss
sehnender Wonne,
doch ewig zu deinem Heil!
O Siegfried! Herrlicher!
Hort der Welt!
Leben der Erde!
Lachender Held!
Lass, ach lass,
lasse von mir!
Nahe mir nicht
mit der wütenden Nähe!
Zwinge mich nicht
mit dem brechenden Zwang,
zertrümmre die Traute dir nicht!
Sahst du dein Bild
im klaren Bach?
Hat es dich Frohen erfreut?
Rührtest zur Woge
das Wasser du auf,
zerflösse die klare
Fläche des Bachs:
dein Bild sähst du nicht mehr,
nur der Welle schwankend Gewog'!
So berühre mich nicht,
trübe mich nicht!
Ewig licht
lachst du selig dann
aus mir dir entgegen,
froh und heiter ein Held!
O Siegfried!
Leuchtender Spross!
Liebe dich,
und lasse von mir:
vernichte dein Eigen nicht!

\Siegfriedspeaks

Dich lieb' ich:
o liebtest mich du!
Nicht hab' ich mehr mich:
o, hätte ich dich!
Ein herrlich Gewässer
wogt vor mir;
mit allen Sinnen
seh' ich nur sie,
die wonnig wogende Welle:
Brach sie mein Bild,
so brenn' ich nun selbst,
sengende Glut
in der Flut zu kühlen;
ich selbst, wie ich bin,
spring' in den Bach:
o, dass seine Wogen
mich selig verschlängen,
mein Sehnen schwänd' in der Flut!
Erwache, Brünnhilde!
Wache, du Maid!
Lache und lebe,
süsseste Lust!
Sei mein! Sei mein! Sei mein!

\Brunnhildespeaks

\direct{sehr innig}

O Siegfried! Dein
war ich von je!

\Siegfriedspeaks

\direct{feurig}

Warst du's von je,
so sei es jetzt!

\Brunnhildespeaks

Dein werd' ich
ewig sein!

\Siegfriedspeaks

Was du sein wirst,
sei es mir heut'!
Fasst dich mein Arm,
umschling' ich dich fest;
schlägt meine Brust
brünstig die deine;
zünden die Blicke,
zehren die Atem sich;
Aug' in Auge,
Mund an Mund:
dann bist du mir,
was bang du mir warst und wirst!
Dann brach sich die brennende Sorge,
ob jetzt Brünnhilde mein?

\direct{Er hat sie umfasst}

\Brunnhildespeaks

Ob jetzt ich dein?
Göttliche Ruhe
rast mir in Wogen;
keuschestes Licht
lodert in Gluten:
himmlisches Wissen
stürmt mir dahin,
Jauchzen der Liebe
jagt es davon!
Ob jetzt ich dein?
Siegfried! Siegfried!
Siehst du mich nicht?
Wie mein Blick dich verzehrt,
erblindest du nicht?
Wie mein Arm dich presst,
entbrennst du mir nicht?
Wie in Strömen mein Blut
entgegen dir stürmt,
das wilde Feuer,
fühlst du es nicht?
Fürchtest du, Siegfried,
fürchtest du nicht
das wild wütende Weib?

\direct{Sie umfasst ihn heftig}

\Siegfriedspeaks

\direct{in freudigem Schreck}

Ha!
Wie des Blutes Ströme sich zünden,
wie der Blicke Strahlen sich zehren,
Wie die Arme brünstig sich pressen,
kehrt mir zurück
mein kühner Mut,
und das Fürchten, ach!
Das ich nie gelernt,
das Fürchten, das du
mich kaum gelehrt:
das Fürchten, mich dünkt
ich Dummer vergass es nun ganz!

\direct{Er hat bei den letzten Worten
Brünnhilde unwillkürlich losgelassen}

\Brunnhildespeaks

\direct{im höchsten Liebesjubel wild auflachend}

O kindischer Held!
O herrlicher Knabe!
Du hehrster Taten
töriger Hort!
Lachend muss ich dich lieben,
lachend will ich erblinden,
lachend lass uns verderben
lachend zu Grunde gehn!
Fahr' hin, Walhalls
leuchtende Welt!
Zerfall in Staub
deine stolze Burg!
Leb' wohl, prangende
Götterpracht!
End' in Wonne,
du ewig Geschlecht!
Zerreisst, ihr Nornen,
das Runenseil!
Götterdämm'rung,
dunkle herauf!
Nacht der Vernichtung,
neble herein!
Mir strahlt zur Stunde
Siegfrieds Stern;
er ist mir ewig,
ist mir immer,
Erb' und Eigen,
ein' und all':
leuchtende Liebe,
lachender Tod!

\Siegfriedspeaks

Lachend erwachst
du Wonnige mir:
Brünnhilde lebt,
Brünnhilde lacht!
Heil dem Tage,
der uns umleuchtet!

Heil der Sonne,
die uns bescheint!

Heil dem Licht,
das ser Nacht enttaucht!

Heil der Welt,
der Brünnhilde lebt!
Sie wacht, sie lebt,
sie lacht mir entgegen.
Prangend strahlt
mir Brünnhildes Stern!
Sie ist mir ewig,
ist mir immer,
Erb' und Eigen,
ein' und all':
leuchtende Liebe,
lachender Tod!
\end{drama}
